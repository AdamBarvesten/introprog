\documentclass{slides}

\setbeamertemplate{footline}[frame number]
\title[Kursutvecklingsmöte pgk]{Kursutvecklingsmöte pgk}
\subtitle{Kickoff}
\author{Björn Regnell, Maj Stenmark, Gustav Cedersjö}
\institute{Datavetenskap, LTH}
\date{11 mars, 2016}

\begin{document}

\frame{\titlepage}
\frame{Agenda:\\\tableofcontents}

\section{Introduktion}
\begin{Slide}{Välkommen!}
\Alert{Fantastiskt!} att så många (ca 35) vill vara med och jobba med \Emph{kursutveckling} (VT2016) och/eller \Emph{undervisning} (HT2016) i nya kursen EDAA45 Programmering, grundkurs \Alert{pgk}

\vspace{2em}
\Emph{Ett nytt sätt att jobba:}\\ Kursmaterial utvecklas open source av lärare \Alert{och} studenter 
\end{Slide}

\begin{Slide}{Upprop}
\begin{itemize}
\item Vem är här? (Se länk till google spreadsheet i mejl)
\item Namnbrickor
\end{itemize}
\end{Slide}

\section{Grov tidplan}

\begin{Slide}{Grov tidplan}
\begin{itemize}
\item I \Emph{mars} fokuserar vi på era studier i grunderna i Scala och studier av de verktyg vi kommer använda (latex, git, med flera). De som är DevLead börjar planera/organisera arbetet med labbarna under mars.

\item Från \Emph{april} startar utveckling \& test av labbar, parallellt med fördjupade Scala-studier. Dessa saker ska utvecklas och testas:
\begin{enumerate}
\item en handledning per lab
\item ett givet kodskellett per lab i studenternas workspace
\item hemlig lablösning för lärare: Obs! får ej delas öppet!!
\item mönsterlösningar till övningarna i kompendiet
\end{enumerate}

\item Senast i slutet av \Alert{maj} ska alla labbarna vara färdiga och testade och alla övningarna ha mönsterlösningar.
\end{itemize}
\end{Slide}

\section{Studera Scala}

\begin{Slide}{Studera Scala}
\begin{itemize}
\item Vi använder denna gratis nätbok: \url{https://www.artima.com/pins1ed}
\item Ni läser själva i förväg
\item Alla hjälps åt på lunchmöten att förklara och förstå
\item Bilda gärna studiegrupper att jobba i 
\item Länk till community-dokumentation: \url{http://www.scala-lang.org/documentation/}
\item Länk till scaladoc: \url{http://www.scala-lang.org/api/current/}

\end{itemize}
\end{Slide}

\begin{Slide}{Om pins1ed}
Första upplagan \url{https://www.artima.com/pins1ed}
\begin{itemize}
\item Ingen nybörjarbok: förutsätter kunskap om OO och Java.
\item Täcker bara fram till Scala 2.8, men det gör inte så mycket då vi koncentrerar oss på grunderna och inte det mest avancerade eller nyaste
\item Saker som vi kan stöta på som kommit sedan Scala 2.8: \\ App, implicit classes, value classes, extension methods, string interpolators, additional collection methods, Try, import scala.language.\_
\item Länk:  \href{https://speakerd.s3.amazonaws.com/presentations/2c9a5a600210013266e732699e109fd5/_.pdf }{språkhistorik}
\end{itemize}
\end{Slide}

\begin{Slide}{Hemläxor -- tidplan scala-studier kommande veckor}
\begin{itemize}
\item Hemläxa till mån lunchmöte 14/3: \\ Läs kapitel 1-2. Gå vidare nedan om du har tid/ork.

\item Hemläxa till tors lunchmöte 17/3: \\ Läs kapitel 3-4 i pins1ed. Börja testa övn1.
\item Hemläxa till mån lunchmöte 21/3: \\
Läs kapitel 5-6 i pins1ed. Börja testa lab1 med Kojo.  

\item Hemläxa tors lunchmöte 24/3: \\
Läs kapitel 7-8 i pins1ed. Test av övn1 och lab 1 klart. 
\end{itemize}
\Alert{OBS!} Håll reda på hur lång tid det tar att göra övn1 och lab1. Jag undrar om de är lagom  omfattande för "medelstudeneten".
Länk till kompendium i pdf (uppdateras kontinuerligt):
\url{https://github.com/lunduniversity/introprog/raw/master/compendium/compendium.pdf }
\end{Slide}

\section{Hur ska vi jobba?}
\subsection{Roller}
\begin{Slide}{Roller}\scriptsize
\begin{itemize}\scriptsize
\item \Emph{DevLead} 

\begin{itemize}\scriptsize
\item övergripande kvalitets- och leveransansvar
\item utveckla handledning i latex i kompendiet
\item utveckla kodskellett i studentworkspace
\item utveckla hemlig lösning
\item samordna insatser från TestLead och Dev
\end{itemize}

\item \Emph{TestLead} 
\vspace{-1em}
\begin{multicols}{2}
För labbar:
\begin{itemize}\scriptsize
\item testa labhandledning
\item testa kodskelett i workspace
\item samordna insatser från Test. 
\end{itemize}

\columnbreak

För övningar:
\begin{itemize}\scriptsize
\item utveckla publik lösning i latex i kompendiet
\item samordna insatser av Test.
\end{itemize}
\end{multicols}

\item \Emph{Dev} 
\begin{itemize}\scriptsize
\item Hjälpa DevLead med det som är lämpligt att delegeras
\end{itemize}

\item \Emph{Test}
\begin{itemize}\scriptsize
\item Hjälpa TestLead med det som är lämpligt att delegeras
\end{itemize}
\end{itemize}
\end{Slide}

\subsection{Lunchmöten}
\begin{Slide}{Lunchmöten mån + tors}
\begin{itemize}
\item Vi kommer att ha en stående lunchmötestid måndagar och torsdagar, med start nu på måndag kl 12:15 alltid i E:2116. 
\item \Alert{Boka in så många mån/tor-luncher du kan till 16 juni.}

\item På lunchmötena hjälps vi åt att identifiera saker att lära mer om och vi varvar plenumdiskussioner med genomgångar och arbete i grupp. 

\item Mötesledaren (oftast någon av Gustav, Maj, Björn, ibland ngn av er) kommer styra upp det hela och ibland ge korta föredrag el. lajvkodning som förklarar begrepp i Scala. 

\item Så småningom kommer vi även använda lunchmötena för att synka arbetet med att utveckla labbarna och  övningarna.
\end{itemize}
\end{Slide}

\subsection{Verktyg}
\begin{Slide}{Verktyg vi ska använda}
\begin{itemize}
\item terminalverktyg, t.ex.:\\ scala, scalac, scaladoc, java, javac, jar, javadoc 
\item din favoriteditor, t.ex gedit
\item Kojo: en lättanvänd IDE för Scala \url{}
\item ScalaIDE: en eclipse-plugin för Scala \url{}
\item git för versionshantering \url{}
\item GitHub för öppen central kodlagring \url{}
\item GitLab för hemlig central kodlagring \url{}
\item sbt: scala build tool för att bygga (autoladda jarfiler, automatisera kompilering och exekvering) \url{}
\item latex för labbhandledning och övningsmönsterlösningar \url{}
\end{itemize} 
\pause
\Alert{Hur många vill redan snart jobba på skolans datorer?}
\end{Slide}

\subsection{Nätplatser}
\begin{Slide}{Var? Nätplatser där vi jobbar}
\begin{itemize}
\item \Emph{''upstream''} på GitHub \\ \url{https://github.com/lunduniversity/introprog}
\item Listor med utsedda roller, mgmt, etc. i Google docs \\ \Emph{''pgk-work''} 
(Se länk till google spreadsheet i mejl)
\item \Emph{''closed upstream''} i GitLab\\ hemliga lösningar, grejer bara för lärare \\ \url{https://git.cs.lth.se/bjornregnell/pgk}
\end{itemize}
\end{Slide}

\subsection{Repo}
\begin{Slide}{Hur jobba med git och GitHub: 1) Sätta upp repo}\footnotesize
Det centrala repot \url{https://github.com/lunduniversity/introprog} kallas \Alert{upstream}
\begin{enumerate}
\item Installera git på din egen dator
\item Registrera användare på GitHub om du inte redan gjort det. \\Tips: tänk efter noga vilket användarnamn du ska ha. Förslag om det inte reda är upptaget: \code{fornamnefternamn} i ett ord utan versaler och konstiga tecken.
\item Installera GitHub-klienten på din dator om du kör windows eller mac \\ \url{https://desktop.github.com/} \\
Kör du linux: använd terminalen eller Atlassian SourceTree
\item Om du är DevLead: Gör \Emph{fork} på GitHub av \url{https://github.com/lunduniversity/introprog}
\item Om du är Dev: \\Be din DevLead tillåta din användare att pusha till hens fork på GitHub.
\item Klona DevLeadens fork till din egen dator. \\Då får du en lokal kopia som du jobbar i.
\end{enumerate}
\end{Slide}

\begin{Slide}{Hur jobba med git och GitHub: 2) Arbetsgång}\footnotesize
Det är \Alert{viktigt} att hålla forken uppdaterad mot upstream.\\
Om man ändrar i föråldrade filer kan det bli onödiga \Alert{merge-konflikter}.
\begin{enumerate}
\item Synka din fork \url{https://help.github.com/articles/syncing-a-fork/}
\item Ändra i de filer som ingår i ditt ansvarsområde enligt överenskommelse med DevLead.
\item Gör \code{git commit -am "msg"} (bekräftar din ändringar lokalt)
\item Gör \code{git pull} (hämtar ev ändringar från forken på GitHub)
\item Gör \code{git push} (skickar dina ändringar till forken på GitHub) \\
(I stället för 3-5 ovan kan du klicka ''synch'' i \emph{GitHub-klienten}.)
\item Om du är DevLead:\\ När en dellösning är färdig gör en \Emph{new pull request} på GitHub.  
\end{enumerate}
\end{Slide}

\subsection{Avslutning: att göra härnäst}
\begin{Slide}{Att göra nu}
\begin{enumerate}
\item Fyll i vad du vill jobba med i google docs spreadsheet pgk-work, i fliken ''work'' kolumnerna E,F,G
\item Anmäll ev. intresse för DevLead, TestLead i google docs spreadsheet pgk-work, i fliken ''lead'' längre ner i högerkolumnerna från C28 \& D28 (skriv inte över andras...)
\item Fyll i nuvarande kunskapsnivå i google docs spreadsheet pgk-work, i fliken ''learn''
\item Börja på hemläxorna: läs minst kap 1-2 i helgen
\item Kom på måndag lunch till E:2116 med frågor om allt möjligt men mest på kap 1-2 
\item Boka in detta event den 20 April kl 10-12 och 18-21: \\
\url{http://cs.lth.se/scala20160420}
\end{enumerate}
Frågor??
\end{Slide}


\end{document}

