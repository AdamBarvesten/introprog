%!TEX root = ../compendium.tex

\Exercise{\ExeWeekTWO}

\begin{Goals}
\item 
\end{Goals}

\begin{Preparations}
\item Studera teorin i kapitel~\ref{chapter:W02}.
\item Bekanta dig med grundläggande terminalkommandon; se appendix~\ref{appendix:terminal}. 
\item Bekanta dig med den editor du vill använda; se appendix~\ref{appendix:edit}.
\end{Preparations}

\BasicTasks %%%%%%%%%%%%%%%%

\Task Skapa med hjälp av en editor en fil med namn \texttt{hello-script.scala} som innehåller denna enda rad:
\begin{Code}
println("hej skript")
\end{Code}
Spara filen och kör kommandot \code{scala hello-script.scala} i terminalen:
\begin{REPL}
> scala hello-script.scala
\end{REPL}

\Subtask Vad händer?

\Subtask Ändra i filen så att högerparentesen saknas. Spara och kör skriptfilen igen. Vad händer?

\Task Skapa med hjälp av en editor en fil med namn \texttt{hello-app.scala}.
\begin{REPL}
> gedit hello-app.scala &
\end{REPL}
Skriv dessa rader i filen:


\scalainputlisting{examples/hello-app.scala}

\Subtask Kompilera med \code{scalac hello-app.scala} och kör koden med \code{scala Hello}.
\begin{REPL}
> scalac hello-app.scala
> ls
> scala Hello
\end{REPL}
Vad heter filerna som kompilatorn skapar?


\Subtask\Pen Vilket alternativ går snabbast att köra igång, ett skript eller en kompilerad applikation? Varför? Vilket alternativ kör snabbast när väl exekveringen är igång?

\Subtask Ändra i din kod så att kompilatorn ger följande felmeddelande: \\
\texttt{Missing closing brace}

\Task Skapa med hjälp av en editor en fil med namn \texttt{Hi.java}.
\begin{REPL}
> gedit Hi.java &
\end{REPL}
Skriv dessa rader i filen:


\javainputlisting{examples/Hi.java}

Kompilera med \code{javac Hi.java} och kör koden med \code{java Hi}.
\begin{REPL}
> javac Hi.java
> ls
> java Hi
\end{REPL}

\Subtask Vad heter filen som kompilatorn skapat?

\Subtask Vad händer om källkodsfilen och klassnamnet inte överensstämmer?

\ExtraTasks %%%%%%%%%%%%%%%%%%%

\Task 

\AdvancedTasks %%%%%%%%%%%%%%%%%

\Task 