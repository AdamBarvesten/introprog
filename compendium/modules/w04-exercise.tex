%!TEX encoding = UTF-8 Unicode
%!TEX root = ../compendium.tex

\Exercise{\ExeWeekFOUR}

\begin{Goals}
\item 
\end{Goals}

\begin{Preparations}
\item 
\end{Preparations}

\BasicTasks %%%%%%%%%%%%%%%%

\Task \emph{En enkel datastruktur: tupel.} Du kan samla flera värden i en tupel. Du kommer åt värdena med en metod som har namnet understreck följt av ordningsnumret.
\begin{REPL}
scala> val namn = ("Pippi", "Långstrump")
scala> namn._1
scala> namn._2
scala> println("Förnamn: " + namn._1 + "\nEfternamn:" + namn._2)
\end{REPL}

\Subtask Definiera en oföränderlig variabel med namnet \code{pt} som representerar en punkt med x-koordinaten 15.9 och y-koordinaten 28.9. Använd sedan \code{math.hypt} för att ta reda på avståndet från origo till punkten. Vad blir svaret?

\Subtask Du kan dela upp en tupel i sina beståndsdelar så här:
\begin{REPLnonum}
scala> val (förnamn, efternamn) = ("Ronja", "Rövardotter")
\end{REPLnonum}
Dela upp din punkt \code{pt} i sina beståndsdelar och kalla delarna \code{x} och \code{y}

\Subtask Värdena i en tupel kan ha olika typ. 
\begin{REPLnonum}
scala> val creature = ("Doktor", "Krokodil", 65.0, false)
scala> val (title, name, weight, isHuman)  = creature
\end{REPLnonum}
Vilken typ har 4-tupeln \code{creature} ovan?

\Subtask \label{subtask:tuplecoll} Tupler kan ingå i samlingar.
\begin{REPLnonum}
scala> val pts = Vector((0.0, 0.0), (1.0, 0.0), (1.0, 1.0), (0.0, 1.0)) 
scala> pts.foreach(println)
\end{REPLnonum}
Vilken typ har vektorn \code{pts} ovan?

\Subtask Funktioner kan ta tupler som parametrar.
\begin{REPL}
scala> def length(pt: (Double, Double)) = math.hypot(pt._1, pt._2) 
scala> length((3.0, 4.0))
scala> length(3.0, 4.0)    //kompilatorn lägger till parenteserna innan anrop
\end{REPL}
Applicera funktionen \code{length} ovan på alla tupler i samlingen pts från uppgift \ref{subtask:tuplecoll} med \code{map}. Vad får resultatet för värde och typ?

\Subtask Funktioner kan ge tupler som resultat.
\begin{REPL}
scala> def div(a: Int, b: Int) = (a / b, a % b)
scala> div(10, 3)
scala> (div(9,2), div(10,2))
scala> (div(9,2)._2, div(10,2)._2)
scala> val nOdd = (1 to 10).map(i => div(i, 2)._2).sum
\end{REPL}
Förklara vad som händer ovan. Använd \code{div} ovan för att ta reda på hur många udda tal finns det i intervallet $[1234, 3456]$.

\Subtask En tupel med $n$ värden kallas $n$-tupel. Om man betraktar enhetsvärdet \code{()} som en tupel, vad kan man då kalla detta värde?

\Task \emph{Objekt med attribut (fält).} Ett objekt kan samla data som hör ihop och på så sätt skapa en datastruktur. Data i ett objekt kallas \emph{attribut} eller \emph{fält}, \Eng{field}. Objekt som samlar enbart data kallas även \emph{post} \Eng{record}.
\begin{REPLnonum}
scala> object mittKonto { var saldo = 0; val nummer = 12345L }
\end{REPLnonum}
\Subtask Skriv en sats som sätter in ett slumpmässigt belopp mellan 0 och en miljon på \code{mittKonto} ovan med hjälp av punktnotation och tilldelning. 

\Subtask Vad händer om du försöker ändra attributet \code{nummer}?

\Task \emph{Klass med attribut.} Om du vill ha många objekt av samma typ, kan du använda en \textbf{klass}. På så sätt kan man skapa många datastrukturer av samma typ men med olika innehåll. Man skapar nya objekt med nyckelordet \code{new} följt av klassens namn. Klassen utgör en ''mall'' för objektet som skapas. Ett objekt som skapas med \code{new Klassnamn} kallas även en \textbf{instans} av klassen \code{Klassnamn}. Nedan skapas en datastruktur \code{Konto} som samlar data om ett bankonto. Poster av typen \code{Konto} håller reda på hur mycket pengar det finns på kontot och vilket kontonumret är:

\begin{REPL}
scala> class Konto {
         var saldo = 0
         var nummer = 0L
       }
scala> val k1 = new Konto
scala> val k2 = new Konto
scala> k1.saldo = 1000
scala> k1.nummer = 12345L
scala> k2.saldo = 2000
scala> k2.nummer = 67890L
scala> println("Konto: " + k1.nummer + " Saldo:" k1.saldo)
scala> println("Konto: " + k2.nummer + " Saldo:" k2.saldo)
\end{REPL}

\Subtask\Pen Rita hur minnessituationen ser ut efter att ovan rader har exekverats.

\Subtask\Pen Vad hade det fått för konsekvenser om attributet \code{nummer} vore oföränderligt i klassen ovan? (Jämför med objektet \code{mittKonto}.)


\Task \emph{Klass med attribut som parametrar.} Om man vill ge attributen initialvärden när objektet skaps med \code{new} kan placera attributen i en parameterlista till klassen. Koden körs när objektet skapas och attributen tilldelas sina initaialvärden, kallas \textbf{konstruktor} \Eng{constructor}.

\begin{REPL}
scala> class Konto(var saldo: Int, val nummer: Long)
scala> val k = new Konto(0, 12345L)
scala> println("Konto: " + k.nummer + " Saldo:" k.saldo)
scala> println(k)
scala> k.toString
\end{REPL}

\Subtask Den två sista raderna ovan skriver ut den identifierare som JVM använder för att hålla reda på objektet i sina interna datastrukturer. Vad skrivs ut?

\Subtask Skapa ännu en instans av klassen Konto  med samma saldo och nummer som \code{k} ovan och spara den i \code{val k2} och undersök dess objektidentifierare. Får objekten \code{k} och \code{k2} olika objektidentifierare?

\Subtask Sätt in olika belopp på respektive konto.

\Subtask Vad händer om du försöker ändra attributet \code{nummer}?

\Subtask\Pen Ibland räcker det fint med en tupel, men ofta vill man ha en klass istället. Beskriv några fördelar med en Konto-klassen ovan jämfört med en tupel av typen \code{(Int, Long)}.

\begin{REPLnonum}
scala> var k3 = (0, 12345L)
scala> k3 = (k3._1 + 100, k3._2)
\end{REPLnonum}

\Task \emph{Publikt versus privat attribut.} Man kan förhindra att ett attribut syns utanför klassen med hjälp av nyckelordet \code{private}.  

\begin{REPL}
scala> class Konto1(val nummer: Long){ var saldo = 0 }
scala> val k1 = new Konto1(12345678901L)
scala> k1.nummer
scala> k1.saldo += 1000
scala> class Konto2(val nummer: Long){ private var saldo = 0 }
scala> val k2 = new Konto2(12345678901L)
scala> k2.nummer
scala> k2.saldo += 1000
\end{REPL}

\Subtask Vad händer ovan?

\Subtask Gör en ny version av klassen \code{Konto} enligt nedan:

\begin{Code}
class Konto(val nummer: Long){ 
  private var saldo = 0
  def in(belopp: Int): Unit = {saldo += belopp}
  def ut(belopp: Int): Unit = {saldo -= belopp}
  def show: Unit = 
    println("Konto Nr: " + nummer + " saldo: " + saldo) 
}

object Main {
  def main(args: Array[String]): Unit = {
    val k = new Konto(1234L)
    k.show
    k.in(1000)
    println("Uttag: " + k.ut(500))
    println("Uttag: " + k.ut(1000))
    k.show
  }
}
\end{Code}

\Subtask Spara koden i en fil, kompilera och kör. Testa även vad som händer om du försöker komma åt attributet \code{saldo} i main-metoden med t.ex. \code{println(k.saldo)} eller \code{k.saldo += 1000}. 

\Subtask Vi ska nu förhindra överuttag. Ändra i metoden \code{ut} så att den får signaturen \code{ut(belopp: Int): (Int Int) = ???} och implementera \code{ut} så att den returnerar både beloppet man verkligen kan ta ut och kvarvarande saldo. Om man försöker ta ut mer än det finns på kontot så ska saldot bli 0 och man får bara ut det som finns kvar. Spara, kompilera, kör. 

\Subtask Förbättra metoderna \code{in} och \code{ut} så att man inte kan sätta in eller ta ut negativa belopp.

\Subtask Vad är fördelen med att göra föränderliga attribut privata och bara påverka deras värden indirekt via metoder?

\Task \emph{Vilken typ har ett objekt?} Objektets typ bestäms av klassen. Vid tilldelning måste typerna passa ihop.

\Subtask Vilka rader nedan ger felmeddelande? Hur lyder felmeddelandet?
\begin{REPL}
scala> class Punkt(val x: Double, val y: Double)
scala> val pt: Punkt = new Punkt(10.0, 10.0)
scala> val i: Int = pt.x
scala> val (x: Double, y: Double) = (pt.x, pt.y)
scala> val p: Double = new Punkt(5.0, 5.0)
scala> val p = new Punkt(5.0, 5.0): Double
scala> val p = new Punkt(5.0, 5.0): Punkt
scala> pt: Punkt
\end{REPL}


\Subtask Man kan undersöka om ett objekt är av en viss typ med metoden \\ \code{isInstanceOf[Typnamn]}. Vad ger nedan anrop av metoden \code{isInstanceOf} för värde?
\begin{REPL}
scala> class Punkt(val x: Double, val y: Double)
scala> val pt: Punkt = new Punkt(1.0, 2.0)
scala> pt.isInstanceOf[Punkt]
scala> pt.isInstanceOf[Double]
scala> pt.x.isInstanceOf[Double]
scala> pt.x.isInstanceOf[Int]
scala> pt.x.isInstanceOf[Punkt]
\end{REPL}

\Task \emph{\code{Any}}. Alla klasser är också av typen \code{Any}. Alla klasser får därmed med sig några gemensamma metoder som finns i den fördefinierade klassen \code{Any}, däribland metoderna  \code{isInstanceOf} och \code{toString}.  Vad blir resultatet av respektive rad nedan? Vilken rad ger ett felmeddelande?

\begin{REPL}
scala> class Punkt(val x: Double, val y: Double)
scala> val pt: Punkt = new Punkt(1.0, 2.0)
scala> pt.isInstanceOf[Punkt]
scala> pt.isInstanceOf[Any]
scala> pt.x.toString
scala> println(pt.x)
scala> val a: Any = pt
scala> println(a.x)
scala> a.toString
scala> p1.y.toString
scala> a.y.toString
\end{REPL}

\Task \emph{Byta ut metoden \code{toString}}. I klassen \code{Any} finns metoden \code{toString} som skapar en strängrepresentation av objektet. Du kan byta ut metoden \code{toString} i klassen \code{Any} mot din egen implementation. Man använder nyckelordet \code{override} när man vill byta ut en metodimplementation.

\begin{REPL}
scala> class Punkt(val x: Double, val y: Double) {
         override def toString: String = "[x=" + x + ",y=" + y + "]"
       }
scala> val pt = new Punkt(1.0, 42.0)
scala> pt.toString
scala> println(pt)
\end{REPL}

\Subtask Vad händer egentligen på sista raden ovan?

\Subtask Omdefiniera toString så att den ger en sträng på formen \code{Punkt(1.0, 42.0)}.

\Subtask Vad hände om du utelämnar nyckelordet \code{override} vid omdefiniering?

\Task \emph{Objektfabrik med \code{apply}-metod.} Man kan ordna så att man slipper skriva \code{new} med ett s.k. \emph{fabriksobjekt} \Eng{factory object}. 
\begin{Code}
class Pt(val x: Double, y: Double) {
  override def toString: String = "Pt(x=" + x + ",y=" + y + ")"
}
object Pt { 
  def apply(x: Double, y: Double): Pt = new Pt(x, y)
}
\end{Code}

\Subtask Skriv satser som använder metoden \code{apply} i fabriksobjektet \code{object Pt} för att skapa flera olika punkter.

\Subtask Ge applymetoden default-argument 0.0 för både x och y så att \code{Pt()} skapar en punkt i origo.

\Subtask Skapa en klass \code{Rational} som representerar rationellt tal som en kvot mellan två heltal. Ge klassen två oföränderliga, publika klassparameterattribut med namnen \code{nom} för täljaren och \code{denom} för nämnaren. 

\Subtask Skapa ett fabriksobjekt med en \code{apply}-metod som tar två heltalsparametrar och skapar en instans av klassen \code{Rational}.

\Subtask Skapa olika instanser av din klass \code{Rational} ovan med hjälp av fabriksobjektet.


\Task \emph{Skapa en case-klass.} Med en case-klass får man \code{toString} och fabriksobjekt på köpet. Man behöver inte skriva \code{val} framför klassparametrar i case-klasser; klassparametrar blir publika, oföränderliga attribut automatiskt när man deklarerar en case-klass.

\begin{REPL}
scala> case class Pt(x: Double, y: Double) 
scala> val p = Pt(1.0, 42.0)
scala> p.toString
scala> println(p)
scala> println(Pt(5,6))
\end{REPL}

\Subtask Implementera din klass \code{Rational} från föregående uppgift, men nu som en case-klass.

\Subtask Skapa en case-klass \code{Complex} som representerar komplext tal. Ge klassen två oföränderliga, publika Double-attribut: \code{re} som lagrar realdelen och \code{im} som lagrar imaginärdelen. 

\Task \label{task:point} \emph{Metoder på datastrukturer.} En datastruktur blir mer användbar om det finns metoder som kan användas på datastrukturen. Metoder i Scala kan även ha (vissa) tecken som namn, t.ex. \code{+} enligt nedan.  
\begin{REPL}
scala> case class Point(x: Double, y: Double) {
         def length: Double = math.hypot(x, y)   
         def add(p: Point): Point = Point(x + p.x, y + p.y)
         def +(p: Point): Point = Point(x + p.x, y + p.y)
       }
\end{REPL}

\Subtask Använd metoden \code{lenght} för att ta reda på vad punkten med koordinaterna (3, 4) har för avstånd till origo?

\Subtask Skriv satser som skapar två punkter (3,4) och (5, 6) och låt variablerna p1 och p2 referera till respektive punkt. Låt variabeln p3 bli summan av p1 och p2. Vad får uttrycken p3.x resp. p3.y för värden?

\Task \emph{Operatornotation.} Vid punktnotation på formen: \\ \code{objekt.metod(parameter)} \\ där man anropar en metod med exakt en parameter, kan man skippa punkten och parenteserna och skriva:\\ \code{objekt metod parameter}  \\
Detta förenklade skrivsätt kallas \textbf{operatornotation}.

\Subtask Använd klassen \code{Point} från uppgift \ref{task:point} och prova nedan satser. Vilka rader använder operatortnotation och vilka rader använder punktnotation?
\begin{REPL}
scala> val p1 = Point(3,4)
scala> val p2 = Point(3,4)
scala> p1.add(p2)
scala> p1 add p2
scala> p1.+(p2)
scala> p1 + p2
scala> 42 + 1
scala> 42.+(1)
scala> 1.to(42)
scala> 1 to 42
\end{REPL}

\Subtask Implementera metoderna \code{sub} och \code{-} i klassen \code{Point} och skriv uttryck som kombinerar add och sub, samt + och - i både punktnotation och operatornotation.



\Task \emph{Föränderlighet och oföränderlighet.} Oföränderliga och föränderliga objekt beter sig olika vid tilldelning.  

\Subtask\Pen Innan du kör nedan kod: Försök lista ut vad som kommer att skrivas ut. Rita minnessituationen efter varje tilldelning. 

\begin{Code}
println("\n--- Example 1: mutable value assigmnent")
var x1 = 42
var y1 = x1
x1 = x1 + 42
println(x1)
println(y1)
\end{Code}

\Subtask\Pen Innan du kör nedan kod: Försök lista ut vad som kommer att skrivas ut. Rita minnessituationen efter varje tilldelning.

\begin{Code}
println("\n--- Example 2: mutable object reference assignment")
class MutableInt(private var i: Int) {
  def +(a: Int): MutableInt = { i = i + a; this }
  override def toString: String = i.toString
}
var x2 = new MutableInt(42)
var y2 = x2
x2 = x2 + 42
println(x2)
println(y2)
\end{Code}

\Subtask\Pen Innan du kör nedan kod: Försök lista ut vad som kommer att skrivas ut. Rita minnessituationen efter varje tilldelning.

\begin{Code}
println("\n--- Example 3: immutable object reference assignment")
class ImmutableInt(val i: Int) {
  def +(a: Int): ImmutableInt = new ImmutableInt(i + a) 
  override def toString: String = i.toString
}
var x3 = new ImmutableInt(42)
var y3 = x3
x3 = x3 + 42
println(x3)
println(y3)
\end{Code}

\Subtask\Pen Vad finns det för fördelar med oföränderliga datastrukturer?


\Task \emph{Några samlingar i Scalas standardbibliotek.} En \textbf{samling} är en datastruktur som samlar många objekt av samma typ. I paketet \code{scala.collection} finns en hel massa samlingar som är användbara till olika saker. De har en stor mängd metoder. De olika samlingarna har en stor mängd metoder gemensamt -- därför har man nytta av det man lärt sig om metoderna i en samling när man ska lära sig använda en annan samling. Vi har redan tidigare sett samlingen \code{Vector}. 

\begin{REPL}
scala> val tärningskast = Vector.fill(10000)((math.random * 6 + 1).toInt)
scala> tä   // tryck TAB
scala> tärningskast.  // tryck TAB
\end{REPL}

\Subtask Ungefär hur många metoder finns det som man kan göra på objekt av typen \code{Vector}?

\Subtask Prova nedan metoder:
\begin{REPL}
scala> val tärningskast = Vector.fill(10000)((math.random * 6 + 1).toInt)
scala> tä   // tryck TAB
scala> tärningskast.  // tryck TAB
\end{REPL}


\Subtask Läs om metoderna här: och försök förklara vad som händer ovan.

\Task Jämföra List och Vector.

\Task Mängd.

\Subtask Keno-bollar.


\Task Map. Huvudstäder.


\ExtraTasks %%%%%%%%%%%%%%%%%%%

\Task 

\begin{Code}
class Account(val number: Long, val maxCredit: Int){ 
  private var balance = 0
  
  def deposit(amount: Int): Int = { 
    if (amount > 0) {balance += amount}
    balance
  }
  
  def withdraw(amount: Int): (Int, Int) = if (amount > 0) { 
    val allowedWithdrawal = 
      if (amount < balance + maxCredit) amount 
      else balance + maxCredit 
    balance = balance - allowedWithdrawal
    (allowedWithdrawal, balance)
  } else (0, balance)
  
  def show: Unit = 
    println("Account Nbr: " + number + " balance: " + balance) 
}

object Main {
  def main(args: Array[String]): Unit = {
    ???
  }
}
\end{Code}

\AdvancedTasks %%%%%%%%%%%%%%%%%

\Task Undersök vilka metoder som finns i klassen Any här: \href{http://www.scala-lang.org/api/current/\#scala.Any}{http://www.scala-lang.org/api/current/\#scala.Any}. Prova några av metoderna i REPL.

\Task Jämför tidsprestanda mellan List och Vector vid hantering i början och i slutet. Hur snabbt går nedan på din dator? (Nedan är kört med en i7-4790K CPU @ 4.00GHz.) %sudo lshw -class processor

\begin{REPLnonum}
scala> :paste

def time(n: Int)(block: => Unit): Double =  {
  def now = System.nanoTime
  var timestamp = now
  var sum = 0L
  var i = 0
  while (i < n) {
    block
    sum = sum + (now - timestamp)
    timestamp = now
    i = i + 1
  }
  val average = sum.toDouble / n
  println("Average time: " + average + " ns")
  average
}

scala> val l = List.fill(100000)(math.random)
scala> val v = Vector.fill(100000)(math.random)

scala> (for(i <- 1 to 20) yield time(100000){l.take(10)}).min
Average time: 47.1852 ns
Average time: 41.64156 ns
Average time: 105.53986 ns
Average time: 41.91562 ns
Average time: 41.73559 ns
Average time: 63.17134 ns
Average time: 52.93756 ns
Average time: 41.58533 ns
Average time: 41.68017 ns
Average time: 60.18881 ns
Average time: 41.69867 ns
Average time: 41.60771 ns
Average time: 60.32759 ns
Average time: 41.62671 ns
Average time: 43.88916 ns
Average time: 70.47824 ns
Average time: 41.68801 ns
Average time: 41.67223 ns
Average time: 41.67262 ns
Average time: 102.84893 ns
res85: Double = 41.58533

scala> (for(i <- 1 to 20) yield time(100000){v.take(10)}).min
Average time: 312.67005 ns
Average time: 88.60023 ns
Average time: 73.21829 ns
Average time: 92.148 ns
Average time: 91.01078 ns
Average time: 87.82874 ns
Average time: 74.04663 ns
Average time: 94.16038 ns
Average time: 88.4243 ns
Average time: 105.88971 ns
Average time: 98.85731 ns
Average time: 72.77369 ns
Average time: 97.04337 ns
Average time: 90.01969 ns
Average time: 88.11196 ns
Average time: 75.20191 ns
Average time: 93.72112 ns
Average time: 110.19777 ns
Average time: 132.4207 ns
Average time: 324.28702 ns
res86: Double = 72.77369

scala> (for(i <- 1 to 20) yield time(1000){l.takeRight(10)}).min
Average time: 247365.43 ns
Average time: 212801.958 ns
Average time: 212335.938 ns
Average time: 212313.427 ns
Average time: 212524.963 ns
Average time: 219525.627 ns
Average time: 223059.563 ns
Average time: 222426.504 ns
Average time: 221838.828 ns
Average time: 223268.567 ns
Average time: 222739.402 ns
Average time: 222685.229 ns
Average time: 223122.599 ns
Average time: 222683.921 ns
Average time: 222865.865 ns
Average time: 222889.118 ns
Average time: 223247.135 ns
Average time: 222016.82 ns
Average time: 223040.299 ns
Average time: 222624.613 ns
res87: Double = 212313.427

scala> (for(i <- 1 to 20) yield time(1000){v.takeRight(10)}).min
Average time: 2665.715 ns
Average time: 190634.043 ns
Average time: 773.111 ns
Average time: 509.008 ns
Average time: 519.04 ns
Average time: 418.172 ns
Average time: 365.54 ns
Average time: 409.016 ns
Average time: 353.115 ns
Average time: 503.679 ns
Average time: 421.369 ns
Average time: 388.685 ns
Average time: 461.725 ns
Average time: 390.791 ns
Average time: 381.83 ns
Average time: 309.667 ns
Average time: 372.09 ns
Average time: 312.254 ns
Average time: 323.925 ns
Average time: 310.261 ns
res88: Double = 309.667

\end{REPLnonum}



\Task Studera skillnader i prestanda mellan olika samlingar här: \\ \href{http://docs.scala-lang.org/overviews/collections/performance-characteristics.html}{docs.scala-lang.org/overviews/collections/performance-characteristics.html} \\
(Mer om detta i kommande kurser.)
    