
%!TEX root = ../compendium.tex

\Exercise{\ExeWeekFOUR}

\begin{Goals}
\item 
\end{Goals}

\begin{Preparations}
\item 
\end{Preparations}

\BasicTasks %%%%%%%%%%%%%%%%

\Task \emph{En enkel datastruktur: tupel.} Du kan samla flera värden i en tupel. Du kommer åt värdena med en metod som har namnet understreck följt av ordningsnumret.
\begin{REPL}
scala> val namn = ("Pippi", "Långstrump")
scala> namn._1
scala> namn._2
scala> println("Förnamn: " + namn._1 + "\nEfternamn:" + namn._2)
\end{REPL}

\Subtask Definiera en oföränderlig variabel med namnet \code{pt} som representerar en punkt med x-koordinaten 15.9 och y-koordinaten 28.9. Använd sedan \code{math.hypt} för att ta reda på avståndet från origo till punkten. Vad blir svaret?

\Subtask Du kan dela upp en tupel i sina beståndsdelar så här:
\begin{REPL}
scala> val (förnamn, efternamn) = ("Ronja", "Rövardotter")
\end{REPL}
Dela upp din punkt \code{pt} i sina beståndsdelar och kalla delarna \code{x} och \code{y}

\Subtask Värdena i en tupel kan ha olika typ. 
\begin{REPL}
scala> val creature = ("Doktor", "Krokodil", 65.0, false)
scala> val (title, name, weight, isHuman)  = creature
\end{REPL}
Vilken typ har 4-tupeln \code{creature} ovan?

\Subtask \label{subtask:tuplecoll} Tupler kan ingå i samlingar.
\begin{REPL}
scala> val pts = Vector((0.0, 0.0), (1.0, 0.0), (1.0, 1.0), (0.0, 1.0)) 
scala> pts.foreach(println)
\end{REPL}
Vilken typ har vektorn \code{pts} ovan?

\Subtask Funktioner kan ta tupler som parametrar.
\begin{REPL}
scala> def length(pt: (Double, Double)) = math.hypot(pt._1, pt._2) 
scala> length((3.0, 4.0))
scala> length(3.0, 4.0)    //kompilatorn lägger till parenteserna innan anrop
\end{REPL}
Applicera funktionen \code{length} ovan på alla tupler i samlingen pts från uppgift \ref{subtask:tuplecoll} med \code{map}. Vad får resultatet för värde och typ?

\Subtask Funktioner kan ge tupler som resultat.
\begin{REPL}
scala> def div(a: Int, b: Int) = (a / b, a % b)
scala> div(10, 3)
scala> (div(9,2), div(10,2))
scala> (div(9,2)._2, div(10,2)._2)
scala> val nOdd = (1 to 10).map(i => div(i, 2)._2).sum
\end{REPL}
Förklara vad som händer ovan. Använd \code{div} ovan för att ta reda på hur många udda tal finns det i intervallet $[1234, 3456]$.

\Task Objekt med attribut (fält). Ett objekt kan samla data som hör ihop. Data i ett objekt kallas \emph{attribut} eller \emph{fält}, \Eng{field}. Objekt som samlar data kallas ibland \emph{post} \Eng{record}.
\begin{REPL}
scala> object mittKonto { var saldo = 0; val nummer = 12345L }
\end{REPL}
\Subtask Skriv en sats som sätter in ett slumpmässigt belopp mellan 0 och en miljon på kontot. 

\Subtask Vad händer om du försöker ändra attributet \code{nummer}?

\Task Klass med attribut. Om man vill kunna ha flera upplagor av ett objekt av samma typ, kan man använda en klass. Man skapar nya objekt med \code{new}.
\begin{REPL}
scala> class Konto(var saldo: Int, val nummer: Long)
scala> val k = new Konto(0, 12345L)
scala> println("Konto: " + k.nummer + " Saldo:" k.saldo)
scala> println(k)
scala> k.toString
\end{REPL}
\Subtask Den två sista raderna ovan skriver ut den identifierare som JVM använder för att hålla reda på objektet i minnet. Skapa ännu en instans av klassen Konto  med samma saldo och nummer som \code{k} ovan och spara den i \code{val k2} och undersök dess objektidentifierare. Får objekten \code{k} och \code{k2} olika objektidentifierare?

\Subtask Vad händer om du försöker ändra attributet \code{nummer}?

\Subtask\Pen Ibland räcker det fint med en tupel, men ofta vill man ha en klass istället. Beskriv några fördelar med en Konto-klassen ovan jämfört med en tupel av typen \code{(Int, Long)}.

\begin{REPL}
scala> var k3 = (0, 12345L)
scala> k3 = (k3._1 + 100, k3._2)

\end{REPL}


\Task Publikt versus privat attribut. 

\begin{REPL}
scala> class Konto1(val nummer: Long){ var saldo = 0 }
scala> val k1 = new Konto1(12345678901L)
scala> k1.nummer
scala> k1.saldo += 1000
scala> class Konto2(val nummer: Long){ private var saldo = 0 }
scala> val k2 = new Konto2(12345678901L)
scala> k2.nummer
scala> k2.saldo += 1000
\end{REPL}

\Subtask Vad händer ovan?

\Subtask Förbättra implementationen av klassen \code{Konto} enligt nedan:

\begin{Code}
class Konto(val nummer: Long){ 
  private var saldo = 0
  def in(belopp: Int): Unit = {saldo += belopp}
  def ut(belopp: Int): Unit = {saldo -= belopp}
  def show: Unit = 
    println("Konto Nr: " + nummer + " saldo: " + saldo) 
}

object Main {
  def main(args: Array[String]): Unit = {
    val k = new Konto(1234L)
    k.show
    k.in(1000)
    println("Uttag: " + k.ut(500))
    println("Uttag: " + k.ut(1000))
    k.show
  }
}
\end{Code}

\Subtask Spara koden i en fil, kompilera och kör. Testa även vad som händer om du försöker komma åt attributet \code{saldo} i main-metoden med t.ex. \code{println(k.saldo)} eller \code{k.saldo += 1000}. 

\Subtask Vi ska nu förhindra överuttag. Ändra i metoden \code{ut} så att den får signaturen \code{ut(belopp: Int): (Int Int) = ???} och implementera \code{ut} så att den returnerar både beloppet man verkligen kan ta ut och kvarvarande saldo. Om man försöker ta ut mer än det finns på kontot så ska saldot bli 0 och man får bara ut det som finns kvar. Spara, kompilera, kör. 

\Subtask Förbättra metoderna \code{in} och \code{ut} så att man inte kan sätta in eller ta ut negativa belopp.

\Subtask Vad är fördelen med att göra föränderliga attribut privata och bara påverka deras värden indirekt via metoder?



\Task Föränderlighet och oföränderlighet.

\Subtask Innan du kör nedan kod: Försök lista ut vad som kommer skrivas ut. Rita minnessituationen efter varje tilldelning.

\begin{Code}
println("\n--- Example 1: mutable value assigmnent")
var x1 = 42
var y1 = x1
x1 = x1 + 42
println(x1)
println(y1)
\end{Code}

\Subtask Innan du kör nedan kod: Försök lista ut vad som kommer skrivas ut. Rita minnessituationen efter varje tilldelning.

\begin{Code}
println("\n--- Example 2: mutable object reference assignment")
class MutableInt(private var i: Int) {
  def +(a: Int): MutableInt = { i = i + a; this }
  override def toString = i.toString
}
var x2 = new MutableInt(42)
var y2 = x2
x2 = x2 + 42
println(x2)
println(y2)
\end{Code}

\Subtask Innan du kör nedan kod: Försök lista ut vad som kommer skrivas ut. Rita minnessituationen efter varje tilldelning.

\begin{Code}
println("\n--- Example 3: immutable object reference assignment")
class ImmutableInt(val i: Int) {
  def +(a: Int): ImmutableInt = new ImmutableInt(i + a) 
  override def toString = i.toString
}
var x3 = new ImmutableInt(42)
var y3 = x3
x3 = x3 + 42
println(x3)
println(y3)
\end{Code}

\Subtask\Pen Vad finns det för fördelar med oföränderliga objekt?


\Task Skapa ett nytt konto

\ExtraTasks %%%%%%%%%%%%%%%%%%%

\Task 

\begin{Code}
class Account(val number: Long, val maxCredit: Int){ 
  private var balance = 0
  
  def deposit(amount: Int): Int = { 
    if (amount > 0) {balance += amount}
    balance
  }
  
  def withdraw	(amount: Int): (Int, Int) = if (amount > 0) { 
    val allowedWithdrawal = 
      if (amount < balance + maxCredit) amount 
      else balance + maxCredit 
    balance = balance - allowedWithdrawal
    (allowedWithdrawal, balance)
  } else (0, balance)
  
  def show: Unit = 
    println("Account Nbr: " + number + " balance: " + balance) 
}

object Main {
  def main(args: Array[String]): Unit = {
    ???
  }
}
\end{Code}

\AdvancedTasks %%%%%%%%%%%%%%%%%

\Task     
    