%!TEX encoding = UTF-8 Unicode

%!TEX root = ../compendium.tex

\Lab{\LabWeekELEVEN}

\begin{Goals}
\item Förstå hur autoboxing fungerar i Java
\item Förstå vad statiska metoder och attribut innebär
\item Förstå skillnad mellan primitiva typer och objekt i listor
\item Kunna byta mellan ArrayLists och Array
\item Kunna hur man läser in från fil
\item Kunna for-sats i Java
\end{Goals}

\begin{Preparations}
\item Scanner
\item ArrayList
\item Enum?
\item Statiskt
\item Autoboxing
\item Arv
\end{Preparations}

\subsection{Bakgrund}
I denna labb skall ni tillverka lthopoly, en variant av det välkända brädspelet Monopol med några simplifieringar. Spelet går ut på att spelarna i tur och ordning slår en tärning, varpå deras pjäs flyttas det antal steg som tärningen visar. Rutorna representeras av tre klasser som alla ärver av klassen BoardSpace, nämligen MoveSpace, MoneySpace och HouseSpace. En stor skillnad här jämfört med tidigare labbar är att det mesta skall skrivas i Java.

Beroende på vilken av de tre rutorna spelaren hamnar på sker olika saker. För att representera rutorn används en ArrayList<BoardSpace> och spelar-objektens positioner används för att hämta motsvarande boardspace från listan. För de tre olika typerna av rutor gäller följande:

\begin{itemize}
\item MoveSpace: Om en spelare hamnar på denna ruta slumpas ett kort från högen av MoveCards, varpå spelaren förflyttas antingen framåt eller bakåt det antal steg som kortet anger.Kortets deskriptiva text skrivs även ut i spelfönstret.
\item MoneySpace: Om en spelare hamnar på denna ruta slumpas ett kort från högen av MoneyCards, varpå spelaren antingen förlorar pengar eller vinnar pengar enl. kortet.Kortets deskriptiva text skrivs även ut i spelfönstret.
\item HouseSpace: Om en spelare hamnar här gäller olika saker beroende på rutans två tillstånd. Om spelaren hamnar på en HouseSpace som ingen annan spelare har köpt upp skall spelaren få möjlighet att köpa denna om den har råd. Är det istället så att rutan ägs av en annan spelare den nuvarande spelaren betala hyra till ägaren (hyran är samma som inköpspriset). 
\end{itemize}


Den visuella representationen av spelet sker via konsolfönstret med hjälp av klassen \code{TextUI}. \code{TextUI} är en färdigskriven klass med statiska metoder som gör det möjligt att skriva ut två kolumner av text i konsolfönstret. Utskriften kan exempelvis se ut såhär:

\begin{REPL}
Studiecentrum(20)(Valthor)
A-huset(25) (Jonas)                                                                                                          
ChansRuta                                                                                                             
ChansRuta                                                                                                              
Moroten och piskan(40)
V-Huset(45)                                                                                                        
RiskRuta (Oskar)                                                                                                         
ChansRuta                                                                                                        
LED-Cafe(70)               Oskar slog en 6:a!                                           
F-Huset(75)                Flytta framåt 6 steg                                        
ChansRuta                  Oskar drog ett kort: Inkassera 50 SEK!
Ideet(80)                  Oskar har avslutat sin runda.                              
ChansRuta                  Nästa spelare: Jonas                                      
E-huset(100)               Jonas slog en 1:a!                                       
RiskRuta                   Grattis, Jonas är nu den stolta ägaren av A-huset!
\end{REPL}

Notera att utskriftens två kolumner är oberoende av varandra och kan uppdateras separat. För att uppdatera UI:t och få input av användaren finns följande metoder:
\subsubsection{Färdigimplementerat}
\begin{JavaSpec}{class TextUI}
     /**Clears the right column (without reprinting). */
    public static void emptyRightColumn();

    /** Clears the left column (without reprinting). */
    public static void emptyLeftColumn();

    /** Adds a new string to be printed in the right column
     * every time it updates. Does not reprint the UI. */
    public static void addToRightColumn(String event);

    /** Adds a new string to be printed in the left column
     * every time it updates. Does not reprint the UI. */
    public static void addToLeftColumn(String event);

    /** Prints a console plot of the given array. */
    public static void printStatistics(int[] moneyLog);

    /** Prints a large number of newlines, visually clearing the console. */
    public static void clearConsole();

    /** Resets the left column with the toString of the GameBoard
     * and reprints both columns.
     * @param board. 
     */
    public static void updateConsole(GameBoard board);

    /**Prompts the user to select one of the given options.
     * @param options An ArrayList containing a list of
     *          descriptions for the possible options.
     * @return the index of the option selected by the user.
     */
    public static int promptForInput(List<String> options);
\end{JavaSpec}

\subsubsection{Implementeras Själv}
\begin{JavaSpec}{class Player}

/**Beskriver en spelare med pengar money, namn name och position pos . */
public Player(int money, String name, int pos)

/**Ger en int med spelarens pengar. */
public int getMoney()

/** Ändrar spelarens pengar. */
public void setMoney(int money)

/** Ger en int med spelarens position. */
public int getPosition()

/** Ändrar spelarens position. */
public void setPosition(int pos)

\end{JavaSpec}

\begin{JavaSpec}{class BoardSpace}
/**Beskriver en generell ruta på spelplanen. */
public BoardSpace()

/* Ger en vektor av möjliga val för spelare som står på rutan. */
public abstract int[] getPossibleActions(GameBoard board);

/* Genomför det val som skickas med. */
public abstract void action(GameBoard board, int action);

/*Ger en strängrepresentation av rutan. */
public abstract String toString();
\end{JavaSpec}

\begin{JavaSpec}{class GameBoard}
/**Beskriver ett spelbräde med en vektor av spelarna players. */
public GameBoard(Player[] players) 

/** Ger en vektor av möjliga val för spelaren vars tur det är. */
public int[] getPossibleActions()

/** Ger sant eller falskt beroende på om spelet är slut eller ej.*/
public boolean isGameOver()

/** Hämtar den för närvarande rikaste spelaren. */
public Player getRichestPlayer()

/** Genomför valet action för den aktuella spelaren. */
public void doAction(int action)

/** Ger spelaren vars tur det är. */
public Player getCurrentPlayer()

/** Ger rutan som den aktuella spelaren står på. */
public BoardSpace getCurrentBoardSpace()

/** Förflyttar nuvarande spelare. */
public void moveCurrentPlayer(int adjustment)

/** Ger en vektor med totala summan pengar för varje runda,
 * där runda ett är index 0 och sista rundan är i sista elementet 
 * i vektorn.  */
public int[] getStatistics()

/** Ger en strängrepresentation av spelbrädet. */
public String toString()
\end{JavaSpec}

\begin{JavaSpec}{class DocumentParser}
/* Ger en representation av spelplanen i en lista. */
public static ArrayList<BoardSpace> getBoard()
\end{JavaSpec}

\begin{JavaSpec}{class MoneyCard}
/* Beskriver ett pengakort. */
public MoneyCard(String description, int money)
\end{JavaSpec}
\begin{JavaSpec}{class MoveCard}
/* Beskriver ett flyttkort. */
public MoveCard(String description, int position)
\end{JavaSpec}

\subsubsection{Spelregler}

\begin{itemize}
\item Om någon spelare har mindre än 0 SEK kvar skall spelet sluta.
\item Om någon spelare hamnar på en husruta som ägs av en annan spelare måste denne betala ägaren husets hyra i SEK.
\item Om en spelare väljer att ta ett kort skall det värde som påverkas av kortet justeras. För MoveCard innebär detta en förflyttning medan för MoneyCard en ändring av pengar.

\end{itemize}



\subsection{Obligatoriska uppgifter}

\Task Textfilern moneycards.txt , movecards.txt och board.txt innehåller information som skall nyttjas för att skapa de olika rutorna. För att kunna använda de färdiskrivna textfilerna behöver ni nu skriva en statisk metod för inläsning av data från fil i klassen DocumentParser. Hur ni väljer att läsa in från fil är valfritt,  så länge det går att anropa getBoard() och få en lista av rutor som motsvarar spelplanen. Även klasserna MoveCard och MoneyCard skall implementeras så att det går bra ihop med inläsningen.
För skapa ett File-objekt med informationen från filerna kan följande kod användas:
\newline
\newline
\code{File f = new File(DocumentParser.class.}
\newline
\code{getResource("/moneycards.txt").getFile());}
\newline
\newline
Därefter kan ni använda er av ett scanner-objekt som itererar över Filen för att hämta ut den aktuella informationen.
\noindent
\textbf{Obs!} Se textfilerna moneycards.txt, movecards.txt och board.txt för förståelse för hur inläsningen bör gå till för korten.  Ni kan nyttja metoden .split(String) för att ta dela upp indatan.

\Subtask Implementera klassen MoveCard.

\Subtask Implementera klassen MoneyCard.

\Subtask Implementera metoderna getMoneyCards() och getMoveCards() i DocumentParser.

\Subtask Implementera klassen Player.

\Task I denna uppgift skall de tre olika typerna av spelrutor implementeras, som alla ärver ifrån klassen \code{BoardSpace}. Ni kommer då behöva implementera de tre abstrakta metoderna i BoardSpace i respeketive subklass. Här behöver subklasserna tillgång till spelbrädet då förändringen av speltillståndet sker till störren delen i metoden \code{action} för de olika rutorna. HouseSpace kommer dessutom behöva ett attribut som håller koll på vem som äger rutan, medan MoveSpace och MoneySpace kommer behöva tillgång till vektorer av de kort som läses in.

\Subtask Implementera en klass för varje typ av spelruta. Tänk på att MoveSpace och MoneySpace behöver tillgång till respektive kortlekar.
För att skriva metoden action kommer ni behöva nyttja att klassen \code{GameBoard} har statiska variabler som representerar de olika spelarvalen.
\newline
\newline
\noindent
\textbf{Obs!} Än så länge kommer logiken inte fungera då inga metoder är implementerade i BoardGame, det går trots detta bra att anropa metoderna utan kompileringsfel (i väntan på att de implementeras).



\Task Nu är det dags att implementera getBoard() i klassen \code{DocumentParser}. I denna metod skall ni läsa in från filen board.txt och nyttja de metoder ni redan skrivit för att nu kunna skapa MoveSpaces och MoneySpaces. Viktigt här är att ordningen i vilken de olika rutorna är representerade i textfilen spelar roll, då den utgör upplägget av spelplanen. Alltså, beroende på vilken ruta som läses in från textfilen skall motsvarande objekt som representerar denna ruta konstrueras och läggas till i en ArrayList<BoardSpace> som slutligen returneras.
\textbf{Fundera på om det behöver skapas flera objekt av varje typ av ruta.} 

\Subtask Implementera getBoard().

\Task För att kunna skriva ut och visa spelplanen använder sig klassen \code{TextUI}  av metoden toString() i klassen \code{GameBoard}. Denna ger en textrepresentation av spelplanen. Övriga metoder i klassen \code{GameBoard} skall nu implementeras, nyttja att  \code{TextUI} har metoder för att lägga till utskrifter i olika kolumner för utskrifter av händelser. Det skall inte förekomma någon direkt utskrift i  \code{GameBoard}, alla utskrifter ska gå via \code{TextUI}. I slutet av varje spelares tur skall dessutom den totala summan av pengar hos alla spelare läggas till i en ArrayList, för att det efter spelets slut skall kunna visas statistik för den totala ekonomin.

\Subtask Implementera GameBoard.
\newline
\newline
\textbf{Tips:}

\begin{itemize}
\item Ni kan använda privata hjälpmetoder för att underlätta implementeringen.
\item Metoden printStatistics i klassen \code{TextUI} tar en vektor av int-värden som inparameter, vilket är opassande då det underlättar att lagra pengahistoriken i en arrayList (eftersom dess storlek inte är bestämd). Det är därför lämpligt att skriva en metod som flyttar över samtliga Integer-objekt från ArrayList<Integers> till en vektor av primitiva int-värden. Detta fungerar trots att de har olika typer p.g.a. autoboxing. 
\item Tänk på att spelarna skall kunna gå runt spelplanen ett obegränsat antal gånger.
\item Tänk på att dela upp arbetet så att det sker en lämpligt arbetsfördelning inom gruppen, men samarbeta så att ni vet hur era olika implementationer interagerar.
\item Glöm inte att skicka strängar till  \code{TextUI} för att generera utskrift när uppdatering sker.
\end{itemize}

\Task Med spelplanen implementerad behövs en main-metod för att kunna starta spelet.  

\Subtask Implementera getAction i Scala. Metoden skall nyttja de statiska variablerna från GameBoard för att ge en lämplig utskrift. Detta är metoden som kommunicerar med TextUI-klassen för att begära input från användarna.

\Subtask Implementera main-metoden i Scala.




\subsection{Frivilliga extrauppgifter}

\Task Här är det dags att utöka spelet för att ge utökad funktionalitet som liknar det riktiga spelet.

\Subtask Implementera funktionalitet för att varje spelare ska få extra pengar då den passerar första spelrutan.

\Subtask Implementera funktionalitet för att varje spelare som hamnar på en ruta de äger sedan tidigare har möjlighet att öka hyran för rutan ifall någon annan spelare skulle hamna på den.

\Subtask Implementera funktionalitet för att spelarna själva måste betala en femtedel värdet av sina hus varje runda (varje gång de passerar ett varv på brädet).
    
