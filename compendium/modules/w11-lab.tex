%!TEX encoding = UTF-8 Unicode

%!TEX root = ../compendium.tex

\Lab{\LabWeekELEVEN}

\begin{Goals}
\item Hur autoboxing fungerar i Java
\item Skillnaden mellan statiska och icke-statiska metoder
\item Skillnaden mellan Objekt och Primitiva typer för listor
\item Hur man läser in från fil
\item Grundläggande hantering av exceptions
\item Skillnader i forloops mellan Scala och Java

\end{Goals}

\begin{Preparations}
\item Läs avsnitt 3.10, om static, i läroboken
\end{Preparations}

\subsection{Bakgrund}
I denna labb skall ni tillverka LTH-o-pol, en variant av det välkända brädspelet Monopol med några simplifieringar. Spelet går ut på att spelarna i tur och ordning slår en tärning, varpå deras pjäs flyttas det antal steg som tärningen visar. Det finns tre typer av rutor, RiskSpace, ChanceSpace och HouseSpace. Beroende på vilken av rutorna spelaren hamnar på sker olika saker. I fallet att man hamnar på RiskSpace får spelaren dra ett s.k. RiskCard, som antingen förflyttar spelaren framåt eller bakåt längs den tänkta spelplanen. Skulle spelaren hamna på ChanceSpace kan denne antingen vinna eller förlora pengar genom att dra ett ChanceCard. Skulle spelaren hamna på en HouseSpace får denne alternativet att köpa rutan, förutsatt att den inte köpts av någon annan spelare, om så är fallet måste spelaren som hamnade på rutan betala hyra till ägaren.
Den visuella representationen av spelet sker via konsolfönstret med hjälp av klassen XXXXXX. 
För att läsa in korten och rutorna används klassen DocumentParser, som har vissa metoder (men inte alla) färdigimplementerade.
Main-klassen skall vara skriven i Scala, men övriga klasser i Java. 
Eftersom labben är en s.k. teamlab är ett förslag till arbetsfördelning att varje person implementerar en utav underuppgifterna.

\subsection{Obligatoriska uppgifter}

\Task Här skall ni skriva en statisk metod för inläsning av data från fil i klassen DocumentParser. Här får filstrukturen väljas helt fritt så länge resultatet av inläsningen blir att man kan anropa metoderna getRiskCards och getChanceCards och få kort enligt den kortstruktur som implementeras. Även klasserna RiskCard och ChanceCard skall implementeras så att det går bra ihop med inläsningen.

\Subtask Implementera klassen RiskCard.

\Subtask Implementera klassen ChanceCard.

\Task 

\subsection{Frivilliga extrauppgifter}

\Task En labbuppgiftsbeskrivning.

\Subtask En underuppgift.

\Subtask En underuppgift.
    