%!TEX encoding = UTF-8 Unicode

%!TEX root = ../compendium.tex

\ExerciseSolution{\ExeWeekTEN}

% Grunduppgifter

\Task %Uppgift 1

\Subtask \begin{enumerate}
\item \code{true}
\item \code{true}
\item \code{true}
\item \code{true}
\item \code{true}
\item \code{false}
\end{enumerate}

\Subtask \emph{s1} kommer först.

\Task %Uppgift 2
\Subtask
\begin{enumerate}
\item \code{String = java.lang.String}
\item \code{Boolean = true}
\item \code{Int = 0}
\end{enumerate}

\Subtask
Exempel på 3 olika uttryck för att testa \code{compareTo}:

\begin{enumerate}
\item \code{"hej".compareTo("Hej) \\ res: Int = 32} (Hej kommer först då H < h)
\item \code{”hej”.compareTo(”hej”) \\ res: Int = 0} (de är ekvivalenta)
\item \code{”hej”.compareTo(”ö”) \\ res: Int = -142} (hej kommer först)
\end{enumerate}

\Subtask
Exempel på 3 olika uttryck för att testa \code{compareToIgnoreCase}:

\begin{enumerate}
\item \code{”hej”.compareToIgnoreCase(”HEj”) \\ res: Int = 0}
\item \code{“hej”.compareToIgnoreCase(“Ö”) \\ res: Int = -142}
\item \code{3.	“hej”.compareToIgnoreCase(“ö”) \\ res: Int = -142} (samma som ovan då Ö omvandlas till ö innan jämförelse)
\end{enumerate}

\Subtask
\begin{REPLnonum}
false
true
0
\end{REPLnonum}


\Task %Uppgift 3

\Subtask
\begin{enumerate}
\item Returnerar en sorterad Vector av \code{double}-värden
\item Skapar en variabel xs och sparar en Array med Int-värden mellan 100000 till 1.
\item Sorterar xs = 1,2,3...
\item Konverterar xs till en Array av String-värden och sorterar dem lexikografiskt: xs = ”1”, ”10”, ”100” etc.
\item Konverterar xs till en Array av Byte-värden (max 127, min -128) och sorterar dem, samt tar bort dubletter: xs = -128, -127, -1...
\item Skapar en ny klass Person som tar 2 String-argument i konstruktorn
\item Sparar en Vector med två Person-objekt i en variabel ps
\item Försöker kalla på sorted-metoden för klassen Person. Eftersom vi skrivit denna klass själva och inte berättat för Scala hur Person-objekt ska sorteras, resulterar detta i ett felmeddelande.
\end{enumerate}

\Subtask 

\begin{enumerate}
\item ---
\item ---
\item Sortera Person-objekten i ps med avseende på värdet i firstName
\item Sortera Person-objekten i ps med avseende på värdet i familyName 
\item \code{sortBy} tar en funktion f som argument. f ska ta ett argument av typen Person och returnera en generisk typ B.
\item Sortera Person-objekten i ps med avseende på firstName i sjunkande ordning (omvänt från tidigare alltså)
\item \code{sortWith} tar en funktion lt som argument. lt ska i sin tur ta två argument av typen Person och returnera ett boolskt värde.
\item Sorterar en vektor så att värdena som är minst delbara med 2 hamnar först, och de mest delbara med 2 hamnar sist. Detta delar alltså upp udda och jämna tal.
\end{enumerate}

\Subtask
Klassens signatur blir då:
\begin{REPLnonum}
case class Person(firstName: String, lastName: String, age: Int)
\end{REPLnonum}

Lägg in dem i en vektor: 
\begin{REPLnonum}
val ps2 = Vector(Person("a", "asson", 34), Person("asson", "assonson", 1234), Person("anna", "Book", 2))
\end{REPLnonum}

Sortera dem på olika sätt:
\begin{REPLnonum}
ps2.sortWith((p1, p2) => p1.age < p2.age)
res40: scala.collection.immutable.Vector[Person] = Vector(Person(anna,Book,2), Person(a,asson,34), Person(asson,assonson,1234))
//Vektorn blir sorterad med avseende på personernas ålder i stigande ordning

ps2.sortWith((p1, p2) => (p1.firstName < p2.firstName) && (p1.age > p2.age))
res42: scala.collection.immutable.Vector[Person] = Vector(Person(a,asson,34), Person(asson,assonson,1234), Person(anna,Book,2))
//Sorterar vektorn med avseende på namn, men också med avseende på ålder (i sjunkande ordning). För att komma före någon i ordningen måste alltså både namnet komma före, och åldern vara högre.
\end{REPLnonum}

\Task %Uppgift 4

\Subtask Exekvera koden och du bör finna att det tar längre tid att hitta värdet 1 i vårt Set s än i vektorn v.

\Subtask Ja varför?!


\Task %Uppgift 5

\Subtask
Förslag på test:
\begin{REPLnonum}
scala> List(1,2,3,35,1,23).indexOfSlice(List(35,1,23))
res73: Int = 3
scala> List(1,2,3,35,1,23).indexOfSlice(List(35,1,3))
res74: Int = -1
\end{REPLnonum}

\Subtask
Förslag på test:
scala> Vector(1,2,3,4,1,2).lastIndexOfSlice(Vector(1,2))
res2: Int = 4
scala> Vector("apa", "banan", "majs", "banan").lastIndexOfSlice(Vector("banan"))
res3: Int = 3
scala> Vector("apa", "banan", "majs", "banan").lastIndexOfSlice(Vector("banand"))
res4: Int = -1
\end{REPLnonum}

\Subtask
Observera att metoden \code{search} antar att samlingen är sorterad i stigande ordning. När vi inverterar ordningen kan \code{search} oftast inte hitta det vi letar efter, eftersom den kommer leta i fel halva av samlingen.

