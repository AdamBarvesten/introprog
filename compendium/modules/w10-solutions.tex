%!TEX encoding = UTF-8 Unicode

%!TEX root = ../compendium.tex

\ExerciseSolution{\ExeWeekTEN}

% Grunduppgifter

\Task %Uppgift 1

\Subtask 
\begin{REPL}
true
true
true
true
true
false
\end{REPL}

\Subtask 
\emph{s1} kommer först.

\Task %Uppgift 2
\Subtask
\begin{REPL}
String = java.lang.String
Boolean = true
Int = 0
\end{REPL}

\Subtask
Exempel på 3 olika uttryck för att testa \code{compareTo}:

\begin{enumerate}
\item 
Hej kommer först då \code{H < h}.
\begin{REPLnonum}
	"hej".compareTo("Hej)
	res: Int = 32
\end{REPLnonum} 

\item 
Dessa är ekvivalenta, så \code{compareTo} returnerar 0.
\begin{REPLnonum}
	"hej".compareTo("hej")
	res: Int = 0
\end{REPLnonum} 

\item 
\emph{h} kommer före \emph{ö}.
\begin{REPLnonum}
	"hej".compareTo("ö")
	res: Int = -142
\end{REPLnonum} 
\end{enumerate}

\Subtask
Exempel på 3 olika uttryck för att testa \code{compareToIgnoreCase}:

\begin{enumerate}

\item
\begin{REPLnonum}
	"hej".compareToIgnoreCase("HEj")
	res: Int = 0
\end{REPLnonum}

\item
\begin{REPLnonum}
	"hej".compareToIgnoreCase("Ö")
	res: Int = -142
\end{REPLnonum}

\item
Samma som ovan, då Ö omvandlas till ö innan jämförelse.
 \begin{REPLnonum}
	"hej".compareToIgnoreCase("ö") \\ res: Int = -142
\end{REPLnonum} 
\end{enumerate}

\Subtask
\begin{REPL}
false
true
0
\end{REPL}


\Task %Uppgift 3

\Subtask
\begin{enumerate}
\item Returnerar en sorterad \code{Vector} av \code{double}-värden
\item Skapar en variabel xs och sparar en \code{Array} med \code{Int}-värden mellan 100000 till 1.
\item Sorterar \code{xs = 1,2,3...}
\item Konverterar xs till en \code{Array} av \code{String}-värden och sorterar dem lexikografiskt: \code{xs = "1", "10", "100"} etc.
\item Konverterar xs till en \code{Array} av \code{Byte}-värden (max 127, min -128) och sorterar dem, samt tar bort dubletter: \code{xs = -128, -127, -1...}
\item Skapar en ny klass \code{Person} som tar 2 \code{String}-argument i konstruktorn
\item Sparar en Vector med två \code{Person}-objekt i en variabel ps
\item Försöker kalla på \code{sorted}-metoden för klassen \code{Person}. Eftersom vi skrivit denna klass själva och inte berättat för Scala hur \code{Person}-objekt ska sorteras, resulterar detta i ett felmeddelande.
\end{enumerate}

\Subtask 

\begin{enumerate}
\item ---
\item ---
\item Sorterar \code{Person}-objekten i ps med avseende på värdet i \code{firstName}
\item Sorterar \code{Person}-objekten i ps med avseende på värdet i \code{familyName} 
\item \code{sortBy} tar en funktion f som argument. f ska ta ett argument av typen \code{Person} och returnera en generisk typ B.
\item Sortera \code{Person}-objekten i ps med avseende på \code{firstName} i sjunkande ordning (omvänt från tidigare alltså)
\item \code{sortWith} tar en funktion lt som argument. lt ska i sin tur ta två argument av typen \code{Person} och returnera ett boolskt värde.
\item Sorterar en vektor så att värdena som är minst delbara med 2 hamnar först, och de mest delbara med 2 hamnar sist. Detta delar alltså upp udda och jämna tal.
\end{enumerate}

\Subtask
Klassens signatur blir då:
\begin{REPLnonum}
case class Person(firstName: String, lastName: String, age: Int)
\end{REPLnonum}

Lägg in dem i en vektor: 
\begin{REPLnonum}
val ps2 = Vector(Person("a", "asson", 34), Person("asson", "assonson", 1234), 
Person("anna", "Book", 2))
\end{REPLnonum}

Sortera dem på olika sätt:
\begin{enumerate}
\item
Vektorn blir sorterad med avseende på personernas ålder i stigande ordning
\begin{REPLnonum}
scala> ps2.sortWith((p1, p2) => p1.age < p2.age)
res40: scala.collection.immutable.Vector[Person] = Vector(Person(anna,Book,2), 
Person(a,asson,34), Person(asson,assonson,1234))
\end{REPLnonum}

\item
Sorterar vektorn med avseende på namn, men också med avseende på ålder (i sjunkande ordning). För att komma före någon i ordningen måste alltså både namnet komma före, och åldern vara högre.
\begin{REPLnonum}
scala> ps2.sortWith((p1, p2) => (p1.firstName < p2.firstName) &&
(p1.age > p2.age))
res42: scala.collection.immutable.Vector[Person] = Vector(Person(a,asson,34), 
Person(asson,assonson,1234), Person(anna,Book,2))
\end{REPLnonum}
\end{enumerate}


\Task %Uppgift 4

\Subtask Exekvera koden och du bör finna att det tar längre tid att hitta värdet 1 i vårt Set s än i vektorn v.

\Subtask Ja varför?!


\Task %Uppgift 5

\Subtask
Förslag på test av \code{indexOfSlice}:
\begin{REPLnonum}
scala> List(1,2,3,35,1,23).indexOfSlice(List(35,1,23))
res73: Int = 3
scala> List(1,2,3,35,1,23).indexOfSlice(List(35,1,3))
res74: Int = -1
\end{REPLnonum}

\Subtask
Förslag på test av \code{lastIndexOfSlice}:
\begin{REPLnonum}
Vector(1,2,3,4,1,2).lastIndexOfSlice(Vector(1,2))
res2: Int = 4
Vector("apa", "banan", "majs", "banan").lastIndexOfSlice(Vector("banan"))
res3: Int = 3
Vector("apa", "banan", "majs", "banan").lastIndexOfSlice(Vector("banand"))
res4: Int = -1
\end{REPLnonum}

\Subtask
Observera att metoden \code{search} antar att samlingen är sorterad i stigande ordning. När vi inverterar ordningen kan \code{search} oftast inte hitta det vi letar efter, eftersom den kommer leta i fel halva av samlingen.

\begin{REPLnonum}
scala> val udda = (1 to 1000000 by 2).toVector
scala> import scala.collection.Searching._ 
scala> udda.search(udda.last)
res18: collection.Searching.SearchResult = Found(499999) 
//Search hittar det sista elementet på plats 499999 i samlingen.

scala> udda.search(udda.last + 1)
res19: collection.Searching.SearchResult = InsertionPoint(500000) 
//Search kan inte hitta udda.last + 1 eftersom det inte existerar i samlingen 
//och returnerar således ett objekt av typen InsertionPoint med värdet 500000. 
//Vårt element udda.last + 1 hade alltså legat på plats 500000 om det funnits.
 
scala> udda.reverse.search(udda(0))
res20: collection.Searching.SearchResult = InsertionPoint(0) 
//Som förklarat innan så förutsätter search att listan är sorterad i stigande
//ordning, så den kan inte hitta elementet udda(0) = 1 när listan är inverterad. 

scala> def lin(x: Int, xs: Seq[Int]) = xs.indexOf(x) 
scala> def bin(x: Int, xs: Seq[Int]) = xs.search(x) match {
	case Found(i) => i
	case InsertionPoint(i) => -i
}
//Definierar en metod bin som använder sig av metoden search på en sekvens. 
//Den ser sedan till med hjälp av "pattern matching" att bara returnera positionen
//i, och inte ett objekt av typen Found eller InsertionPoint.

scala> timed{ lin(udda.last, udda) }
time: 42.294821 ms
res22: (Int, Long) = (499999,42294821)
//För att hitta udda.last = 499999 med linjärsökning tog det ca 42ms.

scala> timed{ bin(udda.last, udda) }
time: 0.147314 ms
res23: (Int, Long) = (499999,147314)
//Binärsökning för att hitta värdet 499999 tog extremt mycket kortare tid. 
//Detta för att vid varje steg i binärsökningen halveras mängden tal som 
//sökningen måste kolla i. Detta är dock ett extremfall eftersom vi söker
//talet längst bak i listan. Om vi istället gjort en linjärsökning efter 
//det första talet 1, hade detta gått minst lika snabbt som binärsökning.
\end{REPLnonum}

\Subtask
Det behövs $log_2(n)$ jämförelser. Detta eftersom att vi hela tiden halverar antalet element i listan vi behöver söka igenom. Så efter första jämförelsen har vi $\frac{n}{2}$ element kvar. Efter andra jämförelsen har vi $\frac{n}{2*2}$ element kvar etc. När vi bara har ett element kvar har vi hittat det vi söker efter, och har då gjort $b$ antal jämförelser. Ekvationen ser då ut på följande vis: 
\begin{equation*}
\frac{n}{2^b} = 1
\end{equation*}
Enligt lagarna för logaritmer kan vi nu komma fram till vad b är:
\begin{equation*}
log_2(n) = b
\end{equation*} 


\Task %Uppgift 6

\Subtask
Den finns som värde för en \emph{td} tagg, på följande vis: \code{<td class="mitt">2</td>}.

\Subtask
Koden laddar ner html-koden för sidan \\ \mbox{\small\url{http://kurser.lth.se/lot/?lasar=16_17&soek_text=&sort=kod&val=kurs&soek=t}} och sparar den i en vektor. Sedan filtreras ut endast de rader som innehåller strängen ”kurskod” så att all onödig HTML-kod försvinner. Sedan konverteras detta, för varje rad, till \code{Course}-objekt med hjälp av metoden \code{fromHtml}. Eftersom variabeln \code{lth2016} är deklarerad som \code{lazy} kommer inte \code{download()} bli anropad förrän vi vill komma åt variabeln. Vi startar alltså processen genom att referera variabeln \code{lth2016} i objektet \code{courses}: 

\begin{REPLnonum}
courses.lth2016
\end{REPLnonum}
Detta generarar en lång lista med \code{Course}-objekt. Antalet kurser är således lika med storleken på vektorn \code{lth2016}.

\begin{REPLnonum}
courses.lth2016.size
res38: Int = 1097
\end{REPLnonum}

\Subtask
\begin{REPL}
scala> def isCS(s: String) = s.startsWith("EDA") || s.startsWith("ETS")
scala> val x = courses.lth2016.find(c => isCS(c.code) && c.level == "G2").get
x: courses.Course = Course(EDA031,C++ - programmering,C++ Programming,7.5,G2)
\end{REPL}
Obs: metoden \code{find} returnerar ett objekt av typen \code{Option}. För att få värdet som är lagrat i detta objekt krävs det att man kallar på \code{get}.

\Subtask
\begin{Code}
def linearSearch[T](xs: Seq[T])(p: T => Boolean): Int = {
   var i = 0
   while(i < xs.size && !p(xs(i))) i += 1 
   if (i < xs.size) i else -1
}
\end{Code}

\Subtask

%%% Nedan bortkommenterat av Björn den 2016-08-18; Vi använder inte Stream i denna kurs.
%%% Dessutom kanske Stream blir deprecated och ersatt med LazyList i Scala 2.13
%%% https://issues.scala-lang.org/browse/SI-9872
%\textbf{Lösning 1}
%
%\begin{Code}
%def rndCode: String = {
%   val f = Vector('0','1','2','3','4','5','6','7','8','9','A'
%      ,'C','F','G','L','M','N','P')
%   val word = Stream.continually((Random.nextInt(26) + 65)
%      .toChar).take(3)
%   val fourth = Stream.continually(f(Random.nextInt(f.length)))
%      .take(1)
%   val num = Stream.continually(Random.nextInt(10)).take(2)
%   word ++ fourth ++ num mkString
%}
%\end{Code}
%
%\newpage
%\textbf{Lösning 2}

\begin{Code}[language=Scala]
def rndCode: String = {
   //randomizes from 0 to n (inclusive)
   def rnd(n: Int) = (math.random * (n + 1)).toInt

   def letter = (rnd('Z' - 'A') + 'A').toChar
   def dig = ('0' + rnd(9)).toChar
   val special = "ACFGLMNP0123456789"
   def digLetter = special(rnd(special.size - 1))
   Seq(letter, letter, letter, digLetter, dig, dig).mkString
}
\end{Code}

\Subtask

\begin{Code}
val lthCourses = courses.lth2016 //avoid including download time
val xs = Vector.fill(500000)(rndCode)
val(ixs, elapsedLin) = timed{
xs.map(x => linearSearch(lthCourses)(_.code == x))}
val found = ixs.filterNot(_== -1).size
\end{Code}

\Subtask

\begin{Code}
def linearSearch[T](xs: Seq[T])(p: T => Boolean): Int = 
  xs.indexWhere(p)
\end{Code}


\Task %Uppgift 7

\Subtask
---

\Subtask
\begin{Code}
def binarySearch(xs: Seq[String], key: String): Int = {

  var (low, high) = (0, xs.size - 1)
  var found = false
  var mid = -1

  while (low <= high && !found) {
    mid = (low + high) / 2
    if (xs(mid) == key) found = true
    else if (xs(mid) < key) low = mid + 1
    else high = mid - 1
  }
  if (found)
    mid
  else
    -(low + 1)
}
\end{Code}

\Subtask
Med en i7-3770K @ 3.50Hz tog sökningarna följande tid:

\begin{itemize}
\item Binärsökning: \code{time: 142.6 ms}
\item Linjärsökning: \code{time: 3316.5 ms}
\end{itemize}

\Subtask
Binärsökningen var ca 23 gånger snabbare.


\Task %Uppgift 8

\Subtask
\begin{Code}
public static boolean isYatzy(int[] dice){
    int col = 1;
    boolean allSimilar = true;
    while(col < dice.length && allSimilar){
        allSimilar = (dice[0] == dice[col]);
        col++; //denna raden saknades
    }
    return allSimilar;
}
\end{Code}

\Subtask

\begin{Code}[language=Java]
public static int findFirstYatzyRow(int[][] m){
    int row = 0;
    int result = -1;
    while(row < m.length){
        if(isYatzy(m[row])){
           result = row;
           break;
        }
        row++;
    }
    return result;
}
\end{Code}


\Task %Uppgift 9

\Subtask
---

\Subtask

\begin{Code}
def insertionSort(xs: Seq[Int]): Seq[Int] = {
  val result = scala.collection.mutable.ArrayBuffer.empty[Int]
  for (e <- xs) {
    var pos = 0
    while (pos < result.size && result(pos) < e) pos += 1
    result.insert(pos,e)
  }
  result.toVector
}
\end{Code}


\Task %Uppgift 10

\begin{Code}
def selectionSortInPlace(xs: Array[String]): Unit = {
  
  def indexOfMin(startFrom: Int): Int = {
    var minPos = startFrom
    var i = startFrom + 1
    while (i < xs.size) {
      if (xs(i) < xs(minPos)) minPos = i
      i += 1
    }
    minPos
  }
  
  def swapIndex(i1: Int, i2: Int): Unit = {
    val temp = xs(i1)
    xs(i1) = xs(i2)
    xs(i2) = temp
  }
  
  for (i <- 0 to xs.size - 1) swapIndex(i, indexOfMin(i))
}
\end{Code}

\Task %Uppgift 11

\Subtask $n + n * log(n)$

\Subtask \TODO

\Subtask \TODO

