%!TEX root = ../compendium.tex

\Exercise{\tt{expressions}}



\begin{Goals}
\item Lär dig detta
\item Lär dig och detta
\end{Goals}

\begin{Preparations}
\item Läs kap.~\ref{intro}
\item Säkerställ att du kan avända de grundläggande terminalkommandona \code{ls}, \code{cd}, \code{rm} och \code{mkdir} för att inspektera, navigera i, och manipulera filträdet, se kap.~\ref{terminal}. 
\item Du behöver en dator med scala installerad. Om du inte har Scala installerad på din maskin, se installationsanvisningar i kap.~\ref{installScala}
\item Starta den editor du vill använda under övningarna, se kap.~\ref{editor}.
\end{Preparations}

\BasicTasks

%\TaskSection{Exekvera kod: REPL, skript, app}

\Task Starta Scala REPL \Eng{Read-Evaluate-Print-Loop} och skriv satsen \code{println("hejsan REPL")} och tryck på \textit{Enter}. 

\begin{REPL}
$ scala
Welcome to Scala version 2.11.7 (Java HotSpot(TM) 64-Bit Server VM, Java 1.8).
Type in expressions to have them evaluated.
Type :help for more information.

scala> println("hejsan REPL")
\end{REPL}

\Subtask Vad händer? 

\Subtask Skriv samma sats igen men ''glöm bort'' att skriva högerparentesen innan du trycker på \textit{Enter}. Vad händer?

\Subtask Evaulera uttrycket \code{"gurka" + "tomat"} i REPL. Vad har uttrycket för värde och typ? Vilken siffra står efter ordet \code{res} i variabeln som lagrar resultatet?

\begin{REPL}
scala> "gurka" + "tomat"   
\end{REPL}

\Subtask Evaluera uttrycket \code{res0 * 42} men byt ut \code{0}:an mot siffran efter \code{res} i utskriften från förra evalueringen. Vad har uttrycket för värde och typ?
\begin{REPL}
scala> res2 * 42
\end{REPL}

\Task Skapa med hjälp av en editor en fil med namn \texttt{hello-script.scala} som innehåller denna enda rad:
\begin{Code}
println("hej skript")
\end{Code}
Spara filen och kör kommandot \code{scala hello-script.scala} i terminalen:
\begin{REPL}
$ scala hello-script.scala
\end{REPL}

\Subtask Vad händer?

\Subtask Ändra i filen så att högerparentesen saknas. Spara och kör skriptfilen igen. Vad händer?

\Task Skapa med hjälp av en editor en fil med namn \texttt{hello-app.scala}.
\begin{REPL}
$ gedit hello-app.scala &
\end{REPL}
Skriv dessa rader i filen:


\scalainputlisting{examples/hello-app.scala}

\Subtask Kompilera med \code{scalac hello-app.scala} och kör koden med \code{scala Hello}.
\begin{REPL}
$ scalac hello-app.scala
$ ls
$ scala Hello
\end{REPL}
Vad heter filerna som kompilatorn skapar?


\Subtask\Pen Vilket alternativ går snabbast att köra igång, ett skript eller en kompilerad applikation? Varför? Vilket alternativ kör snabbast när väl exekveringen är igång?

\Subtask Ändra i din kod så att kompilatorn ger följande felmeddelande: \\
\texttt{Missing closing brace}

\Task Skapa med hjälp av en editor en fil med namn \texttt{Hi.java}.
\begin{REPL}
$ gedit Hi.java &
\end{REPL}
Skriv dessa rader i filen:


\javainputlisting{examples/Hi.java}

\Subtask Kompilera med \code{javac Hi.java} och kör koden med \code{java Hi}.
\begin{REPL}
$ javac Hi.java
$ ls
$ java Hi
\end{REPL}
Vad heter filen som kompilatorn skapat?

%\TaskSection{Uttryck}

\Task\Pen Vad är en \textit{literal}? 

\Task Vilken typ har följande literaler?

\Subtask \code{42} 

\Subtask \code{42L}

\Subtask \code{'*'}

\Subtask \code{"*"}

\Subtask \code{42.0}

\Subtask \code{42D}

\Subtask \code{42d}

\Subtask \code{42F}

\Subtask \code{42f}

\Subtask \code{true}

\Subtask \code{false}


\Task\Pen Vad gör dessa satser? Till vad används klammer och semikolon?
\begin{REPL}
scala> def p = { print("hej"); print("san"); println(42); println("gurka") }
scala> p;p;p;p
\end{REPL}

\Task\Pen Satser versus uttryck. 

\Subtask Vad är det för skillnad på en sats och ett uttryck?

\Subtask Ge exempel på satser som inte är uttryck?

\Subtask Förklara vad som händer för varje evaluerad rad:
\begin{REPL}[numbers=left, numberstyle=\color{black}\ttfamily\scriptsize\selectfont]
scala> def värdeSaknas = ()
scala> värdeSaknas
scala> värdeSaknas.toString
scala> println(värdeSaknas)
scala> println(println("hej"))
\end{REPL}

\Subtask Vilken typ har literalen \code{()}?

\Subtask Vilken returtyp har \code{println}?

\Task Vilken typ och vilket värde har följande uttryck? 

\Subtask \code{1 + 41}

\Subtask \code{1.0 + 41}

\Subtask \code{42.toDouble}

\Subtask \code{(41 + 1).toDouble}

\Subtask \code{"gurk" + 'a'}

\Subtask \code{'A'}

\Subtask \code{'A'.toInt}

\Subtask \linebreak[0] \code{'0'.toInt}

\Subtask \code{'1'.toInt}

\Subtask \code{'9'.toInt}

\Subtask \code{('A' + '0').toChar}

\Subtask \code{"*!%#".charAt(0)}

\Task \textit{De fyra räknesätten}. Vilket värde och vilken typ har följande uttryck?

\Subtask \code{42 * 2}

\Subtask \code{42.0 / 2}

\Subtask \code{42 - 0.2}

\Subtask \code{42L + 2d}

\Task \textit{Precedensregler}. Evalueringsordningen kan styras med parenteser. Vilket värde och vilken typ har följande uttryck?

\Subtask \code{42 + 2 * 2}

\Subtask \code{(42 + 2) * 2}

\Subtask \code{(-(2 - 42)) / (1 + 1 + 1).toDouble}

\Subtask \code{((-(2 - 42)) / (1 + 1 + 1).toDouble).toInt}


\Task \textit{Heltalsdivision}. Vilket värde och vilken typ har följande uttryck?

\Subtask \code{42 / 2}

\Subtask \code{42 / 4}

\Subtask \code{42.0 / 4}

\Subtask \code{1 / 4}

\Subtask \code{1 % 4}

\Subtask \code{2 % 42}

\Subtask \code{42 % 2}


\Task \textit{Hetalsomfång}. För var och en av heltalstyperna i deluppgifterna nedan: undersök i REPL med operationen \code{MaxValue} resp. \code{MinValue}, till exempel \code{Int.MaxValue} vad som är största och minsta värde. 

\Subtask \code{Byte}

\Subtask \code{Short}

\Subtask \code{Int}

\Subtask \code{Long}

\Task Klassen \code{java.lang.Math} och paketobjektet \code{scala.math}.

\Subtask Undersök genom att trycka på Tab-tangenten efter att du skriver nedan, vilka funktioner som finns i \code{Math} och \code{math}. Vad heter konstanten $\pi$ i java.lang.Math respektive scala.math?
\begin{REPL}
scala> java.lang.Math.    //tryck TAB
scala> scala.math.        //tryck TAB
\end{REPL}

\Subtask Undersök dokumentationen för klassen \code{java.lang.Math} här: \\ \url{https://docs.oracle.com/javase/8/docs/api/java/lang/Math.html} \\
Vad gör \code{java.lang.Math.hypot}?

\Subtask Undersök dokumentationen för pakobjektet \code{scala.math} här: \\
\url{http://www.scala-lang.org/api/current/#scala.math.package} \\
Ge exempel på någon funktion i \code{java.lang.Math} som inte finns i \code{scala.math}.

%\TaskSection{Noggranhet och undantag i aritmetiska uttryck}

\Task Vad händer här? Notera undantag \Eng{exceptions} och nogranhetsproblem. 

\Subtask \code{Int.MaxValue} + 1

\Subtask \code{1 / 0}

\Subtask \code{1E8 + 1E-8}

\Subtask \code{1E9 + 1E-9}

\Subtask \code{math.pow(math.hypot(3,6), 2)}

\Subtask \code{1.0 / 0}

\Subtask \code{(1.0 / 0).toInt}

\Subtask \code{math.sqrt(-1)}

\Subtask \code{math.sqrt(Double.NaN)}

\Subtask \code{throw new Exception("PANG!!!")}


\Task \textit{Booelska uttryck}. Vilket värde och vilken typ har följande uttryck?

\Subtask \code{true && true}

\Subtask \code{false && true}

\Subtask \code{true && false}

\Subtask \code{false && false}

\Subtask \code{true || true}

\Subtask \code{false || true}

\Subtask \code{true || false}

\Subtask \code{false || false}

\Subtask \code{42 == 42}

\Subtask \code{42 != 42}

\Subtask \code{42.0001 == 42}

\Subtask \code{42.0000000000000001 == 42}

\Subtask \code{42.0001 > 42}

\Subtask \code{42.0000000000000001 > 42}

\Subtask \code{42.0001 >= 42}

\Subtask \code{42.0000000000000001 <= 42}

\Subtask \code{true == true}

\Subtask \code{true != true}

\Subtask \code{true > false}

\Subtask \code{true < false}


\Task\Pen \textit {Variabler och tilldelning}. Rita en ny bild av datorns minne efter varje evaluerad rad nedan. Bilderna ska visa variablers namn, typ och värde. 
\begin{REPL}[numbers=left, numberstyle=\color{black}\ttfamily\scriptsize\selectfont]
scala> var a = 42
scala> var b = a + 1
scala> var c = (a + b) + 1.0
scala> b = 0
scala> a = 0
scala> c = c + 1
\end{REPL}
Efter första raden ser minnessituationen ut så här:

\vspace{0.5em}
\begin{tikzpicture}[font=\ttfamily]
\matrix [matrix of nodes, row sep=0, column 2/.style={nodes={rectangle,draw,minimum width=4em}}] (mat)
{
a: Int   &  \makebox(16,12){42}\\
};
\end{tikzpicture}

\Task\Pen \textit{Deklarationer: \code{var}, \code{val}, \code{def}}. Evaluera varje rad nedan i tur och ordning i Scala REPL. Förklarar för varje rad vad som händer. Vilka rader ger kompileringsfel och i så fall vilket och varför?
\begin{REPL}[numbers=left, numberstyle=\color{black}\ttfamily\scriptsize\selectfont]
scala> var x = 42
scala> x + 1
scala> x
scala> x = x + 1
scala> x
scala> x == x + 1
scala> val y = 42
scala> y = y + 1
scala> var z = {println("gurka"); 42}
scala> def w = {println("gurka"); 42}
scala> z
scala> z
scala> z = z + 1
scala> w
scala> w
scala> w = w + 1
\end{REPL}

\Task\Pen \code{if}\textit{-sats}. Vad händer nedan?
\begin{REPL}[basicstyle=\color{white}\ttfamily\fontsize{9}{11}\selectfont]
scala> if (true) println("sant") else println("falskt")
scala> if (false) println("sant") else println("falskt")
scala> if (!true) println("sant") else println("falskt")
scala> if (!false) println("sant") else println("falskt")
scala> def kasta = if (math.random > 0.5) println("krona") else println("klave")
scala> kasta; kasta; kasta
\end{REPL}


\Task \code{if}\textit{-uttryck}. Följande variabler är deklarerade med nedan initialvärden:

\begin{REPL}
scala> var grönsak = "gurka"
scala> var frukt = "banan"
\end{REPL}

Vad har följande uttryck för värden och typ?

\Subtask \code{if (grönsak == "tomat") "gott" else "inte gott" }

\Subtask \code{if (frukt == "banan") "gott" else "inte gott" }

\Subtask \code{if (frukt.size == grönsak.size ) "lika stora" else "olika stora" }



\Task \code{for}\textit{-sats}.

\Subtask Vad ger nedan \code{for}-satser för utskrift?

\begin{REPL}
scala> for (i <- 1 to 10) print(i + ", ")
scala> for (i <- 1 until 10) print(i + ", ")
scala> for (i <- 1 to 5) print((i * 2) + ", ")
scala> for (i <- 1 to 92 by 10) print(i + ", ")
scala> for (i <- 10 to 1 by -1) print(i + ", ")
\end{REPL}

\Subtask Skriv en \code{for}-sats som ger följande utskrift:
\begin{REPL}
A1, A4, A7, A10, A13, A16, A19, A22, A25, A28, A31, A34, A37, A40, A43, 
\end{REPL}

\Task \code{while}\textit{-sats}. 

\Subtask Vad ger nedan satser för utskriftert?
\begin{REPL}
scala> var i = 0
scala> while (i < 10) { println(i); i = i + 1 }
scala> var j = 0; while (j <= 10) { println(j); j = j + 2 }; println(j)
\end{REPL}

\Subtask Skriv en \code{while}-sats som ger följande utskrift med hjälp av en variabel \code{k} som initialiseras till 1:
\begin{REPL}
A1, A4, A7, A10, A13, A16, A19, A22, A25, A28, A31, A34, A37, A40, A43, 
\end{REPL}


\Task \textit{Slumptal}. Undersök vad dokumentationen säger om funktionen \code{scala.math.random}:\\
\url{http://www.scala-lang.org/api/current/#scala.math.package} 

\Subtask\Pen Vilken typ har värdet som returneras av funktionen \code{random}? 
 
\Subtask\Pen Vilket är det minsta respektive största värde som kan returneras? 

\Subtask\Pen Är \code{random} en \textit{äkta} funktion \Eng{pure function} i matematisk mening?

\Subtask Anropa funktionen \code{math.random} upprepade gånger och notera vad som händer. Använd pil-upp-tangenten.
\begin{REPL}
scala> math.random
\end{REPL}


\Subtask Vad händer? Använd \textit{pil-upp} och kör nedan \code{for}-sats flera gånger. Förklara vad som sker.

\begin{REPL}
scala> for (i <- 1 to 10) println(math.random)
\end{REPL}

\Subtask Skriv en for-sats som skriver ut 100 slumpmässiga heltal från 0 till och med 9 på var sin rad. 

\begin{REPL}
scala> for (i <- 1 to 100) println(???)
\end{REPL}

\Subtask Skriv en for-sats som skriver ut 100 slumpmässiga heltal från 1 till och med 6 på samm rad. 

\begin{REPL}
scala> for (i <- 1 to 100) print(???)
\end{REPL}


\Subtask Använd \textit{pil-upp} och kör nedan \code{while}-sats flera gånger. Förklara vad som sker.

\begin{REPL}
scala> while (math.random > 0.2) { println("gurka") }
\end{REPL}

\Subtask Ändra i \code{while}-satsen ovan så att sannolikheten ökar att riktigt många  strängar ska skrivs ut.

\Subtask Förklara vad som händer nedan.
\begin{REPL}
scala> var slumptal = math.random
scala> while (slumptal > 0.2) { println(slumptal); slumptal = math.random }
\end{REPL}

\Task\Pen \textit{Logik och De Morgans Lagar}. Förenkla följande uttryck. Antag att \code{poäng} och \code{highscore} är heltalsvariabler medan \code{klar} är av typen \code{Boolean}. 

\Subtask \code{poäng > 100 && poäng > 1000}

\Subtask \code{poäng > 100 || poäng > 1000}

\Subtask \code{!(poäng > highscore)}

\Subtask \code{!(poäng > 0 && poäng < highscore) }

\Subtask \code{!(poäng < 0 || poäng > highscore) }

\Subtask \code{klar == true}

\Subtask \code{klar == false}


\ExtraTasks

\Task \textit{Slumptal}.

\Subtask Ersätt \code{???} nedan med literaler så att \code{tärning} returnerar ett slumpmässigt heltal mellan 1 och 6.
\begin{REPL}
scala> def tärning = (math.random * ??? + ???).toInt 
\end{REPL}

\Subtask Ersätt \code{???} med literaler så att rnd blir ett decimaltal med max en decimal mellan 0.0 och 1.0.
\begin{REPL}
scala> def rnd = math.round(math.random * ???) / ??? 
\end{REPL}

\Subtask Vad blir det för skillnad om \code{math.round} ersätts med \code{math.floor} ovan? (Se dokumentationen av \code{java.lang.Math.round} och \code{java.lang.Math.floor}.)

\AdvancedTasks

\Task Integer.toBinaryString, Integer.toHexString

\Task 0x2a

\Task \code{i += 1; i *= 1; i /= 2}

\Task BigInt, BigDecimal

\Task Vad händer här? 
\begin{REPL}
scala> Math.multiplyExact(2, 42)
scala> Math.multiplyExact(Int.MaxValue, Int.MaxValue)
\end{REPL}

\Task Sök reda på dokumentationen för funktionen multiplyExact i javadoc för klassen java.lang.Math i JDK 8. 

\Task Sök i javadoc för Math efter förekomster av texten \textit{''throwing an exception if the result overflows''}. Vilka fler funktioner finns i java.lang.Math som hjälper en att upptäcka om det blir overflow?

\Task Använda Scala REPL för att undersöka konstanterna nedan. Vilket av dessa värden är negativt? Vad kan man ha för praktisk nytta av dessa värden i ett program som gör flyttalsberäkningar?

\Subtask \code{java.lang.Double.MIN_VALUE}

\Subtask \code{scala.Double.MinValue} 

\Subtask \code{scala.Double.MinPositiveValue}

\Task För typerna \code{Byte}, \code{Short}, \code{Char}, \code{Int}, \code{Long}, \code{Float}, \code{Double}: Undersök hur många bitar som behövs för att representera varje typs omfång? \\*
\textit{Tips:} Några användbara uttryck: \\*
 \code{Integer.toBinaryString(Int.MaxValue + 1).size} \\*
 \code{Integer.toBinaryString((math.pow(2,16) - 1).toInt).size} \\*
 \code{1 + math.log(Long.MaxValue)/math.log(2)}
Se även språkspecifikationen för Scala, kapitlet om heltalsliteraler: \\
\url{http://www.scala-lang.org/files/archive/spec/2.11/01-lexical-syntax.html#integer-literals}

\Subtask Undersök källkoden för pakobjektet \code{scala.math} här: \\
\url{https://github.com/scala/scala/blob/v2.11.7/src/library/scala/math/package.scala} \\
Hur många olika överlagrade varianter av funktionen \code{abs} finns det och för vilka parametertyper är den definierad?



