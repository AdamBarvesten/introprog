%!TEX encoding = UTF-8 Unicode

%!TEX root = ../compendium.tex

\Exercise{\ExeWeekELEVEN}\label{exe:W11}

\begin{Goals}
\item Kunna förklara och beskriva viktiga skillnader mellan Scala och Java.
\item Kunna översätta enkla algoritmer, klasser och singeltonobjekt från Scala till Java och vice versa.
\item Känna till vad en case-klass innehåller i termer av en Javaklass.
%\item Förstå hur autoboxing fungerar.
\item Kunna använda Javatyperna \code{List}, \code{ArrayList}, \code{Set}, \code{HashSet} och översätta till deras Scalamotsvarigheter med \code{JavaConverters}.
\end{Goals}

\begin{Preparations}
\item \StudyTheory{11} 
\end{Preparations}

\BasicTasks %%%%%%%%%%%%%%%%

\Task \emph{Översätta algoritmer och metoder från Java till Scala.} I denna uppgift ska du översätta en Java-klass som används som en modul\footnote{\href{https://en.wikipedia.org/wiki/Modular_programming}{en.wikipedia.org/wiki/Modular\_programming}} och bara innehåller statiska metoder och inget varaktigt tillstånd. (I nästa uppgift ska du sedan översätta klasser med attribut och varaktiga tillstånd.) 

Vi börjar med att göra översättningen från Java till Scala rad för rad och du ska behålla så mycket som möjligt av syntax och semantik så att Scala-koden blir så Java-lik som möjligt. I efterföljande deluppgift ska du sedan omforma översättningen så att Scala-koden blir mer idiomatisk\footnote{\href{https://sv.wikipedia.org/wiki/Idiom_\%28programmering\%29}{sv.wikipedia.org/wiki/Idiom\_\%28programmering\%29}}.

\Subtask Studera klassen \code{Hangman} nedan. Du ska översätta den från Java till Scala enlig de riktlinjer och tips som följer efter koden. Läs igenom alla riktlinjer och tips innan du börjar.

\javainputlisting[numbers=left]{examples/scalajava/Hangman.java}

\noindent\emph{Riktlinjer och tips för översättningen:}

\begin{enumerate}[noitemsep]

\item Skriv Scala-koden med en texteditor i en fil som heter \texttt{hangman1.scala} och kompilera med \code{scalac hangman1.scala} i terminalen; använd alltså \emph{inte} en IDE, så som Eclipse eller IntelliJ, utan en ''vanlig'' texteditor, t.ex. \code{gedit}.

\item Översätt i denna första deluppgift rad för rad så likt den ursprungliga Java-kodens utseende (syntax)  som möjligt, med så få ändringar som möjligt. Du ska alltså ha kvar dessa Scalaovanligheter, även om det inte alls blir som man brukar skriva i Scala: 
\begin{enumerate}[nolistsep, noitemsep]
\item långa indrag, \item onödiga semikolon, \item onödiga \code{()}, \item onödiga \code|{}|, \item onödiga \code{System.out}, och \item onödiga \code{return}. 
\end{enumerate}

\item Försök också i denna deluppgift göra så att betydelsen (semantiken) så långt som möjligt motsvarar den i Java, t.ex. genom att använda \code{var} överallt, även där man i Scala normalt använder \code{val}. 

\item En Javaklass med bara statiska medlemmar motsvara ett singeltonobjekt i Scala, alltså en \code{object}-deklaration innehållande ''vanliga'' medlemmar. 
 
\item För att tydliggöra att du använder Javas \code{Set} och \code{HashSet} i din Scala-kod, använd följande import-satser i \code{hangman1.scala}, som därmed döper om dina importerade namn och gör så att de inte krockar med Scalas inbyggda \code{Set}. Denna form av import går inte att göra i Java.
\begin{Code}
import java.util.{Set => JSet};
import java.util.{HashSet => JHashSet};
\end{Code}

\item Javas \code{i++} fungerar inte i Scala; man får istället skriva \code{i += 1} eller mindre vanliga \code{i = i + 1}.

\item Typparametrar i Java skrivs inom \code{<>} medan Scalas syntax för typparametrar använder \code{[]}.

\item Till skillnad från Java så har Scalas metoddeklarationer ett tilldelningstecken \code{=} efter returtypen, före kroppen.

\item Du kan ladda ner Java-koden till \code{Hangman}-klassen nedan från kursens repo%
\footnote{\href{https://github.com/lunduniversity/introprog/blob/master/compendium/examples/scalajava/Hangman.java}{github.com/lunduniversity/introprog/blob/master/compendium/examples/scalajava/Hangman.java}}. I samma bibliotek ligger även lösningarna till översättningen i Scala, men kolla \emph{inte} på dessa förrän du gjort klart översättningarna och fått dem att kompilera och köra felfritt! Tanken är att du ska träna på att läsa felmeddelande från kompilatorn och åtgärda dem i en upprepad kompilera-testa-rätta-cykel.

\end{enumerate}



\TODO Flytta nedan kod till facit:
\scalainputlisting[numbers=left,basicstyle=\ttfamily\fontsize{10.3}{12}\selectfont]{examples/scalajava/hangman1.scala}



\Subtask Skapa en ny fil \code{hangman2.scala} som till att börja med innehåller en kopia av din direkt-översatta Java-kod från föregående deluppgift. Omforma koden så att den blir mer som man brukar skriva i Scala, alltså mer Scala-idiomatisk. Försök förenkla och förkorta så mycket du kan utan att göra avkall på läsbarheten.

\emph{Tips och riktlinjer:}

\begin{enumerate}[nolistsep, noitemsep]

\item Kalla Scala-objektet för \code{hangman}. När man använder ett Scalaobjekt som en modul (alltså en samling funktioner i en gemensam, avgränsad namnrymd) har man gärna liten begynnelsebokstav, i likhet med konventionen för paketnamn. Ett paket är ju också en slags modul och med en namngivningskonvention som är gemensam kan man senare, utan att behöva ändra koden som använder modulen, ändra från ett singelobjektet till ett paket och vice versa om man så önskar.

\item Gör alla metoder publikt tillgängliga och låt även strängvektorn \code{hangman} vara publikt tillgänglig. Deklarera \code{hangman} som en \code{val} och konstruera den med \code{Vector}. Eftersom \code{Vector} är oföränderlig och man inte kan ärva från singelobjekt och \code{hangman} är deklarerad med \code{val} finns inga speciella risker med att göra den konstanta vektorn publik om  vi inte har något emot att annan kod kan läsa (och eventuellt göra sig beroende av) vår hänggubbetext.

\item I metoden \code{renderHangan} använd \code{take} och \code{mkString}.

\item I metoden \code{hideSecret} använd \code{map} i stället för en \code{for}-sats.

\item Det går att ersätta metoden \code{findAll} med det kärnfulla uttrycket \\ \code{(secret forall found)} där \code{secret} är en sträng och \code{found} är en mängd av tecken (undersök gärna i REPL hur detta fungerar). Skippa därför den metoden helt och använd det kortare uttrycket direkt.

\item I metoden \code{makeGuess}, i stället för \code{Scanner}, använd \code{scala.io.StdIn.readLine}. 

\item Om du vill träna på att använda rekursion i stället för imperativa loopar: Gör metoden \code{makeGuess} rekursiv i stället för att använda \code{do}-\code{while}.

\item I metoden \code{download}, i stället för \code{java.net.URL} och \code{java.util.ArrayList}, använd \code{scala.io.Source.fromURL(address, coding).getLines.toVector} och gör en lokal import av \code{scala.io.Source.fromURL} överst i det block där den används. Det går inte att ha lokala \code{import}-satser i Java.

\item Låt metoden \code{download} returnera en \code{Option[String]} som i fallet att nedladdningen misslyckas returnerar \code{None}.

\item I metoden \code{download}, i stället för \code{try}-\code{catch} använd \code{scala.util.Try} och dess smidiga metoder \code {recover} och \code{toOption}.  

\item Om du vill träna på att använda rekursion i stället för imperativa loopar: Använd, i stället för \code{while}-satsen i metoden \code{play}, en lokal rekursiv funktion med denna signatur: 
\begin{Code}
  def loop(found: Set[Char], bad: Int): (Int, Boolean)
\end{Code}
Funktionen \code{loop} returnerar en 2-tupel med antalet felgissningar och \code{true} om man hittat alla bokstäver eller \code{false} om man blev hängd. 

\end{enumerate}

\TODO Flytta nedan kod till facit:
\scalainputlisting[numbers=left,basicstyle=\ttfamily\fontsize{11.2}{13}\selectfont]{examples/scalajava/hangman2.scala}




\Task \emph{Översätta mellan klasser i Scala och klasser i Java.}  
Klassen \code{Point} nedan är en model av en punkt som kan sparas på begäran i en lista. Listan är privat för kompanjonsobjektet och kan skrivas ut med en metod \code{showSaved}. I koden används en \code{ArrayBuffer}, men i framtiden vill man, vid behov, kunna ändra från \code{ArrayBuffer} till en annan sekvenssamlingsimplementation, t.ex. \code{ListBuffer}, som uppfyller egenskaperna hos supertypen \code{Buffer}, men har andra prestandaegenskaper för olika operationer. Därför är attributet \code{saved} i kompanjonsobjektet deklarerat med den mer generella typen.

\scalainputlisting[numbers=left]{examples/scalajava/Point.scala}

\Subtask Översätt klassen \code{Point} ovan från Scala till Java. Vi ska i nästa deluppgift kompilera både Scala-programmet ovan och ditt motsvarande Java-program i terminalen och testa i REPL att klasserna har motsvarande funktionalitet. 

\emph{Tips och riktlinjer:}
\begin{enumerate}[nolistsep, noitemsep]
\item För att namnen inte ska krocka i våra kommande tester, kalla Javatypen för \code{JPoint}. 
\item  I ställert för Scalas \code{ArrayBuffer} och \code{Buffer}, använd Javas \code{ArrayList} och \code{List} som båda ligger i paketet \code{java.util}. 
\item Undersök dokumentationen för \code{java.util.List} för att hitta en motsvarighet till \code{prepend} för att lägga till i början av listan.
\item I stället för default-argumentet i Scalas primärkonstruktor, använd en extra Java-konstruktor. 
\item Det finns inga singelobjekt och inga kompanjonsobjekt i Java; istället kan man använda statiska klassmedlemmar. Placera kompanjonsobjektets medlemmars motsvarigheter \emph{innuti} Java-klassen och gör dem till \jcode{static}-medlemmar.
\item Kod i klasskroppen i Scalaklassen, så som if-satsen på rad 4, placeras i lämplig konstruktor i Javaklassen.
\item Utskrifter med \code{print} och \code{println} behöver i Java föregås av \code{System.out.}
\item Det finns inget nyckelord \code{override} i Java, men en s.k. annotering som ger samma kompilatorhjälp. Den skrivs med ett snabel-a och stor begynnelsebokstav, så här: \jcode{ @Override }  före metoddeklarationen.
\item I Java används konventionen att börja getter-metoder med ordet \code{get}, t.ex. \code{getX()}.
\item Det finns ingen motsvarighet till \code{mkString} för \code{List} så du behöver själv gå igenom listan och hämta elementreferenser för utskrift med en \jcode{for}-loop. Notera att efter sista elementet ska radbrytning göras i utskriften och att inget komma ska skrivas ut efter sista elementet.
\item I Java behövs en ny \jcode{import}-deklaration om man vill importera ännu en typ från samma paket. Man kan även i Java använda asterisk \code{*}, (motsvarande \code{_} i Scala), för att importera allt i ett paket, men då får man med alla möjliga namn och det vill man kanske inte.
\item Metoder i Java slutar med \code{()} om de saknar parametrar. 
\item Alla satser i Java slutar med lättglömda semikolon. (Efter att man i skrivit mycket Javakod och växlar till Scalakod är det svårt att vänja sig av med att skriva semikolon...)
\end{enumerate}


\TODO Flytta nedan kod till facit:
\javainputlisting[numbers=left]{examples/scalajava/JPoint.java}


\Subtask Starta REPL i samma bibliotek som du kompilerat kodfilerna. Testa så att klasserna \code{Point} och \code{JPoint} beter sig på samma vis enligt nedan. Skriv även testkod i REPL för att avläsa de attributvärden som har getters och undersök att allt funkar som det ska.
\begin{REPLnonum}
$ scalac Point.scala
$ javac JPoint.java
$ scala
scala> val (p, jp) = (new Point, new JPoint)
scala> p.distanceTo(new Point(3, 4))
scala> Point.showSaved
scala> jp.distanceTo(new JPoint(3, 4))
scala> JPoint.showSaved
scala> for (i <- 1 to 10) { new Point(i, i, true) }
scala> Point.showSaved
scala> for (i <- 1 to 10) { new JPoint(i, i, true) }
scala> JPoint.showSaved
\end{REPLnonum}


\Subtask Översätt nedan Javaklass \code{JPerson} till en \code{case class Person} i Scala med  motsvarande funktionalitet.  


\javainputlisting[numbers=left]{examples/scalajava/JPerson.java}


\TODO flytta dena kod till FACIT:
\begin{Code}
case class Person(name: String, age: Int = 0)
\end{Code}

\Subtask\Pen Undersök i REPL vilken funktionalitet i Scala-case-klassen \code{Person} som \emph{inte} är implementerad i Java-klassen \code{JPerson} ovan. Skriv upp namnen på några av case-klassens extra metoder samt deras signatur genom att för en \code{Person}-instans, och för kompansjonsobjektet \code{Person}, trycka på TAB-tangenten. Prova några av de extra metoderna i REPL och förklara vad de gör.

\begin{REPL}
scala> val p = Person("Björn", 49)
scala> p.      // tryck TAB en gång
scala> Person. // tryck TAB en gång
scala> p.copy  // tryck TAB en gång
scala> p.copy()
scala> p.copy(age = p.age + 1)
scala> Person.unapply(p)
\end{REPL}


\Task \emph{Auto(un)boxing.} I JVM måste typparametern för generiska klasser vara av referenstyp. I Scala löser kompilatorn detta åt oss så att vi ändå kan ha t.ex. \code{Int} som argument till en typparameter i Scala, medan man i Java \emph{inte} direkt kan ha den primitiva typen \jcode{int} som typparameter till t.ex. \code{ArrayList}.

I Java och i den underliggande plattformen JVM används s.k. wrapper-klasser för att lösa detta, t.ex. genom wrapper-klassen \code{Integer} som boxar den primitiva typen \jcode{int}. Java-kompilatorn har stöd för att automatiskt packa in värden av primitiv typ i sådana wrapper-klasser för att skapa referenstyper och kan även automatiskt packa upp dem. 

\Subtask Studera hur Scala-kompilatorn låter oss arbeta med en \code{Cell[Int]} även om det underliggande JVM:ens körtidstyp \Eng{runtime type} är en wrapper-klass. Man kan se JVM-körtidstypen med metoderna \code{getClass} och \code{getTypeName} enligt nedan.
\begin{REPL}
scala> class Cell[T](var value: T){
         val typeName: String = value.getClass.getTypeName
         override def toString = "Cell[" + typeName + "](" + value + ")"
       }
scala> val c = new Cell[Int](42)
scala> c.value.getClass.getTypeName
\end{REPL}


\Subtask Vad är körtidstypen för \code{c.value} ovan? Förklara hur det kan komma sig trots att vi deklarerade med typargumentet \code{Int}? 

\Subtask Studera dokumentationen för \code{java.lang.Integer}\footnote{\href{https://docs.oracle.com/javase/8/docs/api/java/lang/Integer.html}{docs.oracle.com/javase/8/docs/api/java/lang/Integer.html}} och testa i REPL några av \emph{klassmetoderna} (de som är \jcode{static} och därmed kan anropas med punktnotation direkt på klassens namn utan \code{new}) och några av \emph{instansmetoderna} (de som inte är \jcode{static}). 
\begin{REPL}
scala> Integer.  //tryck TAB
scala> Integer.
scala> Integer.toBinaryString(42)
scala> Integer.valueOf(42)
scala> val i = new Integer(42)
scala> i.  // tryck TAB
scala> i.toString
scala> i.compareTo  // tryck TAB 2 gånger
scala> i.compareTo(Integer.valueOf(42))
scala> i.compareTo(42)  // varför fungerar detta?      
\end{REPL} 

\Subtask\Pen Enligt dokumentationen\footnote{\href{https://docs.oracle.com/javase/8/docs/api/java/lang/Integer.html\#compareTo-java.lang.Integer-}{docs.oracle.com/javase/8/docs/api/java/lang/Integer.html\#compareTo-java.lang.Integer-}} tar instansmetoden \code{compareTo} i klassen \code{Integer} en \code{Integer} som parameter. Hur kan det då komma sig att sista raden ovan fungerar med en \code{Int}?

\Subtask Studera nedan Java-program och beskriv vad som kommer att skrivas ut \emph{innan} du kompilerar och testkör.

\javainputlisting[numbers=left]{examples/scalajava/Autoboxing.java}

\Subtask Ändra i programmet ovan så att autoboxing och autounboxing utnyttjas på alla ställen där så är möjligt. Utnyttja även att \code{toString}-metoden på \code{Integer} ger samma stränrepresentation som \jcode{int} vid utskrift. Fixa också så att du undviker \emph{fallgropen} att i Java jämföra med referenslikhet i stället för att använda \code{equals}. Testa så att allt fungerar som det borde efter dina ändringar.

\TODO kod till facit:
\javainputlisting[numbers=left]{examples/scalajava/Autoboxing2.java}

\Subtask\Pen Antag att du råkar skriva \jcode{xs.add(0, pos)} på rad 14 i ditt program från föregående uppgift. Förklara hur autoboxingen stjälper dig i en \emph{fallgrop} då. 

\Subtask\Pen Med ledning av de båda tidigare deluppgifterna: sammanfatta de två nämnda fallgropar med autoboxing i Java i två generella punkter, så att du har nytta av att memorera dem inför din framtida Javakodning.


\Task \TODO JavaConverters


\ExtraTasks %%%%%%%%%%%%%%%%%%%

\Task Översätt nedan kod från Java till Scala. Skriv koden i en fil som heter \texttt{showInt.scala} och kalla Scala-objektet med \code{main}-metoden för \code{showInt}. Läs tipsen som följer efter koden innan du börjar. 

\javainputlisting[numbers=left]{examples/scalajava/JShowInt.java}

\emph{Tips:}
\begin{itemize}[nolistsep, noitemsep]
\item En Javaklass med bara statiska medlemmar motsvaras av ett singeltonobjekt i Scala, alltså en \code{object}-deklaration. Scala har därför inte nyckelordet \jcode{static}.
\item Typen \jcode{Object} i Java motsvaras av Scalas \code{Any}.
\item Du kan använda Scalas möjlighet med default-argument (som saknas i Java) för att bara definiera en enda \code{show}-metod med en tom sträng som default \code{msg}-argument.
\item I Scala har objekt av typen \code{Char} en metod \code{def *(n: Int): String} som skapar en sträng med tecknet repeterat \code{n} gånger. Men du kan ju välja att ändå implementera metoden \code{repeatChar} med \code{StringBuilder} som nedan om du vill träna på att översätta en \code{for}-loop från Java till Scala.
\item I stället för \code{Scanner.nextLine} kan använda \code{scala.io.StdIn.readLine} som tar en prompt som parameter, men du kan också använda \code{Scanner} i Scala om du vill träna på det.
\item I Java \emph{måste} man använda nyckelordet \jcode{return} om metoden inte är en \jcode{void}-metod, medan man i Scala faktiskt \emph{får} använda \code{return} även om man brukar undvika det och i stället utnyttja att satser i Scala också är uttryck.
\end{itemize}
Kompilera din Scala-kod och kör i terminalen och testa så att allt funkar. Vill du även kompilera Java-koden så finns den i kursens repo i filen\\ \texttt{compendium/examples/scalajava/JShowInt.java}




\Task \TODO 
Fallgrop med Point som inte har equals och en Polygon som kolla hasVertex som inte hittas...\\ \url{https://github.com/bjornregnell/lth-eda016-2015/blob/master/lectures/examples/eclipse-ws/lecture-examples/src/week10/generics/TestPitfall3.java}


\AdvancedTasks %%%%%%%%%%%%%%%%%

 
\Task \TODO \emph{Gränssnitt i Scala och Java.} 

\Task\Pen Studera fallgropar för hur man skriver en \code{equals}-metod i Java här:
\url{http://www.artima.com/lejava/articles/equality.html}\\
Vilken fallgrop trillar man \emph{inte} i om man endast överskuggar \code{equals} i finala klasser som inte har några superklasser?

\Task\Pen Studera det fullständiga receptet för hur man skriver en välfungerande \code{equals} och \code{hashcode} i Scala här: \url{http://www.artima.com/pins1ed/object-equality.html} 


    