%!TEX encoding = UTF-8 Unicode

%!TEX root = ../compendium.tex

\Exercise{\ExeWeekELEVEN}\label{exe:W11}

\begin{Goals}
\item \TODO
\end{Goals}

\begin{Preparations}
\item \StudyTheory{11} 
\end{Preparations}

\BasicTasks %%%%%%%%%%%%%%%%

\TODO

\Task \emph{Grundläggande syntaxskillnader mellan Scala och Java.} 

\Subtask Översätt nedan kod från Scala till Java. Skriv koden i en editor i två filer, en för Scala och en för Java. Kompilera båda programmen i terminalen och testa så de skriver ut samma sak. 

\begin{Code}
\end{Code}

\Subtask Översätt nedan kod från Java till Scala.


\Task \emph{Klasser i Scala kontra Java.}  

\Subtask Översätt nedan klass från Scala till Java. I ställert för typerna Buffer och ArrayBuffer, använd typerna List och ArrayList.

\begin{Code}
class Point(val x: Int, val y: Int, addToGrid: Boolean = false) {
  import Point._
  
  if (addToGrid) grid.append(this)
  
  def this() = this(0.0, 0.0)
  
  
}
object Point {
  import scala.collection.mutable.ArrayBuffer
  private val grid: Buffer = ArrayBuffer.empty
  
}
\end{Code}

\Subtask Översätt nedan klass från Java till Scala. 


\Task \emph{Auto(un)boxing i JVM.} I JVM måste typparametern för generiska klasser vara av referenstyp. I Scala löser kompilatorn detta åt oss så att vi ändå kan ha t.ex. \code{Int} som argument till en typparameter i Scala. 

I Java och i den underliggande plattformen JVM används s.k. wrapper-klasser användas för att lösa detta, t.ex. genom wrapper-klassen \code{Integer} som boxar den primitiva typen \jcode{int}. Java-kompilatorn har stöd för att automatiskt packa in värden av primitiv typ i sådana wrapper-klasser för att skapa referenstyper och kan även automatiskt packa upp dem. 

\Subtask Studera hur Scala-kompilatorn låter oss arbeta med en Cell[Int] även om det underliggande JVM:ens körtidstyp \Eng{runtime type} är en wrapper-klass. Man kan se JVM-körtidstypen med metoderna \code{getClass} och \code{getTypeName} enligt nedan.
\begin{REPL}
scala> class Cell[T](var value: T){
         val typeName: String = value.getClass.getTypeName
         override def toString = "Cell[" + typeName + "](" + value + ")"
       }
scala> val c = new Cell(42)
scala> c.value.getClass.getTypeName
\end{REPL}

\Subtask Vad är körtidstypen för c.value ovan? Förklara hur det kan komma sig? 


\Subtask 

\ExtraTasks %%%%%%%%%%%%%%%%%%%

\TODO

\Task 

\AdvancedTasks %%%%%%%%%%%%%%%%%

\TODO

\Task     
    