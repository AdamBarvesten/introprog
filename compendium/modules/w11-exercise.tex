%!TEX encoding = UTF-8 Unicode

%!TEX root = ../compendium.tex

\Exercise{\ExeWeekELEVEN}

\begin{Goals}
\item 
\end{Goals}

\begin{Preparations}
\item 
\end{Preparations}

\BasicTasks %%%%%%%%%%%%%%%%

\Task \emph{Auto(un)boxing i JVM.} I JVM måste typparametern för generiska klasser vara av referenstyp. I Scala löser kompilatorn detta åt oss så att vi ändå kan ha t.ex. \code{Int} som argument till en typparameter i Scala. Men i Java och i den underliggande plattformen JVM, så måste s.k. wrapper-klasser användas, t.ex. \code{Integer} som boxar en \jcode{int}.

\Subtask Studera hur Scala-kompilatorn låter oss arbeta med en Cell[Int] även om det inderliggande JVM:ens körtidstyp \Eng{runtime type} är en wrapper-klass. Man kan se JVM-körtidstypen med metoderna \code{getClass} och \code{getTypeName} enligt nedan.
\begin{REPL}
scala> class Cell[T](var value: T){
         val typeName: String = value.getClass.getTypeName
         override def toString = "Cell[" + typeName + "](" + value + ")"
       }
scala> val c = new Cell(42)
scala> c.value.getClass.getTypeName
\end{REPL}

\Subtask Vad är körtidstypen för c.value ovan? Förklara hur det kan komma sig? 


\Subtask 

\ExtraTasks %%%%%%%%%%%%%%%%%%%

\Task 

\AdvancedTasks %%%%%%%%%%%%%%%%%

\Task     
    