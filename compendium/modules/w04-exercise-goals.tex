%!TEX encoding = UTF-8 Unicode
%!TEX root = ../compendium.tex

\item Kunna skapa och använda tupler, som variabelvärden, parametrar och returvärden.

\item Förstå skillnaden mellan ett objekt och en klass och kunna förklara betydelsen av begreppet instans.

\item Kunna skapa och använda attribut som medlemmar i objekt och klasser och som som klassparametrar.

\item Beskriva innebörden av och syftet med att ett attribut är privat.

\item Kunna byta ut implementationen av metoden \code{toString}.

\item Kunna skapa och använda en objektfabrik med metoden \code{apply}.

\item Kunna skapa och använda en enkel case-klass.

\item Kunna använda operatornotation och förklara relationen till punktnotation.

\item Förstå konsekvensen av uppdatering av föränderlig data i samband med multipla referenser.

\item Känna till och kunna använda några grundläggande metoder på samlingar.

\item Känna till den principiella skillnaden mellan \code{List} och \code{Vector}.

\item Kunna skapa och använda en oföränderlig mängd med klassen \code{Set}.

\item Förstå skillnaden mellan en mängd och en sekvens.

\item Kunna skapa och använda en nyckel-värde-tabell, \code{Map}.

\item Förstå likheter och skillnader mellan en \code{Map} och en \code{Vektor}.

