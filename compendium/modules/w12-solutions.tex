%!TEX encoding = UTF-8 Unicode

%!TEX root = ../compendium.tex

\ExerciseSolution{\ExeWeekTWELVE}
                   
\BasicTasks %%%%%%%%%%%

\Task     %%%TODO number  1 %%%starts with: \emph{Trådar.}  %%%

\Subtask  -

\Subtask \code {java.lang.IllegalThreadStateException}. Det går inte att starta en tråd mer än en gång. Tråden kan därför inte startas om när den redan har exekverats.

\Subtask  När \code {start} anropas exekveras koden i \code{run} parallellt. Därför skrivs \code{Gurka} och \code{Tomat} ut omlöpande. Om istället \code{run} anropas direkt blir det inte jämnlöpande exekvering och \code{Gurka} skrivs ut 100 gånger, sedan skrivs \code{Tomat} ut 100 gånger.

\Subtask  \code{Thread.sleep} pausar inte tråden i exakt den tiden som angets. Alltså kommer det skrivas ut \code{zzz snark hej!} i de flesta fall, men det är inte garanterat.


\Task     %%%TODO number  2 %%%starts with: \emph{Jämlöpande variabeluppdat%%%

\Subtask I \code{slösaSpara} hämtas saldot, ändras och placeras tillbaka i minnet -  fördröjs -  upprepas. Om \code{bamse} blir klar med att ladda, ändra och lagra innan skutt gör detsamma med den muterbara variablen hade det inte varit perfekt. Problemet ligger i  när en tråd laddar och innan den kan lagra det förändrade värdet laddar den andra tråden samma värde. Bara en av dessa trådar vinner racet och får lagra sitt ändrade tal. \code{skutt} och \code{bamse} blir alltså upprörda för att inte alla dess uttag och insättningar registreras.

\Task     %%%TODO number  3 %%%starts with: \emph{Jämlöpande exekvering med%%%

Nu är \code{farmor} garanterad att kunna ladda saldot, ta ut pengar/ändra och lagra innan \code{vargen} kan överskriva resultatet. I \code{slösaSpara} pausas tråden i en millisekund så \code{vargen} kan fortfarande ta ut pengar innan \code{farmor} hinner sätta in pengar igen. Dock kommer alla uttag och insättningar registreras eftersom operationerna är atomära.

\Task     %%%TODO number  4 %%%starts with: TODO  %%%%%%%%%%%%%%%%%%%\Advan%%%

\Subtask error: Cannot find an implicit ExecutionContext. Future behöver en ExecutionContext för att kunna köras. \code{f} är av typen Future[Unit].

\Subtask Funktionen \code{printLater} har en Future, vilket innebär att när både \code{printLater} och \code{println} anropas i foreach-loopen exekveras dom jämnlöpande. Eftersom det tar längre tid att starta upp en Future för datorn är \code{println} snabbare och skriver ut att alla är igång först. Sedan skrivs siffrorna från 1 - 42 ut med oregelbundna mellanrum eftersom tråden pausas olika länge.

\Subtask \code{big} är en Future[Int]. Det stora talet har 7 520 383 siffror. \code{r} är av typen Try[Int] (se dokumentationen för Future om du är osäker)

\Subtask Eftersom exekveringen blockas tills den har fått ett resultat går det inte att fortsätta skriva i REPL medan uträkningen pågår. Väntar man för kort tid får man ett TimeOutException och uträkningen avbryts.

\Task     %%%TODO number  5 %%%starts with: Sök upp och studera dokumentati%%%

\Subtask -

\Subtask -

\Subtask Varje sida fördröjs med mellan 2 upp till 3 sekunder (2000-3000 millisekunder). Så i medeltal tar det 2.5 sekunder för varje sida att laddas. Vektorn måste fyllas innan exekveringen kan fortsätta. Därför laddas alla 10 stycken sidor in innan man kan se första websidan. Det tar därför i medeltal 2.5 x 10 = 25 sekunder.

\Subtask \code{f} ger en Vektor fylld med strängar där varje element ges av en rad på hemsidan. Då \code{f} körs i bakgrunden kan programmet fortlöpa medan innehållet räknas ut. Du kan därför skriva \code{f} i REPL:n men det är inte säkert att proccessen är klar och det slutgilltiga resultatet visas.

\Subtask Samma som ovan, förutom att det blir en vektor där varje element är i sig en vektor med strängar. 

\Subtask Laddar in datan parallelt så nedladdingen sker samtidigt, men det går olika snabbt pga metoden seg.

\Subtask Eftersom datan laddas i parallella trådar utan blockering blir dom inte klara i ordning, utan i den ordningen tråden körs klart. Till slut blir alla klara och resultatet visar en vektor med \code{true} värden.

\Subtask Metoden \code{lycka} är väldefinerad och kastar därför inga undantag. Den skriver alltid ut \code{:)}. Metoden \code{olycka} är inte väldefinerad då division med 0 ger \code{java.lang.ArithmeticException}. Detta fångas upp vid callbacken och det skrivs ut \code{:(} samt det specifierade undantaget.

\ExtraTasks %%%%%%%%%%%%

\Task  %%%TODO number  6 %%%

\begin{Code}
def isPrime(n: BigInt): Boolean = n match {
  case _ if (n <= 1) => false
  case _ if (n <= 3) => true
  case _ if n % 2 == 0 || n % 3 == 0 => false
  case _ => 
    var i = BigInt(5)
    while (i * i < n) {
      if (n % i == 0 || n % (i + 2) == 0) false
      i += 6
    }
    true
}

import scala.concurrent._
import ExecutionContext.Implicits.global

val primes = Vector.fill(10)(Future{nextPrime(randomBigInt(16))})
primes.foreach(_.onSuccess{case i => println(i)})
\end{Code}

\Task %%%TODO number  7 %%%

\Subtask Stackoverflow ger följande förklaring: 

A thread is an independent set of values for the processor registers (for a single core). Since this includes the Instruction Pointer (aka Program Counter), it controls what executes in what order. It also includes the Stack Pointer, which had better point to a unique area of memory for each thread or else they will interfere with each other.

\Subtask

\begin{Code}
val thread = new Thread(new Runnable{
	def run(){println(''Det här är en tråd'')}
})
\end{Code}

\Subtask \code{thread.start}

\Subtask Det kan bli kapplöpning(race conditions) om vilken tråds resurser blir sparade. Vilket leder till att de andra trådarnas ändringar blir ignorerade.

\Subtask Trådsäkerhet innebär att flera trådar kan köras parallellt utan felaktigheter i resultatet. Exempelvis får man vara väldigt försiktig om man vill ha en muterbar variabel som alla trådar ska ändra samtidigt.

\Subtask Till exempel slipper man skapa instanser av klassen Thread eftersom man kan placera koden i en Future istället. Den löser även mycket under huven för kodaren.

\Task %%%TODO number  8 %%%

-

\AdvancedTasks %%%%%%%%%

\Task %%%TODO number  9 %%%

\Subtask \code{abbasillen} skrivs ut baklänges till \code{nellisabba}.

\Subtask

\Subtask

\Subtask

\Subtask

\Subtask

\Subtask

\Subtask

\Subtask

Lösningsförslag:
\scalainputlisting[numbers=left,basicstyle=\ttfamily\fontsize{11}{12}\selectfont]{examples/simple-web-server/fibserver-threaded-memcached-while.scala}

\Task %%%TODO number  10 %%%

-

\Task %%%TODO number  11 %%%

-

\Task %%%TODO number  12 %%%

-

\Task %%%TODO number  13 %%%

\Subtask

\Subtask

\Subtask

\Task %%%TODO number  14 %%%

\Subtask -

\Subtask -

\Subtask -

\Subtask -
