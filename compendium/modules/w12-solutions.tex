%!TEX encoding = UTF-8 Unicode

%!TEX root = ../compendium.tex

\ExerciseSolution{\ExeWeekTWELVE}
                   

\Task     %%%TODO number  1 %%%starts with: \emph{Trådar.}  Klassen \code{j%%%

\Subtask  -

\Subtask \code {java.lang.IllegalThreadStateException}. Det går inte att starta en tråd mer än en gång. Tråden kan därför inte startas om när den redan har exekverats.

\Subtask  När \code {start} anropas exekveras koden i \code{run} parallellt. Därför skrivs \code{Gurka} och \code{Tomat} ut omlöpande. Om istället \code{run} anropas direkt blir det inte jämnlöpande exekvering och \code{Gurka} skrivs ut 100 gånger, sedan skrivs \code{Tomat} ut 100 gånger.

\Subtask  \code{Thread.sleep} pausar inte tråden i exakt den tiden som angets. Alltså kommer det skrivas ut \code{zzz snark hej!} i de flesta fall, men det är inte garanterat.


\Task     %%%TODO number  2 %%%starts with: \emph{Jämlöpande variabeluppdat%%%

\Subtask \code{ofta} tar ut - pausar - sätter in. Denna metod körs parallellt av \code{bamse} och \code{skutt} konstant och därför förändras saldot hela tiden. Utskriften visar på denna saldoförändring.

\Task     %%%TODO number  3 %%%starts with: \emph{Jämlöpande exekvering med%%%

\Task     %%%TODO number  4 %%%starts with: TODO  %%%%%%%%%%%%%%%%%%%\Advan%%%

\Subtask

\Subtask

\Subtask

\Subtask

\Task     %%%TODO number  5 %%%starts with: Sök upp och studera dokumentati%%%

\Subtask -

\Subtask -

\Subtask \code{math.random} ger en \code{double} i intervallet [0.0, 1.0), Tidsintevallet för \code{delay} blir därför i medeltal 2.5 sekunder, max 3 sekunder, min 2 sekunder. 
Räkna med tiden det tar att fylla vektorn också!

\Subtask

\Subtask

\Subtask Laddar in datan parallelt så nedladdingen sker samtidigt, men det går olika snabbt pga metoden seg.

\Subtask

\Subtask Metoden \code{lycka} är väldefinerad och kastar därför inga undantag. Den skriver alltid ut \code{:)}. Metoden \code{olycka} är inte väldefinerad då division med 0 ger \code{java.lang.ArithmeticException}. Detta fångas upp vid callbacken och det skrivs ut \code{:(} samt det specifierade undantaget.

\Task  %%%TODO number  6 %%%

\Task %%%TODO number  7 %%%

\Subtask

\Subtask

\Subtask

\Subtask

\Subtask

\Subtask

\Task %%%TODO number  8 %%%

\Task %%%TODO number  9 %%%

\Subtask \code{abbasillen} skrivs ut baklänges till \code{nellisabba}.

\Subtask

\Subtask

\Subtask

\Subtask

\Subtask

\Subtask

\Subtask

\Subtask

\Task %%%TODO number  10 %%%

\Task %%%TODO number  11 %%%

\Task %%%TODO number  12 %%%

\Task %%%TODO number  13 %%%

\Subtask

\Subtask

\Subtask

\Task %%%TODO number  14 %%%

\Subtask

\Subtask

\Subtask

\Subtask
