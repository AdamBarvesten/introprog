%!TEX encoding = UTF-8 Unicode

%!TEX root = ../compendium.tex

\ExerciseSolution{\ExeWeekELEVEN}


\Task     %%%TODO number  1 %%%starts with: \emph{Översätta algoritmer och %%%

\Subtask \scalainputlisting[numbers=left,basicstyle=\ttfamily\fontsize{10.3}{12}\selectfont]{examples/scalajava/hangman1.scala}

\Subtask \scalainputlisting[numbers=left,basicstyle=\ttfamily\fontsize{11.2}{13}\selectfont]{examples/scalajava/hangman2.scala}


\Task     %%%TODO number  2 %%%starts with: \emph{Översätta mellan klasser %%%

\Subtask  \javainputlisting[numbers=left]{examples/scalajava/JPoint.java}

\Subtask  -

\Subtask  \begin{Code}
case class Person(name: String, age: Int = 0)
\end{Code}

\Subtask p.*TAB* - copy, producArity, ProductIterator, productElement, productPrefix

Person.*TAB* - apply, curried, tupled, unapply

\begin{REPLnonum}
scala> p.copy
   def copy(name: String,age: Int): Person

scala> p.copy()
res0: Person = Person(Björn,49)

scala> p.copy(age = p.age + 1)
res1: Person = Person(Björn,50)

scala> Person.unapply(p)
res2: Option[(String, Int)] = Some((Björn,49))
\end{REPLnonum}


\Task     %%%TODO number  3 %%%starts with: \emph{Auto(un)boxing.} I JVM må%%%

\Subtask  -

\Subtask  Cell har typen java.lang.Integer. När man hämtar ut värdet med \code{c.value} hämtas den primitiva typ \code{int} ut.

\Subtask  Med hjälp av autoboxing förvandlas 42 till typen \code{Integer} och kan därför jämföras med en annan \code{Ingeger}.

\Subtask  i.compareTo(42) fungerar på grund av autoboxing. Då JVM packar in den primitiva typ int i en Integer-objekt automatiskt.

\Subtask
\begin{REPLnonum}
0 10 20 30 40 50 60 ... 390 400 410

[0]: 0
[42]: 0
NOT EQUAL
\end{REPLnonum}

\Subtask  \javainputlisting[numbers=left]{examples/scalajava/Autoboxing2.java}

\Subtask  42 kommer läggas längst fram i listan istället för längst bak, då autounboxing kommer göra Integer(0) till 0 och tvärtom med variablen \code{pos}.

\Subtask  Om man ska undersöka om två int-variabler är lika ska man använda ==, men om variablerna är av typen Integer måste man använda \code{equals}.

JVM kommer inte varna om man vänder på \code{Integer} och \code{int}, som i \code{xs.add(0, pos)}.


\Task     %%%TODO number  4 %%%starts with: \emph{JavaConverters.} Med \cod%%%

\Subtask  Vector[Int] - java.util.List[Int]
Set[Char] - java.util.Set[Char]
Map[String, Int] - java.util.Map[String, Int]

\Subtask  ArrayList[Int] - scala.collection.mutable.Buffer[Int]
HashSet[Char] - scala.collection.mutable.Set[Char]

Båda blir föränderliga motsvarigheter. Det visas genom att de till hör \code{scaka.collection.mutable} och både \code{ArrayList} och \code{HashSet} är förändrliga i Java.

\Subtask  \code{scala.collection.immutable.Set}

\Subtask  \code{sm.asJava.asScala} ger typen \code{scala.collection.mutable.Map[String,Int]}

\code{sm.asJava.asScala.toMap} ger typen \code{scala.collection.immutable.Map[String,Int]}

\Subtask  -


\Task

\begin{Code}
object showInt {
  def show(obj: Any, msg: String = ""): Unit = println(msg + obj)

  def repeatChar(ch: Char, n: Int): String = ch.toString * n

  def showInt(i: Int): Unit = {
    import java.lang.Integer
    val leading = Integer.numberOfLeadingZeros(i)
    val binaryString = repeatChar('0', leading) + i.toBinaryString
    show(i,               "Heltal : ")
    show(i.asInstanceOf[Char],         "Tecken : ")
    show(binaryString,    "Binärt : ")
    show(i.toHexString,   "Hex    : ")
    show(i.toOctalString, "Oktal  : ")
  }


  import scala.io.StdIn.readLine
  import scala.util.{Try,Success,Failure}

  def loop: Unit =
    Try { readLine("Heltal annars pang: ").toInt } match {
      case Failure(e) => show(e); show("PANG!")
      case Success(i) => showInt(i); loop
    }

  def main(args: Array[String]): Unit =
    if(args.length > 0) args.foreach(i => showInt(i.toInt))
    else loop
}
\end{Code}


\Task     %%%TODO number  6 %%%starts with: \TODO Fallgrop med Point som in%%%


\Task     %%%TODO number  7 %%%starts with: \TODO \emph{Gränssnitt i Scala %%%


\Task     %%%TODO number  8 %%%starts with: \Pen Studera fallgropar för hur%%%


\Task     %%%TODO number  9 %%%starts with: \Pen Studera det fullständiga r%%%
