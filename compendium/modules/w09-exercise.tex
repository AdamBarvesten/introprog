%!TEX encoding = UTF-8 Unicode

%!TEX root = ../compendium.tex

\Exercise{\ExeWeekNINE}

\begin{Goals}
\item Kunna skapa och använda matriser med nästlade strukturer av \code{Vector}.
\item Kunna iterera över elementen i en matris med nästlade \code{for}-satser och \code{for}-\code{yield}-uttryck, samt nästlad applicering av \code{map} respektive \code{foreach}.
\item Kunna skapa och använda funktioner som tar matriser som parametrar.
\item Känna till generiska funktioner.
\item Känna till generiska klasser.
\item Kunna använda enkla matriser i Javas inbyggda arrayer.
\item Kunna anända nästlade \code{for}-satser i Java för att iterera över elementen i en matris.
\end{Goals}

\begin{Preparations}
\item Läs teorin i kapitel \ref{chapter:W09}.
\end{Preparations}

\BasicTasks %%%%%%%%%%%%%%%%

\Task \emph{Skapa matriser med hjälp av nästlade samlingar.} Man kan i ett datorprogram, med hjälp av samlingar som innehåller samlingar, skapa nästlade strukturer som kan indexeras i två dimensioner och på så sätt representera en matematisk \textbf{matris}.\footnote{\href{https://sv.wikipedia.org/wiki/Matris}{sv.wikipedia.org/wiki/Matris}} 
\begin{Background}
En \textbf{matris} inom matematiken innehåller ett antal rader och kolumner (även kallade kolonner). I en matematisk matris har alla rader lika många element och även alla kolumner har lika många element. En matris av dimension $m\times{}n$ har $m \cdot n$ stycken element, där $m$ är antalet rader och $n$ är antalet kolumner. En matris $A_{m,n}$ av dimension $m\times{}n$ ritas ofta så här:

\[
A_{m,n} = 
 \begin{pmatrix}
  a_{1,1} & a_{1,2} & \cdots & a_{1,n} \\
  a_{2,1} & a_{2,2} & \cdots & a_{2,n} \\
  \vdots  & \vdots  & \ddots & \vdots  \\
  a_{m,1} & a_{m,2} & \cdots & a_{m,n} 
 \end{pmatrix}
\]

\noindent Exempel: En heltalsmatris $M_{2,5}$ av dimension $2\times{}5$ där element $m_{2,5}=7$:

\[
M=
  \begin{pmatrix}
    5 & 2 & 42 & 4 & 5 \\
    3 & 4 & 18 & 6 & 7
  \end{pmatrix}
\]
\end{Background}

\Subtask\Pen Rita minnessituationen efter tilldelningen på rad 1 nedan. Vad har \code{m} för typ och värde? Vad har \code{m} för dimensioner? Hur sker indexeringen i ett datorprogram jämfört med i matematiken?

\begin{REPL}
scala> val m = Vector((1 to 5).toVector, (3 to 7).toVector)
scala> m.apply(0).apply(1)
scala> m(1)
scala> m(1)(4)
\end{REPL}

\Subtask Vad ger uttrycken på raderna 2, 3 och 4 ovan för värden och typ? 

\Subtask Man kan i ett datorprogram mycket väl skapa tvådimensionella, nästlade strukturer där raderna \emph{inte} innehåller samma antal element. Det blir då ingen äkta matris i strikt matematisk mening, men man kallar ofta ändå en sådan struktur för en ''matris''. Vilken typ har variablerna \code{m2}, \code{m3}, \code{m4} och \code{m5} nedan? 

\begin{REPL}
scala> val m2 = Vector(Vector(1,2,3),Vector(4,5),Vector(42))
scala> val m3 = Vector(Vector(1,2), Vector(1.0, 2.0, 3.0))
scala> m3(1)
scala> val m4 = m3(1) +: Vector("a") +: m3
scala> val m5 = Vector.fill(42){ m2(1).map(e => (e * math.random).toInt) }
\end{REPL}

\Subtask\Pen Rita minnessituationen efter tilldelingen av \code{m2} på rad 1 ovan.

\Subtask\Pen Vilken av variablerna \code{m2}, \code{m3}, \code{m4} och \code{m5} ovan representerar en äkta matris i matematisk mening? Vilken är dess dimensioner?



\Task \emph{Skapa och itererar över matriser.} Vi ska skapa matriser där varje rad representerar 5 kast med en tärning spelet Yatzy.\footnote{\href{https://sv.wikipedia.org/wiki/Yatzy}{sv.wikipedia.org/wiki/Yatzy}}


\Subtask Definiera i REPL en funktion \code{def throwDice: Int = ???} som returnerar ett slumptal mellan 1 och 6.
\begin{Code}
// kod till facit
def throwDice: Int = (math.random * 6).toInt + 1
\end{Code}


\Subtask Skapa nedan heltalsmatris i REPL. Vilken dimension får matrisen?
\begin{REPL}
val ds1 = for (i <- 1 to 1000) yield { 
            for (j <- 1 to 5) yield throwDice 
          }
\end{REPL}

\Subtask\Pen Man kan också använda nedan varianter för att skapa en heltalsmatris. Vilken av varianterna \code{ds1} ... \code{ds5} tycker du är lättast att läsa och förstå? Prova respektive variant i REPL och ange vilken typ på \code{ds1} ... \code{ds5} som härleds av kompilatorn.
\begin{REPL}
val ds2 = (1 to 1000).map(i => (1 to 5).map(j => throwDice))  
val ds3 = (1 to 1000).map(i => Vector.fill(5)(throwDice)) 
val ds4 = for (i <- 1 to 1000) yield Vector.fill(5)(throwDice) 
val ds5 = Vector.fill(1000)(Vector.fill(5)(throwDice))
\end{REPL}


\Subtask Definiera en funktion \\ \code{def roll(n: Int): Vector[Int] = ???}\\ som ger en heltalsvektor med $n$ stycken slumpvisa tärningskast. Kasten ska vara sorterade i växande ordning; använd för detta ändamål samlingsmetoden \code{sorted}.
\begin{Code}
// kod till facit
def roll(n: Int) = Vector.fill(n)(throwDice).sorted
\end{Code}


\Subtask Definera i REPL en funktion \code{isYatzy(xs: Vector[Int]): Boolean = ???} som testar om alla elementen i en heltalsvektor är samma. Använd samlingsmetoden \code{forall}. 
\begin{Code}
// kod till facit
def isYatzy(xs: Vector[Int]): Boolean = xs.forall(_ == xs(0))
\end{Code}

\Subtask Implementera \code{isYatzy} igen med ett imperativt angreppssätt som använder en \code{while}-sats (alltså utan att använda funktionella  \code{forall}). Ta hjälp av en variabel \code{i} som håller reda på index och en variabel \code{foundDiff} som håller reda på om ett avvikande värde upptäcks. Funktionen blir ca 10 rader, så det kan vara lämpligt att öppna en editor att skriva i medan du klurar ut lösningen. Börja med att skriva pseudokod, gärna med penna på papper. Prova genom att klistra in i REPL. 
\begin{Code}
// kod till facit
def isYatzy(xs: Vector[Int]): Boolean = {
  var foundDiff = false
  var i = 0
  while (i < xs.size  && !foundDiff){
    foundDiff = xs(i) != xs(0)
    i += 1
  } 
  !foundDiff
}
\end{Code}


\Subtask Skapa en funktion  \\ \code{def diceMatrix(m: Int, n: Int): Vector[Vector[Int]] = ???} \\ som med hjälp av funktionen \code{roll} skapar en matris med \code{m} st vektorer med vardera \code{n} slumpvisa tärningskast.
\begin{Code}
// kod till facit
def diceMatrix(m: Int, n: Int): Vector[Vector[Int]] = 
  Vector.fill(m)(roll(n))
\end{Code}

\Subtask Skapa en funktion som returnerar en utskriftsvänlig sträng \\ \code{def diceMatrixToString(xss: Vector[Vector[Int]]): String = ???} \\med hjälp av \code{map} och \code{mkString}, som fungerar enligt nedan. 
\begin{REPL}
scala> println(diceMatrixToString(diceMatrix(10, 5)))
4 5 5 3 3
1 4 1 3 1
1 3 1 5 5
6 4 4 5 5
2 1 5 6 5
1 2 2 3 6
1 3 2 4 5
2 2 3 2 2
2 6 3 4 6
4 5 5 2 3

\end{REPL}
\begin{Code}
// kod till facit
def diceMatrixToString(xss: Vector[Vector[Int]]): String = 
  xss.map(_.mkString(" ")).mkString("\n")
\end{Code}


\Subtask\Pen Ett imperativt sätt\footnote{Imperativa anreppssätt är nödvändiga att kunna när du stöter på samlingar och/eller språk som saknar funktionsprogrammeringsmöjligheter med \code{map}, \code{mkString} etc.} att göra detta på visas nedan. Förklara hur nedan kod fungerar. Vad händer om \code{xss} är tom? Vad händer om \code{xss} bara innehåller tomma vektorer? Nämn en fördel och en nackdel med att använda \code{val sb: StringBuilder} och \code{append}, jämfört med en vanlig \code{var s: String} och \code{+} för tillägg i slutet.
\begin{Code}
def diceMatrixToString(xss: Vector[Vector[Int]]): String = {
  val sb = new StringBuilder()
  for(m <- 0 until xss.size) {
    for(n <- 0 until xss(m).size) { 
      sb.append(xss(m)(n))
      if (n < xss(m).size - 1) sb.append(" ") 
      else if (m < xss.size - 1) sb.append("\n")
    }
  }
  sb.toString
}
\end{Code}

\Subtask Implementera funktionen \\ \code{def filterYatzy(xss: Vector[Vector[Int]]): Vector[Vector[Int]]} \\ som filtrerar fram alla yatzy-rader i matrisen \code{xss} enligt nedan. Använd din funktion \code{isYatzy} och samlingsmetoden \code{filter}. 
\begin{REPL}
scala> println(diceMatrixToString(filterYatzy(diceMatrix(10000, 5))))
2 2 2 2 2
3 3 3 3 3
1 1 1 1 1
3 3 3 3 3
4 4 4 4 4
6 6 6 6 6
2 2 2 2 2
3 3 3 3 3
2 2 2 2 2
6 6 6 6 6
4 4 4 4 4
2 2 2 2 2
4 4 4 4 4

\end{REPL}

\begin{Code}
// kod till facit
def filterYatzy(xss: Vector[Vector[Int]]): Vector[Vector[Int]] = 
  xss.filter(isYatzy)
\end{Code}

\Subtask Gör som träning en imperativ implementation av \code{filterYatzy} med en \code{for}-sats (alltså utan att använda \code{filter}, och utan att använda \code{yield}). 
\begin{CodeSmall}
// kod till facit
def filterYatzy(xss: Vector[Vector[Int]]): Vector[Vector[Int]] = {
  var result: Vector[Vector[Int]] = Vector()
  for (i <- 0 until xss.size) {
    if (isYatzy(xss(i))) result = result :+ xss(i) 
  } 
  result
}
\end{CodeSmall}

\Subtask\Pen Tycker du din imperativa lösning är lättare eller svårare att läsa och förstå jämfört nedan funktionella lösning med ett \code{for}-uttryck och \code{yield}?
\begin{CodeSmall}
def filterYatzy(xss: Vector[Vector[Int]]): Vector[Vector[Int]] = {
  for (i <- 0 until xss.size if isYatzy(xss(i))) yield xss(i)
}.toVector  
\end{CodeSmall}

\Subtask Implementera funktionen \\ 
\code{def yatzyPips(xss: Vector[Vector[Int]]): Vector[Int]} \\ som ger en vektor med tärningsvärdena för de kast i matrisen \code{xss} som gav yatzy enligt nedan. Använd din funktion \code{isYatzy} och samlingsmetoden \code{filter}. 
\begin{REPL}
scala> yatzyPips(diceMatrix(10000, 5))
res42: Vector[Int] = Vector(3, 5, 6, 6, 3, 3, 2, 6, 1, 3)
\end{REPL}

\begin{Code}
// kod till facit
def yatzyPips(xss: Vector[Vector[Int]]): Vector[Int] = 
  xss.filter(isYatzy).map(_.head)
\end{Code}



\Task \emph{Strängtabell med rubrikrad.} Denna övning utgör en början på det du ska göra under veckans laboration.

\Subtask Implementera case-klassen \code{Table} enligt nedan specifikation. Du kan förutsätta att alla rader har lika många kolumner som antalet element i \code{headings}, samt att alla rubrikerna i \code{headings} är unika. Detta förutsätts också gälla för indatafiler som läses in med \code{fromFile}. 
\\ \noindent \emph{Tips:} 
\begin{itemize}[nolistsep,noitemsep]
\item Värdet \code{indexOfHeading} kan skapas med hjälp av metoden \code{zipWithIndex} som fungerar på alla sekvenssamlingar, samt metoden \code{toMap} som fungerar på sekvenser av 2-tupler. Undersök först hur metoderna fungerar i REPL och sök upp deras dokumentation.
\item Skapa en indatafil som du kan använda för att testa att \code{Table} fungerar. 
\end{itemize}

\begin{ScalaSpec}{Table}
case class Table(
  data: Vector[Vector[String]], 
  headings: Vector[String], 
  sep: String){
  /** A 2-tuple with (number of rows, number of columns) in data */
  val dim: (Int, Int) = ???

  /** The element in row r an column c of data, counting from 0 */
  def apply(r: Int, c: Int): String = ???

  /** The row-vector r in data, counting from 0 */
  def row(r: Int): Vector[String]= ???

  /** The column-vector c in data, counting from 0 */
  def col(c: Int): Vector[String] = ???

  /** A map from heading to index counting from 0 */
  lazy val indexOfHeading: Map[String, Int] = ???

  /** The column-vector with heading h in data */
  def col(h: String): Vector[String] = ???

  /** A vector with the distinct, sorted values of col with heading h */ 
  def values(h: String): Vector[String] = ???

  /** Headings and data with columns separated by sep */
  override lazy val toString: String = ???
}
object Table {
  /** Creates a new Table from fileName with columns split by sep */
  def fromFile(fileName: String, separator: Char = ';'): Table = ???
}
\end{ScalaSpec}


\begin{CodeSmall}
// kod till facit

case class Table(
  data: Vector[Vector[String]], 
  headings: Vector[String], 
  sep: String){

  val dim: (Int, Int) = (data.size, headings.size)

  def apply(r: Int, c: Int): String = data(r)(c)

  def row(r: Int): Vector[String]= data(r)

  def col(c: Int): Vector[String] = data.map(r => r(c))

  lazy val indexOfHeading: Map[String, Int] = headings.zipWithIndex.toMap

  def col(h: String): Vector[String] = col(indexOfHeading(h)) 

  def values(h: String): Vector[String] = col(h).distinct.sorted

  override lazy val toString: String = 
    headings.mkString(sep) + "\n" +data.map(_.mkString(sep)).mkString("\n")
}
object Table {
  def fromFile(fileName: String, separator: Char = ';'): Table = {
    val lines = scala.io.Source.fromFile(fileName).getLines.toVector
    val matrix= lines.map(_.split(separator).toVector) 
    new Table(matrix.tail, matrix.head, separator.toString)
  }
}
\end{CodeSmall}

\Subtask Skapa med hjälp av \code{Table} ett program som kan köras från terminalen med \texttt{scala regtable infile.csv ';'} som ger en utskrift av antalet förekomster av respektive unika värde i respektive kolumn (alltså en variant av registrering).

%%%%%%%%%%%%%%%%%%%%%%%%%%%%%%%%%%%%%%%%%%%%%%%%%%%%%%%%%%%%%

\Task \emph{Generiska funkioner.} En generisk funktion har (minst) en typparameter inom klammerparenteser efter namnet, till exempel \code{[T]}. Denna typ förekommer sedan som typ på (någon av) parametrarna i parameterlistan. Kompilatorn härleder en konkret typ vid kompileringstid och ersätter typparametern med denna konkreta typ. På så sätt kan en funktion fungera för många olika typer. 

\Subtask Förklara för varje rad nedan vad som händer.

\begin{REPL}
scala> def tnirp[T](x: T): Unit = println(x.toString.reverse)
scala> tnirp(42)
scala> tnirp("hej")
scala> case class Gurka(vikt: Int)
scala> tnirp(Gurka(42))
scala> tnirp[String](42)
scala> tnirp[Double](42)
\end{REPL}

\Subtask Man kan kombinera generiska funktioner med funktioner som tar funktioner som parametrar. Det är så \code{map} och \code{foreach} är implementerade. Förklara för varje rad nedan vad som händer.

\begin{REPL}
scala> def compose[A, B, C](f: A => B, g: B => C)(x: A): C = g(f(x))
scala> def inc(x: Int): Int = x + 1
scala> def half(x: Int): Double = x / 2.0
scala> compose(inc, half)(42)
scala> compose(half, inc)(42)
\end{REPL}

\Subtask Hur lyder felmeddelandet på sista raden ovan? Ändra \code{inc} och/eller \code{half} så att typerna passar.


\Task \emph{Generiska klasser.} Även klasser kan vara generiska. En generisk klass har (minst) en typparameter inom klammerparenteser efter klassens namn. 

\Subtask Testa nedan generiska klass \code{Cell[T]} i REPL. Skapa instanser av klassen \code{Cell[T]} där typparametern \code{T} binds till olika konkreta typer och förklara vad som händer.

\begin{REPL}
scala> class Cell[T](var value: T){
         override def toString = "Cell(" + value + ")"
       }
scala> new Cell(42)
scala> new Cell("hej")
scala> new Cell(new Cell(math.Pi))
scala> new Cell[String](42)
scala> new Cell[Double](42)
\end{REPL}

\Subtask Lägg till metoden \code{def concat[U](that: Cell[U]):Cell[String]} i klassen \code{Cell} som konkatenerar strängrepresentationerna av de båda cellvärdena.

\begin{REPL}
scala> val a = new Cell("hej")
scala> val b = new Cell(42)
scala> a concat b
\end{REPL}

\begin{Code}
// kod till facit
class Cell[T](var value: T){
  override def toString = "Cell(" + value + ")"
  def concat[U](that: Cell[U]): Cell[String] = 
    new Cell(value.toString + that.value.toString)
}
\end{Code}

\Subtask\Pen Vad händer om du i stället för typparameternamnet \code{U} i \code{concat} använder namnet \code{T}?

\Task \emph{Skapa en generisk, ofäränderlig matrisklass.} Med hjälp av en typparameter kan vi skapa en matrisklass som kan innehålla vilka element som helst. Implementera nedan specifikation. Testa din matrisklass i REPL för olika typer av element.

\begin{ScalaSpec}{Matrix[T]}
case class Matrix[T](data: Vector[Vector[T]]) {

  /** The element at row r and column c */
  def apply(r: Int, c: Int): T = ???

  /** Gives Some[T](element) at row r and column c 
   *  if r and c are within index bounds, else None */
  def get(r: Int, c: Int): Option[T] = ???

  /** The row vector of row r */ 
  def row(r: Int): Vector[T] = ???

  /** The column vector of column c */ 
  def col(c: Int): Vector[T] = ???

  /** A new Matrix with element at row r and col c updated */
  def updated(r: Int, c: Int, value: T): Matrix[T] = ???
}
object Matrix {
  def empty[T]: Matrix[T] = new Matrix[T](Vector())
  def ofRowSize[T](rowSize: Int)(elements: T*): Matrix[T] =
    new Matrix(elements.toVector.grouped(rowSize).toVector)
}
\end{ScalaSpec}


\Task \label{task:arraymatrix-java} \emph{Matriser med array i Java.} Om man redan vid allokering vet hur många element en matris ska ha, använder man i Java gärna en array av arrayer. En heltalsmatris (en array av array av heltal) skrivs i Java med dubbla hakparentespar \jcode{int[][]} direkt efter typen. Vid allokering använder man nyckelordet \code{new} och antalet element i respektive dimension anges inom hakperenteserna; t.ex. så ger \jcode{new int[42][21]} en matris med 42 rader och 21 kolumner, vilket motsvarar att man i Scala skriver\footnote{Ett annat, längre men kanske tydligare, sätt att skriva detta i Scala där initialvärdet framgår explicit: \code{Array.fill(42)(Array.fill(21)(0))}}  \code{Array.ofDim[Int](42,21)}. Alla element får defaultvärdet för typen, som är \code{0} för typen \code{Int} i Scala, motsvarande \jcode{int} i Java.

\Subtask Skriv nedan program i en editor och spara koden i filen \texttt{ArrayMatrix.java} och kompilera med \texttt{javac ArrayMatrix.java} och kör i terminalen med \texttt{java ArrayMatrix} och undersök utskriften. Förklara vad som händer. Notera några skillnader i hur matriser används i Scala och Java.


\begin{Code}[language=Java]
// ArrayMatrix.java

public class ArrayMatrix {

    public static void showMatrix(int[][] m){
        System.out.println("\n--- showMatrix ---");
        for (int row = 0; row < m.length; row++){
            for (int col = 0; col < m[row].length; col++) {
                System.out.print("[" + row + "]");
                System.out.print("[" + col + "] = ");
                System.out.print(m[row][col] + "; ");
            }
            System.out.println();
        }
    }
    
    public static void main(String[] args) {
        System.out.println("ArrayMatrix test");
        int[][] xss = new int[10][5];
        showMatrix(xss);
    }
}
\end{Code}

\Subtask Implementera nedan metod \code{fillRnd} med kod som fyller matrisen \code{m} med slumptal mellan \code{1} och \code{n}.
\begin{Code}[language=Java]
    public static void fillRnd(int[][] m, int n){
        /* ??? * /
    }
\end{Code}
\noindent \emph{Tips:} med detta uttryck skapas ett slumptal mellan 1 och 42 i Java:\\
\jcode{(int) (Math.random() * 42 + 1)} \\
där typkonverteringen \jcode{(int)} ger samma effekt som ett anrop av metoden \code{toInt} i Scala; alltså att dubbelprecisionsflyttal omvandlas till heltal genom avkortning av alla eventuella decimaler.


Ändra huvudprogrammet till:
\begin{Code}[language=Java]
    public static void main(String[] args) {
        System.out.println("ArrayMatrix test");
        int[][] xss = new int[10][5];
        showMatrix(xss);
        fillRnd(xss, 6);
        showMatrix(xss);
    }
\end{Code}

Programmet ska ge en utskrift som liknar följande:
\begin{REPL}
$ javac ArrayMatrix.java 
$ java ArrayMatrix 
ArrayMatrix test

--- showMatrix ---
[0][0] = 0; [0][1] = 0; [0][2] = 0; [0][3] = 0; [0][4] = 0; 
[1][0] = 0; [1][1] = 0; [1][2] = 0; [1][3] = 0; [1][4] = 0; 
[2][0] = 0; [2][1] = 0; [2][2] = 0; [2][3] = 0; [2][4] = 0; 
[3][0] = 0; [3][1] = 0; [3][2] = 0; [3][3] = 0; [3][4] = 0; 
[4][0] = 0; [4][1] = 0; [4][2] = 0; [4][3] = 0; [4][4] = 0; 
[5][0] = 0; [5][1] = 0; [5][2] = 0; [5][3] = 0; [5][4] = 0; 
[6][0] = 0; [6][1] = 0; [6][2] = 0; [6][3] = 0; [6][4] = 0; 
[7][0] = 0; [7][1] = 0; [7][2] = 0; [7][3] = 0; [7][4] = 0; 
[8][0] = 0; [8][1] = 0; [8][2] = 0; [8][3] = 0; [8][4] = 0; 
[9][0] = 0; [9][1] = 0; [9][2] = 0; [9][3] = 0; [9][4] = 0; 

--- showMatrix ---
[0][0] = 6; [0][1] = 2; [0][2] = 6; [0][3] = 3; [0][4] = 5; 
[1][0] = 2; [1][1] = 4; [1][2] = 6; [1][3] = 1; [1][4] = 1; 
[2][0] = 5; [2][1] = 4; [2][2] = 4; [2][3] = 1; [2][4] = 5; 
[3][0] = 4; [3][1] = 6; [3][2] = 6; [3][3] = 1; [3][4] = 3; 
[4][0] = 4; [4][1] = 6; [4][2] = 2; [4][3] = 3; [4][4] = 2; 
[5][0] = 2; [5][1] = 4; [5][2] = 5; [5][3] = 5; [5][4] = 3; 
[6][0] = 6; [6][1] = 5; [6][2] = 2; [6][3] = 4; [6][4] = 3; 
[7][0] = 1; [7][1] = 6; [7][2] = 1; [7][3] = 6; [7][4] = 2; 
[8][0] = 1; [8][1] = 1; [8][2] = 5; [8][3] = 3; [8][4] = 2; 
[9][0] = 1; [9][1] = 1; [9][2] = 1; [9][3] = 5; [9][4] = 4; 

\end{REPL}

 

\begin{Code}[language=Java]
// kod till facit:
public class ArrayMatrix {

    public static void showMatrix(int[][] m){
        System.out.println("\n--- showMatrix ---");
        for (int row = 0; row < m.length; row++){
            for (int col = 0; col < m[row].length; col++) {
                System.out.print("[" + row + "]");
                System.out.print("[" + col + "] = ");
                System.out.print(m[row][col] + "; ");
            }
            System.out.println();
        }
    }
    
    public static void fillRnd(int[][] m, int n){
        for (int row = 0; row < m.length; row++){
            for (int col = 0; col < m[row].length; col++) {
              m[row][col] = (int) (Math.random() * n + 1);
            }
        }
    }

    public static void main(String[] args) {
        System.out.println("ArrayMatrix test");
        int[][] xss = new int[10][5];
        showMatrix(xss);
        fillRnd(xss, 6);
        showMatrix(xss);
    }
}
\end{Code}




\ExtraTasks %%%%%%%%%%%%%%%%%%%

\Task \emph{Skapa ett yatzy-spel för användning i terminalen.}

\Subtask Skapa med en editor en klass enligt nedan specifikation. Läs om hur de olika predikaten för att kolla olika giltiga kombinationer i Yatzy ska fungera här: \href{https://en.wikipedia.org/wiki/Yahtzee}{en.wikipedia.org/wiki/Yahtzee}. Bygg ett huvudprogram som testar dina funktioner. Kompilera och testa i terminalen allteftersom du lägger till nya funktioner.

\begin{ScalaSpec}{YatzyRows}
/** En skiss på en klass som kan användas till ett förenklat yatzy-spel */
case class YatzyRows(val rows: Vector[Vector[Int]]) {
  /** A new YatzyRows with a new row of 5 dice rolls appended to rows  */
  def roll: YatzyRows = ???

  /** A new YatzyGame with some indices of the last row re-rolled  */
  def reroll(indices: Vector[Int]): YatzyRows = ???
}

object YatzyRows {
  def isYatzy(xs: Vector[Int]): Boolean = ???
  def isThreeOfAKind(xs: Vector[Int]): Boolean = ???
  def isFourOfAKind(xs: Vector[Int]): Boolean = ???
  def isFullHouse(xs: Vector[Int]): Boolean = ???
  def isSmallStraight(xs: Vector[Int]): Boolean = ???
  def isLargeStraight(xs: Vector[Int]): Boolean = ???
}
\end{ScalaSpec}


\Subtask Använd \code{YatzyRows} för att med hjälp av många tärningskast beräkna sannolikheter för några olika giltiga kombinationer. Använd, om du vill, möjligheten som reglerna ger att slå om tärningar i två yterliggare kast, där de tärningar som slås om väljs slumpmässigt.

\Subtask Bygg ett förenklat yazty-spel i terminalen där användaren kan bestämma vilka tärningar som ska slås om. Använd \code{Scanner} för att läsa indata från användaren. Börja med något riktigt enkelt och bygg sedan vidare på ditt spel genom att införa fler och fler funktioner.



\AdvancedTasks %%%%%%%%%%%%%%%%%


\Task \emph{Klasser för täta och glesa matematiska matriser med flyttal.} \\TODO!!!

\Subtask Skapa en oföränderlig, final klass \code{DenseMatrix} för matematiska matriser med dubbelprecisionsflyttal. \code{DenseMatrix} ska internt lagra elementen i en privat \emph{endimensionell} array av flyttal av typen \code{Array[Double]}. Klassen ska inte vara en case-klass. Det ska gå att skapa matriser med uttrycket DenseMatrix.ofDim(3,7)(1.0,42,3.2,1.0,2.2,3) tack vare ett kompanjonsobjekt med lämplig fabriksmetod som anropar den privata konstruktorn. 

\Subtask Överskugga metoderna equals och hashcode och ge \code{DenseMatrix} innehållslikhet i stället för referenslikhet.

\Subtask Implementera egna innehålllikhetsmetoder med namnet \code{===} på \code{DenseMatrix} som är typsäker, d.v.s. bara tillåter jämförelse mellan matriser.

\Subtask \emph{SparseMatrix med private mutable.Map[(Int, Int), Double]} som bara lagrar index som inte är noll.

\Subtask Implementera addition, subtraktion och multiplikation av matriser.

\Subtask Skapa ett trait Matrix som både DenseMatrix och SparseMatrix ärver, med lämpliga abstrakta och konkreta medlemmar.

\Task \emph{Matriser med \jcode{ArrayList} i Java.} Om man i Java inte vet antalet element i matrisen från början kan man använda en lista av typen \jcode{ArrayList}, där varje element i sin tur innehåller en lista av typen\jcode{ArrayList}. Javas \jcode{ArrayList} är en generisk samling som motsvaras av Scalas \code{ArrayBuffer}. Generiska samlingar i Java kan endast innehålla referenstyper; vill man ha en primitiv typ, t.ex. \jcode{int}, behöver man packa in denna i en s.k. wrapper-klass, t.ex.  klassen \jcode{Integer}. Det finns en wrapper-klass för varje primitiv typ i Java. Matristypen för en heltalstyp i Java skrivs \jcode{ArrayList<ArrayList<Integer>>} där alltså \code{<T>} motsvarar Scalas hakparenteser \code{[T]} för typparametern T.


\Subtask TODO!! Hitta på deluppgifter med \jcode{ArrayList<ArrayList<Integer>>} som illustrerar ovan. Peka framåt till scalajava-veckan.

    