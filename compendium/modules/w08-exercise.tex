%!TEX encoding = UTF-8 Unicode

%!TEX root = ../compendium.tex

\Exercise{\ExeWeekEIGHT}

\begin{Goals}
\item 
\end{Goals}

\begin{Preparations}
\item 
\end{Preparations}

\BasicTasks %%%%%%%%%%%%%%%%

\Task \label{task:switch} \emph{Hur funkar en \jcode{switch}-sats i Java (och flera andra språk)?} Det händer ofta att man vill testa om ett värde är ett av många olika alternativ. Då kan man använda en sekvens av många \code{if}-\code{else}, ett för varje alternativ. Men det finns ett annat sätt i Java och många andra språk: man kan använda \jcode{switch} som kollar flera alternativ i en och samma sats, se t.ex. \href{https://en.wikipedia.org/wiki/Switch_statement}{en.wikipedia.org/wiki/Switch\_statement}.

\Subtask Skriv in nedan kod i en kodeditor. Spara med namnet \texttt{Switch.java} och kompilera filen med kommandot \texttt{javac Switch.java}. Kör den med \texttt{java Switch} och ange din favoritgrönsak som argumnet till programmet. Vad händer? Förklara hur \jcode{switch}-satsen fungerar.

\javainputlisting[numbers=left,basicstyle=\ttfamily\fontsize{11}{12}\selectfont]{examples/Switch.java}

\Subtask \label{subtask:break} Vad händer om du tar bort \jcode{break}-satsen på rad 16?




\Task \label{task:vegomatch} \emph{Matcha värden.} I Scala finns ingen \jcode{switch}-sats. I stället har Scala ett \code{match}-uttryck som är mer kraftfullt. Dock saknar Scala nyckelordet \jcode{break} och Scalas \code{match}-uttryck kan inte ''falla igenom'' som skedde i uppgift \ref{task:switch}\ref{subtask:break}.

\Subtask \label{subtask:vegomatch} Skriv nedan program med en kodeditor och spara i filen \texttt{Match.scala}. Kompilera med \texttt{scalac Match.scala}. Kör med \texttt{scala Match} och ge som argument din favoritgrönsak. Vad händer? Förklara hur ett \code{match}-uttryck fungerar.

\scalainputlisting[numbers=left,basicstyle=\ttfamily\fontsize{11}{12}\selectfont]{examples/Match.scala}

\Subtask Vad blir det för felmeddelande om du tar bort case-grenen för defaultvärden och indata väljs så att inga case-grenar matchar? Är det ett körtidsfel eller ett kompileringsfel?


\Subtask\Pen Beskriv några skillnader i syntax och semantik mellan Javas flervalssats \jcode{switch} och Scalas flervalsuttryck \code{match}.




\Task \emph{Gard i case-grenar.} Med hjälp en gard \Eng{guard} i en case-gren kan man begränsa med ett villkor om grenen ska väljas. 

Utgå från koden i uppgift \ref{task:vegomatch}\ref{subtask:vegomatch} och byt ut case-grenen för \code{'g'}-matchning till nedan variant med en gard med nyckelordet \code{if} (notera att det inte behövs parenteser runt villkoret):
\begin{Code}
    case 'g' if math.random > 0.5 => "gurka är gott ibland..."
\end{Code}
Kompilera om och kör programmet upprepade gånger med olika indata tills alla grenar i \code{match}-uttrycket har exekverats. Förklara vad som händer.

\Task \label{task:match-caseclass} \emph{Matcha på case-klasser.} Scalas \code{match}-uttryck är extra kraftfulla om de används tillsammans med \code{case}-klasser: då kan fält extraheras automatiskt och bindas till lokala namn direkt i case-grenen som nedan exempel visar (notera att \code{v} och \code{rutten} inte behöver deklareras explicit).   

\Subtask Vad skrivs ut nedan? Varför? Prova att byta namn på \code{v} och \code{rutten}.
\begin{REPL}
scala> case class Gurka(vikt: Int, ärRutten: Boolean)
scala> val g = Gurka(100, true)
scala> g match { case Gurka(v,rutten) => println("G" + v + rutten) }
\end{REPL}

\Subtask Skriv sedan nedan i REPL och tryck TAB två gånger efter punkten. Vad har \code{unapply}-metoden för resultattyp?  
\begin{REPL}
scala> Gurka.unapply   // Tryck TAB två gånger
\end{REPL}
Bakgrund för kännedom: Case-klasser får av kompilatorn automatiskt ett s.k. kompanjonsobjekt \Eng{companion object}, i detta fallet \code{object Gurka}. Det objektet får av kompilatorn automatiskt en \code{unapply}-metod. Det är \code{unapply} som anropas ''under huven'' när case-klassernas fält extraheras vid matchning med \code{match}, men detta sker alltså automatiskt och man behöver inte explicit nyttja \code{unapply} om man inte själv vill implementera s.k. extraherare \Eng{extractors}, se fördjupningsuppgift \ref{task:extractor}. 

\Subtask Anropa \code{unapply}-metoden enligt nedan. Vad blir resultatet?
\begin{REPL}
scala> Gurka.unapply(g)   
\end{REPL}
Vi ska i senare uppgifter undersöka hur typerna \code{Option} och \code{Some} fungerar och hur man kan ha nytta av dessa i andra sammanhang.

\Subtask Spara programmet nedan i filen \texttt{vegomatch.scala} och kompilera med \code{scalac vegomatch.scala} och kör med \code{scala vegomatch.Main 1000} i terminalen. Förklara hur predikatet \code{ärÄtvärd} fungerar. 
\scalainputlisting[numbers=left,basicstyle=\ttfamily\fontsize{11}{12}\selectfont]{examples/vegomatch.scala}



\Task Man kan åstadkomma urskiljningen av de ätbara grönsakerna i uppgift \ref{task:match-caseclass} med polymorfism i stället för \code{match}. 

\Subtask Gör en ny variant av ditt program enligt nedan riktlinjer och spara den modifierade koden i filen \texttt{vegopoly.scala} och kompilera och kör.
\begin{itemize}[noitemsep]
\item Ta bort predikatet \code{ärÄtvärd} i objektet \code{Main} och inför i stället en abstrakt metod \code{def ärÄtbar: Boolean} i traiten \code{Grönsak}.
\item Inför konkreta värden i respektive grönsak som definierar ätbarheten.
\item Ändra i huvudprogrammet enligt ovan ändringar så att \code{ärÄtvärd} anropas som en metod på de skördade grönsaksobjekten när de ätvarda ska filtreras ut.
\end{itemize} 

\Subtask Lägg till en ny grönsak \code{case class Brocolli} och definiera dess ätbarhet. Ändra i slump-funktionerna så att brocolli blir ännu ovanligare än gurka.

\Subtask\Pen Jämför lösningen med \code{match} i uppgift \ref{task:match-caseclass} och lösningen ovan med polymorfism. Vilka är för- och nackdelarna med respektive lösning. Diskutera två olika situationer: 1) att uppsättningen grönsaker inte ändras särskilt ofta medan definitionerna av ätbarhet ändras väldigt ofta och 2) att uppsättningen grönsaker ändras väldigt ofta men att ätbarhetsdefinitionerna inte ändras särskilt ofta.



\Task \emph{Matcha på case-objekt och nyttan med \code{sealed}.} Skapa nedan kod i en editor, och klistra in i REPL med kommandot \code{:pa}. Notera nyckelordet \code{sealed}.
\begin{Code}
sealed trait Färg
object Färg { 
  val values = Vector(Spader, Hjärter, Ruter, Klöver) 
}
case object Spader  extends Färg
case object Hjärter extends Färg
case object Ruter   extends Färg
case object Klöver  extends Färg
\end{Code}

\Subtask Skapa en funktion \code{def parafärg(f: Färg): Färg} i en editor, som med hjälp av ett match-uttryck returnerar parallellfärgerna: paralellfärgen till \code{Hjärter} är \code{Ruter} och vice versa, medan parallellfärgen till \code{Klöver} är \code{Spader} och vice versa. Klistra in funktionen i REPL.
\begin{REPL}
scala> val xs = Vector.fill(5)(Färg.values((math.random * 4).toInt))  
scala> xs.map(parafärg)
\end{REPL}

\begin{Code}
// kod till facit
def parafärg(f: Färg): Färg = f match {
  case Spader  => Klöver
  case Hjärter => Ruter
  case Ruter   => Hjärter
  case Klöver  => Spader
}
\end{Code}

\Subtask Vi ska nu undersöka vad som händer om man glömmer en av case-grenarna i matchningen i \code{parafärg}? ''Glöm'' alltså avsiktligt en av case-grenarna och klistra in den nya \code{parafärg} med den ofullständiga matchningen. Hur lyder varningen? Kommer varningen vid körtid eller vid kompilering? 

\Subtask Anropa \code{parafärg} med den ''glömda'' färgen. Hur lyder felmeddelandet? Är det ett kompileringsfel eller ett körtidsfel?

\Subtask\Pen Förklara vad nyckelordet \code{sealed} innebär och vilken nytta man kan ha av att \textbf{försegla} en supertyp.


\Task \emph{Betydelsen av små och stora begynnelsebokstaver vid matchning.} TODO



\Task \emph{TODO Använda och matcha på Option.}  

\Task \emph{TODO Fånga undantantag i Java} Integer.parseInt

\Task \emph{TODO Fånga undantantag i Scala med \code{try}.} Integer.parseInt

\Task \emph{TODO Fånga undantantag i Scala med \code{scala.util.Try}.} Integer.parseInt  

\Task \emph{TODO flatten och flatMap med Option och Try}

\Task \emph{TODO partiella funktioner och metoden collect på samlingar} TODO Ska detta vara fördjupning???


\Task \emph{Överskugga metoden \code{equals}}  Om man överskuggar den befintliga metoden \code{equals} så kommer metoden \code{==} att fungera annorlunda. Vi börjar att studera den befintliga equals med referenslikhet.

\begin{REPL}
scala> class Gurka(val vikt: Int)
scala> val g1, g2 = new Gurka(42)
scala> val g3 = new Gurka(42)
scala> g1 == g2
scala> g1 == g3
scala> g1.equals  // tryck TAB två gånger
\end{REPL}

\Subtask Om du trycker TAB \emph{två} gånger efter ett metodnamn får du se metodens signatur. Vilken signatur har metoden \code{equals}?

\Subtask Byt ut equals enligt nedan och förklara vad som händer.

\begin{REPL}
scala> class Gurka(val vikt: Int) { 
         override def equals(other: Any): Boolean = other match {
           
         } 
scala> val g = new Gurka(42)
scala> g.equals  // tryck TAB två gånger
\end{REPL}



\Task \emph{Klassen \code{Complex} och metoden \code{equals}.} Implementera klassen \code{Complex} som representerar ett komplext tal\footnote{\href{https://sv.wikipedia.org/wiki/Komplexa_tal}{sv.wikipedia.org/wiki/Komplexa\_tal}} med realdel och imaginärdel. I stället för att använda en case-klass och en \code{equals}-metod som automatiskt ger innehållslikhet, ska du träna på att implementera en egen \code{equals}.

\begin{Code}
class Complex(re: Double, im: Double) extends  {
  def abs: Double = math.hypot(re, im)
  override def toString = s"Complex($re, $im)"
  //override def equals(other: Any): Boolean = ???
}
case object Complex {
  def apply(re: Double, im: Double): Complex = new Complex(re, im)
}
\end{Code}

\ExtraTasks %%%%%%%%%%%%%%%%%%%

\Task 

\AdvancedTasks %%%%%%%%%%%%%%%%%

\Task Överskugga equals vid arv. Ska detta vara fördjupning???     

\begin{Code}
trait Number
class Complex(re: Double, im: Double) extends Number {
  override def equals(other: Any): Boolean = ???
}
class Rational(numerator: Int, denominator: Int) extends Number {
  override def equals(other: Any): Boolean = ???
}
\end{Code}
    