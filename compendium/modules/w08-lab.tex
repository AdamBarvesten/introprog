%!TEX encoding = UTF-8 Unicode

%!TEX root = ../compendium.tex

\Lab{\LabWeekEIGHT}

\begin{Goals}
\item Kunna använda mönstermatchning
\item Känna till exceptions
\item Förstå hur Try fungerar

\end{Goals}

\begin{Preparations}
\item Gör övning {\tt \ExeWeekEIGHT} i kapitel \ref{exe:W08}.
\item Läs om exceptions och felhantering.
\item Bonus: ha tillgång till en dator där ni kan spela upp ljud.
\end{Preparations}

\subsection{Bakgrund}
Inom musik utgår man från en skala med 12 olika toner (C, C\#, D, D\#, E, F, F\#,
G, G\#, A, A\#, B). Nästa ton efter B är C och tillhör nästa oktav.
Ett ackord är uppbyggt av ett antal olika toner som spelas tillsammans.
Laborationen kommer att utgå från två olika instrumant: gitarr och ukulele.
Skillnaden för dessa två instrument är antalet strängar och vilken tonart
de är stämda i. På båda instrumenten har en grepbräda med ett antal olika band.
Ett ackord spelas genom att man med ett finger trycker ner på strängen på band
$i$. Om strängen spelas kommer tonen att vara ett halvt tonsteg högre än om man
håller ner strängen på plats $i-1$.

Laborationen kommer bestå av ett textbaserat användargränssnitt
där man kommer ha möjlighet bl.a. att lägga till nya ackord, rita upp ackord och
spela ackord. En ton anges på följande format "E2", vilket innebär andra oktaven
tonen E. Rekommenderad stämning för gitarr resp. ukulele är: E2, A2, D3, G3, B3,
E4 resp. G4, C4, E4, A4. Denna stämning kommer fungera för de fördefinierad
ackorden i filen chords.txt.

\subsection{Obligatoriska uppgifter}

\Task \code{Notes.} Objektet ska kunna omvandla en tons namn (t.ex. "E2") till
en heltalsrepresentation och tvärtom.

\Subtask Implementera först metoden \code{fromNbrToNote} med hjälp av
\code{%}-operatorn

\Subtask Implementera metoden \code{unapply} som ska ta in en sträng (t.ex. "E2")
och tonens nummer. "E2\" kommer översättas till 16 och "C1\" till 0. Utnyttja
attributet \code{toNumber} för att översätta en ton utan oktav ("C\#", "E") till
ett nummer. Tittar gärna på vad metoden \code{zipWithIndex.toMap} gör och
utnyttja den i \code{toNumber}.

\Subtask Skriv ett program som testar de olika metoderna

\Task \code{Chord.} Representation av de två olika ackorden. Lägga märken till
hur stämning (eng. \textit{tuning}) och ett grepp (eng. \textit{grip})
representeras i \code{Chord}. $-1$ i ett grep betyder att strängen inte ska
spelas.

\Subtask Implementera \code{toString} så att den matchar utskriften i filen
\textit{chords.txt}. I objektet \code{Chord} ska \code{toString} vara på formen
D:-1 -1 0 2 3 2

\Task textui

\Task Databas

\Task ChordPlayer

\Task ChordDraw

\subsection{Frivilliga extrauppgifter}

\Task Grov plan

\Subtask Använd ChordPlayer för att få datorn att spela hela låtar

\Subtask Använd NotePlayer för att få datorn att kunna plocka ackord (alltså
spela en sträng i taget)
