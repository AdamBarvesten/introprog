%!TEX encoding = UTF-8 Unicode

%!TEX root = ../compendium.tex

\Exercise{\ExeWeekFIVE}

\begin{Goals}
\item 
\end{Goals}

\begin{Preparations}
\item 
\end{Preparations}

\BasicTasks %%%%%%%%%%%%%%%%

\Task Föränderliga, intexerbara sekvenser: \code{Array} och \code{ArrayBuffer}

\Subtask Samlingen \code{scala.Array} har speciellt stöd i JVM och är extra snabb att allokera och indexera i. Dock kan man inte ändra storleken efter att en Array allokerats. Behöver man mer plats kan man kopiera den till en ny, större array. Koden nedan visar hur det kan gå till.
\begin{REPL}
scala> val xs = Array(42, 43, 44)
scala> val ys = new Array[Int](4)
scala> for (i <- 0 until xs.size){ys(i) = xs(i)}
scala> ys(3) = 45
\end{REPL}
Definiera funktionen \code{def copyAppend(xs: Array[Int], x): Array[Int]} som implementerar denna algoritm, \emph{efter} att du rätta de \textbf{två buggarna} i algoritmens while-loop:

\begin{algorithm}[H]
 \SetKwInOut{Input}{Indata}\SetKwInOut{Output}{Resultat}
 
 \Input{Heltalsarray $xs$ och heltalet $x$}
 \Output{En ny array som som är en kopia av $xs$ men med $x$ tillagt på slutet som extra element.}
 $n \leftarrow$ antalet element i $xs$ \\
 $ys \leftarrow$ en ny array med plats för $n + 1$ element\\
 $i \leftarrow 0$  \\
 \While{$i \leq n$}{
  $ys(i) \leftarrow xs(i)$
 }
 $ys(n) \leftarrow x$ 
\end{algorithm}



\Subtask Samlingen \code{scala.collection.mutable.ArrayBuffer} är inte riktigt lika snabb i alla lägen som Array men storleksändring hanteras automatiskt. Ofta använder ArrayBuffer om man stegvis vill bygga upp en sekvens. Vad händer nedan?
\begin{REPL}
scala> 

\end{REPL}


\Task \emph{Kopiering och uppdatering.} Metoder på oföränderliga samlingar skapar nya samlingar istället för att ändra. Därför behöver man inte själv skapa kopior. När en föränderlig samling uppdateras på plats, syns denna förändring via alla referenser till samlingen.

\begin{REPL}
scala> val xs = Vector(1, 2, 3)
scala> val ys = xs.toArray
scala> ys(1) = 42
scala> xs
scala> ys
scala> val zs = ys.toArray
scala> zs(1) = 84
scala> xs
scala> ys
scala> zs
\end{REPL}

\Subtask Syns updateringen av objektet som \code{ys} refererar till via referensen \code{xs}? Varför?

\Subtask Syns updateringen av objektet som \code{zs} refererar till via referensen \code{ys}? Varför? 

\Subtask Syns updateringen av objektet som \code{zs} refererar till via referensen \code{xs}? Varför?

\Task \emph{Skapa kopior av föränderliga samlingar.} Om man inte vill att en uppdatering av en föränderlig samling ska få oönskad påverkan på andra koddelar som refererar till samlingen, behöver man göra kopior av samlingen före uppdatering. Det finns färdiga metoder för kopiering av objekt av typen Array i paketet \code{java.util.Arrays}. 

\Subtask\Pen Studera dokumentationen för metoden \code{java.util.Arrays.copyOf} här:\\ \href{https://docs.oracle.com/javase/8/docs/api/java/util/Arrays.html\#copyOf-int:A-int-}{docs.oracle.com/javase/8/docs/api/java/util/Arrays.html\#copyOf-int:A-int-} \\
Notera att syntaxen för arrayer i Java är annorlunda: När det står \code{int[]} i Java så motsvarar det \code{Array[Int]} i Scala. Vad används den andra parametern till?

\Subtask\Pen Rita en bild av hur minnet ser ut efter varje tilldelning nedan. Vad har \code{xs}, \code{ys} och \code{zs} för värden efter exekveringen av raderna 1--5? Varför? 
\begin{REPL}
scala> val xs = Array(1, 2, 3, 4)
scala> val ys = xs
scala> val zs = java.util.Arrays.copyOf(xs, xs.size - 1)
sxala> xs(0) = 42
scala> zs(0) = 84
scala> ys
scala> xs
scala> zs
\end{REPL}







\Task Keno-dragningar under ett år -> Registrering...

\ExtraTasks %%%%%%%%%%%%%%%%%%%

\Task 

\AdvancedTasks %%%%%%%%%%%%%%%%%

\Task     
    