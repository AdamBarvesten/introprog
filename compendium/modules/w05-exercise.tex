%!TEX encoding = UTF-8 Unicode

%!TEX root = ../compendium.tex

\Exercise{\ExeWeekFIVE}

\begin{Goals}
\item 
\end{Goals}

\begin{Preparations}
\item 
\end{Preparations}

\BasicTasks %%%%%%%%%%%%%%%%

\Task \emph{Variabelt antal argument.} Det går fint att deklarera en funktion som tar en argumentsekvens av godtycklig längd. Syntaxen består ev en asterisk \code{*} efter typen.

\Subtask Vad händer nedan?
\begin{REPL}
scala> def printAll(xs: Int*) = xs.foreach(println)
scala> printAll(42)
scala> printAll(1, 2, 7, 42)
scala> def printStrings(wa: String*) = println(wa)
scala> printStrings("hej","på","dej")
\end{REPL}

\Subtask Vad har parametern \code{wa} i \code{printStrings} ovan för typ?

\Subtask Ändra i \code{printAll} så att även längden på \code{xs} skrivs ut före utskriften av alla element. Testa att anropa \code{printAll} med olika antal parametrar. 

\Subtask Vad händer om du anropar \code{printAll} med noll parametrar?

\Task \emph{Oföränderliga sekvenser med föränderliga objekt.} 

\Subtask Vad får xs för värde efter att attributet i objektet som \code{c2} refererar till ändras på rad 4 nedan? Förklara vad som händer.
\begin{REPL}
scala> class IntCell(var x: Int){override def toString = "[Int](" + x + ")"}
scala> val (c1, c2, c3) = (new IntCell(7), new IntCell(8), new IntCell(9))
scala> val xs = Vector(c1, c2, c3)
scala> c2.x = 42
scala> xs
\end{REPL}

\Subtask\Pen Rita en bild av minnessituationen efter rad 4 ovan.

\Subtask\Pen Vad krävs för att allt innehåll i en oföränderlig samling garanterat ska förbli oförändrat? 

\Task Föränderliga, indexerbara sekvenser: \code{Array} och \code{ArrayBuffer}

\Subtask Samlingen \code{scala.Array} har speciellt stöd i JVM och är extra snabb att allokera och indexera i. Dock kan man inte ändra storleken efter att en Array allokerats. Behöver man mer plats kan man kopiera den till en ny, större array. Koden nedan visar hur det kan gå till.
\begin{REPL}
scala> val xs = Array(42, 43, 44)
scala> val ys = new Array[Int](4)  //plats för 4 heltal, från början nollor
scala> for (i <- 0 until xs.size){ys(i) = xs(i)}
scala> ys(3) = 45
\end{REPL}
Definiera funktionen \code{def copyAppend(xs: Array[Int], x): Array[Int]} som implementerar nedan algoritm, \emph{efter} att du rätta de \textbf{\color{red}{två buggarna}} i algoritmens while-loop:

\begin{algorithm}[H]
 \SetKwInOut{Input}{Indata}\SetKwInOut{Output}{Resultat}
 
 \Input{Heltalsarray $xs$ och heltalet $x$}
 \Output{En ny array som som är en kopia av $xs$ men med $x$ tillagt på slutet som extra element.}
 $n \leftarrow$ antalet element i $xs$ \\
 $ys \leftarrow$ en ny array med plats för $n + 1$ element\\
 $i \leftarrow 0$  \\
 \While{$i \leq n$}{
  $ys(i) \leftarrow xs(i)$
 }
 $ys(n) \leftarrow x$ 
\end{algorithm}



\Subtask Samlingen \code{scala.collection.mutable.ArrayBuffer} är inte riktigt lika snabb i alla lägen som \code{scala.Array} men storleksändring hanteras automatiskt, vilket är en stor fördel då man slipper att själv implementera algoritmer liknande \code{copyAppend} ovan. Speciellt använder man ofta \code{ArrayBuffer} om man stegvis vill bygga upp en sekvens. Vad händer nedan?
\begin{REPL}
scala> val xs = scala.collection.mutable.ArrayBuffer.empty[Int]
scala> xs.append(1, 2)
scala> while (xs.last < 100) {xs.append(xs.takeRight(2).sum); println(xs)}
scala> xs.last
scala> xs.length
\end{REPL}

\Subtask Talen i sekvensen som produceras ovan kallas Fibonaccital\footnote{\href{https://sv.wikipedia.org/wiki/Fibonaccital}{sv.wikipedia.org/wiki/Fibonaccital}}. Hur lång ska en Fibonacci-sekvens vara för att det sista elementet ska komma så nära (men inte över) \code{Int.MaxValue} som möjligt?



\Task \emph{Kopiering och uppdatering.} Metoder på oföränderliga samlingar skapar nya samlingar istället för att ändra. Därför behöver man inte själv skapa kopior. När en \emph{föränderlig} samling uppdateras på plats, syns denna förändring via alla referenser till samlingen.

\begin{REPL}
scala> val xs = Vector(1, 2, 3)
scala> val ys = xs.toArray
scala> ys(1) = 42
scala> xs
scala> ys
scala> val zs = ys.toArray
scala> zs(1) = 84
scala> xs
scala> ys
scala> zs
\end{REPL}

\Subtask Syns updateringen av objektet som \code{ys} refererar till via referensen \code{xs}? Varför?

\Subtask Syns updateringen av objektet som \code{zs} refererar till via referensen \code{ys}? Varför? 

\Subtask Syns updateringen av objektet som \code{zs} refererar till via referensen \code{xs}? Varför?

\Task \emph{Färdig metod för att skapa kopia av array.} Om man inte vill att en uppdatering av en föränderlig samling ska få oönskad påverkan på andra koddelar som refererar till samlingen, behöver man göra kopior av samlingen före uppdatering. Det finns färdiga metoder för kopiering av objekt av typen Array i paketet \code{java.util.Arrays}. 

\Subtask\Pen Studera dokumentationen för metoden \code{java.util.Arrays.copyOf} här:\\ \href{https://docs.oracle.com/javase/8/docs/api/java/util/Arrays.html\#copyOf-int:A-int-}{docs.oracle.com/javase/8/docs/api/java/util/Arrays.html\#copyOf-int:A-int-} \\
Notera att syntaxen för arrayer i Java är annorlunda: När det står \code{int[]} i Java så motsvarar det \code{Array[Int]} i Scala. Vad används den andra parametern till?

\Subtask\Pen Rita en bild av hur minnet ser ut efter varje tilldelning nedan. Vad har \code{xs}, \code{ys} och \code{zs} för värden efter exekveringen av raderna 1--5 nedan? Varför? 
\begin{REPL}
scala> val xs = Array(1, 2, 3, 4)
scala> val ys = xs
scala> val zs = java.util.Arrays.copyOf(xs, xs.size - 1)
sxala> xs(0) = 42
scala> zs(0) = 84
scala> ys
scala> xs
scala> zs
\end{REPL}

\Task \emph{Algortim: SEQ-REVERSE-COPY.} Implementera nedan algoritm:

\begin{algorithm}[H]
 \SetKwInOut{Input}{Indata}\SetKwInOut{Output}{Resultat}
 
 \Input{Heltalsarray $xs$ och heltalet $x$}
 \Output{En ny heltalsarray med elementen i $xs$ i omvänd ordning.}
 $n \leftarrow$ antalet element i $xs$ \\
 $ys \leftarrow$ en ny heltalsarray med plats för $n$ element\\
 $i \leftarrow 0$  \\
 \While{$i < n$}{
  $ys(n - i - 1) \leftarrow xs(i)$ \\
  $i \leftarrow i + 1$
 }
 \Return $ys$
\end{algorithm}

\Subtask\Pen Skriv implementation med penna och papper. Använd en \code{while}-sats på samma sätt som i algoritmen. Prova sedan din implementation på dator och kolla så att den fungerar.

\Subtask\Pen \label{subtask:for-seq-copy} Skriv implementationen med penna och papper igen, men använd nu istället en \code{for}-sats som räknar baklänges. Prova sedan din implementation på dator och kolla så att den fungerar. 

\Subtask Definiera en funktion i REPL med namnet \code{reverseCopy} med din implementation i uppgift \ref{subtask:for-seq-copy}.  


\Task \emph{Algoritm: SEQ-REVERSE.} Strängar av typen \code{String} är oföränderliga. Vill man ändra i en sträng utan att skapa en ny kopia kan man använda en \code{StringBuilder} enligt nedan algoritm som vänder bak-och-fram på en sträng. 

\begin{algorithm}[H]
 \SetKwInOut{Input}{Indata}\SetKwInOut{Output}{Resultat}
 
 \Input{En sträng $s$ av typen \texttt{String}}
 \Output{En ny sträng av typen \texttt{String}}
 $sb \leftarrow$ en ny \texttt{StringBuilder} som innehåller $s$ \\
 $n \leftarrow$ antalet tecken i $s$\\
 $i \leftarrow 0$  \\
 \For{$i \leftarrow 0$ \KwTo $\frac{n}{2} - 1$}{
  $temp \leftarrow sb(i)$ \\
  $sb(i) \leftarrow sb(n - i - 1)$ \\
  $sb(n - i - 1) \leftarrow temp$ \\
 }
 \Return $sb$ omvandlad till en \texttt{String}
\end{algorithm}

\Subtask Implementera algoritmen ovan i en funktion med signaturen: \\
 \code{def reverseString(s: String): String}

\begin{Code}
// Kod till facit:
def reverseString(s: String): String = {
  val sb = new StringBuilder(s)
  val n = sb.length
  for (i <-0 until n / 2) { 
    val temp = sb(i)
    sb(i) = sb(n - i - 1)
    sb(n - i - 1) = temp
  }
  sb.toString     
}
\end{Code}

\Subtask Använd din funktion \code{reverseString} från föregående deluppgift i en ny funktion med signaturen:\\
 \code{def isPalindrome(s: String): Boolean} \\ som avgör om en sträng är en palindrom.\footnote{\href{https://sv.wikipedia.org/wiki/Palindrom}{sv.wikipedia.org/wiki/Palindrom}} 

\Subtask\Pen Man kan med en \code{while}-sats och indexering direkt i en \code{String} avgöra om en sträng är en palindrom utan att kopiera den till en \code{StringBuilder}. Implementera en ny variant av \code{isPalindrome} som använder denna metod. Skriv först algoritmen på papper i pseudo-kod.

\begin{Code}
// Kod till facit:
def isPalindrome(s: String): Boolean = {
  val n = s.length
  var foundDiff = false
  var i = 0
  while (i < n/2 && !foundDiff)  { 
    foundDiff = s(i) != s(n - i - 1)
    i += 1
  }
  !foundDiff
}
\end{Code}

\Task \label{task:seq-reg} \emph{Algoritm: SEQ-REGISTER.} Algoritmer för registrering löser problemet att räkna förekomst av olika saker, till exempel antalet tärningskast som gav en sexa. Antag att vi har följande vektor \code{xs} som representerar 13 st tärningskast:
\begin{REPL}
scala> val xs =  Vector(5, 3, 1, 6, 1, 3, 5, 1, 1, 6, 3, 2, 6)
\end{REPL}

\Subtask Använd metoderna \code{filter} och \code{size} på \code{xs} för att filtrera ut alla 6:or och räkna hur många de är.

\Subtask Använd metoderna \code{filter} och \code{size} på \code{xs} för att filtrera ut alla jämna kast och räkna hur många de är.

\Subtask Metoden \code{groupBy} på en samling tar en funktion \code{f} som parameter och skapar en ny \code{Map} med nycklar \code{k} som är associerade till samlingar som utgör grupper av värden där 
\code{f(x) == k}.  Vad händer här:
\begin{REPL}
scala> xs.groupBy(x => x % 2)
scala> xs.groupBy(_ % 2)
scala> xs.groupBy(_ % 3)
scala> xs.groupBy(_ % 3).foreach(println)
scala> val freqEvenOdd = xs.groupBy(_ % 2).map(p => (p._1, p._2.size)) 
scala> val nEven = freqEvenOdd(0)
scala> val nOdd = freqEvenOdd(1)
\end{REPL}

\Subtask Använd metoden \code{groupBy} på \code{xs} med den s.k. identitetsfunktionen \code{i => i} som returnerar sitt eget argument. Vad händer? 

\Subtask Definiera en \code{val freq: Map[Int, Int]} som räknar antalet olika tärningsutfall i \code{xs}. Använd metoden \code{groupBy} på \code{xs} med identitetsfunktionen följt av en \code{map} med funktionen \code{p => (p._1, p._2.size)}.

\Subtask Du ska nu själv implementera en registreringsalgoritm. Skriv en funktion:
\begin{Code}
def tärningsRegistrering(xs: Array[Int]): Array[Int] = ???
\end{Code}
som implementerar nedan algoritm (som alltså inte använder \code{groupBy} eller andra färdiga metoder på samlingar förutom \code{size} och \code{apply}).


\begin{algorithm}[H]
 \SetKwInOut{Input}{Indata}\SetKwInOut{Output}{Resultat}
 
 \Input{En array $xs$ med heltal mellan 1 och 6 som representerar utfall av många tärningskast.}
 \Output{En array $f$ med 7 st element där $f(0)$ innehåller totala antalet kast, $f(1)$ anger antalet ettor, $f(2)$ antalet tvåor, etc. till och med $f(6)$ som anger antalet sexor.}
 $f \leftarrow$ en ny array med $7$ element där alla element initaliseras till 0.\\
 $f(0) \leftarrow$ antalet element i $xs$ \\
 $i \leftarrow 0$  \\
 \While{$i < f(0)$}{
  $f(xs(i)) \leftarrow f(xs(i)) + 1$ \\
  $i \leftarrow i + 1$
 }
 \Return $f$ 
\end{algorithm}

Testa din funktion med nedan funktionsanrop:
\begin{REPL}
scala> tärningsRegistrering(Array.fill(1000)((math.random * 6).toInt +1))
res12: Array[Int] = Array(1000, 174, 174, 167, 171, 145, 169)
\end{REPL}

\begin{Code}
// kod till facit:
def tärningsRegistrering(xs: Array[Int]): Array[Int] = {
  val f = Array.fill(7)(0)
  f(0) = xs.size
  var i = 0
  while (i < f(0)) {
    f(xs(i)) += 1
    i += 1
  }
  f 
}
\end{Code}

\Task \emph{Algoritm: SEQ-REMOVE-COPY.} Ibland vill man kopiera alla element till en ny \code{Array} \emph{utom} ett element på en viss plats \code{pos}.   

\Subtask\Pen Skriv algoritmen SEQ-REMOVE-COPY i pseudokod med penna på papper.

\Subtask Implementera algoritmen SEQ-REMOVE-COPY i en funktion med denna signatur:
\begin{Code}
def removeCopy(xs: Array[Int], pos: Int): Array[Int]
\end{Code}

\begin{Code}
// kod till facit
def removeCopy(xs: Array[Int], pos: Int): Array[Int] = {
  val n = xs.size
  val ys = Array.ofDim[Int](n - 1)
  for (i <- 0 until pos) ys(i) = xs(i)
  for (i <- pos+1 until n) ys(i - 1) = xs(i)
  ys
} 
\end{Code}

\Task \emph{Algoritm: SEQ-REMOVE.} Ibland vill man ta bort ett element på en viss position i en befintlig \code{Array} utan att kopiera alla element till en ny \code{Array}. Ett sätt att göra detta är att flytta alla efterföljande element ett steg mot lägre index och låta sista platsen bli 0.   

\Subtask\Pen Skriv algoritmen SEQ-REMOVE i pseudokod med penna på papper.

\Subtask Implementera algoritmen SEQ-REMOVE i en funktion med denna signatur:
\begin{Code}
def remove(xs: Array[Int], pos: Int): Unit
\end{Code}

\begin{Code}
// kod till facit
def remove(xs: Array[Int], pos: Int): Unit = {
  val n = xs.size
  for (i <- pos+1 until n) xs(i - 1) = xs(i)
  xs(n-1) = 0
} 
\end{Code}
 

%%%%%%%%%%%%%%%%%%%%%%%%%%%%%%%%%%%%%%%%%%%5


\Task \emph{Deterministiska pseudoslumptalssekvenser med \code{java.util.Random}.} Klassen \code{java.util.Random} ger möjlighet att generera en sekvens av tal som verkar slumpmässiga. Genom att välja ett visst s.k. \textbf{frö} \Eng{seed} kan man få samma sekvens av pseudoslumptal varje gång.

\Subtask\Pen Sök upp och studera dokumentationen för \code{java.util.Random}. Hur skapar man en ny instans av klassen \code{Random}? Vad gör operationen \code{nextInt} på \code{Random}-objekt. 

\Subtask Förklara vad som händer nedan?
\begin{REPL}
import java.util.Random
val frö = 42L
val rnd = new Random(frö)
rnd.nextInt(10)
(1 to 100).foreach(print(rnd.nextInt(10)))
val rnd1 = new Random(frö)
val rnd2 = new Random(frö)
val rnd3 = new Random(System.nanoTime)
val rnd4 = new Random((math.random * Long.MaxValue).toLong)
def flip(r: Random) = if (r.nextInt(2) > 0) "krona" else "klave" 
val xs = (1 to 100).map{i => (flip(rnd1), flip(rnd2), flip(rnd3), flip(rnd4))} 
xs foreach println
xs.exists(q => q._1 != q._2)
xs.exists(q => q._1 != q._3)
\end{REPL} 

\Subtask\Pen Nämn några sammanhang då det är användbart att kunna bestämma fröet till en slumptalssekvens.

\Subtask Blir det samma sekvens om du använder fröet \code{42L} som argument till konstruktorn vid skapandet av en instans av \code{java.util.Random} på en \emph{annan} dator?

\Subtask Sök reda på dokumentationen för \code{java.math.random} och undersök hur denna sekvens skapas. 

\Subtask Vad blir det för frö till slumpalssekvensen om man skapar ett \code{Random}-objekt med hjälp av konstruktorn utan parameter?

\Task \emph{Undersök om tärningskast är rektangelfördelade.} \\ Skriv en funktion \\ \code{def testRandom(r: Random, n: Int): Unit = ???} \\ som ger följande utskrift.
\begin{REPL}
scala> val rnd = new Random(42L)
scala> testRandom(rnd, 1000)
Antal kast: 1000
Antal 1:or: 178
Antal 2:or: 187
Antal 3:or: 167
Antal 4:or: 148
Antal 5:or: 155
Antal 6:or: 165
\end{REPL}

\emph{Tips:} 
Anropa din funktion \code{tärningsRegistrering} från uppgift \ref{task:seq-reg}.


\begin{Code}
// kod till facit
def testRandom(r: Random, n: Int): Unit = {
  val xs = Array.fill(n)(r.nextInt(6) + 1)
  val f = tärningsRegistrering(xs)
  println("Antal kast: " + f(0))
  for (i <- 1 to 6) println(s"Antal $i:or: " + f(i)) 
}
\end{Code}

\Task \emph{Array och \code{for}-sats i Java.} 

\Subtask Skriv nedan program i en editor och spara i filen \code{DiceReg.java}:

\javainputlisting{examples/DiceReg.java} 

\Subtask Kompilera med \code{javac DiceReg.java} och kör med \code{java DiceReg 10000 42} och förklara vad som händer.

\Subtask\Pen Beskriv skillnaderna mellan Scala och Java, vad gäller syntaxen för array och \code{for}-sats. Beskriv några andra skillnader mellan språken som syns i programmet ovan.

\Subtask Ändra i programmet ovan så att loop-variabeln \code{i} skrivs ut i varje runda i varje \code{for}-sats. Kompilera om och kör.

\Subtask Skriv om programmet ovan genom att abstrahera huvudprogrammets delar till de statiska metoderna \code{parseArguments}, \code{registerPips} och \code{printReg} enligt nedan skelett. Notera speciellt hur \code{private} och \code{public} är angivet. Spara programmet i filen \code{DiceReg2.java}.

\begin{Code}[language=Java]
// DiceReg2.java
import java.util.Random;

public class DiceReg2 {
    public static int[] diceReg = new int[6];  
    private static Random rnd = new Random();

    public static int parseArguments(String[] args) {
        // ???
        return n;        
    }
    
    public static void registerPips(int n){
        // ???
    }
    
    public static void printReg() {
        // ???
    }

    public static void main(String[] args) {
        int n = parseArguments(args);
        registerPips(n);
        printReg();
    }
}
\end{Code}

\Subtask Starta Scala REPL i samma bibliotek som filen \texttt{DiceReg2.class} ligger i och kör nedan satser och förklara vad som händer:
\begin{REPL}
scala> DiceReg2.main(Array("1000","42"))
scala> DiceReg2.diceReg
scala> DiceReg2.registerPips(1000)
scala> DiceReg2.printReg
scala> DiceReg2.registerPips(1000)
scala> DiceReg2.printReg
scala> DiceReg2.rnd
\end{REPL}

\Subtask Växla synligheten på attributen mellan \jcode{private} och \jcode{public}, kompilera om och studera effekten i Scala REPL. Hur lyder felmeddelandet om du försöker komma åt en privat medlem?

\Subtask\Pen Ange en viktig anledning till att man kan vilja göra medlemmar privata.



\Task \emph{Läsa in tal med \code{java.util.Scanner}.} Med \jcode{new Scanner(System.in)} skapas ett objekt som kan läsa in tal som användaren skriver i terminalfönstret.

\Subtask Sök upp och studera dokumentationen för \code{java.util.Scanner}. Vad gör metoderna \jcode{hasNextInt()} och \jcode{nextInt()}?

\Subtask Skriv nedan program i en editor och spara i filen \code{DiceScanBuggy.java}:

\javainputlisting{examples/DiceScanBuggy.java} 

\Subtask Kompilera och kör med indatasekvensen \texttt{1 2 3 4 -1} och notera hur registreringen sker.

\Subtask Programmet fungerar inte som det ska. Du behöver korrigera 3 saker för att programmet ska göra rätt. Rätta buggarna och spara det rättade programmet som \texttt{DiceScan.java}. Kompilera och testa att det rättade programmer fungerar med olika indata.

 
\ExtraTasks %%%%%%%%%%%%%%%%%%%

\Task \emph{Algoritm: SEQ-INSERT-COPY.} 

\begin{algorithm}[H]
 \SetKwInOut{Input}{Indata}\SetKwInOut{Output}{Resultat}
 
 \Input{En sekvens $xs$ av typen \texttt{Array[Int]} och heltalen $x$ och $pos$}
 \Output{En ny sekvens av typen \texttt{Array[Int]} som är en kopia av $xs$ men där $x$ är infogat på plats $pos$}
 $n \leftarrow$ antalet element $xs$\\ 
 $ys \leftarrow$ en ny \texttt{Array[Int]} med plats för $n+1$ element \\
 \For{$i \leftarrow 0$ \KwTo $pos - 1$}{
  $ys(i) \leftarrow xs(i)$
 }
 $ys(pos) \leftarrow x$ \\
 \For{$i \leftarrow pos$ \KwTo $n - 1$}{
  $ys(i + 1) \leftarrow xs(i)$
 }
 \Return $ys$ 
\end{algorithm}

\Subtask Implementera ovan algoritm i en funktion med denna signatur:
\begin{Code}
def insertCopy(xs: Array[Int], x: Int, pos: Int): Array[Int]
\end{Code}

\begin{Code}
// kod till facit
def insertCopy(xs: Array[Int], x: Int, pos: Int): Array[Int] = {
  val n = xs.size
  val ys = Array.ofDim[Int](n + 1)
  for (i <- 0 until pos) ys(i) = xs(i)
  ys(pos) = x
  for (i <- pos until n) ys(i + 1) = xs(i)
  ys
} 
\end{Code}

\Subtask Vad måste \code{pos} vara för att det ska fungera med en tom array som argument?

\Subtask Vad händer om din funktion anropas med ett negativt argument för \code{pos}?

\Subtask Vad händer om din funktion anropas med \code{pos} lika med \code{xs.size}?

\Subtask Vad händer om din funktion anropas med \code{pos} större än \code{xs.size}?



\Task \emph{Algoritm: SEQ-INSERT.} Man kan implementera algoritmen SEQ-INSERT på plats i en \code{Array[Int]} så att alla elementen efter \code{pos} flyttas fram ett steg och att sista elementet ''försvinner''.

\Subtask\Pen Skriv algoritment SEQ-INSERT i pseudokod med penna och papper.

\Subtask Implemtera SEQ-INSERT i en funktion med denna signatur:
\begin{Code}
def insert(xs: Array[Int], x: Int, pos: Int): Unit
\end{Code}

\Task Implementera funktionen \code{tärningsRegistrering} från uppgift \ref{task:seq-reg} på nytt, men nu med en \code{for}-sats istället.



\AdvancedTasks %%%%%%%%%%%%%%%%%

\Task Sök reda på dokumentationen för metoden \code{patch} på klassen \code{Array}.

\Subtask Använd metoden \code{patch} för att implementera SEQ-INSERT-COPY: 
\begin{Code}
def insertCopy(xs: Array[Int], x: Int, pos: Int): Array[Int] = 
  xs.patch(???, ???, ???)
\end{Code}

\Subtask Använd metoden \code{patch} för att  implementera SEQ-REMOVE-COPY: 
\begin{Code}
def removeCopy(xs: Array[Int], pos: Int): Array[Int] = 
  xs.patch(???, ???, ???)
\end{Code}

\Task Studera skillnader och likheter mellan 

\Subtask \code{Array}

\Subtask \code{WrappedArray}  

\Subtask \code{ArraySeq} 

\noindent genom att läsa mer om dessa arrayvarianter här: \\
\href{http://docs.scala-lang.org/overviews/collections/concrete-mutable-collection-classes}{docs.scala-lang.org/overviews/collections/concrete-mutable-collection-classes} \\  
\href{http://docs.scala-lang.org/overviews/collections/arrays.html}{docs.scala-lang.org/overviews/collections/arrays.html}  \\ 
\href{http://stackoverflow.com/questions/5028551/scala-array-vs-arrayseq}{stackoverflow.com/questions/5028551/scala-array-vs-arrayseq}   
    
    
\Task Studera vad metoden \code{java.util.Arrays.deepEquals} gör här:\\
\href{https://docs.oracle.com/javase/8/docs/api/java/util/Arrays.html#deepEquals-java.lang.Object:A-java.lang.Object:A-}{Arrays.html\#deepEquals-java.lang.Object:A-java.lang.Object:A-} \\
Vad skiljer ovan metod från metoden \code{java.util.Arrays.equals}?

\Task Keno-dragningar under ett år -> Registrering...

\Task \emph{Använda \code{jline} istället för \code{Scanner} i REPL.} Om du använder  \code{java.util.Scanner} i Scala REPL så ekas inte de tecken som skrivs, så som sker om du använder scannern med \code{System.in} i en kompilerad applikation. Om du vill se vad du skriver vid indata i REPL kan du använda \code{jline}\footnote{
\href{https://github.com/jline/jline2}{github.com/jline/jline2}
} och klassen \code{jline.console.ConsoleReader}\footnote{
\href{http://jline.github.io/jline2/apidocs/reference/jline/console/ConsoleReader.html}{jline.github.io/jline2/apidocs/reference/jline/console/ConsoleReader.html}}. 
Då får du dessutom editeringsfunktioner vid inmatning med t.ex. Ctrl+A och Ctrl+K så som i en vanlig unixterminal. Med pil upp och pil ner kan du bläddra i inmatningshistoriken.
\begin{REPL}
val scan = new java.util.Scanner(System.in)
scan.next
scan.nextInt
val cr = new jline.console.ConsoleReader
cr.readLine
cr.readLine("> ")
cr.readLine("Ange tal: ").toInt
scala.util.Try{cr.readLine("Ange tal: ").toInt}.toOption
\end{REPL} 

\Subtask Prova ovan rader i REPL. Vad händer om du matar in bokstäver i stället för siffror på sista raden ovan? (Mer om \code{Option} i kapitel \ref{chapter:W08}).

\Subtask Skriv ett funktion \code{readPalindromLoop} som låter användaren mata in strängar och som kollar om de är palindromer så som nedan REPL-körning indikerar. Skriv funktionen i en editor och klistra in den i REPL enligt nedan istället för \code{???}

\begin{REPL}
scala> val cr = new jline.console.ConsoleReader
scala> def isPalindrome(s: String): Boolean = s == s.reverse
scala> :paste 
// Entering paste mode (ctrl-D to finish)

def readPalindromLoop: Unit = ???

// Exiting paste mode, now interpreting.

readPalindromLoop: Unit

scala> readPalindromLoop
Ange sträng följt av <Enter>
Programmet avslutas med tom sträng + <Enter>
> gurka
gurka är ingen palindrom
> dallassallad
dallassallad är en palindrom!
> 
Tack och hej!
scala>
\end{REPL}

\Subtask Skapa ett objekt med inläsningsstöd enligt nedan specifikation. Objektet ska delegera implementationerna till ett attribut \code{private val reader} som innehåller en referens till ett \code{ConsoleReader}-objekt.
\begin{ScalaSpec}{termutil}
object termutil {
  /** Reads one line from terminal input. */
  def readLine: String = ???

  /** Prints prompt and reads one line. */
  def readLine(prompt: String): String = ???

  /** Reads one line and converts it to an Int. 
   *  If a non-integer is input, a NumberFormatException is thrown.  */
  def readInt: Int = ???

  /** Prints prompt, reads one line and converts it to an Int. 
   *  If a non-integer is input, a NumberFormatException is thrown.  */
  def readInt(prompt: String): Int = ???

  /** Reads one line and converts it to an Option[Int]
   *  with Some integer or None if the input cannot be converted.  */
  def readIntOpt: Option[Int] = ???

  /** Prints prompt, reads one line and converts it to an Option[Int]
   *  with Some integer or None if the input cannot be converted.  */
  def readIntOpt(prompt: String): Option[Int] = ???
}
\end{ScalaSpec}