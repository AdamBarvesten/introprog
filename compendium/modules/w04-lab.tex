%!TEX root = ../compendium.tex
%!TEX encoding = UTF-8 Unicode

\Lab{\LabWeekFOUR}

\begin{Goals}
\item {\"O}va p{\aa} att anv{\"a}nda datastrukturerna tuple, lista, map.
\item Att spara data till fil.
\item Att l{\"a}sa in data fr{\aa}n fil.
\item Att spara objekt till fil.
\item Att l{\"a}sa in objekt fr{\aa}n en fil.
\end{Goals}

\begin{Preparations}
\item G{\"o}r {\"o}vning 4. % ref
\item {\"O}ppna Scala IDE i Eclipse enligt intruktionerna XX.
\item Skapa ett projekt och skapa ett \code{object Hello} med en \code{main}-metod enligt XY.
\item Skriv ut en h{\"a}lsning till terminalen med \code{println("...")} och testk{\"o}r programmet genom att markera filnamnet i projektmenyn och trycka p{\aa} den gr{\"o}na pilen. Kontrollera att h{\"a}lsningen skrivs ut!
\end{Preparations}

\subsection{Obligatoriska uppgifter}
Efter en rad olyckliga omständigheter har du blivit pirat i 1700-talets Karibien. Nu beh{\"o}ver du undvika galgen, hitta en skatt och f{\"o}rs{\"o}ka f{\"o}rutse dina f{\"o}r{\"a}diska skeppskamraters n{\"a}sta steg.

\Task \emph{Save your crew}. 

\Subtask Kung George {\"a}r villig att ben{\aa}da fem personer ur din bes{\"a}ttning! Skapa en lista d{\"a}r personerna sparas med f{\"o}rnamn, efternamn och befattning genom att l{\"a}sa in dem fr{\aa}n terminalen (kallad Console i Eclipse). Inl{\"a}sning kan g{\"o}ras med kodraden 
\begin{Code}
val förnamn = scala.io.StdIn.readLine("Förnamn: "). 
\end{Code}
Namnen och befattningen kan sparas som en tuppel som skapas med koden \code{ (förnamn, efternamn, befattning)}. En tom f{\"o}r{\"a}nderlig lista med s{\aa}dana tuppler kan skapas genom att ange typerna till listan: 
\begin{Code}
var enlist = List[Tuple3[String, String, String]]()
\end{Code}
 Exempel 

\Subtask En underuppgift.


\Task \emph{Read the map}. 

\Subtask En underuppgift.

\subsection{Frivilliga extrauppgifter}

\Task En labbuppgiftsbeskrivning.

\Subtask En underuppgift.

\Subtask En underuppgift.
    