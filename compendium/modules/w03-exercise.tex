%!TEX root = ../compendium.tex

\Exercise{\ExeWeekTHREE}

\begin{Goals}
\item 
\end{Goals}

\begin{Preparations}
\item 
\end{Preparations}

\BasicTasks %%%%%%%%%%%%%%%%

\Task \label{task:funcall}\emph{Definiera och anropa funktioner.} En funktion med två parametrar definieras med följande syntax i Scala: \vspace{0.5em} \\  \texttt{\code{def} \textit{namn}(\textit{parameter1}: \textit{Typ1}, \textit{parameter2}: \textit{Typ2}): \textit{Returtyp} = \textit{returvärde}}

\Subtask Definiera en funktion med namnet \code{öka} som har en heltalsparameter \code{x} och som returnerar \code{x + 1}. Ange returtypen explicit. Testa funktionen i REPL med argumentet 42.

\begin{REPL}
scala> ???  // definiera funktionen öka
scala> öka(42)
43
\end{REPL}

\Subtask\Pen Vad har funktionen \code{öka} i föregående uppgift för returtyp?

\Subtask\Pen Vad gör kompilatorn om du utelämnar returtypen?

\Subtask\Pen Varför kan det vara bra att ange returtypen explicit?

\Subtask\Pen Vad är det för skillnad mellan parameter och argument?
 
\Subtask Vad har uttrycket \code{öka(öka(öka(öka(42))))} för värde?

\Subtask Definera funktionen \code{minska(x: Int): Int} med returvärdet \code{x - 1}.

\Subtask Vad är värdet av uttrycket \code{öka(minska(öka(öka(minska(minska(42))))))}


\Task \emph{Funktion med flera parametrar.} Definiera i REPL två funktioner \code{sum} och \code{diff} med två heltalsparametrar som returnerar summan respektive differensen av argumenten: \\
\code{def sum(x: Int, y: Int): Int = x + y} \\
\code{def diff(x: Int, y: Int): Int = x - y} \\
Vad har nedan uttryck för värden? Förklara vad som händer.

\Subtask \code{diff(0, 100)}

\Subtask \code{diff(100, add(42, 43))}

\Subtask \code{sum(sum(42, 43), diff(100, sum(0, 0))}

\Subtask \code{sum(diff(Byte.MaxValue, Byte.MinValue),1)}

\Task \emph{Funktion med default-argument.} Förklara vad som händer här?
\begin{REPL}
scala> def inc(i: Int, j: Int = 1) = i + j
scala> inc(42, 2)
scala> inc(42, 1)
scaka> inc(42)
\end{REPL}

\Task \emph{Funktionsanrop med namngivna argument.} 
\begin{REPL}
scala> def skrivNamn(förnamn: String, efternamn: String) = 
         println("Namn: " + efternamn + ", " + förnamn)
scala> skrivNamn("Kim", "Robinson")
scala> skrivNamn(förnamn = "Viktor", efternamn = "Oval")
scaka> skrivNamn(efternamn = "Triangelsson", förnamn = "Stina")
\end{REPL}

\Subtask Förklara vad som händer ovan?

\Subtask\Pen Vad är fördelen med namngivna argument?



\Task \emph{Applicera en funktion på elementen i en samling.} Använd dina funktioner \code{öka} och \code{minska} från uppgift \ref{task:funcall}. Vad har nedan uttryck för värde? Förklara vad som händer.

\Subtask \code{for (i <- 0 to 4) yield öka(i)}

\Subtask \code{for (i <- 1 to 5) yield minska(i)}

\Subtask \code{(0 to 4).map(i => öka(i))}

\Subtask \code{(1 to 5).map(i => minska(i))}

\Subtask \code{(0 to 4).map(öka)}

\Subtask \code{(1 to 5).map(minska)}

\Subtask \code{Vector(12, 3, 41, -8).map(öka)}

\Subtask \code{Vector(12, 3, 41, -8).map(öka).map(minska).map(minska)}



\Task En funktion som inte returnerar något intressant värde, men som anropas för det den \emph{gör} kallas \textbf{procedur}. Definiera följande procedur i REPL: \\ 
\code{def tUvirks(msg: String) = println(msg.reverse)} \\
Vad skriver nedan satser ut? Förklara vad som händer.

\Subtask \code{println("sallad".reverse)}

\Subtask \code{tUvirks("sallad")}

\Subtask \code{val x = tUvirks("sallad"); println(x)}

\Subtask \code{def enhetsvärdet = (); println(enhetsvärdet)}

\Subtask \code{def bortkastad: Unit = 1 + 1; println(bortkastad)}

\Subtask \code|def bortkastad2 = {val x = 1 + 1}; println(bortkastad2)|

\Subtask\Pen Varför är det bra att explicit ange \code{Unit} som returtyp för procedurer?


\Task \emph{Värdeanrop och namnanrop (fördröjd evaluering, ''lata'' argument).} Deklarera nedan funktioner i REPL eller Kojo.

\begin{Code}
def snark = {print("snark "); Thread.sleep(1000); 42}
def callByValue(x: Int) = x + x
def callByName(x: => Int) = x + x
\end{Code}

Evaluera nedan uttryck. Förklara vad som händer.

\Subtask \code{snark}

\Subtask \code{snark; snark; snark}

\Subtask \code{callByValue(1)}

\Subtask \code{callByName(1)}

\Subtask \code{callByValue(snark)}

\Subtask \code{callByName(snark)}


\Subtask Förklara vad som händer här:
\begin{REPL}
scala> def görDetta(block: => Unit) = block
scala> görDetta(println("hej"))
scala> görDetta{println("goddag")}
scala> görDetta{println("hej"); println("svejs")}
scala> def görDettaTvåGånger(block: => Unit) = {block; block}
scala> görDettaTvåGånger{println("goddag")}
\end{REPL}


\Task \emph{Uppdelad parameterlista.} Man kan dela upp parametrarna till en funktion i flera parameterlistor. Förklara vad som händer här:
\begin{REPL}
scala> def add(a: Int)(b: Int) = a + b
scala> add(22)(20)
scala> add(22)(add(1)(19))
\end{REPL}


\Task \emph{Skapa dina egen kontrollstruktur.} Använd fördröjd evaluering och stegad funktion och skapa din egen loop-konstruktion.
\begin{REPL}
scala> def upprepa(n: Int)(block: => Unit) = {
         var i = 0
         while (i < n) {block; i = i + 1}
       }
scala> upprepa(10)(println("hej"))
scala> upprepa(1000){
  val tärning = (math.random * 6 + 1).toInt
  print(tärning + " ")
}
\end{REPL}
Förklara vad som händer ovan. (Det är så som \code{upprepa} i Kojo är definierad.)


\Task \emph{Funktion som värde.} Funktioner är äkta värden i Scala.

\Subtask \label{subtask:funcval} Förklara vad som händer nedan. Notera understrecket på rad 4:

\begin{REPL}[numbers=left, numberstyle=\color{black}\ttfamily\scriptsize\selectfont]
scala> def inc(x: Int): Int = x + 1
scala> inc(42)
scala> Vector(12, 3, 41, -8).map(inc)
scala> val f = inc _
scala> Vector(12, 3, 41, -8).map(f)
\end{REPL}

\Subtask Vad händer om du bara skriver \code{val f = inc} utan understreck?

\Subtask På liknande sätt som i uppgift \ref{subtask:funcval}: definiera en funktion \code{dec} som i stället \emph{minskar} med 1. Deklarera ett funktionsvärde \code{g} som tilldelas funktionen \code{dec} och kör sedan \code{g} på varje element i \code{Vector(12, 3, 41, -8)} med metoden \code{map}.

\Subtask\Pen Vad har variablerna \code{f} och \code{g} ovan för typ?

\Subtask Förklara vad som händer nedan. Vad får \code{d} och \code{h} för värde?

\begin{REPL}
scala> def räkna(x: Int, f: Int => Int) = f(x,y)
scala> def dubbla(x: Int) = 2 * x
scala> def halva(x: Int) = x / 2
scala> val d = räkna(42, dubbla)
scala> val h = räkna(42, halva)
\end{REPL}

\Task\emph{Stegade funktioner (''Curry-funktioner'').} Förklara vad som händer nedan.
\begin{REPL}
scala> def sum(a: Int)(b: Int) = a + b
scala> sum(1)(2)
scala> val f = sum(42) _
scala> f(1)
scala> val inc = sum(1) _
scala> val dec = sum(-1) _
scala> inc(42)
scala> dec(42)
\end{REPL}

\Task \emph{Objekt som moduler.} 

\Subtask Lär dig följande terminologi utantill: 

\begin{itemize}[noitemsep, nolistsep]
\item Ett objekt som samlar funktioner och variabler kallas även en \textbf{modul}. 
\item Funktioner i objekt kallas även \textbf{metoder}. 
\item Variabler och metoder i objekt kallas \textbf{medlemmar}. 
\item Moduler kan i sin tur innehålla moduler, i godtyckligt nästlingsdjup. 
\item Man kommer åt innehållet i en modul med \textbf{punktnotation}. 
\item Med \textbf{import} slipper man punktnotation. 
\item Ett objekt med variabler sägs ha ett \textbf{tillstånd}.
\end{itemize}

\Subtask Deklarera modulerna \code{stringstat} och \code{Test} nedan i REPL eller i Kojo. 

\begin{Code}
object stringstat {
  object stringfun {
    def sentences(s: String): Array[String] = s.split('.')
    def words(s: String): Array[String] = s.split(' ')
    def countWords(s: String): Int = words(s).size
    def countSentences(s: String): Int = sentences(s).size
  }
  
  object statistics {
    var history = ""
    def printFreq(s: String): Unit = {
      println("\n---- Frekvenser ----")
      println("Antal tecken:   " + s.size)
      println("Antal ord:      " + stringfun.countWords(s))
      println("Antal meningar: " + stringfun.countSentences(s))
      history = history + " " + s
    }
    def printTotal: Unit = printFreq(history)
  }
}
  
object Test {
  import stringstat._
  def apply(n: Int = 42): Unit = {
    val s1 = "Fem myror är fler än fyra elefanter. Ät gurka."
    val s2 = "Galaxer i mina braxer. Tomat är gott. Hejsan."
    statistics.printFreq(s1 * n)
    statistics.printFreq(s2 * n)
    statistics.printTotal
  }
}
\end{Code}

\Subtask Anropa \code{Test()} och förklara vad som händer. Vad skrivs ut?

\Subtask Vilket av objekten i modulen \code{stringstat} har tillstånd och vilket av objekten är tillståndslöst? Vad består tillståndet av?


\Task \emph{Äkta funktioner.} En \textbf{äkta funktion} ger alltid samma resultat med samma argument.  

\begin{Code}
object inSearchOfPurity {
  var x = 0
  val y = x
  def inc(i: Int) = i + 1
  def oink(i: Int) = {x = x + i; "Pig says oink " + x}
  def addX(i: Int): Int = x + i
  def addY(i: Int): Int = y + i
  def isPalindrome(s: String): Boolean = s == s.reverse
  def rnd(min: Int, max: Int) = math.random * max + min
}
\end{Code}

\Subtask\Pen Vilka funktioner i objektet \code{inSearchOfPurity} är äkta funktioner?

\Subtask \label{subtask:nonpure} Anropa de funktioner som inte är äkta i REPL och demonstrera med exempel att de kan ge olika resultat för samma argument.

\Subtask Vad objektets tillstånd efter dina körningar i uppgift \ref{subtask:nonpure}? 

\Subtask Vilken del av tillståndet i objektet är oföränderligt?



\Task Funktioner är objekt med en \code{apply}-metod.



\Task \emph{Fördröjd initialisering (''lata'' variabler).} 

\Subtask \label{subtask:delayalloc} Förklara vad som händer här:
\begin{REPL}
scala> val olat = 42
scala> lazy val lat = 42
scala> println(lat)
scala> val nu = {Thread.sleep(1000); println("nu"); 42}
scala> lazy val sen = {Thread.sleep(1000); println("sen"); 42}
scala> def igen = {Thread.sleep(1000); println("hver gang"); 42}
scala> println(nu)
scala> println(sen)
scala> println(igen)
scala> println(nu)
scala> println(sen)
scala> println(igen)
scala> object m {lazy val stor = Array.fill(1e9.toInt)(liten); val liten = 42}
scala> m.liten
scala> m.stor
\end{REPL}

\Subtask\Pen Vad är skillnaden mellan \code{val}, \code{lazy val} och \code{def}, vad gäller \emph{när} evalueringen sker?


\Subtask \label{subtask:forwardref} Förklara vad som händer här:
\begin{REPL}
scala> object objektÄrLata { val sen = { println("nu!"); 42 } }
scala> objektÄrLata
scala> objektÄrLata.sen
scala> {val x = y; val y = 42}
scala> object buggig {val a = b; val b = 42}
scala> buggig.a
scala> object funkar {lazy val a = b; val b = 42}
scala> funkar.a
scala> object nowarning {val many = Array.fill(10)(one); val one = 1}
scala> nowarning.many
\end{REPL}

\Subtask\Pen Med ledning av uppgift \ref{subtask:delayalloc} och uppgift \ref{subtask:forwardref}, beskriv två olika situationer när kan man ha nytta av \code{lazy val}?


\Task \emph{Aktiveringspost.} Antag att vi bara kan addera eller subtrahera med ett. Då kan man ändå skapa en additionsfunktion på nedan (ganska omständliga) sätt. Skriv nedan program i en editor, kompilera och exekvera. 
\begin{Code}
object Count {
  def inc(x: Int) = {println("inc[x = " + x + "]"); x + 1}
  def dec(x: Int) = {println("dec[x = " + x + "]"); x - 1}

  def add(x: Int, y: Int) = {
    println("add[x = " + x + ", y = " + y + "]")
    var result = x; 
    var i = 0; 
    while (i < math.abs(y)){
      result = if (y >= 0) inc(result) else dec(result)
      i = i + 1
    }
    result
  }

  def main(args: Array[String]): Unit = {
    val x =  inc(dec(inc(0)))
    println(x)
    val y = add(1, add(1, add(1, -2)))
    println(y)
  }
}
\end{Code}

\Subtask Vad skrivs ut? Förklara vad som händer.

\Subtask\Pen Rita hur anropsstacken förändras under exekveringen av main-metoden.


\Task \emph{Lokala funktioner.} Skapa nedan program i en editor, kompilera och exekvera. I programmet nedan har metoden \code{add} två lokala funktioner som skiljer sig från metoderna med samma namn. 
\begin{Code}
object Count {
  def inc(x: Int) = x + 1
  def dec(x: Int) = x - 1

  def add(x: Int, y: Int) = {
    def inc(x: Int) = {println("inc[x = " + x + "]"); x + 1}
    def dec(x: Int) = {println("dec[x = " + x + "]"); x - 1}
    println("add[x = " + x + ", y = " + y + "]")
    var result = x; 
    var i = 0; 
    while (i < math.abs(y)){
      result = if (y >= 0) inc(result) else dec(result)
      i = i + 1
    }
    result
  }

  def main(args: Array[String]): Unit = {
    val x =  inc(dec(inc(0)))
    println(x)
    val y = add(1, add(1, add(1, -2)))
    println(y)
  }
}
\end{Code}

\Subtask Vad skrivs ut? Förklara vad som händer.

\Subtask\Pen Vad är fördelen med lokala funktioner?



\Task \emph{Rekursion.} 

\Subtask Förklara vad som händer nedan. 

\begin{REPL}
scala> def countdown(x: Int): Unit = if (x > 0) {println(x); countdown(x -1)}
scala> countdown(10)
scala> countdown(-1)
scala> def finalCountdown(x: Byte): Unit = 
         {println(x); Thread.sleep(100); finalCountdown((x-1).toByte); 1 / x}
scala> finalCountdown(Byte.MaxValue)
\end{REPL}

\Subtask Vad händer om du gör satsen som riskerar division med noll \emph{före} det rekursiva anropet i funktionen \code{finalCountdown} ovan?

\Subtask Förklara vad som händer nedan. Varför tar sista raden längre tid än näst sista raden?
\begin{REPL}
scala> def signum(a: Int): Int = if (a >= 0) 1 else -1 
scala> def add(x: Int, y: Int): Int = 
         if (y == 0) x else add(x + 1, y - signum(y))
scala> add(100,100)
scala> add(Int.MaxValue, 0)
scala> add(0, Int.MaxValue)
\end{REPL}



\ExtraTasks %%%%%%%%%%%%%%%%%%%

\Task Visa anropsstacken genom att kasta undantag.

\AdvancedTasks %%%%%%%%%%%%%%%%%



\Task Kolla bajtkoden.
\begin{REPL}
scala> def plusxy(x: Int, y: Int) = x + y
scala> :javap plusxy
\end{REPL}

\Subtask Leta upp raden \code{public int plusxy(int, int);} och studera koden efter \code{Code:} och försök gissa vilken instruktion som utför själva additionen.

\Subtask Lägg till en parameter till: \\ \code{def plusxyz(x: Int, y: Int, z: Int) = x + y + z}
\\ och studera bajtkoden med \code{:javap plusxyz}. Vad skiljer bajtkoden mellan \code{plusxy} och \code{plusxyz}?

\Subtask\Pen Läs om bajtkod här: \href{https://en.wikipedia.org/wiki/Java\_bytecode}{en.wikipedia.org/wiki/Java\_bytecode}. Vad betyder den inledande bokstaven i additionsinstruktionen?


\Task Undersök svansrekursion genom att kasta undantag. Förklara vad som händer. Kan du hitta bevis för att kompilatorn kan optimera rekursionen till en vanlig loop?

\begin{REPL}
scala> def explode = throw new Exception("BANG!!!")
scala> explode
scala> lastException.printStackTrace
scala> def countdown(n: Int): Unit = 
         if (n == 0) explode else countdown(n-1)
scala> countdown(10)
scala> lastException.printStackTrace
scala> def countdown2(n: Int): Unit = 
         if (n == 0) explode else {countdown2(n-1); print("no tailrec")}
scala> countdown2(10)
scala> countdown2(1000)
scala> lastException
scala> lastException.getStackTrace.size
scala> :javap countdown
scala> :javap countdown2
\end{REPL}

\Task \emph{\code{@tailrec}-annotering.} Du kan be kompilatorn att ge felmeddelande om den inte kan optimera koden till en loop utan aktiveringsposter på stacken. Prova nedan i REPL och förklara vad som händer. 
\begin{REPL}
scala> def countNoTailrec(n: Long): Unit = 
         if (n <= 0L) println("Klar! " + n) else {countNoTailrec(n-1L); ()}
scala> countNoTailrec(1000L)
scala> countNoTailrec(100000L) 
scala> import scala.annotation.tailrec
scala> @tailrec def countNoTailrec(n: Long): Unit = 
         if (n <= 0L) println("Klar! " + n) else {countNoTailrec(n-1L); ()}        
scala> @tailrec def countTailrec(n: Long): Unit = 
         if (n <= 0L) println("Klar! " + n) else countTailrec(n-1L)
scala> countTailrec(1000L)
scala> countTailrec(100000L)
scala> countTailrec(Int.MaxValue.toLong * 2L)
\end{REPL}



