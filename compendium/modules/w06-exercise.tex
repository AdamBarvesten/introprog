%!TEX encoding = UTF-8 Unicode

%!TEX root = ../compendium.tex

\Exercise{\ExeWeekSIX}\label{exe:W06}

\begin{Goals}
%!TEX encoding = UTF-8 Unicode

%!TEX root = ../compendium.tex

\item Kunna deklarera klasser med klassparametrar.
\item Kunna skapa objekt med \code{new} och konstruktorargument.
\item Förstå innebörden av referensvariabler och värdet \code{null}.
\item Förstå innebörden av begreppen instans och referenslikhet.
\item Kunna använda nyckelordet \code{private} för att styra synlighet i primärkonstruktor.
\item Förstå i vilka sammanhang man kan ha nytta av en privat konstruktor.
\item Kunna implementera en klass utifrån en specikation.
\item Förstå skillnaden mellan referenslikhet och strukturlikhet.
\item Känna till hur case-klasser hanterar likhet.
\item Förstå nyttan med att möjliggöra framtida förändring av attributrepresentation.
\item Känna till begreppen getters och setters. 
\item Känna till accessregler för kompanjonsobjekt.
\item Känna till skillnaden mellan \code{==} och \code{eq}, samt \code{!=} versus \code{ne}.
    
\end{Goals}

\begin{Preparations}
\item \StudyTheory{06}
\end{Preparations}

\BasicTasks %%%%%%%%%%%%%%%%

\Task \emph{Instansiering med \code{new} och värdet \code{null}.} Man skapar instanser av klasser med \code{new}. Då anropas konstruktorn och plats reserveras i datorns minne för objektet. Variabler av referenstyp som inte refererar till något objekt har värdet \code{null}. 

\Subtask Vad händer nedan? Vilka rader ger felmeddelande och i så fall hur lyder felmeddelandet?

\begin{REPL}
scala> class Gurka(val vikt: Int)
scala> var g: Gurka = null
scala> g.vikt
scala> g = new Gurka(42)
scala> g.vikt
scala> g = null
scala> g.vikt
\end{REPL}

\Subtask\Pen Rita minnessituationen efter raderna 2, 4, 6.

\Task \emph{Klasser och instanser.} 

\Subtask Vad händer nedan?
\begin{REPL}
scala> :pa
class Arm(val ärTillVänster: Boolean)  
class Ben(val ärTillVänster: Boolean)
class Huvud(val harHår: Boolean)
class Rymdvarelse {
  var arm1 = new Arm(true)
  var arm2 = new Arm(false)
  var ben1 = new Ben(true)
  var ben2 = new Ben(false)
  var huvud1 = new Huvud(false)
  var huvud2 = new Huvud(true)
  def ärSkallig = !huvud1.harHår && !huvud2.harHår
}
scala> val alien = new Rymdvarelse
scala> alien.ärSkallig
scala> val predator = new Rymdvarelse
scala> predator.ärSkallig
scala> predator.huvud2 = alien.huvud1
scala> predator.ärSkallig
\end{REPL}

\Subtask\Pen Rita minnessituationen efter rad 18.

\Subtask\Pen Vad händer så småningom med det ursprungliga huvud2-objektet i predator efter tilldelningen på rad 18? Går det att referera till detta objekt på något sätt?


\Task \emph{Synlighet i primärkonstruktorer.} Undersök nedan vad nyckelorden \code{val} och \code{private} får för konsekvenser. Förklara vad som händer. Vilka rader ger vilka felmeddelanden?

\begin{REPL}
scala> class Gurka1(vikt: Int)
scala> new Gurka1(42).vikt
scala> class Gurka2(val vikt: Int)
scala> new Gurka2(42).vikt
scala> class Gurka3(private val vikt: Int)
scala> new Gurka3(42).vikt
scala> class Gurka4(private val vikt: Int, kompis: Gurka4){
         def kompisVikt = kompis.vikt
       }
scala> val ingenGurka: Gurka4 = null
scala> new Gurka4(42, ingenGurka).kompisVikt
scala> new Gurka4(42, new Gurka4(84, null)).kompisVikt
scala> class Gurka5(private[this] val vikt: Int, kompis: Gurka5){
         def kompisVikt = kompis.vikt
       }
scala> class Gurka6 private (vikt: Int)
scala> new Gurka6(42)
scala> :pa
class Gurka7 private (var vikt: Int)
object Gurka7 { 
  def apply(vikt: Int) = {
    require(vikt >= 0, s"negativ vikt: $vikt")
    new Gurka7(vikt)
  }
}
scala> new Gurka7(-42)
scala> Gurka7(-42)
scala> val g = Gurka7(42)
scala> g.vikt
scala> g.vikt = -1
scala> g.vikt 
\end{REPL}


\Task \emph{Egendefinierad setter kombinerat med privat konstruktor.} 

\Subtask Förklara vad som händer nedan. Vilka rader ger vilka felmeddelanden?
\begin{REPL}
scala> :pa
class Gurka8 private (private var _vikt: Int) {
  def vikt = _vikt
  def vikt_=(v: Int): Unit = {
    require(v >= 0, s"negativ vikt: $v")
    _vikt = v
  }
}

object Gurka8 { 
  def apply(vikt: Int) = {
    require(vikt >= 0, s"negativ vikt: $vikt")
    new Gurka8(vikt)
  }
}
scala> val g = Gurka8(-42)
scala> val g = Gurka8(42)
scala> g.vikt
scala> g.vikt = 0
scala> g.vikt = -1
scala> g.vikt += 42
scala> g.vikt -= 1000
\end{REPL}

\Subtask\Pen Vad är fördelen med möjligheten att skapa egendefienerade setters?

\Task \label{task:Square} \emph{En oföränderlig kvadrat med alternativ fabriksmetod.} 

\Subtask Implementera klassen \code{Square} enligt nedan specifikation. Gör  implementationen i en kodeditor, så som \code{gedit}, och klistra in klassen i Scala REPL efter kommandot \code{:pa} (förkortning av \code{:paste}). På så sätt blir \code{object Square} ett kompanjonsobjekt till \code{class Square}.

\begin{ScalaSpec}{Square}
/** A class representing a square object with position and side. */
class Square(val x: Int, val y: Int, val side: Int) {
  /** The area of this Square */
  val area: Int = ???
  
  /** Creates a new Square moved to position (x + dx, y + dy) */
  def move(dx: Int, dy: Int): Square = ???
  
  /** Tests if this Square has equal size as that Square */
  def isEqualSizeAs(that: Square): Boolean = ???
  
  /** Multiplies the side with factor and rounded to nearest integer */
  def scale(factor: Double): Square = ???
  
  /** A string representation of this Square */
  override def toString: String = ???
}

object Square {
  /** A square placed in origin with size 1 */
  val unit: Square = ??? 
  
  /** Constructs a new Square object at (x, y) with size side */
  def apply(x: Int, y: Int, side: Int): Square = ???

  /** Constructs a new Square object at (0, 0) with side 1 */
  def apply(): Square = ???
}
\end{ScalaSpec}

\Subtask Testa din kvadrat enligt nedan. Förklara vad som händer.

\begin{REPL}
scala> val (s1, s2) = (Square(), Square(1, 10, 1))
scala> val s3 = s1.move(1,-5)
scala> s1 isEqualSizeAs s3
scala> s2 isEqualSizeAs s1
scala> s1 isEqualSizeAs Square.unit
scala> s2.scale(math.Pi) isEqualSizeAs s2
scala> s2.scale(math.Pi) isEqualSizeAs s2.scale(math.Pi)
\end{REPL}

\Task \emph{Referenslikhet versus strukturlikhet.} Metoden \code{==} på case-klasser ger \textbf{strukturlikhet} (även kallad innehållslikhet) så att \emph{innehållet} i klassens klassparametrar jämförs om de har lika värde, medan för vanliga klasser ger metoden \code{==} \textbf{referenslikhet} där olika objekt är olika även om de har samma innehåll (om man inte överskuggar metoden \code{equals} som anropas av \code{==} vilket vi ska titta närmare på i kapitel \ref{chapter:W08}).

\begin{REPL}
scala> class GurkaRef(val vikt: Int)
scala> case class GurkaStrukt(val vikt: Int)
scala> val a = new GurkaRef(42)
scala> val b = new GurkaRef(42)
scala> val c = new GurkaStrukt(42)
scala> val d = new GurkaStrukt(42)
scala> a == b
scala> c == d 
\end{REPL}

\Subtask Förklara vad som händer ovan.

\Subtask Istället för \code{==}, prova metoden \code{eq} på objekten ovan. Metoden \code{eq} ger alltid referenslikhet (även om byter ut metoden \code{equals}).

\Task \label{task:Point} \emph{Klassen \code{Point} med case-klass.} 

\Subtask Implementera klassen \code{Point} som en oföränderlig case-klass med heltalsattributen \code{x} och \code{y}. 

\Subtask Lägg till metoden \code{distanceTo(that: Point): Double} som räknar ut avståndet till en annan punkt med hjälp av \code{math.hypot}.

\Subtask Lägg till metoden \code{distanceTo(x: Int, y: Int): Double} som räknar ut avståndet till koordinaterna x och y med hjälpa av metoden i föregående deluppgift.

\Subtask Lägg till metoden \code{move(dx: Int, dy: Int): Point} som skapar en ny punkt på translaterad position enligt delta-koordinaterna \code{dx} och {dy}.

\Subtask Lägg till ett kompanjonsobjekt med medlemmen \code{val origin} som ger en punkt i origo.

\Subtask Undersök metoderna \code{==}, \code{!=}, \code{eq} och \code{ne} och förklara vad som händer nedan:
\begin{REPL}
scala> Point(1, 2) == Point(1, 3)
scala> Point(1, 2) != Point(1, 3)
scala> Point(1, 2) == Point(1, 2)
scala> Point(1, 2) != Point(1, 2)
scala> Point.origin.move(1, 1) == Point.origin.move(1, 1)
scala> Point.origin.move(1, 1).move(1, 1) != Point(2, 2)
scala> Point(0, 0) eq Point(0, 0)
scala> Point(0, 0) ne Point(0, 0)
scala> Point.origin eq Point.origin
scala> Point.origin ne Point.origin
scala> val p1 = Point(0, 0)
scala> val p2 = p1
scala> p1 eq p2
\end{REPL}

\Subtask Vad ger \code{Point.origin eq Point.origin} för resultat om \code{origin} istället  implementeras som \code{def origin: Point = Point(0, 0)}

\Subtask\Pen Vad är det för skillnad på strukturlikhet och referenslikhet?

\Task \label{task:PointSquare}Ändra representationen av positionen i klassen \code{Square} från deluppgift \ref{task:Square} till att vara en \code{Point} från deluppgift \ref{task:Point}.


\Task \label{task:PointTuple} \emph{Case-klassen \code{Point} med 2-tupel.} I ett utvecklingsprojekt vill man ändra representationen av positionen i den gamla klassen  \\ \code{case class Point(x: Int, y: Int)} så att positionen istället i den uppdaterade klassen representeras av en 2-tupel. Man kan då vid konstruktion utnyttja att n-tupler som parameter även kan skrivas som en parameterlista med n argument, varför både \code{Point(1,2)} och \code{Point((1,2))} fungerar fint. Samtidigt vill man att befintlig kod som fortfarande använder \code{x} och \code{y} ska fungera utan ändringar.  Implementera den nya \code{Point} enligt specifikationen nedan. 
\begin{ScalaSpec}{Point}
/** A 2-dimensional immutable position p in an integer coordinate system */ 
case class Point(p:(Int, Int)) {
  /** The x-axis position of this Point */
  val x: Int = ???

  /** The y-axis position of this Point */
  val y: Int = ???

  /** The distance to another Point that */
  def distanceTo(that: Point): Double = ???

  /** The distance to another 2-tuple that representing (x, y). */
  def distanceTo(that: (Int, Int)): Double = ???

  /** A new Point that is moved (dx, dy) */
  def move(dxdy: (Int, Int)): Point = ???
}

object Point {
  /** A Point object at position (0, 0) */ 
  val origin: Point = ???
}
\end{ScalaSpec}

\Task\Pen Vad behöver du ändra i klassen \code{Square} från uppgift \ref{task:PointSquare} för att den ska fungera med en \code{Point} med 2-tupel från uppgift \ref{task:PointTuple}?

\Task \emph{Objekt med föränderligt tillstång \Eng{mutable state}.} Du ska implementera en modell av en hoppande groda som uppfyller följande krav: 
\begin{enumerate}[nolistsep, noitemsep]
\item Varje grodobjekt ska hålla reda på var den är.
\item Varje grodobjekt ska hålla reda på hur långt grodan hoppat totalt.
\item Varje grodobjekt ska kunna beräkna hur långt det är mellan grodans nuvarande position och utgångsläget.
\item Alla grodor börjar sitt hoppande i origo.
\item En groda kan hoppa enligt två metoder: 
  \begin{itemize} [nolistsep, noitemsep]
  \item relativ förflyttning enligt parametrarna \code{dx} och \code{dy},  
  \item slumpmässig förflyttning $[1, 10]$ i x-led och $[1, 10]$ i y-led. 
  \end{itemize}
\end{enumerate}

\Subtask Implementera klassen \code{Frog} enligt nedan specifikation och ovan krav. \\  \emph{Tips:}
  \begin{itemize} [nolistsep, noitemsep]
  \item Om namnet man vill ge ett privat föränderligt attribut ''krockar'' med ett metodnamn, är det vanligt att man börjar attributets namn med understreck, t.ex. \code{private var _x } för att på så sätt undkomma namnkonflikten. 
  \item Inför en metod i taget och klistra in den nya grodan i REPL efter varje utvidgning och testa. 
  \end{itemize}

\begin{ScalaSpec}{Frog}
class Frog private (initX: Int = 0, initY: Int = 0) {
  def jump(dx: Int, dy: Int): Unit = ???
  def x: Int = ??? 
  def y: Int = ???
  def randomJump: Unit = ???
  def distanceToStart: Double = ???
  def distanceJumped: Double = ???
  def distanceTo(that: Frog): Double = ???
}
object Frog {
  def spawn(): Frog = ???
}
\end{ScalaSpec}

\Subtask Skriv ett testhuvudprogram som kontrollerar så att alla krav är uppfyllda och att alla metoder fungerar som de ska. 

\Subtask\Pen Vad kallas en metod som enbart returnerar värdet av ett privat attribut?

\Subtask\Pen Hur kan man från en metods signatur få en ledtråd om att ett objekt har föränderligt tillstånd \Eng{mutable state}?

\Subtask Inför setters för attributen som håller reda på x- och y-postitionen. Förändringar av positionen i x- eller y-led ska räknas som ett hopp och alltså registreras i det attribut som håller reda på det ackumulerade hoppavståndet.

\Subtask Simulera ett massivt grodhoppande med krockdetektering genom att skapa 100 grodor som till att börja med är placerade på x-axeln med avståndet $8$ längdenheter mellan sig. Låt grodorna i en \code{while}-sats hoppa slumpmässigt tills någon groda befinner sig närmare än $0.5$ längdenheter som är definitionen på att de har krockat. Räkna hur många looprundor som behövs innan något grodpar krockar och skriv ut antalet. \\ \emph{Tips:} Börja med pseudokod på papper. Använd en grodvektor.

\clearpage

\ExtraTasks %%%%%%%%%%%%%%%%%%%

\Task \emph{En kvadratklass med föränderligt tillstånd \Eng{mutable state}.} Webbshoppen UberSquare säljer flyttbara kvadrater. I affärsmodellen ingår att ta betalt per förflytttning. Du ska hjälpa UberSquare med att utveckla en enkel systemprototyp. 

\Subtask Implementera \code{Square} enligt nedan specifikation, under uppfyllandet av följande krav:

\begin{enumerate}[nolistsep, noitemsep]
\item Till skillnad från uppgift \ref{task:Square} ska du nu göra en kvadrat med föränderligt tillstånd \Eng{mutable state}. I stället för att vid förflyttning returnera ett nytt kvadratobjekt, returneras \code{Unit} i samband med att privata attribut uppdateras.
\item Du ska införa funktionalitet som räknar antalet förflyttningar som gjorts för varje kvadrat som skapats och även räkna ut det totala antalet förflyttningar som någonsin gjorts.
\item Varje gång förflyttning sker adderas en kostnad till den ackumulerade kostnaden för respektive kvadrat. Kostnaden för varje förflyttning är avståndet till ursprungsläget multiplicerat med storleken på kvadraten.
\end{enumerate}

\begin{ScalaSpec}{Square}
/** A mutable and expensive Square. */
class Square private (val initX: Int, val initY: Int, val initSide: Int) {
  
  private var nMoves = 0;
  private var sumCost = 0.0;
  private var _x = initX;
  private var _y = initY;
  private var _side = initSide;
  
  private def addCost: Unit = { 
   sumCost += ???
  }
  
  /** The current position on the x axis */
  def x: Int = ???

  /** The current position on the y axis */
  def y: Int = ???

  /** The size of this Square */
  def side = ???

  /** Scales the side of this square and rounds it to nearest integer */
  def scale(factor: Double): Unit = ???

  /** Moves this square to position (x + xd, y + dy) */
  def move(dx: Int, dy: Int): Unit = ???

  /** Moves this square to position (x, y) */
  def moveTo(x: Int, y: Int): Unit = ???
  
  /** The accumulated cost of this Square */
  def cost: Double = ???

  /** Reset the cost of this Square */
  def pay: Unit = ???
  
  /** A string representation of this Square */
  override def toString: String = 
    s"Square[($x, $y), side: $side, #moves: $nMoves times, cost: $sumCost]"
}

object Square {
  private var created = Vector[Square]()
 
  /** Constructs a new Square object at (x, y) with size side */
  def apply(x: Int, y: Int, side: Int): Square = {
    require(side >= 0, s"side must be positive: $side")
    ???
  }

  /** Constructs a new Square object at (0, 0) with side 1 */
  def apply(): Square = apply(0, 0, 1)

  /** The total number of moves that have been made for all squares. */
  def totalNumberOfMoves: Int = ???
  
  /** The total cost of all squares. */  
  def totalCost: Double = ???
} 
\end{ScalaSpec}

\Subtask Testa din kvadratprototyp i REPL enligt nedan:
\begin{REPL}
scala> val xs = Vector.fill(10)(Square())
scala> xs.foreach(_.move(2,3))
scala> xs.foreach(_.scale(2.9))
scala> val (m, c) = (Square.totalNumberOfMoves, Square.totalCost)
m: Int = 10
c: Double = 36.055512754639885
\end{REPL}


\clearpage

\AdvancedTasks %%%%%%%%%%%%%%%%%


\Task \label{task:aux-constructor} \emph{Hjälpkonstruktor.} I uppgift \ref{task:Square} erbjöds ett alternativt sätt att skapa \code{Square} med en extra fabriksmetod med namnet \code{apply} i kompanjonsobjektet. Ett annat sätt att göras detta på, som i Scala är mindre vanligt (men i Java är desto vanligare), är att definiera flera konstruktorer innuti klassen. I Scala kallas en sådan extra konstruktor för \textbf{hjälpkonstruktor} \Eng{auxiliary constructor}. 

En hjälpkonstruktor skapar man i Scala genom att definiera en metod som har det speciella namnet \code{this}, alltså en deklaration \code{def this(...) = ...} Hjälponstruktorer måste börja med att anropa en annan konstruktor, antingen den primära konstruktorn eller en tidigare definierad  hjälpkonstruktor.  

\Subtask Läs mer om hjälpkonstruktorer här: \\ \href{http://www.artima.com/pins1ed/functional-objects.html#6.7}{www.artima.com/pins1ed/functional-objects.html\#6.7}

\Subtask Hitta på en egen uppgift med hjälpkonstruktorer, baserat på någon av klasserna i tidigare övningar.


%\Task \TODO \\ \code{class Rational private (numerator: BigInt, denominator: BigInt)} \\
%Inspirerat av Rational i pins1ed med GCD
    
