%!TEX encoding = UTF-8 Unicode

%!TEX root = ../compendium.tex

\Exercise{\ExeWeekTEN}

\begin{Goals}
\item 
\end{Goals}

\begin{Preparations}
\item 
\end{Preparations}

\BasicTasks %%%%%%%%%%%%%%%%

\Task \emph{Jämföra strängar i Scala.} I Scala kan strängar jämföras med operatorerna \code{< <= > >= == !=}  där likhet/olikhet avgörs av om alla tecken i strängen är lika eller inte, medan större/mindre avgörs av sorteringsordningen i enlighet med varje teckens Unicode\footnote{\href{https://sv.wikipedia.org/wiki/Unicode}{sv.wikipedia.org/wiki/Unicode}}-värde. 

\Subtask Vad ger följande jämförelser för värde?
\begin{REPL}
scala> 'a' < 'b'
scala> "aaa" < "aaaa"
scala> "aaa" < "bbb"
scala> "AAA" < "aaa"
scala> "ÄÄÄ" < "ÖÖÖ"
scala> "ÅÅÅ" < "ÄÄÄ"
\end{REPL}
Tyvärr så följer ordningen av ÅÄÖ inte svenska regler; om du är intresserad av hur man kan fixa  detta, gör uppgift \ref{task:swedish-letter-ordering}. 

\Subtask\Pen Vilken av strängarna $s1$ och $s2$ kommer först (d.v.s. är mindre) om $s1$ utgör början av $s2$ och $s2$ innehåller fler tecken än $s1$? 


\Task \emph{Jämföra strängar i Java.} I Java kan man \emph{inte} jämföra strängar med operatorerna \code{< <= > >=}. Dessutom ger operatorerna \code{==} och \code{!=} inte innehålls(o)likhet utan referens(o)likhet. Istället får man använda metoderna \code{equals} och \code{comareTo}, vilka också fungerar i Scala eftersom strrängar i Scala är av samma typ som i Java, nämligen \code{}.


\Subtask Vad ger följande uttryck för värde?

\begin{REPL}
scala> "hej".getClass.getTypeName
scala> "hej".equals("hej")
scala> "hej".compareTo("hej")
\end{REPL}


\Subtask Studera dokumentationen för metoden \code{compareTo} i \code{java.lang.String}\footnote{\href{https://docs.oracle.com/javase/8/docs/api/java/lang/String.html\#compareTo-java.lang.String-}{docs.oracle.com/javase/8/docs/api/java/lang/String.html\#compareTo-java.lang.String-}} och skriv minst 3 olika uttryck i Scala REPL som testar hur metoden fungerar i olika fall. 

\Subtask Studera dokumentationen \code{compareToIgnoreCase} \footnote{\href{https://docs.oracle.com/javase/8/docs/api/java/lang/String.html\#compareToIgnoreCase-java.lang.String-}{docs.oracle.com/javase/8/docs/api/java/lang/String.html\#compareToIgnoreCase-java.lang.String-}} och skriv minst 3 olika stränguttryck i Scala REPL som testar hur metoden fungerar i olika fall. 

\Subtask Vad skriver följande Java-program ut?
\javainputlisting{examples/StringEqTest.java}

\Task \emph{Sökning med inbyggda sökmetoder.} TODO!!!

\Subtask \emph{Linjärsökning framifrån med \code{indexOfSlice}}. Studera dokumentationen för Scalas samlingsmetod \code{indexOfSlice}\footnote{\href{http://docs.scala-lang.org/overviews/collections/seqs.html}{docs.scala-lang.org/overviews/collections/seqs.html}} och skriv 8 olika uttryck som, både med en sträng och med en vektor med heltal, provar 4 olika fall: (1) finns i börja, (2) finns någonstans i mitten, (3) finns i slutet, samt (4) finns ej.

\Subtask \emph{Linjärsökning bakifrån med \code{lastIndexOfSlice}}. Studera dokumentationen för Scalas samlingsmetod \code{lastIndexOfSlice}\footnote{\href{http://docs.scala-lang.org/overviews/collections/seqs.html}{docs.scala-lang.org/overviews/collections/seqs.html}} och skriv 8 olika uttryck som, både med en sträng och med en vektor med heltal, provar 4 olika fall: (1) finns i börja, (2) finns någonstans i mitten, (3) finns i slutet, samt (4) finns ej.

\Subtask \emph{Sökning med inbyggd binärsökning.} Om en samling är sorterad kan man utnyttja detta för att göra snabbare sökning. Vid \textbf{binärsökning} \Eng{binary search}\footnote{\href{https://en.wikipedia.org/wiki/Binary_search_algorithm}{en.wikipedia.org/wiki/Binary\_search\_algorithm}} börjar man på mitten och kollar vilken halva att  söka vidare i, sedan delar man upp denna halva på mitten och kollar vilken fjärdedel att söka vidare i etc. 

I objektet \code{scala.collection.Searching}\footnote{\href{http://www.scala-lang.org/api/current/\#scala.collection.Searching$}{http://www.scala-lang.org/api/current/\#scala.collection.Searching\$}} finns en metod \code{search} som, om den importeras, erbjuder binärsökning för alla sekvenssamlingar. Om samlingen är sorterad ger den ett objekt av case-klassen \code{Found} som innehåller indexet med platsen; alternativt om det som eftersöks ej finns, ges ett objekt av case-klassen \code{InsertionPoint} som innehåller indexet där den borde ha varit placerad om den fanns. Observera att om samlingen inte är sorterad är resultatet ''odefinierat'', d.v.s. något heltal returneras men det är inte att lita på; man måste alltså först sortera samlingen eller vara helt säker på att den är sorterad. 

Undersök hur \code{search} fungerar genom att förklara vad som händer nedan.

\begin{REPL}
scala> val xs = Vector(TODO!!!
scala> xs == xs.sorted
\end{REPL}



\Task \emph{Sortering med inbyggda sorteringsmetoder.} TODO!!!

\Subtask \emph{\code{sorted}}

\Subtask \emph{\code{sortWith}}

\Subtask \emph{\code{sortBy}}


\Task \emph{Sök bland LTH:s kurser med linjärsökning} 

\Subtask Surfa till denna URL:\\ {\nolinebreak[4]\footnotesize\url{http://kurser.lth.se/lot/?lasar=16_17&soek_text=&sort=kod&val=kurs&soek=t}}
\\
och inspektera html-koden i din webbläsare genom att trycka \emph{Ctrl+U} (fungerar i Firefox och Chrome). Rulla ner till rad 171 och framåt. Var finns antalet poäng för resp kurs i html-koden?

\Subtask \label{subtask:download-lthcourses} Klistra in objektet \code{courses} med kommandot \code{:paste} i REPL.\footnote{Du kan ladda ner koden från: \\ \href{https://raw.githubusercontent.com/lunduniversity/introprog/master/compendium/examples/lth-courses/courses.scala}{github.com/lunduniversity/introprog/tree/master/compendium/examples/lth-courses/courses.scala}} Vad gör koden?

\scalainputlisting{examples/lth-courses/courses.scala}

\Subtask \emph{Linjärsökning med find.} Teknologen Oddput Clementina vill gå första bästa datavetenskapskurs som är på G2-nivå. Hjälp Oddput med att söka upp första bästa kurs genom linjärsökning med samlingsmetoden \code{find}. Kurskoder vid datavetenskap börjar på EDA eller ETS. \emph{Tips:} Du har nytta av att definiera predikatet \code{def isCS(s: String): Boolean} som i sin tur lämpligen nyttjar strängmetoden \code{startsWith}.

\begin{REPL}
// kod till facit
scala> def isCS(s: String) = s.startsWith("EDA") || s.startsWith("ETS")
scala> val x = courses.lth2016.find(c => isCS(c.code) && c.level == "G2").get
x: courses.Course = Course(EDA031,C++ - programmering,C++ Programming,7.5,G2)
\end{REPL}

\Subtask \emph{Implementera linjärsökning.} Som träning ska du nu implementera en egen linjärsökningsfunktion med signaturen: \\ \code{def linearSearch[T](xs: Seq[T])(p: T => Boolean): Int = ???}
\\ Funktionen ska ta en sekvenssamling \code{xs} och ett predikat \code{p} och returnera index för första hittade elementet i \code{xs} där \code{p} gäller. Om det inte finns något element som uppfyller predikatet ska -1 returneras. Skriv först pseudokod för funktionen med penna och papper. Använd \code{while}. 



\Subtask Definiera en funktion \code{def rndCode: String} som genererar slumpmässiga kurskoder (tre bokstäver mellan A och Z och tre siffror mellan 0 och 9). Använd \code{rndCode} för att fylla en vektor kallad \code{xs} med en halv miljon slumpmässiga kurskoder. För varje slumpkod i \code{xs} sök med din funktion \code{linearSearch} efter index i vektorn \code{courses.lth2016} från deluppgift \ref{subtask:download-lthcourses}. Mät totala tiden för de $500000$ linjärsökningarna med hjälp av funktionen \code{time} nedan.
\begin{Code}
def time[T](code: => T): (T, Long) = {
  val now = System.currentTimeMillis
  val result = code
  val elapsed = System.currentTimeMillis - now
  println(s"time: $elapsed ms") 
  (result, elapsed)
}
\end{Code}
 

\begin{Code}
// kod till facit
def rndCode: String = {
  import math.{random => r}
  def ch: Char = ((r * ('Z' - 'A')).toInt + 'A').toChar
  def dig: Int   = (r * 9).toInt + 1
  val xs = ((1 to 3).map(_ => ch) ++ (1 to 3).map(_ => dig))
  xs.mkString
}

val xs = Vector.fill(500000)(rndCode)
val (ix, elapsed) = 
  time{xs.map(x => linearSearch(courses.lth2016)(_.code == x))}
ix.filterNot(_ == -1).size
\end{Code}



\Subtask Hur kan du implementera \code{linearSearch} med den inbyggda samlingsmetoden \code{indexWhere}?

\begin{Code}
// kod till facit
def linearSearch[T](xs: Seq[T])(p: T => Boolean): Int = 
  xs.indexWhere(p)
\end{Code}



\Task \emph{Sök bland LTH:s kurser med binärsökning.} 



\Subtask Implementera binärsökning. TODO!!!



\Task \emph{Linjärsökning i Java.} Denna uppgift bygger vidare på uppgift \ref{task:arraymatrix-java} i kapitel \ref{chapter:W09}. Du ska göra en variant på linjärsökning som innebär att leta upp första yatzy-raden i en matris där varje rad innehåller utfallet av 5 tärningskast. 

\Subtask Lägg till metoderna \code{isYatzy} och \code{findFirstYatzyRow} klassen \code{ArrayMatrix} i uppgift \ref{task:arraymatrix-java} i kapitel \ref{chapter:W09} enligt nedan. Börja med metoden  \code{isYatzy}. Det finns en bug i koden nedan -- rätta bugen och testa så att metoden fungerar. 

\begin{Code}[language=Java]
    public static boolean isYatzy(int[] dice){
        int col = 1;
        boolean allSimilar = true;
        while (col < dice.length && allSimilar) {
          allSimilar = dice[0] == dice[col];
        }
        return allSimilar;
    }
    
    /** Finds first yatzy row in m; returns -1 if not found */
    public static int findFirstYatzyRow(int[][] m, int n){
        int row = 0;
        int result = -1;
        while (???) {
             /* linjärsökning  */ 
        }
        return result;
    }
\end{Code}

\Subtask Implementera sedan \code{findFirstYatzyRow}. Skapa först pseudo-kod för länjärsökningsalgoritmen innan du skriver implementationen i Java. 
Testa ditt program genom att lägga till följande rader i huvudprogrammet:
\begin{Code}[language=Java]
        int[][] yss = new int[2500][5];
        fillRnd(yss, 6);
        int i = findFirstYatzyRow(yss);
        System.out.println("First Yatzy Index: " + i);
\end{Code}




\begin{Code}[language=Java]
// kod till facit
    public static boolean isYatzy(int[] dice){
        int col = 1;
        boolean allSimilar = true;
        while (col < dice.length && allSimilar) {
          allSimilar = dice[0] == dice[col];
          col++;
        }
        return allSimilar;
    }
    
    public static int findFirstYatzyRow(int[][] m){
        int row = 0;
        int result = -1;
        while (row < m.length && result < 0){
            if (isYatzy(m[row])) {
              result = row;
            } else {
              row = row + 1;    
            }
        }
        return result;
    }
\end{Code}



\Task Implementera sortering till ny Vector med instickssortering. TODO!!!


\Task Implementera sortering på plats i en Array med urvalssortering. TODO!!!



\ExtraTasks %%%%%%%%%%%%%%%%%%%


\Task Implementera sortering på plats i en Array med instickssortering. TODO!!!


\Task \label{task:swedish-letter-ordering}Implementera sortering till ny Vector med urvalssortering. TODO!!!


\AdvancedTasks %%%%%%%%%%%%%%%%%

\Task  Ordering och Order TODO!!!

\Task java.util.Arrays.binarySearch  TODO!!! \\ \url{https://docs.oracle.com/javase/8/docs/api/java/util/Arrays.html}

\Task java.util.Arrays.sort  TODO!!!

\Task java.util.Comparator  TODO!!!

\Task Svenska bokstäver TODO!!

\url{http://stackoverflow.com/questions/24860138/sort-list-of-string-with-localization-in-scala} 

\begin{REPL}
scala> val fel = Vector("ö","å","ä","z").sorted
fel: scala.collection.immutable.Vector[String] = Vector(z, ä, å, ö)

scala val svColl = java.text.Collator.getInstance(new java.util.Locale("sv"))

scala> val svOrd = Ordering.comparatorToOrdering(svColl)
svOrd: scala.math.Ordering[Object] = scala.math.LowPriorityOrderingImplicits$$anon$7@64dbfba5

scala> val rätt = Vector("ö","å","ä","z").sorted(svOrd)
rätt: scala.collection.immutable.Vector[String] = Vector(z, å, ä, ö)
\end{REPL}

    