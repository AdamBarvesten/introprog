%!TEX encoding = UTF-8 Unicode

%!TEX root = ../compendium.tex

\Lab{\LabWeekTEN}

\begin{Goals}
\item Implementera sorteringsalgoritmerna insättnings- (insertion) och urvalssortering (selection) i Java.
\item Analysera hur algoritmernas komplexitet påverkar exekveringstiden.
\item In- och utläsning av fil på disk.
%\item Implementera sökalgoritmerna linear-search och binary-search.
%\item Serialisera data för transport och lagring.
\end{Goals}

\begin{Preparations}
\item Svara på enkäten: % TODO
\item \DoExercise{\ExeWeekTWO}{02}
\item Veckans övningar, främst 1-3, 9-13.
\item Uppgift 3 från övningen Matrices, vecka 9.%Uppgift 6 också?
\item Om det behövs, läs om eller kolla på de grafiska exemplen av insättningssortering: \url{https://en.wikipedia.org/wiki/Insertion_sort} och urvalssortering: \url{https://en.wikipedia.org/wiki/Selection_sort}
\end{Preparations}

\subsection{Om Labben}

Veckans labb handlar om sortering och statistisk registrering av data. Del 1 handlar om sorteringsalgoritmerna insättnings- och urvalssortering, vilka kommer implementeras i Java. Del 2 är en uttökning av Scalaklassen \code{Table} från uppgift 3, övning Matrices. Utgående från det som implementerades på övningen ska metoder för sortering och registrering med avseende på en viss kolumn läggas till. Rådatan som kommer behandlas i del 2 är enkätsvaren från förberedelseuppgifterna och är av typen CSV (comma-separated values) och ser ut något såhär:
%formattering??
\begin{Code}
rad1kolumn1,rad1kolumn2,rad1kolumn3,rad1kolumn4,rad1kolumn5
rad2kolumn1,rad2kolumn2,rad2kolumn3,rad2kolumn4,rad2kolumn5
rad3kolumn1,rad3kolumn2,rad3kolumn3,rad3kolumn4,rad3kolumn5
...
\end{Code}

\subsection{Kodstruktur}
Koden är uppdelad i följande delar:

\begin{REPL}
src # tree .
.
└── main
    ├── java
    │   ├── GenericInsertionSort.java
    │   ├── GenericSelectionSort.java
    │   ├── InsertionSort.java
    │   └── SelectionSort.java
    └── scala
        ├── StringMatrix.scala
        ├── StringSort.scala
        └── Table.scala
.
src
 `-- main
    |-- java
    |   |-- GenericInsertionSort.java
    |   |-- GenericSelectionSort.java
    |   |-- InsertionSort.java
    |   `-- SelectionSort.java
    `-- scala
        |-- StringMatrix.scala
        |-- StringSort.scala
        `-- Table.scala
\end{REPL}

\begin{itemize}
\item Klassen \code{Table} är datastrukturen som har hand om datan och tillhandahåller de nödvändiga metoderna som kommer användas i implementeringen av sök-, registrering och sorteringsalgoritmerna.
\item OSV
\end{itemize}

\subsection{Given kod}


\subsection{Obligatoriska uppgifter}

\Task Implementera datastrukturen

\Subtask En underuppgift.

\Subtask En underuppgift.

\subsection{Frivilliga extrauppgifter}

\Task Implementera sorteringsfunktionen generellt

\Subtask En underuppgift.

\Subtask En underuppgift.
    