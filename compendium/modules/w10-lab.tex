%!TEX encoding = UTF-8 Unicode

%!TEX root = ../compendium.tex

\Lab{\LabWeekTEN}

\begin{Goals}
\item Kunna använda inbyggda sorteringsfunktioner
\item Kunna använda inbyggda sökfunktioner.
\item Känna till hur strängar ordnas.
\item Kunna läsa text i tabellform från fil och webbaddress.
\item Kunna använda registrering (frekvensräkning) för enkla statistikberäkningar.
\item Kunna skriva till fil.
\item Kunna omvandla startargument till kommandon.
\item ... \TODO mer här
\end{Goals}

\begin{Preparations}
\item \StudyTheory{10}
\item \DoExercise{\ExeWeekTEN}{10}
\item \ReadTheLab
\item Fyll i denna enkät: \url{https://goo.gl/forms/hC6JK2UQXVpbGECc2}  \\
I enkäten ska du svarar på frågan: \textit{Vilket är ditt favoritalternativ?} \\
gällande följande områden och svarsalternativen:
\begin{itemize}[nolistsep,noitemsep]
\item \textbf{lthprogram} (D, W, C, E, F, I, Bio, K, L, M, Bme, Nano, V), 
\item \textbf{os} (Win7, Win10, macOS X, Linux, Android, IOS, ChromeOS), 
\item \textbf{editor} (gedit, vim, emacs, vi, notepad++, sublime text, atom)
\item \textbf{ide} (Eclipse, IntelliJ/AndroidStudio, VisualStudio, xcode), 
\item \textbf{socialnät} (facebook, snapchat, linkedin, instagram, github), 
\item \textbf{webbläsare} (firefox, chrome, safari, edge, vivaldi, opera)
\item \textbf{sorteringsalgoritm} (insättningssortering, urvalssortering)
\item \textbf{språk} (Java, Python, PHP, C\#, Javascript, C++, C, Objective-C, R, Swift, Matlab, Ruby, Visual Basic, VBA, Scala, Perl, lua, Delphi)  \\
Listan enligt \url{http://pypl.github.io/PYPL.html} i Aug 2016
% Alternativet är TIOBE, men den är längre...:
%(Java, C, C++, C\#, Python, PHP, Javascript, Visual Basic, Perl, Pascal, Ruby, Swift, Groovy, R, Matlab, SQL, Go, Dart, Fortram, Lua, Ada, Lisp, Scala, Prolog, Haskell, Erlang, Rust)
\end{itemize}
\end{Preparations}


\subsection{Bakgrund}

I den här veckans laboration ska du utveckla ett program som analyserar svar på enkäter med flervalsfrågor. Indata utgörs av text i form av \textbf{kolumnseparerade värden}, där varje persons svar finns på en egen rad och varje svarsrad innehåller svarsalternativ separerade med en \textbf{kolumnseparator} som till exempel kan vara \code{\t} eller \code{,}. Första raden i textfilen anger kolumnernas namn.

Exempel på indatafil:%\footnote{\TODO gör indataexemplet lite längre så analysexempel blir lite roligare och låt indatafilen finnas med som testfil i workspace}
\begin{CodeSmall}[language=, ]
Program,Indent,UI,Lang,OS,DevEnv
D,Spaces,Terminal,C,BSD,Emacs
C,Spaces,Terminal,Javascript,Windows 7,notepad++
D,Spaces,GUI,Java,macOS,Gedit
I,Tabs,Terminal,PHP,Windows 10,notepad++
C,Spaces,GUI,Java,Windows 8,Eclipse
D,Spaces,Terminal,Java,Windows 8,Eclipse
F,Spaces,Terminal,C,Linux,Emacs
D,Spaces,GUI,C,Linux,Vim
Nano,Tabs,Terminal,Javascript,macOS,Vim
C,Tabs,Terminal,C#,Windows 10,Visual Studio
D,Tabs,GUI,Javascript,macOS,Emacs
D,Spaces,GUI,Python,Windows 7,notepad++
E,Spaces,Terminal,Java,Linux,Eclipse
I,Tabs,Terminal,Python,Windows 10,notepad++
K,Tabs,GUI,C#,Windows 7,Visual Studio
F,Spaces,Terminal,C,Linux,Vim
D,Tabs,GUI,C,Linux,Gedit
\end{CodeSmall}

Ditt program ska bestå av följande delar:
\begin{itemize}
\item En case-klass för strängmatriser som heter \code{Table} med funktioner för inläsning av tabellformatterad text, sortering, filtrering och registrering med avseende på en viss kolumn.
\item Ett objekt för argumentparsning och kommandoexekvering som heter \code{Command}. Ett \code{Command} är också en egen typ: en funktion som tar emot ett \code{Table}, gör något med det och returnerar tillbaka det. På så vis kan flera kommandon exekveras som en kedja där varje kommando tar emot det föregående kommandots utdata som indata.
\item Funktioner för att presentera statistik från enkätdata med hjälp av registrering.
\end{itemize}

Utöver \code{Table} och \code{Command} finns I mappen resources exempeldatan från tidigare i filen favorit.csv, använd den när du testar ditt program.
Main är det som kör det slutgiltiga programmet, kolla på den för att få bättre uppfattning om hur programmet är uppbyggt.

\scalainputlisting{../workspace/w10_survey/src/main/scala/stats/Main.scala}



\subsection{Obligatoriska uppgifter}

\Task Implementera \code{Table} enligt specifikation:

\ScalaSpecInputListing{Table}{../workspace/w10_survey/src/main/scala/stats/Table.scala}

\Subtask Implementera klassen \code{Table} förutom \code{register}. I \code{sort} och \code{filter} ska inbyggda funktioner för sortering och filtrering användas. Eftersom \code{Table} är omuterbar måste en förändrad kopia av table skapas för att . Fundera på hur det ska gå till.

\Subtask Implementera \code{register}. Fundera på huruvida kolumnrubriken ska vara med i returdatan. Funktionen kommer senare användas för att skapa diagram som presenterar den registrerade datan. Om du fastnar: \footnote{Läs om funktionen groupBy \code{groupBy}.}.

\Subtask Implementera objektet \code{Table}. Funktionen \code{fromFile} ska kunna ta emot antingen en webbadress eller en lokal sökväg. I ditt program räcker det med att undersöka ifall \code{uri} börjar på \code{"http"}.

Datan läses in till vektorn på detta vis:
\begin{CodeSmall}[language=, ]
Vector(Vector(rad1kolumn1,rad1kolumn2,rad1kolumn3),
       Vector(rad2kolumn1,rad2kolumn2,rad2kolumn3),
       Vector(rad3kolumn1,rad3kolumn2,rad3kolumn3))
\end{CodeSmall}
I \code{toFile} kan valfritt paketet användas för att skriva till fil.

\Subtask Testa att köra \code{Table}. Borde ge utskriften (inklusive kolumnrubrik och totala antalet på första raden):
\begin{REPLnonum}
(DevEnv,5)
(Vim,2)
(Emacs,1)
(Gedit,1)
(Eclipse,1)
\end{REPLnonum}

\TODO ok med footnotes tips?



\Task Komplettera \code{Command} enligt specifikation:

\ScalaSpecInputListing{Command}{../workspace/w10_survey/src/main/scala/stats/Command.scala}

\Subtask Implementera \code{parseOne}. Observera att likt \code{parseAll} är parametertypen \code{Vector[String]}, men skillnaden är att \code{parseOne} ska matcha och returnera exakt ett kommando. I annat fall kastas ett undantag (Exception). Till exempel \code{"-sort 4"} funkar men \code{"-sort 4 asdf"} funkar inte. Kommandona som ska kunna tas emot finns i \code{List}.

\Subtask Implementera \code{parseAll} som en rekursiv funktion. Använd minustecknet som markerar starten för ett nytt kommando för att dela upp argumenten \code{args} i delar som sedan var för sig skickas med i anrop till \code{parseOne}. Tips: \footnote{Kolla på \code{span} och \code{startsWith}.}.

\Subtask Implementera slutligen \code{runAllWith}, även den är rekursivt.

\Subtask Testa ditt program med \code{Main}-filen. \TODO exempeltestfall

\url{https://docs.google.com/spreadsheets/d/1mgx4tqOYTdZVNkzXAENx7LVhxPGXugWEoyqzRtuf1JM/pub?gid=1841525127&single=true&output=csv}

\subsection{Frivilliga extrauppgifter}
    
\Task En labbuppgiftsbeskrivning.

\Subtask En underuppgift.

\Subtask En underuppgift.