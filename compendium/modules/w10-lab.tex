%!TEX encoding = UTF-8 Unicode

%!TEX root = ../compendium.tex

\Lab{\LabWeekTEN}

\begin{Goals}
\item Implementera sorteringsalgoritmerna insättnings- (insertion) och urvalssortering (selection) i Java.
\item Analysera hur algoritmernas komplexitet påverkar exekveringstiden.
\item In- och utläsning av fil på disk.
%\item Implementera sökalgoritmerna linear-search och binary-search.
%\item Serialisera data för transport och lagring.
\end{Goals}

\begin{Preparations}
\item Svara på enkäten: % TODO
\item \DoExercise{\ExeWeekTWO}{02}
\item Veckans övningar, främst 1-3, 9-13.
\item Uppgift 3 från övningen Matrices, vecka 9.%Uppgift 6 också?
\item Om det behövs, läs om eller kolla på de grafiska exemplen av insättningssortering: \url{https://en.wikipedia.org/wiki/Insertion_sort} och urvalssortering: \url{https://en.wikipedia.org/wiki/Selection_sort}
\end{Preparations}

\subsection{Om Labben}

Veckans labb handlar om sortering och statistisk registrering av data. Del 1 handlar om sorteringsalgoritmerna insättnings- och urvalssortering, vilka kommer implementeras i Java. Del 2 är en uttökning av Scalaklassen \code{Table} från uppgift 3, övning Matrices. Utgående från det som implementerades på övningen ska metoder för sortering och registrering med avseende på en viss kolumn läggas till. Rådatan som kommer behandlas i del 2 är enkätsvaren från förberedelseuppgifterna och är av typen CSV (comma-separated values) och ser ut något såhär:
%formattering??
\begin{Code}
rad1kolumn1,rad1kolumn2,rad1kolumn3,rad1kolumn4,rad1kolumn5
rad2kolumn1,rad2kolumn2,rad2kolumn3,rad2kolumn4,rad2kolumn5
rad3kolumn1,rad3kolumn2,rad3kolumn3,rad3kolumn4,rad3kolumn5
...
\end{Code}

\subsection{Kodstruktur}
Koden är uppdelad i följande delar:

\begin{REPL}
src # tree .
.
└── main
    ├── java
    │   ├── GenericInsertionSort.java
    │   ├── GenericSelectionSort.java
    │   ├── InsertionSort.java
    │   └── SelectionSort.java
    └── scala
        ├── StringMatrix.scala
        ├── StringSort.scala
        └── Table.scala
.
src
 `-- main
    |-- java
    |   |-- GenericInsertionSort.java
    |   |-- GenericSelectionSort.java
    |   |-- InsertionSort.java
    |   `-- SelectionSort.java
    `-- scala
        |-- StringMatrix.scala
        |-- StringSort.scala
        `-- Table.scala
\end{REPL}

\begin{itemize}
\item Klassen \code{Table} är datastrukturen som har hand om datan och tillhandahåller de nödvändiga metoderna som kommer användas i implementeringen av sök-, registrering och sorteringsalgoritmerna.
\item OSV
\end{itemize}

\subsection{Given kod}

\item Kunna använda inbyggda sorteringsfunktioner.
\item Kunna använda inbyggda sökfunktioner.
\item Känna till hur strängar ordnas.
\item Kunna läsa text i tabellform från fil.
\item Kunna använda registrering (frekvensräkning) för enkla statistikberäkningar.
\item ... \TODO mer här
\end{Goals}

\begin{Preparations}
\item \StudyTheory{10}
\item \DoExercise{\ExeWeekTEN}{10}
\item \ReadTheLab
\item Svara på denna enkät $<<$ \TODO Länk till google forms-enkät $>>$  \\
\TODO förslag till Anton på innehåll i google forms-enkät:\\ \textit{Vilket är ditt favoritalternativ?}
\begin{itemize}[nolistsep,noitemsep]
\item \textbf{program} (D, W, C, E, F, I, Bio, K, L, M, Bme, Nano, V), 
\item \textbf{OS} (Win7, Win10, macOS X, Linux, Android, IOS, ChromeOS), 
\item \textbf{editor} (gedit, vim, emacs, vi, notepad++, sublime text, atom)
\item \textbf{IDE} (Eclipse, IntelliJ/AndroidStudio, VisualStudio, xcode), 
\item \textbf{socialnät} (facebook, snapchat, linkedin, instagram, github), 
\item \textbf{webbläsare} (firefox, chrome, safari, edge, vivaldi, opera)
\item \textbf{sorteringsalgoritm} (insättningssortering, urvalssortering)
\item \textbf{språk} (Java, Python, PHP, C\#, Javascript, C++, C, Objective-C, R, Swift, Matlab, Ruby, Visual Basic, VBA, Scala, Perl, lua, Delphi)  \\
Listan ordnad enligt \url{http://pypl.github.io/PYPL.html} i Aug 2016
% Alternativet är TIOBE, men den är längre...:
%(Java, C, C++, C\#, Python, PHP, Javascript, Visual Basic, Perl, Pascal, Ruby, Swift, Groovy, R, Matlab, SQL, Go, Dart, Fortram, Lua, Ada, Lisp, Scala, Prolog, Haskell, Erlang, Rust)
\end{itemize}
\end{Preparations}

\item Kunna använda inbyggda sorteringsfunktioner.
\item Kunna använda inbyggda sökfunktioner.
\item Känna till hur strängar ordnas.
\item Kunna läsa text i tabellform från fil.
\item Kunna använda registrering (frekvensräkning) för enkla statistikberäkningar.
\item ... \TODO mer här
\end{Goals}

\begin{Preparations}
\item \StudyTheory{10}
\item \DoExercise{\ExeWeekTEN}{10}
\item \ReadTheLab
\item Svara på denna enkät $<<$ \TODO Länk till google forms-enkät $>>$  \\
\TODO förslag till Anton på innehåll i google forms-enkät:\\ \textit{Vilket är ditt favoritalternativ?}
\begin{itemize}[nolistsep,noitemsep]
\item \textbf{program} (D, W, C, E, F, I, Bio, K, L, M, Bme, Nano, V), 
\item \textbf{OS} (Win7, Win10, macOS X, Linux, Android, IOS, ChromeOS), 
\item \textbf{editor} (gedit, vim, emacs, vi, notepad++, sublime text, atom)
\item \textbf{IDE} (Eclipse, IntelliJ/AndroidStudio, VisualStudio, xcode), 
\item \textbf{socialnät} (facebook, snapchat, linkedin, instagram, github), 
\item \textbf{webbläsare} (firefox, chrome, safari, edge, vivaldi, opera)
\item \textbf{sorteringsalgoritm} (insättningssortering, urvalssortering)
\item \textbf{språk} (Java, Python, PHP, C\#, Javascript, C++, C, Objective-C, R, Swift, Matlab, Ruby, Visual Basic, VBA, Scala, Perl, lua, Delphi)  \\
Listan ordnad enligt \url{http://pypl.github.io/PYPL.html} i Aug 2016
% Alternativet är TIOBE, men den är längre...:
%(Java, C, C++, C\#, Python, PHP, Javascript, Visual Basic, Perl, Pascal, Ruby, Swift, Groovy, R, Matlab, SQL, Go, Dart, Fortram, Lua, Ada, Lisp, Scala, Prolog, Haskell, Erlang, Rust)
\end{itemize}
\end{Preparations}

\subsection{Bakgrund}

I denna laboration ska du utveckla ett program som analyserar svar på enkäter med flervalsfrågor. Indata utgörs av text i form av \textbf{kolumnseparerade värden}, där varje persons svar finns på en egen rad och varje svarsrad innehåller svarsalternativ separerade med en \textbf{kolumnseparator} som till exempel kan vara \code{;} eller \code{\t}. Första raden i textfilen anger kolumnernas namn.

Exempel på indatafil: \footnote{\TODO gör indataexemplet lite längre så analysexempel blir lite roligare och låt indatafilen finnas med som testfil i workspace}
\begin{CodeSmall}[language=, ]
program;OS;editor;IDE;socialnät;webbläsare;sorteringsalgoritm;språk
W;Gedit;Eclipse;Facebook;Firefox;insättningssortering;Java
D;Atom;Intellij;GitHub;Vivaldi;urvalssortering;Scala
E;emacs;Eclipse;Snapchat;Edge;urvalssortering;C
\end{CodeSmall}

Ditt program ska innehålla följande delar:
\begin{itemize}
\item En case-klass för strängmatriser 
\item Funktioner för inläsning av tabellformatterad text.
\item Funktioner för att presentera statistik från enkätdata med hjälp av registrering.
\item \TODO mer här ?
\end{itemize}

\TODO mer här...

\subsection{Obligatoriska uppgifter}

\Task Implementera datastrukturen

\Subtask En underuppgift.

\Subtask En underuppgift.

\subsection{Frivilliga extrauppgifter}

\Task Implementera sorteringsfunktionen generellt

\Subtask En underuppgift.

\Subtask En underuppgift.
    
