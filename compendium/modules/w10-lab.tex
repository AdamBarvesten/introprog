%!TEX encoding = UTF-8 Unicode

%!TEX root = ../compendium.tex

\Lab{\LabWeekTEN}

\begin{Goals}
\item Kunna använda inbyggda sorteringsfunktioner.
\item Kunna använda inbyggda sökfunktioner.
\item Känna till hur strängar ordnas.
\item Kunna läsa text i tabellform från fil och webbaddress.
\item Kunna använda registrering (frekvensräkning) för enkla statistikberäkningar.
\item Kunna skriva till fil.
\item Kunna omvandla startargument till kommandon.
\item Kunna skriva rekursiva funktioner.
\end{Goals}

\begin{Preparations}
\item \StudyTheory{10}
\item \DoExercise{\ExeWeekTEN}{10}
\item \ReadTheLab
\item Fyll i denna enkät: \url{https://goo.gl/forms/hC6JK2UQXVpbGECc2}  \\
I enkäten ska du för olika flervalsalternativlistor besvara frågan: \\ \textit{Vilket är ditt favoritalternativ?}
\end{Preparations}


\subsection{Bakgrund}

I den här veckans laboration ska du utveckla ett program som analyserar svar på enkäter med flervalsfrågor. Indata utgörs av text i form av \textbf{kolumnseparerade värden}, där varje persons svar finns på en egen rad och varje svarsrad innehåller svarsalternativ separerade med en \textbf{kolumnseparator} som till exempel kan vara ``\code{\t}'' eller ``\code{,}''. Första raden i textfilen anger kolumnernas namn.
Exempelindata finns i filen \code{favorit.csv} i mappen resources:
\lstinputlisting[basicstyle=\ttfamily\fontsize{10}{12}\selectfont]{../workspace/w10_survey/src/main/resources/favorit.csv}

\noindent Ditt program ska bestå av följande delar:
\begin{itemize}
\item En case-klass för strängmatriser som heter \code{Table} med funktioner för inläsning av tabellformatterad text, sortering, filtrering och registrering med avseende på en viss kolumn.
\item Ett objekt för argumentparsning och kommandoexekvering som heter \code{Command}.
\item En given Main-fil som ska kunna köra det slutgiltiga programmet.
\item Funktion för att presentera statistik från enkätdata med hjälp av registrering.
\end{itemize}

\subsection{Given kod}

Given kod finns här: \\
\href{https://github.com/lunduniversity/introprog/tree/master/workspace/w10_survey/src/main}{https://github.com/lunduniversity/introprog/tree/master/\\workspace/w10\_survey/src/main}

\noindent Så här ser huvudprogrammet ut:
\scalainputlisting{../workspace/w10_survey/src/main/scala/stats/Main.scala}



\subsection{Obligatoriska uppgifter}

\Task Implementera \code{Table} enligt specifikation:

\ScalaSpecInputListing{Table}{../workspace/w10_survey/src/main/scala/stats/Table.scala}

\Subtask Börja med att implementera klassen \code{Table} men vänta med \code{register} till nästa uppgift. I \code{sort} och \code{filter} ska inbyggda funktioner för sortering och filtrering användas. Eftersom \code{Table} är omuterbar måste nya instanser eller kopior skapas av \code{Table} vid varje filtrering eller sortering.

\Subtask Implementera \code{register} som senare kommer användas för att skapa diagram som presenterar den registrerade datan. För att registrera antalet förekomster av varje unikt värde kan funktionerna \code{groupBy} och \code{mapValues} användas. 

Omvandlingen går från \code{Vector[String]} till \code{Map[String, Vector[String]]} med \code{groupBy} och efter det till \code{Map[String, Int]} med \code{mapValues}.
Exempel med \code{groupBy}:
\begin{REPLnonum}
scala> Vector("ICA", "COOP", "Konsum", "ICA", "ICA", "COOP").groupBy(v => v)

res0: Map[String,Vector[String]] = Map(COOP -> Vector(COOP, COOP),
                                       Konsum -> Vector(Konsum),
                                       ICA -> Vector(ICA, ICA, ICA))
\end{REPLnonum}

\Subtask Implementera funktionen \code{fromFile} i objektet \code{Table}. Parametern \code{uri} är antingen en webbadress eller en lokal sökväg. Det räcker med att anta att en webbadress börjar på ``\code{http}''.

En fil som innehåller tre rader med 3 kolumner på varje rad läses in till vektorn såhär:
\begin{CodeSmall}[language=, ]
Vector(Vector(rad1kolumn1,rad1kolumn2,rad1kolumn3),
       Vector(rad2kolumn1,rad2kolumn2,rad2kolumn3),
       Vector(rad3kolumn1,rad3kolumn2,rad3kolumn3))
\end{CodeSmall}

\Subtask Implementera \code{toFile} med valfri metod för att skriva till fil.

\Subtask Testa \code{Table} genom att köra \code{main}-funktionen. Om allt står rätt till finns det nu en fil \code{out.csv} i mappen resources och i konsollen lyder utskriften:
\begin{REPLnonum}
(DE,5)
(Vim,2)
(Eclipse,1)
(Emacs,1)
(Gedit,1)

\end{REPLnonum}

\clearpage %Hack to fix pagination

\Task Komplettera \code{Command} enligt specifikation:

\ScalaSpecInputListing{Command}{../workspace/w10_survey/src/main/scala/stats/Command.scala}

Observera att i objektet \code{Command} definieras typen \code{Command}. En instans av typen \code{Command} är det som \textit{kommando} hädanefter kommer åsyfta. Det är helt enkelt en funktion som tar emot en instans av \code{Table}, gör något med det och returnerar tillbaka en instans av \code{Table}. På så vis kan flera kommandon exekveras som en kedja där nästa kommandos indata är det föregående kommandots utdata. Exempel på ett kommando som lägger till ett utropstecken på kolumnrubrikerna:
\begin{CodeSmall}[language=, ]
t: Table => t.copy(headings = t.headings.map(_ + "!"))
\end{CodeSmall}
Metoderna \code{Command.parseOne} och \code{Command.parseAll} omvandlar argument till kommandon.

\Subtask Implementera \code{parseOne}. Lägg märke till att likt \code{parseAll} är parametertypen \code{Vector[String]}, men skillnaden är att \code{parseOne} ska matcha och returnera exakt ett kommando. I annat fall kastas ett generellt undantag: \code{throw new Exception("meddelande")}. Till exempel ``\code{-sort 4}'' funkar men ``\code{-sprt 4}'' och ``\code{-sort 4 asdf}'' funkar inte. De giltiga argumenten som ska kunna tas emot finns i \code{Command.list}. Undantaget som kastas då något annat än ett strängheltal ska göras om till en \code{Int} tas hand om i \code{Main} och behöver inte hanteras.

Om ``\code{-printchart 6}'' körs på datan från \code{favorit.csv} ska följande diagram produceras:
\begin{REPLnonum}
Column: DE (17)
notepad++: 4 occurrences
****
Eclipse: 3 occurrences
***
Emacs: 3 occurrences
***
Vim: 3 occurrences
***
Gedit: 2 occurrences
**
Visual Studio: 2 occurrences
**

\end{REPLnonum}
Om flera kolumner efterfrågas, till exempel ``\code{-printchart 3 5 6}'', skrivs diagrammen ut direkt efter varandra.

\Subtask Implementera \code{parseAll} som en rekursiv funktion. Använd minustecknet som markerar starten för ett nytt kommando för att dela upp argumenten \code{args} i delar som sedan var för sig skickas med i anrop till \code{parseOne}. Kanske funktionen \code{span} kan vara till nytta:
\begin{REPLnonum}
scala> Vector("myra", "panda", "", "ekorre").span(_.nonEmpty)

res1: (Vector[String], Vector[String]) = (Vector(myra, panda),
                                         Vector("", ekorre))
\end{REPLnonum}

\Subtask Implementera \code{runAllWith}. Eftersom kommandona ska exekveras i en kedja passar det väldigt bra att även göra den här funktionen rekursiv.

\Subtask Testa ditt program med \code{Main}-filen och \code{favorit.csv}.
\begin{REPLnonum}
>scala stats.Main favorit.csv , -filter 2 Terminal -sort 3 -sep | -print
-printchart 5
Loading "favorit.csv" "," to table ...

Done. Size: 17x7.

Program|Indent|UI|Lang|OS|Browser|DE
D|Spaces|Terminal|C|BSD|Firefox|Emacs
F|Spaces|Terminal|C|Linux|Chrome|Emacs
F|Spaces|Terminal|C|Linux|Firefox|Vim
C|Tabs|Terminal|C#|Windows 10|Edge|Visual Studio
D|Spaces|Terminal|Java|Windows 8|Edge|Eclipse
E|Spaces|Terminal|Java|Linux|Chromium|Eclipse
C|Spaces|Terminal|Javascript|Windows 7|Chrome|Notepad++
Nano|Tabs|Terminal|Javascript|macOS|Safari|Vim
I|Tabs|Terminal|PHP|Windows 10|Edge|Notepad++
I|Tabs|Terminal|Python|Windows 10|Chrome|Notepad++

Column: Browser (10)
Chrome: 3 occurrences
***
Edge: 3 occurrences
***
Firefox: 2 occurrences
**
Chromium: 1 occurrence
*
Safari: 1 occurrence
*

\end{REPLnonum}

\Task Avslutningsvis, testa att programmet fungerar med länken till enkätsvaren: \url{https://goo.gl/qPcuAO}.

\subsection{Frivilliga extrauppgifter}
    
\Task Inför lämpliga argument till programmet som medför att fina tårtdiagram och/eller stapeldiagram i SimpleWindow med automatiska färger ritas ut som illustrationer till frekvenserna i de olika kolumnerna.
