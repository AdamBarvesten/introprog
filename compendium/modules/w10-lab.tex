%!TEX encoding = UTF-8 Unicode

%!TEX root = ../compendium.tex

\Lab{\LabWeekTEN}

\begin{Goals}
\item Kunna använda inbyggda sorteringsfunktioner
\item Kunna använda inbyggda sökfunktioner.
\item Känna till hur strängar ordnas.
\item Kunna läsa text i tabellform från fil och webbaddress.
\item Kunna använda registrering (frekvensräkning) för enkla statistikberäkningar.
\item Kunna skriva till fil.
\item Kunna omvandla startargument till kommandon.
\item Kunna skriva rekursiva funktioner.
\end{Goals}

\begin{Preparations}
\item \StudyTheory{10}
\item \DoExercise{\ExeWeekTEN}{10}
\item \ReadTheLab
\item Fyll i denna enkät: \url{https://goo.gl/forms/hC6JK2UQXVpbGECc2}  \\
I enkäten ska du i sju olika kategorier besvara frågan: \textit{Vilket är ditt favoritalternativ?} \\
Kategorierna är:
\begin{itemize}[nolistsep,noitemsep]
\item Program - LTH-program
\item Indent - Spaces vs Tabs.
\item UI - GUI vs Terminal.
\item Lang - Programmeringsspråk.
\item OS - Operativsystem
\item Browser - Webbläsare
\item DE - Development Environment, editorer och IDEer.
\end{itemize}
%gällande följande områden och svarsalternativen:
%\begin{itemize}[nolistsep,noitemsep]
%\item \textbf{Program} (D, W, C, E, F, I, Bio, K, L, M, Bme, Nano, V), 
%\item \textbf{OS} (Win7, Win10, macOS X, Linux, Android, IOS, ChromeOS), 
%\item \textbf{editor} (gedit, vim, emacs, vi, notepad++, sublime text, atom)
%\item \textbf{ide} (Eclipse, IntelliJ/AndroidStudio, VisualStudio, xcode), 
%\item \textbf{socialnät} (facebook, snapchat, linkedin, instagram, github), 
%\item \textbf{webbläsare} (firefox, chrome, safari, edge, vivaldi, opera)
%\item \textbf{sorteringsalgoritm} (insättningssortering, urvalssortering)
%\item \textbf{språk} (Java, Python, PHP, C\#, Javascript, C++, C, Objective-C, R, Swift, Matlab, Ruby, %Visual Basic, VBA, Scala, Perl, lua, Delphi)  \\
%Listan enligt \url{http://pypl.github.io/PYPL.html} i Aug 2016
% Alternativet är TIOBE, men den är längre...:
%(Java, C, C++, C\#, Python, PHP, Javascript, Visual Basic, Perl, Pascal, Ruby, Swift, Groovy, R, Matlab, SQL, Go, Dart, Fortram, Lua, Ada, Lisp, Scala, Prolog, Haskell, Erlang, Rust)
%\end{itemize}
\end{Preparations}


\subsection{Bakgrund}

I den här veckans laboration ska du utveckla ett program som analyserar svar på enkäter med flervalsfrågor. Indata utgörs av text i form av \textbf{kolumnseparerade värden}, där varje persons svar finns på en egen rad och varje svarsrad innehåller svarsalternativ separerade med en \textbf{kolumnseparator} som till exempel kan vara ``\code{\t}'' eller ``\code{,}''. Första raden i textfilen anger kolumnernas namn.

Exempelindata finns i filen \code{favorit.csv} i mappen resources, såhär ser den ut:
\lstinputlisting[basicstyle=\ttfamily\fontsize{10}{12}\selectfont]{../workspace/w10_survey/src/main/resources/favorit.csv}

Ditt program ska bestå av följande delar:
\begin{itemize}
\item En case-klass för strängmatriser som heter \code{Table} med funktioner för inläsning av tabellformatterad text, sortering, filtrering och registrering med avseende på en viss kolumn.
\item Ett objekt för argumentparsning och kommandoexekvering som heter \code{Command}.
\item En given Main-fil som ska kunna köra det slutgiltiga programmet.
\item Funktion för att presentera statistik från enkätdata med hjälp av registrering.
\end{itemize}

\subsection{Given kod}
\ScalaSpecInputListing{Main}{../workspace/w10_survey/src/main/scala/stats/Main.scala}



\subsection{Obligatoriska uppgifter}

\Task Implementera \code{Table} enligt specifikation:

\ScalaSpecInputListing{Table}{../workspace/w10_survey/src/main/scala/stats/Table.scala}

\Subtask Börja med att implementera alla funktioner förutom \code{register} i klassen \code{Table}. I \code{sort} och \code{filter} ska inbyggda funktioner för sortering och filtrering användas. Eftersom \code{Table} är omuterbar måste nya instanser eller kopior skapas av \code{Table} vid varje filtrering eller sortering.

\Subtask Implementera \code{register}. Returdatan är av typen \code{Vector[(String, Int)]} där första elementet är kolumnrubriken tillsammans med det totala antalet registrerade uppkomster. Funktionen kommer senare användas för att skapa diagram som presenterar den registrerade datan. Använd gärna funktionen \code{groupBy} för att enkelt gruppera en vektor med kolumnvärden:
\begin{REPLnonum}
scala> Vector("ICA", "COOP", "Konsum", "ICA", "ICA", "COOP").groupBy(v => v)

res0: Map[String,Vector[String]] = Map(COOP -> Vector(COOP, COOP),
                                       Konsum -> Vector(Konsum),
                                       ICA -> Vector(ICA, ICA, ICA))
\end{REPLnonum}
Omvandla sedan \code{Map[String, Vector[String]]} till \code{Map[String, Int]} där \code{Int}-värdet är vektorns storlek. Och sedan till \code{Vector[(String, Int)]} i nästa steg. Innan datan returneras ska den sorteras så det alternativ med flest uppkomster är andra elementet och det med minst är sista elementet i vektorn.
\TODO är den här uppgiften för klurig?

\Subtask Implementera funktionen \code{fromFile} i objektet \code{Table}. Den ska kunna ta emot antingen en webbadress eller en lokal sökväg. I ditt program räcker det med att undersöka fallet då \code{uri} börjar på ``\code{http}''.

En fil som innehåller tre rader med 3 kolumner på varje rad läses in till vektorn såhär:
\begin{CodeSmall}[language=, ]
Vector(Vector(rad1kolumn1,rad1kolumn2,rad1kolumn3),
       Vector(rad2kolumn1,rad2kolumn2,rad2kolumn3),
       Vector(rad3kolumn1,rad3kolumn2,rad3kolumn3))
\end{CodeSmall}

\Subtask Implementera \code{toFile} med valfri metod för att skriva till fil.

\Subtask Testa \code{Table} genom att köra \code{main}-funktionen. Om allt står rätt till finns det nu en fil \code{out.csv} i mappen resources och i konsollen lyder utskriften:
\begin{REPLnonum}
(DevEnv,5)
(Vim,2)
(Emacs,1)
(Gedit,1)
(Eclipse,1)
\end{REPLnonum}



\Task Komplettera \code{Command} enligt specifikation:

\ScalaSpecInputListing{Command}{../workspace/w10_survey/src/main/scala/stats/Command.scala}

Observera att i objektet \code{Command} definieras typen \code{Command}: en instans av typen \code{Command} är helt enkelt en funktion som tar emot en instans av \code{Table}, gör något med det och returnerar tillbaka en instans av \code{Table}. På så vis kan flera kommandon exekveras som en kedja där varje kommandos indata är det föregående kommandots utdata.

\Subtask Implementera \code{parseOne}. Observera att likt \code{parseAll} är parametertypen \code{Vector[String]}, men skillnaden är att \code{parseOne} ska matcha och returnera exakt ett kommando. I annat fall kastas ett generellt undantag: \code{throw new Exception("meddelande")}. Till exempel ``\code{-sort 4}'' funkar men ``\code{-sort 4 asdf}'' funkar inte. Kommandona som ska kunna tas emot finns i värdet \code{list}.

Exempel på kommando: ``\code{-printchart 7}''-kommandot tar emot en instans \code{t: Table}, anropar \code{t.register(7)} och skriver ut datan. Om kommandot anropas med datan från \code{favorit.csv} ska följande utskrift produceras:
\begin{REPLnonum}
Column: DE (17)
notepad++: 4 occurrences
****
Eclipse: 3 occurrences
***
Emacs: 3 occurrences
***
Vim: 3 occurrences
***
Gedit: 2 occurrences
**
Visual Studio: 2 occurrences
**

\end{REPLnonum}

\Subtask Implementera \code{parseAll} som en rekursiv funktion. Använd minustecknet som markerar starten för ett nytt kommando för att dela upp argumenten \code{args} i delar som sedan var för sig skickas med i anrop till \code{parseOne}. Kanske funktionen \code{span} kan vara till nytta:
\begin{REPLnonum}
scala> Vector("myra", "panda", "", "ekorre").span(_.nonEmpty)

res1: (Vector[String], Vector[String]) = (Vector(myra, panda),Vector("", ekorre))
\end{REPLnonum}
\TODO behövs pseudokod här eller ska parseAll rent av vara given?


\Subtask Implementera \code{runAllWith}. Eftersom kommandona ska exekveras i en kedja passar det väldigt bra att även göra den här funktionen rekursiv.

\Subtask Testa ditt program med \code{Main}-filen och \code{favorit.csv}.
\begin{REPLnonum}
>scala stats.Main favorit.csv , -filter 2 Terminal -sort 3 -sep "|" -print -printchart 6
Loading "favorit.csv" with separator "," ...

Done. Size: 17x7.

Program|Indent|UI|Lang|OS|Browser|DE
D|Spaces|Terminal|C|BSD|Firefox|Emacs
F|Spaces|Terminal|C|Linux|Chrome|Emacs
F|Spaces|Terminal|C|Linux|Firefox|Vim
C|Tabs|Terminal|C#|Windows 10|Edge|Visual Studio
D|Spaces|Terminal|Java|Windows 8|Edge|Eclipse
E|Spaces|Terminal|Java|Linux|Chromium|Eclipse
C|Spaces|Terminal|Javascript|Windows 7|Chrome|Notepad++
Nano|Tabs|Terminal|Javascript|macOS|Safari|Vim
I|Tabs|Terminal|PHP|Windows 10|Edge|Notepad++
I|Tabs|Terminal|Python|Windows 10|Chrome|Notepad++

Column: Browser (10)
Chrome: 3 occurrences
***
Edge: 3 occurrences
***
Firefox: 2 occurrences
**
Chromium: 1 occurrence
*
Safari: 1 occurrence
*

\end{REPLnonum}

\Task Avslutningsvis kan du testa ditt program med enkätsvaren på länken 
\url{https://goo.gl/qPcuAO}.

\subsection{Frivilliga extrauppgifter}
    
\Task En labbuppgiftsbeskrivning. \TODO

\Subtask En underuppgift.

\Subtask En underuppgift.