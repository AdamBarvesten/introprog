%!TEX encoding = UTF-8 Unicode
%!TEX root = ../compendium.tex

\Lab{\LabWeekTHREE}

\begin{Goals}
\item Kunna definiera och anropa funktioner.
\item Kunna definiera default-argument.
\item Kunna ange parametervärden med parameternamn.
\end{Goals}

\begin{Preparations}
\item Använd övningarna i kapitel~\ref{exe:W03} för att ta reda på hur man skriver en \code{main}-funktion i Scala.
\end{Preparations}



\subsection{Obligatoriska uppgifter}


\begin{quote}
\textbf{Blockmullvad} (\textit{Talpa laterculus}) är ett fantasidjur i familjen mullvadsdjur. 
Den är känd för sitt karaktäristiska kvadratiska utseende.
Den lever mest ensam i sina underjordiska gångar som till skillnad från mullvadens (\emph{Talpa europaea}) har helt raka väggar.
\end{quote}

\begin{figure}
\end{figure}

\Task
Du ska skriva ett Scala-program med en vanlig texteditor och kompilera ditt program med kommandot \texttt{scalac} och sedan köra programmet med kommandot \texttt{scala}.

\Subtask
Öppna en texteditor, till exempel gedit eller Atom (se appendix~\ref{appendix:edit} för hjälp).
Skapa en ny fil med namnet \texttt{Mole.scala} och spara den i en ny katalog i din hemkatalog, till exempel \texttt{\textasciitilde/pgk/mole/Mole.scala}, där \texttt{\textasciitilde} är din hemkatalog.

\Subtask
Öppna ett terminalfönster (se appendix~\ref{appendix:terminal} för hjälp).
Navigera till din nya katalog med \texttt{cd}-kommandot \Eng{change directory} och kontrollera med \texttt{ls}-kommandot \Eng{list} att din nya fil finns där.
\begin{REPLnonum}
> cd ~/pgk/mole
> ls
\end{REPLnonum}
Om allt går bra ska \texttt{ls}-kommandot skriva ut \texttt{Mole.scala}.

\Subtask
Gå tillbaka till din texteditor och skriv in ett object med namnet \code{Mole} i din fil.
Lägg till en \code{main}-funktion i objektet som skriver ut texten \emph{Keep on digging!} med hjälp av funktionen \code{println}.
Behöver du hjälp kan du gå tillbaka till övningarna i kapitel~\ref{exe:W03}.

\Subtask
Kör kommandot \texttt{scalac Mole.scala} i terminalfönstret för att kompilera ditt program.
Om kompilatorn rapporterar några fel rättar du till det i din texteditor kompilerar igen.
Kontrollera sedan med \texttt{ls}-kommandot att några filer som slutar på \texttt{class} har skapats.

\Subtask
Kör kommandot \texttt{scala Mole} för att köra ditt program.
Om att går bra ska texten du angivit skrivas ut i terminalfönstret.


\Task
Nu har du skrivit ett Scala-program som skriver ut en uppmaning till en mullvad att fortsätta gräva.
Det programmet är inte så användbart, eftersom mullvadar inte kan inte läsa.
Nästa steg är att skriva ett grafiskt program, snarare än ett textbaserat.

Funktionen \code{println} som anropas i \code{main}-funktionen ingår i Scalas standardbibliotek.
Ett programbibliotek innehåller kod eller kompilerade programsnuttar som kan användas av andra program, och för de flesta programspråk ingår ett standardbiblitek som alla program kan nyttja.
Till grafiken i denna uppgift ska du använda ett biblitek som kallas \emph{cslib} och som kommer att användas även i senare labbar.

\Subtask
Gå till kursens webbsida \url{https://cs.lth.se/pgk/} och ladda ner \href{https://cs.lth.se/pgk/cslib.jar}{cslib.jar} och lägg den i samma katalog som ditt Scala program.
En \texttt{jar}-fil är egenligen en \texttt{zip}-fil som innehåller färdiga \texttt{class}-filer.

\Subtask
Byt ut \code{println}-anropet så att \code{main}-funktionen innehåller följande rader:
\begin{Code}
val w = new cslib.window.SimpleWindow(300, 500, "Digging")
w.moveTo(10, 10)
w.lineTo(10, 20)
w.lineTo(20, 20)
w.lineTo(20, 10)
w.lineTo(10, 10)
\end{Code}
Den första raden skapar ett nytt \code{SimpleWindow} som ritar upp ett fönster som är 300 bildpunkter brett och 500 bildpunkter högt med titeln \emph{Digging}.
\code{SimpleWindow} har en \emph{penna} som kan flyttas runt och rita linjer.
Anropet \code{w.moveTo(10, 10)} flyttar pennan för fönstret \code{w} till position $(10,10)$ utan att rita något, och anropet \code{w.lineTo(10, 20)} ritar en linje därifrån till position $(10, 20)$.

\Subtask
Nu ska du kompilera ditt program, men eftersom \code{SimpleWindow} inte finns i Scalas standardbiblitek utan i \texttt{cslib.jar} behöver du visa kompilatorn var den ska leta.
Det gör du genom att ange en \emph{classpath}, dvs. en sökväg till \texttt{class}-filer, när du kompilerar.
Använd flaggan \texttt{-cp cslib.jar} för att ange \texttt{cslib.jar} som classpath och kompilera ditt Scala-program igen:
\begin{REPLnonum}
> scalac -cp cslib.jar Mole.scala
\end{REPLnonum}

\Subtask
Nu ska du köra ditt program, och då behöver du också ange var \texttt{class}-filerna ligger.
Du ska ange den katalog där \texttt{class}-filerna för \code{Mole} ligger, som du just kompilerat, men du ska också ange \texttt{cslib.jar}, och det gör du med en kolon-separerad lista, till exempel \texttt{sökväg1:sökväg2:sökväg3}.
Katalogen du står i, där dina \texttt{class}-filer ligger, kan anges med en punkt (\texttt{.}).
Kör programmet med följande kommando:
\begin{REPLnonum}
> scala -cp .:cslib.jar Mole
\end{REPLnonum}
Du ska nu få upp ett fönster med en liten kvadrat utritad i övre vänstra hörnet.


\Task
Hela ditt program är för tillfället samlat i en och samma funktion, vilket fungerar bra för väldigt små program.
Nu ska vi dela upp programmet så det blir lättare att återanvända samma kodsnuttar.

\Subtask
Lägg till ett objekt med namnet \code{Graphics} i \texttt{Mole.scala} och flytta dit deklarationen av fönstret \code{w}.
Skapa en ny funktion med namnet \code{square} i det nya objektet och flytta dit koden som ritar kvadraten.
Anropa \code{square} i din \code{main}-funktion.
Filen \texttt{Mole.scala} ska se ut såhär:
\begin{Code}
object Graphics {
	val w = new cslib.window.SimpleWindow(300, 500, "Digging")
	def square(): Unit = ???
}
object Mole {
	def main(args: Array[String]): Unit = {
		Graphics.square()
	}
}
\end{Code}
Observera att du inte kan anropa \code{square} direkt i funktionen \code{main}, utan måste ange att det är \code{square} metoden inuti \code{Graphics} du vill anropa.

\Subtask
Kompilera \texttt{Mole.scala} med \texttt{scalac}.
Glöm inte att ange korrekt classpath.
(Tips: Du kan trycka uppåtpil för att komma till tidigare kommandon i terminalen.)
Kontrollera med \texttt{ls} att det nu också finns \texttt{class}-filer för \code{Graphics}-objektet.

\Subtask
Kör programmet \code{Mole} med \texttt{scala}.
Glöm inte att ange korrekt classpath.
Om allt fungerar ska programmet göra samma sak som innan.

\Task
Nu har du gjort ett grafiskt program, men ännu syns ingen mullvad.
Det är dags att ta reda på hur koordinatsystemet fungerar i denna grafiska miljö, så vi kan få mullvaden att hitta rätt.

\Subtask
Ändra i \code{Graphics.square} så att kvadraten ritas upp i \emph{övre högra} hörnet istället.
Prova dig fram för att ta reda på hur koordinatsystemet fungerar genom att ändra i koden, kompilera och köra programmet tills du får rätt på det.

\Subtask
För att kunna rita ut flera kvadrater på olika ställen ska du lägga till två parametrar till \code{Graphics.square}.
Funktionen ska nu ha följande form:
\begin{Code}
def square(x: Int, y: Int) = ???
\end{Code}
Ändra i koden så kvadraten ritas ut med övre vänstra hörnet i den position som parametrarna \code{x} och \code{y} anger.
Ändra också i \code{Mole.main} så att \code{Graphics.square} anropas med lämpliga parametrar.
Kompilera och kör programmet och tills du får det att fungera.
Se till att du kan rita ut kvadraten på olika ställen i fönstret genom att ange olika parametrar.


\Task
\code{???}

\Subtask
\code{???}


\subsection{Frivilliga extrauppgifter}

\Task En labbuppgiftsbeskrivning.

\Subtask En underuppgift.

\Subtask En underuppgift.
