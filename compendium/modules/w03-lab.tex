%!TEX encoding = UTF-8 Unicode
%!TEX root = ../compendium.tex

\Lab{\LabWeekTHREE}

\begin{Goals}
\item Kunna kompilera Scalaprogram med \texttt{scalac}.
\item Kunna köra Scalaprogram med \texttt{scala}.
\item Kunna definiera och anropa funktioner.
\item Kunna använda och förstå default-argument.
\item Kunna ange argument med parameternamn.
\item Kunna definiera objekt med medlemmar.
\item Förstå kvalificerade namn och import.
\item Förstå synlighet och skuggning.
\end{Goals}

\begin{Preparations}
\item \DoExercise{\ExeWeekTWO}{02}
\item \DoExercise{\ExeWeekTHREE}{03} 
\end{Preparations}



\subsection{Obligatoriska uppgifter}


\begin{quote}
\textbf{Blockmullvad} (\textit{Talpa laterculus}) är ett fantasidjur i familjen mullvadsdjur. 
Den är känd för sitt karaktäristiska kvadratiska utseende.
Den lever mest ensam i sina underjordiska gångar som till skillnad från mullvadens (\emph{Talpa europaea}) har helt raka väggar.
\end{quote}

\begin{figure}
\end{figure}

\Task
Du ska skriva ett Scala-program med en vanlig texteditor och kompilera ditt program med kommandot \texttt{scalac} och sedan köra programmet med kommandot \texttt{scala}.

\Subtask
Öppna en texteditor, till exempel gedit eller Atom (se appendix~\ref{appendix:edit} för hjälp).
Skapa en ny fil med namnet \texttt{Mole.scala} och spara den i en ny katalog i din hemkatalog, till exempel \texttt{\textasciitilde/pgk/mole/Mole.scala}, där \texttt{\textasciitilde} är din hemkatalog.

\Subtask
Öppna ett terminalfönster (se appendix~\ref{appendix:terminal} för hjälp).
Navigera till din nya katalog med \texttt{cd}-kommandot \Eng{change directory} och kontrollera med \texttt{ls}-kommandot \Eng{list} att din nya fil finns där.
\begin{REPLnonum}
> cd ~/pgk/mole
> ls
\end{REPLnonum}
Om allt går bra ska \texttt{ls}-kommandot skriva ut \texttt{Mole.scala}.

\Subtask
Gå tillbaka till din texteditor och skriv in ett objekt med namnet \code{Mole} i din fil.
Lägg till en \code{main}-funktion i objektet som skriver ut texten \emph{Keep on digging!} med hjälp av funktionen \code{println}.
Behöver du hjälp kan du gå tillbaka till övningarna i kapitel~\ref{chapter:W03}.

\Subtask
Kör kommandot \texttt{scalac Mole.scala} i terminalfönstret för att kompilera ditt program.
Om kompilatorn rapporterar några fel rättar du till det i din texteditor kompilerar igen.
Kontrollera sedan med \texttt{ls}-kommandot att några filer som slutar på \texttt{class} har skapats.

\Subtask
Kör kommandot \texttt{scala Mole} för att köra ditt program.
Om att går bra ska texten du angivit skrivas ut i terminalfönstret.


\Task
Nu har du skrivit ett Scala-program som skriver ut en uppmaning till en mullvad att fortsätta gräva.
Det programmet är inte så användbart, eftersom mullvadar inte kan inte läsa.
Nästa steg är att skriva ett grafiskt program, snarare än ett textbaserat.

Funktionen \code{println} som anropas i \code{main}-funktionen ingår i Scalas standardbibliotek.
Ett programbibliotek innehåller kod eller kompilerade programsnuttar som kan användas av andra program, och för de flesta programspråk ingår ett standardbiblitek som alla program kan nyttja.
Till grafiken i denna uppgift ska du använda ett biblitek som kallas \emph{cslib} och som kommer att användas även i senare labbar.

\Subtask
Gå till kursens webbsida \url{https://cs.lth.se/pgk/} och ladda ner \href{https://cs.lth.se/pgk/cslib.jar}{cslib.jar} och lägg den i samma katalog som ditt Scala program.
En \texttt{jar}-fil är egenligen en \texttt{zip}-fil som innehåller färdiga \texttt{class}-filer.

\Subtask
Byt ut \code{main}-funktionens kropp mot följande block:
\begin{Code}
{
	val w = new cslib.window.SimpleWindow(300, 500, "Digging")
	w.moveTo(10, 10)
	w.lineTo(10, 20)
	w.lineTo(20, 20)
	w.lineTo(20, 10)
	w.lineTo(10, 10)
}
\end{Code}
Den första raden skapar ett nytt \code{SimpleWindow} som ritar upp ett fönster som är 300 bildpunkter brett och 500 bildpunkter högt med titeln \emph{Digging}.
\code{SimpleWindow} har en \emph{penna} som kan flyttas runt och rita linjer.
Anropet \code{w.moveTo(10, 10)} flyttar pennan för fönstret \code{w} till position $(10,10)$ utan att rita något, och anropet \code{w.lineTo(10, 20)} ritar en linje därifrån till position $(10, 20)$.

\Subtask
Nu ska du kompilera ditt program, men eftersom \code{SimpleWindow} inte finns i Scalas standardbiblitek utan i \texttt{cslib.jar} behöver du visa kompilatorn var den ska leta.
Det gör du genom att ange en \emph{classpath}, dvs. en sökväg till \texttt{class}-filer, när du kompilerar.
Använd flaggan \texttt{-cp cslib.jar} för att ange \texttt{cslib.jar} som classpath och kompilera ditt Scala-program igen:
\begin{REPLnonum}
> scalac -cp cslib.jar Mole.scala
\end{REPLnonum}

\Subtask
Nu ska du köra ditt program, och då behöver du också ange var \texttt{class}-filerna ligger.
Du ska ange den katalog där \texttt{class}-filerna för \code{Mole} ligger, som du just kompilerat, men du ska också ange \texttt{cslib.jar}, och det gör du med en kolon-separerad lista\footnote{kolon används i Linux och MacOSX, medan windows använder semikolon.}, till exempel \code{"sökväg1:sökväg2:sökväg3"}.
Katalogen du står i, där dina \texttt{class}-filer ligger, kan anges med en punkt (\texttt{.}).
Kör programmet med följande kommando (om Windows använd semikolon):
\begin{REPLnonum}
> scala -cp ".:cslib.jar" Mole
\end{REPLnonum}
Du ska nu få upp ett fönster med en liten kvadrat utritad i övre vänstra hörnet.


\Task
Hela ditt program är för tillfället samlat i en och samma funktion, vilket fungerar bra för väldigt små program.
Nu ska vi strukturera programmet så det blir lättare att återanvända samma kodsnuttar.

\Subtask
Lägg till ett objekt med namnet \code{Graphics} i \texttt{Mole.scala} och flytta dit deklarationen av fönstret \code{w}.
Skapa en ny funktion med namnet \code{square} i det nya objektet och flytta dit koden som ritar kvadraten.
Anropa \code{square} i din \code{main}-funktion.
Filen \texttt{Mole.scala} ska se ut såhär (förutom \code{???}):
\begin{Code}
object Graphics {
	val w = new cslib.window.SimpleWindow(300, 500, "Digging")
	def square(): Unit = ???
}
object Mole {
	def main(args: Array[String]): Unit = {
		Graphics.square()
	}
}
\end{Code}
Observera att du inte kan anropa \code{square} direkt i funktionen \code{main}, utan måste ange att det är \code{square} funktionen inuti \code{Graphics} du vill anropa.

\Subtask
Kompilera \texttt{Mole.scala} med \texttt{scalac}.
Glöm inte att ange korrekt classpath.
(\emph{Tips:} Du kan trycka uppåtpil för att komma till tidigare kommandon i terminalen.)
Kontrollera med \texttt{ls} att det nu också finns \texttt{class}-filer för \code{Graphics}-objektet.

\Subtask
Kör programmet \code{Mole} med \texttt{scala}.
Glöm inte att ange korrekt classpath.
Om allt fungerar ska programmet göra samma sak som innan.

\Task
Nu har du gjort ett grafiskt program, men ännu syns ingen mullvad.
Det är dags att ta reda på hur koordinatsystemet fungerar i denna grafiska miljö, så vi kan få mullvaden att hitta rätt.

\Subtask
Ändra i \code{Graphics.square} så att kvadraten ritas upp i \emph{övre högra} hörnet istället.
Prova dig fram för att ta reda på hur koordinatsystemet fungerar genom att ändra i koden, kompilera och köra programmet tills du får rätt på det.

\Subtask\Checkpoint
Visa kvadraten för din labbhandledare och förklara vad de två parametrarna gör genom att peka ut ungefär var positionerna $(0,0)$, $(300, 0)$, $(0, 300)$ och $(300, 300)$ ligger.

\Subtask
Ta bort anropet till funktionen \code{square} när du har visat den för din labbhandledare.

\Task
Nu ska du skapa ett nytt koordinatsystem för \code{Graphics} som har \emph{stora} bildpunkter.
Vi kallar \code{Graphics} stora bildpunkterna för \emph{block} för att lättare skilja dem från \code{SimpleWindow}s bildpunkter.
Om blockstorleken är $b$, så ligger koordinaten $(x, y)$ i \code{Graphics} på koordinaten $(bx, by)$ i \code{SimpleWindow}.

\Subtask
Lägg till följande deklarationer överst i objektet \code{Graphics}.
\begin{Code}
val width = 30
val height = 50
val blockSize = 10
\end{Code}
Ändra bredden på ditt \code{SimpleWindow} till \code{width * blockSize} och ändra höjden till \code{height * blockSize}.

\Subtask
Skapa en ny funktion i \code{Graphics} med namnet \code{block} och två parametrar \code{x} och \code{y} av typen \code{Int} och returtypen \code{Unit}.
Metodens \emph{kropp} ska se ut såhär:
\begin{Code}
{
    val left = x * blockSize
    val right = left + blockSize - 1
    val top = y * blockSize
    val bottom = top + blockSize - 1

    for (row <- top to bottom) {
      w.moveTo(left, row)
      w.lineTo(right, row)
    }
}
\end{Code}

\Subtask\Pen
Metoden \code{block} ritar ett antal linjer.
Hur många linjer ritas ut?
I vilken ordning ritas linjerna?

\Subtask
Anropa funktionen \code{Graphics.block} några gånger i \code{Mole.main} så att några block ritas upp i fönstret när programmet körs.
Kompilera och kör ditt program.


\Task
Det finns många sätt att beskriva färger.
I naturligt språk har vi olika namn på färgerna, till exempel \emph{vitt}, \emph{rosa} och \emph{magenta}.
I datorn är det vanligt att beskriva färgerna som en blandning av \emph{rött}, \emph{grönt} och \emph{blått} i det så kallade RGB-systemet.
\code{SimpleWindow} använder typen \code{java.awt.Color} för att beskriva färger och \code{java.awt.Color} bygger på RGB.
Det finns några fördefinierade färger i \code{java.awt.Color}, till exempel \code{java.awt.Color.black} för svart och \code{java.awt.Color.green} för grönt.
Andra färger kan skapas genom att ange mängden rött, grönt och blått.

\Subtask
Skapa ett nytt objekt i \texttt{Mole.scala} med namnet \code{Colors} och lägg in följande definitioner:
\begin{Code}
val mole = new java.awt.Color(51, 51, 0)
val soil = new java.awt.Color(153, 102, 51)
val tunnel = new java.awt.Color(204, 153, 102)
\end{Code}
% val sky = new java.awt.Color(51, 51, 204)
% val grass = new java.awt.Color(51, 204, 51)
Den tre parametrarna till \code{new java.awt.Color(r, g, b)} anger hur mycket \emph{rött}, \emph{grönt} respektive \emph{blått} som färgen ska innehålla, och mängderna ska vara i intervallet 0--255.
Färgen $(153, 102, 51)$ innebär ganska mycket rött, lite mindre grönt och ännu mindre blått och det upplevs som brunt.
Objektet \code{Colors} är en färgpallett, men vi har inte ritat något med färg ännu.
Kompilera och kör ditt program ändå, för att se så programmet fungerar lika dant som sist.

\Subtask
Lägg till en parameter till \code{Graphics.block} sist i parameterlistan med namnet \code{color} och typen \code{java.awt.Color}.
Låt \emph{default-argumentet} för den nya parametern vara \code{java.awt.Color.black}.
(Kommer du inte ihåg hur man gör default-argument kan du titta på övningarna i kapitel~\ref{chapter:W03}.)
För att ändra färgen på blocket kan du byta linjefärg innan du ritar.
Lägg till följande rad i början på \code{Graphics.block}:
\begin{Code}
w.setLineColor(color)
\end{Code}
Kompilera och kör ditt program igen för att set om det fortfarande fungerar.

\Subtask\Pen
Funktionen \code{Graphics.block} har tre parametrar, men den anropas bara med två parametrar i \code{Mole.main}.
Varför är det tillåtet?
Vilket värde har den tredje parametern om ingen anges?

\Subtask
Ändra i \code{Mole.main} och lägg till en av definitionerna från objektet \code{Colors} som tredje parameter till \code{Graphics.block}.
Kompilera och kör ditt program och upplev världen i färg.

\Task
I programmet används många långa namn med punkter, som till exempel \code{java.awt.Color} och \code{Graphics.block}.
Dessa punkt-separerade namn kallas \emph{kvalificerade} namn.
För att slippa skriva dessa långa namn hela tiden kan man \emph{importera} en definition och sen använda bara den sista delen av namnet.

\Subtask
Importera namnet \code{java.awt.Color} i objektet \code{Colors}. Ändra sen alla \code{new java.awt.Color(...)} i objektet till \code{new Color(...)}.
(Har du glömt hur man importerar ett namn kan du gå tillbaka till övningarna i kapitel~\ref{chapter:W02}.)

\Subtask\Pen
I vilka av objekten \code{Mole}, \code{Colors} och \code{Graphics} kan du använda det korta respektive det kvalificerade namnet av \code{java.awt.Color}?

\Subtask
Importera namnet \code{java.awt.Color} så att det korta namnet \code{Color} kan användas i objekten \code{Colors} \code{Graphics} men inte i \code{Mole}.
Åndra också i \code{Graphics} så att bara det korta namnet används.

\Task
Nu ska du skriva en funktion för att rita en rektangel och rektangeln ska ritas med hjälp av funktionen \code{block}.
Sen ska du rita upp mullvadens underjordiska värld med hjälp av denna funktion.

\Subtask
Lägg till en funktion i objektet \code{Graphics} med namnet \code{rectangle} som tar fem parametrar \code{x}, \code{y}, \code{width} och \code{height} av typen \code{Int} och \code{color} av typen \code{Color}.
Parametrarna \code{x} och \code{y} anger \code{Graphics}-koordinaten för rektangelns övre vänstra hörn och \code{width} och \code{height} anger bredden respektive höjden.
Använd följande \code{for}-satser för att rita ut rektangeln.
\begin{Code}
for (yy <- y until (y + height)) {
	for (xx <- x until (x + width)) {
		block(xx, yy, color)
	}
}
\end{Code}

\Subtask\Pen
I vilken ordning ritas blocken ut?

% \Subtask\Pen (Fråga något om skuggning gällande \code{width} och \code{height}.)

\Subtask
Skriv en funktion i objektet \code{Mole} med namnet \code{drawWorld} som ritar ut mullvadens värld, det vill säga en massa jord där den kan gräva sina tunnlar.
\code{Mole.drawWorld} ska inte ha några parametrar och returtypen ska vara \code{Unit} och den ska anropa \code{Graphics.rectangle} för att rita en rekltangel med färgen \code{Colors.soil} som precis täcker fönstret.
Eftersom funktionen har många parametrar som lätt kan blandas ihop ska du använda namngivna argument vid anropet.
(Om du har glömt hur man använder namngivna argument kan du titta på övningarna i kapitel~\ref{chapter:W03}.)

\Subtask
Anropa \code{Mole.drawWorld} i \code{Mole.main} och testa så att det fungerar genom att kompilera och köra.

\Task
I \code{SimpleWindow} finns funktioner för att känna av tangenttryckningar och musklick.
Du ska använda de funktionerna för att styra en liten blockmullvad.

\Subtask
Importera \code{cslib.window.SimpleWindow} i objektet \code{Graphics} och lägg till följande funktion:
\begin{Code}
def waitForKey(): Char = {
	do {
		w.waitForEvent()
	} while (w.getEventType() != SimpleWindow.KEY_EVENT)
	w.getKey()
}
\end{Code}
Det finns olika sorters händelser som ett \code{SimpleWindow} kan reagera på, till exempel tangenttryckningar och musklick.
Funktionen som du precis lagt in väntar på en händelse i ditt \code{SimpleWindow} (\code{w.waitForEvent}) ända tills det kommer en tangenttryckning (\code{KEY_EVENT}).
När det kommit en tangenttryckning anropas \code{w.getKey} för att ta reda på vilken bokstav eller vilket tecken det blev, och det resultatet blir också resultatet av \code{waitForKey}, eftersom det ligger sist i det yttre \texttt{\{\}}-blocket.

\Subtask
Lägg till en funktion i objektet \code{Mole} med namnet \code{dig}, som har två parametrar \code{x} och \code{y} av typen \code{Int} och returtypen \code{Unit}.
Funktionens kropp ska se ut såhär (fast utan \code{???}):
\begin{Code}
{
	Graphics.block(x, y, Colors.mole) // ritar ut mullvaden
	val key = Graphics.waitForKey // väntar på tangenttryckning
	if (key == 'w') dig(???, ???)
	else if (key == 'a') dig(???, ???)
	else if (key == 's') dig(???, ???)
	else if (key == 'd') dig(???, ???)
	else dig(x, y)
}
\end{Code}
Fyll i alla \code{???} så att \code{'w'} styr mullvaden ett steg uppåt, \code{'a'} ett steg åt vänster, \code{'s'} ett steg nedåt och \code{'d'} ett steg åt höger.

\Subtask
Ändra \code{Mole.main} att innehålla bara två anrop: Ett till \code{drawWorld} och ett till \code{dig} med lämpliga parametrar.
Kompilera och kör ditt program för att se om programmer reagerar på w, a, s och d.

\Subtask
Om programmet fungerar kommer det bli många mullvad som tillsammans bildar en lång mask, och det är ju lite underligt.
Lägg till ett anrop i \code{Mole.dig} som ritar ut en bit tunnel på position $(x, y)$ efter anropet till \code{Graphics.waitForKey} men innan \code{if}-uttrycket.
Kompilera och kör ditt program för att gräva tunnlar med din blockmullvad.

\subsection{Frivilliga extrauppgifter}

\Task
\TODO Begränsa mullvaden att inte kunna gräva sig utanför fönstret.

\Task
\TODO Lägg till himmel och gräs i ovandelen av fönstret i funktionen \code{Mole.drawWorld}.

\Task
\TODO Begränsa mullvaden till ''gravitationen''.

\Task
\TODO Använd timeout för att få mullvaden att springa själv.

