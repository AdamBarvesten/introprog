%!TEX encoding = UTF-8 Unicode

%!TEX root = ../compendium.tex

\Lab{\LabWeekFIVE}

\begin{Goals}
\item Att öva på att använda vektorer
\item Att använda SHUFFLE-algoritmen för kortblandning
\item Att räkna frekvenser
\end{Goals}

\begin{Preparations}
\item Läs igenom så att du förstår SHUFFLE-algoritmen.
\end{Preparations}

\subsection{Bakgrund}

Denna labb kommer att handla mycket om kortblandning. Att blanda kort så att varje möjlig permutation är lika sannolik är ganska knepigt; om man inte blandar systematiskt kommer det leda till en skev fördelning. Ett bra sätt att blanda en kortlek på, förutsatt att man har en bra slumpgenerator, är att lägga alla kort i en hög och sedan ta ett slumpvist kort från högen och lägga det överst i leken, tills alla kort ligger i leken. Ungefär på detta sätt fungerar den s.k. Fisher-Yates-algoritmen, ibland även kallad en Knuth-shuffle, här endast kallad SHUFFLE.

\begin{algorithm}[H]
 \SetKwInOut{Input}{Indata}
 \Input{Array $xs$ som ska blandas}
 $len \leftarrow$ antalet element i $xs$ \\
 \For{$i \leftarrow (len - 1)$ \KwTo $0$}{
  $r \leftarrow$ slumptal mellan $0$ och $i$ \\
  $temp \leftarrow xs(i)$ \\
  $xs(i) \leftarrow xs(r)$ \\
  $xs(r) \leftarrow temp$ \\
 }
\end{algorithm}

\subsection{Obligatoriska uppgifter}

\Task Den första uppgiften är att implementera metoden \code{shuffle} i klassen \code{CardDeck}. Följ algoritmen noga, och använd \code{cards.length} för att få fram längden på kortleken. 

\Subtask Implementera metoden.

\Subtask Kör \code{TestDeck} för att testa att blandningen är jämnt fördelad. Du bör få sex ungefär lika långa staplar.

\Task Fyll i resten av klassen \code{CardDeck}, skriv kod för att skapa en array innehållande en 52-korts standardlek. Använd case-klasserna i Cards. Tänk på att en en \code{for}/\code{yield}-sats inte nödvändigtvis ger en \code{Array}, men att alla samlingar kan omvandlas till en sådan med \code{toArray}.

\Task Använd den färdiga \code{CardDeck}-klassen för att ta fram sannolikheterna för att kortkombinationerna ``royal flush'', ``straight flush'' ``straight'' eller ``flush'' dyker upp bland 5 kort dragna från en blandad kortlek. Simulera detta genom att upprepade gånger blanda kortleken, dra 5 kort och registrera vilka kombinationer som uppstått. 

\subsection{Frivilliga extrauppgifter}

\Task Utöka simuleringen till att även ta reda på antalet fyrtal, triss, kåk, dubbelpar och enkelpar.
