\documentclass[a4paper]{compendium}

\usepackage[swedish]{babel}
\addto\captionsswedish{%
  \renewcommand{\appendixname}{Appendix}%
}

%\addto\captionsswedish{\renewcommand{\chaptername}{Vecka}} % http://tex.stackexchange.com/questions/30757/change-the-word-chapter-to-something-else

%TODO: Glossary
%http://tex.stackexchange.com/questions/5821/creating-a-standalone-glossary/5837#5837

\setlength{\columnsep}{16mm}

%\usepackage{chngcntr} %http://tex.stackexchange.com/questions/54383/how-to-reset-chapter-and-section-counter-with-part
%\counterwithin*{chapter}{part}

\title{
{\bf\Huge\sffamily  Programmering, grundkurs} 
\\ \vspace{2em}
{\sffamily  Kompendium}
}

%\author{Redaktör: Björn Regnell}
\date{EDAA45, Lp1-2, HT 2016 \\ 
Datavetenskap, LTH \\ 
Lunds Universitet  \\~\\
\url{http://cs.lth.se/pgk}}

% http://tex.stackexchange.com/questions/136527/section-numbering-without-numbers
\usepackage{titlesec}
\usepackage{titletoc}
\usepackage{booktabs}

\begin{document}

\maketitle

%%% 
\clearpage\null\thispagestyle{empty}
\vfill

{\setlength{\parindent}{0pt}
\emph{Editor}: Björn Regnell, Faculty of Engineering LTH, Lund University. \\ 

\emph{Contributors}: 
Björn Regnell,
Per Holm,
Sandra Nilsson,
Patrik Andersson,
Gustav Cedersjö,
Maj Stenmark,
Anna Axelsson,
Roy Andersson,
Markus Borg,
Anton Klarén.
\\

\emph{Repo}: \url{https://github.com/lunduniversity/introprog} \\ \newline

This manuscript is on-going work. Contributions are welcome! \\ 
\emph{Contact}: \url{bjorn.regnell@cs.lth.se}
\\ \newline

\emph{LICENCE}: CC BY-NC-SA 4.0 \\
\url{http://creativecommons.org/licenses/by-nc-sa/4.0/}
\\ \newline
Copyright \copyright~Computer Science, LTH \& Björn Regnell. 2016. Lund. Sweden.\\

}

%%% 
\ChapterUnnum{Framstegsprotokoll} 


\subsubsection*{Genomförda övningar}

\vspace{1em}\noindent 
{Till varje laboration hör en övning med uppgifter som utgör förberedelse inför labben. Du behöver minst behärska de grundläggande övningarna för att klara labben inom rimlig tid. Om du känner att du behöver öva mer på grunderna, gör då även extrauppgifterna. Om du vill fördjupa dig, gör fördjupningsuppgifterna som är på mer avancerad nivå. Genom att du kryssar för nedan vilka övningar du har gjort, blir det lättare för handledaren att förstå vilka förkunskaper du har inför labben.}

\newcommand{\TickBox}{\raisebox{-.50ex}{\Large$\square$}}
\newcommand{\ExeRow}[1]{\texttt{#1} & \TickBox  &  \TickBox &  \TickBox  \\ \addlinespace }

\begin{table}[h]
\centering
\vspace{2em}
\begin{tabular}{lccc}
\toprule \addlinespace 
{\sffamily\small Övning} & 
{\sffamily\small Grundläggande} &	
{\sffamily\small Extra} &
{\sffamily\small Fördjupning}\\ \addlinespace \midrule \\[-0.7em]
\ExeRow{hello} 
\ExeRow{expressions}
\ExeRow{functions}
\bottomrule
\end{tabular}
\end{table}

\newpage

\subsubsection*{Godkända obligatoriska moment}

\vspace{1em}\noindent 
{För att bli godkänd på laborationsuppgifterna måste du lösa deluppgifterna och diskutera dina lösningar med en handledare. Denna diskussion är din möjlighet att få feedback på dina lösningar. Ta vara på den!
Se till att handledaren noterar när du blivit godkänd på detta blad, som är ditt kvitto. Spara detta blad tills du fått slutbetyg i kursen. Det kan vara bra att fotografera sidan om du skulle tappa bort ditt kompendium}


\vspace{2.5em}\noindent Namn: \dotfill\\

\vspace{1em}\noindent Namnteckning: \dotfill\\

\newcommand{\LabRow}[1]{\\[-1.1em] \texttt{#1} & \dotfill &  \dotfill  \\ \addlinespace }

\begin{table}[h]
\centering
\vspace{1em}
\begin{tabular}{lcc}
\toprule \addlinespace 
{\sffamily\bfseries\small Lab} & {\sffamily\small Datum gk} &	{\sffamily\small Handledares namnteckning}\\ \addlinespace \midrule \\[-0.5em]
\LabRow{textgame} 
\LabRow{eclipse} 
\LabRow{anv-square}
\LabRow{impl-square}
\LabRow{gissa-tal}
\LabRow{turtle}
\LabRow{maze}
\LabRow{vektor}
\LabRow{teamlab-turtlerace}
\LabRow{life}
\LabRow{teamlab-imagefilters}
%\toprule 
\addlinespace \midrule \addlinespace
{\sffamily\small {\bfseries Inlämningsuppgift} (välj en)	} & {\sffamily\small Datum gk} &	{\sffamily\small Handledares namnteckning}\\ \addlinespace\addlinespace %\midrule
\texttt{( ) bank}  & \dotfill & \dotfill \\
\texttt{( ) mandelbrot} \\  
\texttt{( ) draw}  \\
\texttt{( ) egendefinerad}  \\
\textit{\small Om egen, ge kort beskrivning:}\\
%\dotfill  \\
\bottomrule
\end{tabular}
\end{table}


\ChapterUnnum{Förord} 
\lipsum[1-3]
\mainmatter
\tableofcontents

\part{Om kursen}      %%%%%%%%%%%%%%%%%%%%%%%%%%%%%

\ChapterUnnum{Kursens arkitektur}
\resizebox{\columnwidth}{!}{%
\begin{tabular}{l|l|l|l|l|l|l}
\textit{W} & \textit{Datum} & \textit{Lp V} & \textit{Tema} & \textit{Förel} & \textit{Resurstid} & \textit{Lab} \\ \hline \hline
W01 & 29/8-2/9    & Lp1V1 & Introduktion      & F01 F02 & Ö01        & Lab01     \\
W02 & 5/9-9/9     & Lp1V2 & Kodstrukturer     & F03 F04 & Ö02        & --        \\
W03 & 12/9-16/9   & Lp1V3 & Funktioner        & F05 F06 & Ö03        & Lab02     \\
W04 & 19/9-23/9   & Lp1V4 & Typer, Arv        & F07 F08 & Ö04        & Lab03     \\
W05 & 26/9-30/9   & Lp1V5 & Datastrukturer    & F09 F10 & Ö05        & Lab04     \\
W06 & 3/10-7/10   & Lp1V6 & Sekvenser         & F11 F12 & Ö06        & Lab05     \\
W07 & 10/10-14/10 & Lp1V7 & Mönster, Undantag & F13 F14 & Ö07        & Lab06     \\
W08 & ksdatum     & TP1   & KONTROLLSKRIVN.   & --      & --         & --        \\
W09 & 31/10-4/11  & Lp2V1 & Java + Scala      & F15 F16 & Ö08        & Lab07     \\
W10 & 7/11-11/11  & Lp2V2 & Matriser          & F17 F18 & Ö09        & Lab08     \\
W11 & 14/11-18/11 & Lp2V3 & Sortering         & F19 F20 & Ö10        & Lab09     \\
W12 & 21/11-25/11 & Lp2V4 & Web, Android      & F21 F22 & Ö11        & Lab10     \\
W13 & 28/11-2/12  & Lp2V5 & Trådar            & F23 F24 & Ö12        & Lab11     \\
W14 & 5/12-9/12   & Lp2V6 & Designexempel     & F25 F26 & Uppsamling & Inl.Uppg. \\
W15 & 12/12-16/12 & Lp2V7 & Tentaträning      & F27 F28 & Extenta    & --        \\
W16 & tentadatum  & TP2   & TENTAMEN          & --      & --         & --        \\
\end{tabular}

}
\ChapterUnnum{Anvisningar}

\SectionUnnum{Föreläsningar}
\SectionUnnum{Övningar}
\SectionUnnum{Laborationer}
\SectionUnnum{Resurstider}
\SectionUnnum{Kontrollskrivning}
\SectionUnnum{Tentamen}

\ChapterUnnum{Hur lära att programmera?}
\SectionUnnum{Lära genom att göra}
\SectionUnnum{Vilken är din lärandestil?}

\ChapterUnnum{Hur bidra till kursmaterialet?}

\part{Moduler} %%%%%%%%%%%%%%%%%%%%%%%%%%%%%

%!TEX root = ../compendium.tex

\chapter{Introduktion}

\begin{itemize}[nosep]
\item om kursen
\item sekvens
\item alternativ
\item repetition
\item abstraktion
\item programmeringsparadigmer
\item editera-kompilera-exekvera
\item datorns delar
\item virtuell maskin
\item värde
\item uttryck
\item variabel
\item typ
\item tilldelning
\item val
\item var
\item alternativ
\item if
\item else
\item true
\item false
\item logik
\end{itemize}

\section{Vad är programmering?}
\lipsum

\section{Hur fungerar en dator?}
\lipsum

%!TEX root = ../compendium.tex

\Exercise{Hello}

\begin{Goals}
\item Lär dig detta
\item Lär dig och detta
\end{Goals}

\begin{Preparations}
\item Läs detta
\item Läs och detta
\end{Preparations}

\BasicTasks

\TaskSection{Avdelning ditten}

\Task Starta Scala REPL och skriv ut en sträng. Om du inte har Scala installerad på din maskin, se installationsanvisningar i Kapitel~\ref{installScala}

\begin{REPL}
$ scala
Welcome to Scala version 2.11.7 (Java HotSpot(TM) 64-Bit Server VM, Java 1.8.0_66).
Type in expressions to have them evaluated.
Type :help for more information.

scala> println("hej")
\end{REPL}

\Subtask Vad händer?

\Subtask Vad händer sen?


\Task Gör sedan detta och detta. 

\Subtask Vad händer?

\Subtask Vad händer sen?

\Task Gör sedan detta och detta. 

\TaskPen Gör sedan detta och detta med papper och penna.

\TaskSection{Avdelning datten}
\lipsum[7]

\subsection{Extrauppgifter: öva mer på grunderna}
\lipsum[2]


\subsection{Fördjupningsuppgifter: avancerad nivå}
\lipsum[2]
%!TEX root = ../compendium.tex

\Lab{Quiz}

\subsubsection{Mål}
\begin{itemize}[nosep]
\item Lär dig detta
\item Lär dig och detta
\end{itemize}

\subsubsection{Förberedelser}
\begin{itemize}[nosep]
\item Läs detta
\item Läs och detta
\end{itemize}

\subsection{Obligatoriska uppgifter}


\Task Gör först detta

\Task Gör sedan detta

\subsection{Frivilliga extrauppgifter}

\Task Gör först detta

\Task Gör sedan detta





\chapter{Kodstrukturer}
\lipsum
%!TEX root = ../compendium.tex

\Exercise{Hello}

\subsubsection{Mål}
\begin{itemize}[nosep]
\item Lär dig detta
\item Lär dig och detta
\end{itemize}

\subsubsection{Förberedelser}
\begin{itemize}[nosep]
\item Läs detta
\item Läs och detta
\end{itemize}

\subsection{Grundläggande uppgifter}

\subsubsection{Avdelning ditten}

\Task Starta Scala REPL och skriv ut en sträng. Om du inte har Scala installerad på din maskin, se installationsanvisningar i Kapitel~\ref{installScala}

\begin{REPL}
$ scala
Welcome to Scala version 2.11.7 (Java HotSpot(TM) 64-Bit Server VM, Java 1.8.0_66).
Type in expressions to have them evaluated.
Type :help for more information.

scala> println("hej")
\end{REPL}

\noindent Vad händer?


\Task Gör sedan detta och detta. 

\Task Gör sedan detta och detta. 

\TaskPencil Gör sedan detta och detta med papper och penna.

\subsubsection{Avdelning datten}
\lipsum[7]

\subsection{Extrauppgifter: öva mer på grunderna}
\lipsum[2]


\subsection{Fördjupningsuppgifter: överkurs}
\lipsum[2]
%!TEX root = ../compendium.tex

\Lab{Quiz}

\subsubsection{Mål}
\begin{itemize}[nosep]
\item Lär dig detta
\item Lär dig och detta
\end{itemize}

\subsubsection{Förberedelser}
\begin{itemize}[nosep]
\item Läs detta
\item Läs och detta
\end{itemize}

\subsection{Obligatoriska uppgifter}


\Task Gör först detta

\Task Gör sedan detta

\subsection{Frivilliga extrauppgifter}

\Task Gör först detta

\Task Gör sedan detta





\part{Verktyg}     %%%%%%%%%%%%%%%%%%%%%%%%%%%%%

\appendix

\chapter{Terminalfönster och kommandoskal}

\section{Vad är ett terminalfönster?}

I ett terminalfönster kan man skriva kommandon som till exempel kör program och hanterar filer på din dator. När man programmerar använder man ofta terminalkommando för att kompilera och exekvera sina program.   
 
\subsubsection{Terminal i Linux}

\subsubsection{PowerShell i Microsoft Windows}
Microsoft Windows är inte Unix-baserat, men i kommandotolken PowerShell finns alias definierat för en del vanliga unix-kommandon. Du startar Powershell t.ex. genom att genom att trycka på Windows-knappen och skriva \texttt{powershell}.

\subsubsection{Terminal i Apple OS X}
Apple OS X är ett Unix-baserat operativsystem. Många kommandon som fungerar under Linux fungerar också under Apple OS X.

\section{Några viktiga terminalkommando}

\chapter{Editera}
\section{Vad är en editor?}
\section{Välj editor}

\chapter{Kompilera och exekvera}
\section{Vad är en kompilator?}
\section{Java JDK}
\subsection{Installera Java JDK}
\section{Scala}
\subsection{Installera Scala-kompilatorn}
\section{Read-Evaluate-Print-Loop (REPL)}
För många språk, t.ex. Scala och Python, finns det en interaktiv tolk som gör det möjligt att exekvera enstaka programrader och direkt se effekte. En sådan tolk kallas Read-Evaluate-Print-Loop eftersom den läser en rad i taget och översätter till maskinkod som körs direkt.    
\subsection{Scala REPL}
\subsubsection{Kommandon i REPL}
:paste

Kortkommandon: Ctrl+K etc.

\chapter{Dokumentation}
\section{Vad gör ett dokumentationsverktyg?}
\section{scaladoc}
\section{javadoc}

\chapter{Integrerad utvecklingsmiljö}
\section{Vad är en IDE?}
\section{ScalaIDE och Eclipse}
\subsection{Installera ScalaIDE}
\section{Handledning ScalaIDE}

\chapter{Byggverktyg}
\section{Vad gör ett byggverktyg?}
\section{Byggverktyget sbt}
\subsubsection{Installera sbt}

\chapter{Versionshantering}
\section{Vad är versionshantering?}
\section{Versionshanteringsverktyget git}
\subsubsection{Installera git}



\end{document}

