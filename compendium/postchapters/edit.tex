%!TEX encoding = UTF-8 Unicode
%!TEX root = ../compendium.tex

\chapter{Editera}\label{appendix:edit}
\section{Vad är en editor?}

En editor används för att redigera programkod. Det finns många olika editorer att välja på. Erfarna utvecklare lägger ofta mycket energi på att lära sig att använda favoriteditorns kortkommandon och specialfunktioner, eftersom detta påverkar stort hur snabbt kodredigeringen kan göras. 

En bra editor har \textbf{syntaxfärgning} för språket du använder, så att olika delar av koden visas i olika färger. Då går det mycket lättare att läsa och hitta i koden. 

Nedan listas några viktiga funktioner som man använder många gånger dagligen när man kodar:

\begin{itemize}
\item \textbf{Navigera}. Det finns flera olika sätt att flytta markören och bläddra genom koden. Alla editorer erbjuder sökmöjligheter, och de flesta editorer har även mer avancerade sökfunktioner där kodmönster kan identifieras och multipla sökträffar markeras över flera kodfiler. 

\item \textbf{Markera}. Att markera kod kan göras på många sätt: med piltangenter+Shift, med olika speciella menyalternativ, med mus + dubbelklick eller trippelklick, etc. I vissa editorer finns även möjlighet att ha multipla markörer så att flera rader kan editeras samtidigt.

\item \textbf{Kopiera}. Genom Copy-Paste slipper du du skriva samma sak många gånger. Kortkommandona Ctrl+C för Copy och Ctrl+V för Paste sitter i fingrarna efter ett tag. Man ska dock vara medveten om att det lätt blir fel när man kopierar en stor del som sedan ska ändras lite; många Copy-Paste-buggar kommer av att man inte är tillräckligt noggrann och ofta är det bättre att skriva från grunden i stället för att kopiera så att du hinner tänka efter medan du skriver.

\item \textbf{Klipp ut}. Genom Ctrl+X för Cut och Ctrl+V för Paste, kan du lätt flytta kod. Att skriva kod är en stegvis process där man gör många förändringar under resans gång för att förbättra och vidareutveckla koden. Att flytta på kod för att skapa en bättre struktur är mycket vanligt.

\item \textbf{Formatering}. Med indragningar, radbrytningar och nästlade block i flera nivåer får koden struktur. Många editorer kan hjälpa till med detta och har speciella kortkommandon för att ändra indragningsnivå inåt eller utåt. 

\item \textbf{Parentesmatchning}. Olika former av parenteser, \code+ ( { [ ) } ] +,  behöver matchas för att koden ska fungera; annars går kompilatorn ofta helt vilse och konstiga felmeddelanden kan peka på helt fel plats i koden. En bra kodeditor kan hjälpa dig att markera vilka parentespar som hör ihop så att du undviker att spendera för mycket tid med att leta efter en parentes som saknas eller är står i vägen.
    
\end{itemize}

I en integrerad utvecklingsmiljö, en s.k. IDE, (se appendix \ref{appendix:ide}) finns en inbyggd editor som, tack vare ett mer intimt samarbete med kompilatorn, kan erbjuda ännu fler avancerade funktioner som hjälper dig i kodarbetet. Men även när du lärt dig använda en IDE kommer du fortfarande ha stor nytta av en ''vanlig'' editor. Ofta har man flera terminalfönster igång, tillsammans med flera editorfönster och en IDE. 

\section{Välj editor}

I tabell \ref{edit:popular-editors} visas en lista med några populära editorer. Det är en stor fördel om din favoriteditor finns på flera plattformar så att du har nytta av dina förvärvade färdigheter när du behöver växla mellan olika operativsystem. 

Om du inte vet vilken du ska välja, börja med \textit{gedit}, som inte är så avancerad, men därför lätt att komma igång med. När du sedan är redo att investera din lärotid i en mer avancerad editor rekommenderas \textit{Atom}, eftersom den är öppen, gratis och finns för Linux, Windows och macOS. 

Det är är också bra att lära sig åtminstone de mest basala kommandona i editorn \textit{vim} eftersom denna  editor kan köras direkt i terminalen, även vid fjärrinloggning, och finns förinstallerad i de flesta Linux-system.


\begin{table}[t]

\renewcommand{\arraystretch}{1.25}

\begin{tabular}{@{}r | p{0.75\textwidth}}
\textit{Editor} & \textit{Beskrivning} \\ \hline

Gedit & öppen, fri och gratis; lätt att lära men inte så avancerad; finns för Linux, Windows \& macOS; är förinstallerad på LTH:s Linux-datorer och startas med kommandot \verb+gedit+ i ett terminalfönster\newline  
 \url{https://wiki.gnome.org/Apps/Gedit} \\

Atom & öppen, fri och gratis; finns för Linux, Windows, \& macOS; är förinstallerad på LTH:s Linux-datorer och startas med kommandot \verb+atom+ i ett terminalfönster; öppenkällkodsprojektet startades nyligen av GitHub och därför är Atom-editorn ännu inte lika mogen som övriga i denna lista\newline \url{https://atom.io/} \newline 
Installera Ensime som ger kraftfullt Scala-stöd i Atom: \newline \url{http://ensime.github.io/editors/atom}\\


Vim & öppen, fri och gratis; lång historik, hög inlärningströskel; finns för Linux, Windows, \& Mac; är förinstallerad på LTH:s Linux-datorer och startas med kommandot \verb+vim+ i ett terminalfönster\newline \url{http://www.vim.org/} \\

Emacs & öppen, fri och gratis; lång historik, hög inlärningströskel; finns för Linux, Windows, \& Mac; är förinstallerad på LTH:s Linux-datorer och startas med kommandot \verb+emacs+ i ett terminalfönster\newline \url{http://www.gnu.org/software/emacs/} \\

Sublime Text& stängd kod; gratis att prova på, men programmet föreslår då och då att du köper en licens; finns för Windows, Mac, Linux. \newline
 \url{http://www.sublimetext.com/3} \\


Notepad++ & öppen, fri och gratis; finns endast för Windows; \newline \url{https://notepad-plus-plus.org/} \\


Textwrangler & stängd kod, gratis; lätt att lära men inte så avancerad; finns endast för macOS  
\newline \url{http://www.barebones.com/products/textwrangler/} \\

\end{tabular}
    \caption{Några populära editorer. Om du inte vet vilken du ska välja, börja med att installera Gedit.}
    \label{edit:popular-editors}
\end{table}