%!TEX encoding = UTF-8 Unicode
%!TEX root = ../compendium.tex

\chapter{Nyckelord}\label{appendix:keywords}

\section{Vad är ett nyckelord ord?}

Nyckelord är ord i ett programmeringsspråk som som har speciell betydelse och reserverade för endast ett användningsområde. Nyckelord kallas även \emph{reserverade ord}\footnote{Läs mer här: \href{https://en.wikipedia.org/wiki/Reserved\_word}{en.wikipedia.org/wiki/Reserved\_word}}. 
Man kan till exempel inte använda nyckelordet \code{def} som namn på en variabel. Nyckelord ges ofta en speciell färg av de kodeditorer som erbjuder \emph{syntaxstyrd färgning}. 

\section{Nyckelord i Scala} 

\begin{Code}
abstract    case        catch       class       def
do          else        extends     false       final
finally     for         forSome     if          implicit
import      lazy        macro       match       new
null        object      override    package     private
protected   return      sealed      super       this
throw       trait       try         true        type
val         var         while       with        yield
_    :    =    =>    <-    <:    <%     >:    #    @	
\end{Code}


\section{Nyckelord i Java}

Here is a list of keywords in the Java programming language. You cannot use any of the following as identifiers in your programs. The keywords const and goto are reserved, even though they are not currently used. true, false, and null might seem like keywords, but they are actually literals; you cannot use them as identifiers in your programs.

\begin{Code}[language=Java]
abstract 	continue 	for 	new 	switch
assert *** 	default 	goto * 	package 	synchronized
boolean 	do 	if 	private 	this
break 	double 	implements 	protected 	throw
byte 	else 	import 	public 	throws
case 	enum **** 	instanceof 	return 	transient
catch 	extends 	int 	short 	try
char 	final 	interface 	static 	void
class 	finally 	long 	strictfp ** 	volatile
const * 	float 	native 	super 	while
* 	  	not used
** 	  	added in 1.2
*** 	  	added in 1.4
**** 	  	added in 5.0
\end{Code}

