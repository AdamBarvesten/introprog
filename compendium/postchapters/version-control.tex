%!TEX encoding = UTF-8 Unicode
%!TEX root = ../compendium.tex


\chapter{Versionshantering och kodlagring}

\section{Vad är versionshantering?}

\textbf{Versionshantering}\footnote{\href{https://en.wikipedia.org/wiki/Version_control}{en.wikipedia.org/wiki/Version\_control}} \Eng{version control eller revision control} av mjukvara innebär att hålla koll på olika versioner av koden i ett utvecklingsprojekt allteftersom koden ändras. Versionshantering är en deldisciplin inom \textbf{konfigurationshantering} \Eng{software configuration managament} som inbegriper allt i processen för att identifiera, besluta, genomföra och följa upp ändringar.

En viktig del av versionshantering är att \textit{lagra} olika versioner av koden allt eftersom den utvecklas, så att tidigare versioner kan \textit{återskapas} vid behov. Ett bra verktygsstöd och en väldefinierad arbetsprocess för versionshanteringen, som alla i utvecklingsprojektet följer, möjliggör att flera utvecklare kan \textit{arbeta parallellt} med att sammanfoga \Eng{merge} varandras tillägg och ändringar i den gemensamma kodbasen utan att det blir kaos och förvirring.

God versionshantering är helt avgörande för utvecklarnas produktivitet, speciellt för stora projekt med många utvecklare som jobbar parallellt mot en omfattande kodbas med många olika interna och externa komponenter. 
Men även ett litet projekt med en enda utvecklare kan ha god nytta av ett versionshanteringsverktyg och ett disciplinerat förfarande för att namge versioner, t.ex. för att kunna återskapa tidigare versioner av projektets olika kodfiler när en ändring visar sig mindre lyckad.   

Det finns flera olika modeller för hur kodlagringen sker:
\begin{itemize}
\item \textbf{lokal}; alla utvecklare jobbar i samma, lokala filsystem där alla olika versioner lagras.
\item \textbf{centraliserad}; ett repositorium (förk. repo), alltså en databas med koden, finns centralt på en server som alla jobbar mot med hjällp av en versionshanteringsklient.
\item \textbf{distribuerad}; alla utvecklare har sitt eget lokala repo och varje utvecklare initierar enskilt delning av ändringar mellan olika repo. 
\end{itemize}


\section{Versionshanteringsverktyget Git}

Det finns många olika versionshanteringsverktyg\footnote{\href{https://en.wikipedia.org/wiki/List_of_version_control_software}{https://en.wikipedia.org/wiki/List\_of\_version\_control\_software}}
 som använder olika modeller för kodlagring; lokal, centraliserad, distribuerad eller kombinationer därav. 
På senare tid har verktyget \textbf{Git}\footnote{\href{https://en.wikipedia.org/wiki/Git_(software)}{https://en.wikipedia.org/wiki/Git\_(software)}} fått en stark ställning, speciellt i öppenkällkodsvärlden. Git utvecklades ursprungligen av Linus Torvalds för att versionshantera Linuxkärnan, men har växt till ett omfattande öppenkällkodsprojekt med stor spridning och många användare och bidragsgivare. 

Git är skapad för \textbf{distribuerad} versionshantering där var och en kan jobba snabbt och smidigt i sitt eget lokala repo, utan att behöva vänta på att en klient ska synkronisera koden med ett centralt repo på en server över nätverket. Ändringar delas mellan repo på begäran ev enskilda utvecklare. 

Varje ny version av koden lagras som en avgränsad mängd ändringar sedan förra versionen, en s.k. \textbf{commit}%
\footnote{På svenska kan t.ex. ''inlämning'' användas, men låneordet commit är redan etablerat.}%
, och hanteras internt av Git i en lokal databas i katalogen \code{.git} som ligger överst i din projektkatalog. Genom olika kommandon i terminalen, eller via en klient med ett grafiskt användargränssnitt, kan din kod överföras till och från den lokala koddatabasen, alternativt delas med andra repon via nätet. 

Det finns en välskriven bok kallad \textit{''Pro Git''} som förklarar Git på djupet och är tillgänglig fritt här: 
\url{https://git-scm.com/book/en/v2}.
Läs kapitel 1 och 2 så får du en bra grund att stå på. 

Dessa termer är bra att kunna utantill innan du kör igång med Git:
\newcommand{\TermItem}[3]{\item \textbf{#1} (\textit{substantiv}: #2, \textit{verb}: #3).}
\begin{itemize}

\item \textbf{repo} (\textit{substantiv}: ett repositorium, \textit{eng. a repository}) En koddatabas med ändringshistorik. 

\TermItem{commit}{en inlämning}{att lämna in} 
  En avgränsad mängd nya ändringar lämnas in i det lokala repot. Repots ändringshistorik utgörs av sekvensen av alla inlämningar.

\TermItem{push}{en leverans}{att leverera, att trycka upp} En eller flera inlämningar trycks upp till ett annat repo.

\TermItem{pull}{en hämtning}{att hämta, att dra ner} En eller flera inlämningar dras ner från ett annat repo.

\TermItem{merge}{en ihopslagning}{att sammanfoga} En eller flera inlämningar slås samman till en ny inlämning. 

\item \textbf{merge conflict} (\textit{substantiv}: en sammanfogningskonflikt, \textit{eng. a merge conflict}) Problem vid sammanfogning; ändringar kan inte enkelt sammanfogas på ett entydigt sätt.

\item \textbf{pull request} (förk. PR, \textit{substantiv}: en hämtningsbegäran, \textit{verb}: att begära en hämtning). Utvecklare A ber en annan utvecklare B att hämta en eller flera inlämningar från A:s repo och sammanfoga med B:s repo.

\end{itemize}

\subsection{Installera git}\label{subsection:install-git}

Git finns förinstallerat på LTH:s Linuxdatorer. Du kan kolla om Git redan finns på din maskin genom att skriva \code{git help} i terminalen. 

Det finns bra instruktioner om hur du installerar Git på din egen maskin här: \url{https://git-scm.com/book/en/v2/Getting-Started-Installing-Git}

Om du vill ha en Git-klient med grafiskt användargränssnitt finns det många att välja på, se här:  \url{https://git-scm.com/downloads/guis} 

Om du inte vet vilken du ska välja, prova GitKraken som är gratis (men stängd) och finns för alla plattformar: \url{https://www.gitkraken.com/}.


\subsection{Anpassa Git}

Innan du börjar använda git, konfigurera ditt användarnamn och din email med nedan terminalkommando, där du anger ditt användarnamn i stället för \code{fornamnefternamn} och din mejladress i stället för \code{mejladr@plats.se}:
\begin{REPLnonum}
$ git config --global user.name fornamnefternamn
$ git config --global user.email mejladr@plats.se
\end{REPLnonum}
Det är bra att välja \textit{ett} användarnamn, för \textit{alla} repo, även kodlagringsplatser på nätet; förslagsvis \code{fornamnefternamn} utan svenska tecken,  så att du blir lätt att känna igen, speciellt om du jobbar med öppen källkod där ditt namn kommer associerat med alla de kodbidrag du gör under ditt yrkesliv.

Läs mer om hur du gör andra inställningar här, t.ex. hur du anger vilken editor som git startar när du ska skriva commit-beskrivningar: \\ \url{https://git-scm.com/book/en/v2/Getting-Started-First-Time-Git-Setup}
  
  
\subsection{Använda git}

Nedan listas några vanliga terminalkommandon i Git.

\begin{itemize}[leftmargin=*]

\item Skapa ett repo i en katalog:
\begin{REPLnonum}
$ cd myproject
$ git init
\end{REPLnonum} 

\item Se vilka filer som ändrats och ännu ej lämnats in:
\begin{REPLnonum}
$ git status
$ git status -s
\end{REPLnonum} 

\item Se vilka ändringar som gjorts i filer som ännu ej lämnats in:
\begin{REPLnonum}
$ git diff 
\end{REPLnonum} 

\item Se vilka inlämningar som finns i ändringshistoriken:
\begin{REPLnonum}
$ git log 
$ git log --oneline -5
\end{REPLnonum} 

\item Lägg till filer som ska ingå i nästa inlämning och gör sedan inlämningen; ge inlämningen en bra beskrivning som förklarar vad inlämningen omfattar:
\begin{REPLnonum}
$ git add *.scala
$ git commit -m 'initial project version'
\end{REPLnonum} 

\item Ångra alla tillägg inför inlämning (ändringarna finns kvar och kan läggas till igen om du vill):
\begin{REPLnonum}
$ git reset 
\end{REPLnonum} 

\item Du kan skippa de senaste, ännu ej commitade, ändringar i filen \code{filename}, och göra ''\textit{undo}'', med kommandot \code{git checkout} på filen enligt nedan. Gör bara detta om du är helt säker på att du vill ångra dina senaste ändringar.
\\ \mbox{\colorbox{red!30}{VARNING!} Dina senaste ändringar i filen förloras för alltid; kan ej ångras!}   
\begin{REPLnonum}
$ git checkout filename 
\end{REPLnonum} 

\item Man vill förhindra versionshantering av vissa filer, t.ex. binärkodsfiler så som \code{.class}-filer och andra genererade filer. Detta gör du genom att skapa en fil med namnet \code{.gitignore} och lägga in filändelser enligt nedan syntax, där \code{**/} avser alla kataloger och underkataloger och \code{*} kan vara vilken början på ett filnamn som helst. Symbolen \code{#} föregår en kommentarsrad.
\begin{Code}[language=]
# this is my .gitignore

# Java / Scala
**/*.class

# Sbt
**/target

\end{Code} 


\end{itemize}
 

\clearpage 
  
\section{Kodlagringsplatser på nätet}\label{section:code-hosting}

Många utvecklare använder kodlagringsplatser på nätet (''i molnet''). \Eng{code hosting} för att underlätta samarbete kring kod och för att dela med sig av öppen källkod. Det finns många olika kodlagringsplatser som kan användas gratis under vissa förutsättningar eller mot betalning med tillhörande extratjänster. 

Nedan beskrivs några vanliga nätplatser för öppen och sluten kodlagring, som alla är Git-baserade:

\begin{itemize}
\item  \textbf{GitHub}, \url{https://github.com}, är en av de mest populära kodlagringsplatserna för öppen källkod, men har även blivit en populär plats för jobbsökande utvecklare att visa upp sina  kodarbetsprover för framtida arbetsgivare. GitHub är gratis att använda för dig som privatperson om du låter ditt repo vara öppet att läsa för alla. Det kostar pengar om du vill ha ett slutet repo. Många företag betalar GitHub för att lagra sin stängda kod med tilläggstjänster för att testa, bygga och driftsätta kod etc. Koden som styr själva kodlagringsplatsen GitHub är stängd, till skillnad från GitLab.

\item \textbf{BitBucket}, \url{https://bitbucket.org}, är en populära kodlagringsplats både för öppen och stängd källkod och drivs av det australiensiska företaget Atlassian. Det är gratis för privatpersoner och små team att ha både öppna och slutna repon, men bara om det är få bidragsgivare. Kostnader tillkommer om antalet bidragsgivare kommer över en viss nivå. Universitetsanställda och studenter kan få mer gynnsamma villkor efter ansökan. Atlassian erbjuder en hel verktygssvit för att hantera buggar och samarbeta över nätet. BitBucket stödjer, förutom Git, även andra versionshanteringsverktyg.

\item \textbf{GitLab}, \url{https://gitlab.com}, erbjuder gratis kodlagring för öppen källkod, men det är även gratis för privatpersoner och gemenskapsprojekt att ha stängda repo. Företag kan betala för stängd kodlagring med extratjänster för att testa, bygga och driftsätta kod etc. GitLab är i sig ett öppenkällkodsprojekt och koden som styr kodlagringsplatsen är öppen och fri. Detta innebär att du själv kan ladda ner koden och starta en kodlagringsplats. LTH har en GitLab-baserad kodlagringsplats här: \url{https://git.cs.lth.se}

\end{itemize}

\subsubsection{Använda kodlagringsplatser}

Det är bra att registrera ditt användarnamn, förslagsvis \code{fornamnefternamn} som ett ord utan svenska tecken, på någon eller alla av ovan sajter, dels för att paxa ditt namn och dels för börja samarbeta med utvecklarvänner världen över. Om du inte vet vilken du ska välja, börja med \url{https://github.com}. Om du vill ha både öppna och slutna repon gratis, testa \url{https://gitlab.com}. 

Med en Git-baserad kodlagringsplats för du möjlighet att synka ditt lokala repo mot en server på nätet med hjälp av \code{git}-kommandon i terminalen eller via en Git-klient med grafiskt användargränssnitt, se avsnitt \ref{subsection:install-git}. 

Innan du börjar använda en kodlagringsplats är det bra att sätta sig in i begreppen nedan.

\begin{itemize}
\TermItem{clone}{en klon är kopia av ett (nätlagrat) repo}{att klona, att skapa en kopia} Genom att klona ett repo som ligger på en nätlagringsplats kan du bygga, undersöka och vidareutveckla koden lokalt på din dator. Om du har rättigheter att lämna in kod till det centrala orginalet kan du pusha dina commits direkt via terminalkommando eller Git-klient.

\TermItem{fork}{en förgrening av ett helt repo}{att förgrena ett repo, att ''forka''} Genom att förgrena ett repo skapar du en kopia, normalt även den nätlagrad på en kodlagringsplats, som du kan utveckla separat från orginalet. Det blir då möjligt för dig att lämna in ändringar och trycka upp dem, även om du inte har rättigheter att leverera (''pusha'') till originalet. Gör en ändringsbegäran (Pull Request, PR) om du vill bidra med dina ändringar, så kan ägaren av orginalet sedan välja att sammanfoga (''merga'') dina ändringar med orginalet. Många nätlagringsplatser, så som GitHub, har en speciell knapp som du trycker på för att enkelt skapa en fork av ett repo under din användare. 

\item \textbf{upstream} (\textit{preposition}: uppströms, \textit{substantiv}: uppströmsrepo) Ett uppströmsrepo utgör orginal till ett förgrenat repo (en ''fork''). 
\begin{itemize}[noitemsep,nolistsep]

\item Här beskrivs hur du länkar en förgrening uppströms: \\ 
{\small\url{https://help.github.com/articles/configuring-a-remote-for-a-fork/}}

\item Här beskrivs hur du synkar en förgrening uppströms:\\
{\small\url{https://help.github.com/articles/syncing-a-fork/}}

\end{itemize}

\end{itemize}

Om du vill bidra till ett öppenkällkodsprojekt, börja med att forka repot på kodlagringsplatsen och sedan klona repot till din lokala dator. Därefter kan du commita ändringar och pusha till din fork och slutligen gör en pull request från din fork till upstream. Läs om hur ett bidrag kan gå till i avsnitt \ref{section:OSS-contribution-example}.

Här följer några användbara kommandon:

\begin{itemize}
\item Skapa en lokal kopia av ett fjärran \Eng{remote} repo; här visas hur du klonar kursens repo från GitHub:
\begin{REPLnonum}
$ git clone --depth 1 https://github.com/lunduniversity/introprog
\end{REPLnonum} 

\item Dra ner nya inlämningar från ett fjärran repo:
\begin{REPLnonum}
$ git pull 
\end{REPLnonum} 

\item Trycka upp nya lokala inlämning till ett fjärran repo:
\begin{REPLnonum}
$ git push 
\end{REPLnonum} 

\end{itemize}


