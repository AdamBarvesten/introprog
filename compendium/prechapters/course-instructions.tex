%!TEX encoding = UTF-8 Unicode
%!TEX root = ../compendium.tex

\ChapterUnnum{Anvisningar}

Detta kapitel innehåller anvisningar och riktlinjer för kursens olika delar. Läs noga så att du inte missar obligatoriska regler och annan viktig information om syftet bakom kursmomenten och vad som förväntas av dig. 

\SectionUnnum{Samarbetsgrupper}

Systemutveckling sker i samverkan mellan många utvecklare i ett globalt sammanhang under ständig teknologisk evolution, och lärandet om programmering fördjupas om det sker i dialog med andra. Därför är kursdeltagarna indelade i \emph{samarbetsgrupper} om 4-6 personer där deltagarna samarbetar med syftet att alla i gruppen ska nå så långt som möjligt i sina studier och hjälpas åt att uppfylla lärandemålen. 

För att hantera och dra nytta av skillnader i förkunskaper är samarbetsgrupperna indelade så att deltagarnas har \emph{varierande förkunskaper} baserat på en förkunskapsenkät. De som redan har provat på att programmera får då chansen att träna på sin pedagogiska förmåga som är så viktig för systemutvecklare, medan de som ännu inte kommit lika långt kan dra nytta av gruppmedlemmarnas samlade kompetens i sitt lärande. Kompetensvariationen kommer att förändras under kursens gång då olika individer lär sig olika snabbt i olika skeden; de som till att börja med har ett försprång kan senare få kämpa mer för att komma vidare.

Samarbetsgrupperna organiserar själva sitt arbete och varje grupp får finna sina samarbetsformer. Här är några tips:

\begin{itemize}
\item Träffas så fort som möjligt i hela gruppen och lär känna varandra. Ju snabbare ni kommer samman som grupp i ett fungerande socialt sammanhang desto bättre. Ni kommer att ha nytta av denna investering under hela terminen och kanske under resten av er studietid. 
\item Kom överens om stående mötestider och mötesplatser. Det är viktigt med kontinuiteten i arbetet för att samarbetet i gruppen ska utvecklas och fördjupas. Träffas minst en gång i veckan. Ha en stående agenda, t.ex. en runda runt bordet där var och en berättar hur långt hen kommit och listar de begreppen som hen för tillfället behöver fokusera på. 
\item Hjälps åt att tillsammans identifiera och diskutera era olika individuella studiebehov och studieambitioner. När man ska lära sig att programmera stöter man på olika lärandetrösklar som man kan få hjälp att ta sig över av någon som redan är förbi tröskeln. Men det gäller då för den som hjälper att först förstå vad exakt det är som är svårt, eller vilka specifika pusselbitar som saknas, för att kunna underlätta för en medstudent att ta sig över tröskeln.
\item Var en schysst kamrat och agera professionellt, speciellt i situationer där gruppdeltagarna vill olika. Kommunicera på ett respektfullt sätt och sök konstruktiva kompromisser. Att utvecklas socialt är viktigt för er framtida yrkesutövning som systemutvecklare och i samarbetsgruppen kan du träna och utveckla din samarbetsförmåga.
\end{itemize}

\subsection*{Samarbetskontrakt}

Ni ska upprätta ett samarbetskontrakt redan under första veckan och visa för en handledare. Alla gruppmedlemmarna ska skriva under kontraktet. Handledaren ska också skriva under som bekräftelse på att ni visat kontraktet. 

Syftet med kontraktet är att ni ska diskutera igenom i gruppen hur ni vill arbeta och vilka grundläggande regler ni tycker är rimliga för arbetet. Ni bestämmer själva vad kontraktet ska innehålla. Nedan finns ett förslag på punkter som kan ingå i ett kontrakt. Ni kan ladda ner en kontraktsmall här om ni vill träna på att använda latex: \url{} 

\begin{tcolorbox}%[width=1.05\textwidth,  grow to right by=0.03\textwidth,grow to left by=0.03\textwidth,breakable, enhanced]
\subsubsection*{Samarbetsregler}
Vi lovar att göra vårt bästa för att följa dessa regler när vi samarbetar för att alla ska lära sig så mycket som möjligt: 
\begin{enumerate}
\item Komma i tid till gruppmöten
\item Vara väl förberedda genom självstudier inför gruppmöten
\item Hjälpa varandra att förstå, men inte ta över och lösa allt åt någon annan
\item Ha ett respektfullt bemötande även om vi har olika åsikter
\item Inkludera alla i gemenskapen
\item ...
\end{enumerate}
\end{tcolorbox}
\subsection*{Grupplaborationer}

\subsection*{Samarbetsbonus}

\SectionUnnum{Föreläsningar}

\SectionUnnum{Övningar}

\SectionUnnum{Laborationer}
 
\begin{itemize}
\item TODO!!! Skriv om pennan och ögat och bocken
\item TODO!!! Skriv att om man inte gör föreberedelserna hinner man inte labben på 2h
\end{itemize}

\SectionUnnum{Resurstider}

\SectionUnnum{Kontrollskrivning}

\SectionUnnum{Tentamen}