%!TEX encoding = UTF-8 Unicode
%!TEX root = ../lect-week06.tex

%%%

\begin{Slide}{Vad är en klass?}
\begin{itemize} 
\item En mall för att skapa objekt.
\item Objekt skapade med \code{new Klassnamn} kallas för\\  \Emph{instanser} av klassen \code{Klassnamn}.
\item En klass innehåller medlemmar \Eng{members}: 
  \begin{itemize} 
  \item \Emph{attribut}, kallas även fält \Eng{field}: \code{val}, \code{lazy val}, \code{var} 
  \item \Emph{metoder}, kallas även operationer: \code{def}
  \end{itemize}
\item Varje instans har sin uppsättning värden på attributen (fälten).
\end{itemize}

\end{Slide}

%\begin{Slide}{Designexempel: Klassen Complex}\small
%TODO:
%  \begin{itemize} 
%  \item Bygg upp \code{case class Complex(re: Double, im: Double)} steg för steg inspirerat av Pins3ed kap 6 i likhet med hur de gör med Rational
%  \item Illustrera följande begrepp: this (behövs i max(that)), method overloading behövs för att plussa med både Complex och Double
%  \item Till fördjupningsövning: dekorera Double med metoderna im och re samt (Double, Double) med metoden ir (för irrational) med implicit klass
%  \item Till extrauppgift: implementera klassen Polar(r, fi) med polära koordinater \url{https://sv.wikipedia.org/wiki/Pol%C3%A4ra_koordinater}
%  \end{itemize}
%\end{Slide}


\begin{Slide}{Specifikationer av klasser i Scala}\footnotesize
\begin{itemize}
\item Specifikationer av klasser innehåller information som \emph{den som ska implementera} klassen behöver veta.
\item Specifikationer innehåller liknande information som dokumentationen av klassen (scaladoc), som beskriver vad \emph{användaren} av klassen behöver veta.  
\end{itemize}
\begin{ScalaSpec}{Person}
/** Encapsulate immutable data about a Person: name and age. */ 
case class Person(name: String, age: Int = 0){
  /** Tests whether this Person is more than 17 years old. */
  def isAdult: Boolean = ???
}
\end{ScalaSpec}
\begin{itemize}
\item Specifikationer av Scala-klasser utgör i denna kurs ofullständig kod som kan kompileras utan fel. 
\item Saknade implementationer markeras med \code{???}
\item Kommentarer utgör krav på implementationen.
\end{itemize}

\end{Slide}


\begin{Slide}{Specifikationer av klasser och objekt}
\begin{ScalaSpec}{MutablePerson}
/** Encapsulates mutable data about a person. */
class MutablePerson(initName: String, initAge: Int){
  /** The name of the person. */
  def getName: String = ???
  
  /** Update the name of the Person */
  def setName(name: String): Unit = ???

  /** The age of this person. */
  def getAge: Int = ???

  /** Update the age of this Person */
  def setAge(age: Int): Unit = ???

  /** Tests whether this Person is more than 17 years old. */
  def isAdult: Boolean = ???

  /** A string representation of this Person, e.g.: Person(Robin, 25) */
  override def toString: String = ???
}
object MutablePerson {
  /** Creates a new MutablePerson with default age. */
  def apply(name: String): MutablePerson = ???
}
\end{ScalaSpec}

\end{Slide}

\ifkompendium
Man brukar inte använda \code{get} och \code{set} i metodnamn i Scala. Mer senare om principen om enhetlig access \Eng{uniform access principle} och hur man gör ''setters'' som möjliggör tilldelningssyntax.
\fi


\begin{Slide}{Specifikationer av Java-klasser}
\begin{itemize}\small
\item Specificerar signaturer för konstruktorer och metoder. 
\item Kommentarerna utgör krav på implementationen.  
\item Används flitigt på extentor i EDA016, EDA011, EDA017...
\item Javaklass-specifikationerna behöver kompletteras med metodkroppar och klassrubriker innan de kan kompileras.
\end{itemize}
\begin{JavaSpec}{class Person}
/** Skapar en person med namnet name och åldern age. */
Person(String name, int age);

/** Ger en sträng med denna persons namn. */
String getName();

/** Ändrar denna persons ålder. */
void setAge(int age);

/** Anger åldersgränsen för när man blir myndig. */
static int adultLimit = 18;
\end{JavaSpec}
\end{Slide}









