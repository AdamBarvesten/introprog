%!TEX encoding = UTF-8 Unicode
%!TEX root = ../lect-week06.tex

%%%

\ifkompendium\else

\Subsection{Grumligt-lådan}
\begin{Slide}{Grumligt-lådan}
Veckans skörd av lappar i ''grumligtlådan'':
\begin{multicols}{2}
\begin{lstlisting}[basicstyle=\SlideFontSize{7}{9},language=]
12	case class
8	Map och map
8	private, public
5	override
3	toString
3	kompanjonsobjekt
2	typparametrar [Int]
2	Specialfall, sekvensalgoritmer
\end{lstlisting}

\columnbreak

\begin{lstlisting}[basicstyle=\SlideFontSize{5}{6},language=]
1	lab pirates
1	Hur ska jag träna datastrukturer?
1	underscore i olika sammanhang
1	Stränginterpolator s"\$x" 
1	heap
1	Assume
1	Mutable / immutable
1	Vad är en typ och hur kan klass bli en typ?
1	tomma parenteser ()
1	skillnad mellan argument och parameter
1	pseudokod
1	Hur hitta buggar?
1	w04 datastrukturer
1	skillnad på olika parenteser \{[(
1	terminologi allmänt
1	uppdatering av variabler som refererar till varandra
1	val, lazy val, var
1	när använda terminal, editor, IDE?
1	Formattering/upplägg av kod, indrag, var ska objekt vara?
-	Bättre instruktioner på labbarna
-	bra med sammanfattning på slutet av föreläsningarna
\end{lstlisting}
\end{multicols}
\end{Slide}

\fi

