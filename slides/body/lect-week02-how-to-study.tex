%!TEX encoding = UTF-8 Unicode
%!TEX root = ../lect-week02.tex

\Subsection{Studieteknik}

\begin{Slide}{Hur studerar du?}
\begin{itemize}
\item Vad är bra \Emph{studieteknik}?
\item Hur lär \Alert{du} dig bäst? Olika personer har olika preferenser. 
\begin{itemize}
\item Ta reda på vad som funkar bäst för dig. 
\item En kombination av flera sinnen är bäst: läsa+prata+skriva... 
\item Aktivera dig! Inte bara passivt läsa utan också aktivt göra.
\end{itemize}

\item Hur skapa \Emph{struktur}? 
Du behöver ett sammanhang, ett \Emph{system av begrepp}, att \Alert{placera in} din nya kunskap i.
\item Hur uppbåda \Emph{koncentration}? Steg 1: Stäng av mobilen! 
\item Hur vara \Emph{disciplinerad}? Studier först, nöje sen!
\item Du måste \Emph{planera och omplanera} för att säkerställa \Alert{tillräckligt mycket egen pluggtid} då du är pigg och koncentrerad för att det ska funka! 
\item Programmering \Alert{kräver} en \Emph{pigg och koncentrerad hjärna}!
\end{itemize}
\end{Slide}


\begin{Slide}{Hur ska du studera programmering?}
\begin{itemize}
\item När du gör \Emph{övningarna}:
\begin{itemize}
\item Ta fram föreläsningsbilderna i pdf och kolla igenom dem.
\item Är det något i föreläsningsbilderna du inte förstår: ta upp det i samarbetsgrupperna eller på resurstiderna.
\item Om något är knepigt: 
\begin{itemize}
\item Hitta på egna REPL-experiment och undersök hur det funkar.
\item Följ ev. länkar i föreläsningsbilderna, eller googla själv på wikipedia, stackoverflow, ...
\end{itemize}
\end{itemize}

\item Innan du gör \Emph{laborationerna}:
\begin{itemize}
\item Kolla igenom målen för veckans \Alert{övning} och dubbelkolla så att du har uppnått dem.
\item Gör förberedelserna \Alert{i god tid innan} labben.
\item Läs igen \Alert{hela} labbinstruktionen \Alert{innan} labben.
\item Om du tror att du behöver det för att hinna med: \\ gör delar av labben redan innan redovisningstillfället.
\end{itemize}
\end{itemize}

\end{Slide}

\begin{Slide}{Det går inte att förstå allt på en gång!}
\begin{itemize}
\item Vi nosar på ett visst begrepp på ytan i en vecka ... 

\item ... för att i senare vecka återkomma till det, men djupare.

\item Förståelse kommer efter hand och kräver bearbetning.

\item Vi måste iterera begreppen innan vi kan nå djup.

\pause\item Det är svårt för dig nu att se vad som är \Alert{detaljer} som du inte ska hänga upp dig på, och vad som är \Emph{det viktiga} i detta läget. Men det kommer! Ha tålamod! 
\end{itemize}

\end{Slide}


\begin{Slide}{På rasten: träffa din samarbetsgrupp}
\begin{itemize}

\item Träffas i samarbetsgrupperna och bestäm/gör/diskutera:
\begin{enumerate}
\item När ska ni träffas nästa gång?
\item Bläddra igenom föreläsningsbilderna från w01 i pdf.
\item Vilka \Emph{koncept} är fortfarande (mest) \Alert{grumliga}? \\Alltså: Vilka koncept från förra veckan vill ni på nästa möte jobba mer med i gruppen för att alla ska förstå grunderna?
\end{enumerate}

\end{itemize}

\end{Slide}


%%%