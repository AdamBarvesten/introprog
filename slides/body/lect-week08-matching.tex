%!TEX encoding = UTF-8 Unicode
%!TEX root = ../lect-week08.tex

%%%


\begin{Slide}{TODO: Begrepp att förklara}
  Tänk igenom ordningen:
  \begin{itemize}
    \item java switch, scala match ... 
  \end{itemize}
\end{Slide}


\begin{Slide}{Javas switch-sats}
A switch in Java works with the byte, short, char, and int primitive data types. It also works with enumerated types (discussed in Enum Types), the String class, and a few special classes that wrap certain primitive types: Character, Byte, Short, and Integer
\end{Slide}


\begin{Slide}{Javas switch-sats}

\begin{Code}[language=Java]
public class Switch {
    public static void main(String[] args) {
        String favorite = "selleri";
        if (args.length > 0) {
            favorite = args[0];
        }
        System.out.println("Din favoritgrönsak: " + favorite);
        char firstChar = Character.toLowerCase(favorite.charAt(0));
        System.out.print("Jag tycker ");
        switch (firstChar) {
        case 'g': 
            System.out.println("gurka är gott!");
            break;
        case 't': 
            System.out.println("tomat är gott!");
            break;
        case 'b': 
            System.out.println("brocolli är gott!");
            break;
        default:
            System.out.println(favorite + " är äckligt!");
            break;
        }
    }
}
\end{Code}
\end{Slide}












