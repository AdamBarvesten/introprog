%!TEX encoding = UTF-8 Unicode
%!TEX root = ../lect-week05.tex

%%%

\Subsection{Vad är en sekvensalgoritm?}

\begin{Slide}{Vad är en sekvensalgoritm?}
\begin{itemize} 
\item En algoritm är en stegvis beskrivning av hur man löser ett problem. 
\item En sekvensalgoritm är en algoritm där dataelement i sekvens utgör en viktig del av problembeskrivningen och/eller lösningen.   

\item Exempel: sortera en sekvens av personer efter deras ålder.

\item Två olika principer:
\begin{itemize} 
\item Skapa \Emph{ny sekvens} utan att förändra indatasekvensen
\item Ändra \Emph{på plats} \Eng{in place} i den \Alert{förändringsbara} indatasekvensen
\end{itemize}
\end{itemize}

\end{Slide}


\Subsection{SEQ-COPY}

\begin{Slide}{Algoritm: SEQ-COPY}
\Emph{Pseudokod} för algoritmen SEQ-COPY som kopierar en sekvens, här en Array med heltal:\\
\noindent\hrulefill
\begin{algorithm}[H]
 \SetKwInOut{Input}{Indata}\SetKwInOut{Output}{Resultat}
 \Input{Heltalsarray $xs$} 
 \Output{En ny heltalsarray som är en kopia av $xs$. \\ \vspace{1em}}
 $result \leftarrow$ en ny array med plats för $xs.length$ element\\
 $i \leftarrow 0$  \\
 \While{$i < xs.length$}{
  $result(i) \leftarrow xs(i)$\\
  $i \leftarrow i + 1$\\
 }
 \Return $result$
\end{algorithm}
\noindent\hrulefill
\end{Slide}

\ifkompendium\else

\begin{Slide}{Implementation av SEQ-COPY med \texttt{while}}
\lstinputlisting[numbers=left]{../compendium/examples/workspace/w05-seqalg/src/seqCopy.scala}
\end{Slide}

\begin{Slide}{Implementation av SEQ-COPY med \texttt{for}}
\lstinputlisting[numbers=left]{../compendium/examples/workspace/w05-seqalg/src/seqCopyFor.scala}
\end{Slide}

\begin{Slide}{Implementation av SEQ-COPY med \texttt{for-yield}}
\lstinputlisting[numbers=left]{../compendium/examples/workspace/w05-seqalg/src/seqCopyForYield.scala}
\end{Slide}

\begin{Slide}{Implementation av SEQ-COPY i Java med \texttt{for}-sats}
\vspace{-0.6em}
\javainputlisting[numbers=left,numberstyle=,basicstyle=\fontsize{6.5}{8}\ttfamily\selectfont]{../compendium/examples/workspace/w05-seqalg/src/seqCopyForJava.java}
\end{Slide}


\Subsection{SEQ-INSERT-COPY}

\Subsection{SEQ-INSERT-IN-PLACE}


\Subsection{StringBuilder}
%Nannanannan Batman


\Subsection{Registrering}



\fi







