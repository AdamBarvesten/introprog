%!TEX encoding = UTF-8 Unicode
%!TEX root = ../lect-week03.tex


\Subsection{Deklarera funktioner}




\begin{Slide}{Deklarera funktioner}
\begin{itemize}
\item En parameter
\begin{REPLnonum}
scala> def öka(a: Int): Int = a + 1
scala> def öka(a: Int, b: Int) = a + b
\end{REPLnonum}
\item Två parametrar
\end{itemize}
Båda ovan funktioner kan finnas samtidigt: De är olika funktioner, eftersom parameterlistan har olika antal parametrar med olika typer, trots att de har samma namn. Detta kallas överlagring \Eng{overloading} av funktioner.
\end{Slide} 


\begin{Slide}{Funktioner med default-argument}
\begin{itemize}
\item 
\end{itemize}
\end{Slide} 


\begin{Slide}{Värdeanrop och namnanrop}
\begin{itemize}
\item 
\end{itemize}
\end{Slide} 

\begin{Slide}{Uppdelad parameterlista}
\begin{itemize}
\item 
\end{itemize}
\end{Slide} 


\begin{Slide}{Skapa din egen kontrollstruktur}
\begin{itemize}
\item 
\end{itemize}
\end{Slide} 


\Subsection{Använda funktioner}

\begin{Slide}{Funktioner med namngivna argument}
\begin{itemize}
\item 
\end{itemize}
\end{Slide} 


\begin{Slide}{Applicera funktioner på element i samlingar}
\begin{itemize}
\item 
\end{itemize}
\end{Slide} 

\begin{Slide}{Funktioner är äkta värden}
\begin{itemize}
\item 
\end{itemize}
\end{Slide} 



\begin{Slide}{Stegade funktioner, ''Curry-funktioner''}
\begin{itemize}
\item 
\end{itemize}
\end{Slide} 




\Subsection{Objekt}

\begin{Slide}
Objekt kan användas som moduler som samlar \texttt{val}, \texttt{var} och \texttt{def} som hör ihop.
\end{Slide}

\begin{Slide}{Vad är egentligen skillnaden mellan \texttt{val}, \texttt{var} och \texttt{def}?} 
\end{Slide} 

\begin{Slide}{Vad är ett tillstånd?} 
\end{Slide} 






