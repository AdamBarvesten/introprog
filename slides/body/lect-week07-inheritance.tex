%!TEX encoding = UTF-8 Unicode
%!TEX root = ../lect-week07.tex

%%%


%\begin{Slide}{TODO: Begrepp att förklara}
%  Tänk igenom ordningen:
%  \begin{itemize}
%    \item OO, arv, supertyp, subtyp, bastyp, polymorfism, ... 
%  \end{itemize}
%\end{Slide}


\begin{Slide}{Medlemmar och arv}\SlideFontSmall
\begin{multicols}{2}
Olika sorters medlemmar i \Emph{Scala}:
\begin{itemize}
\item \code{def}
\item \code{val}
\item \code{lazy val}
\item \code{var}
%\item \code{type}

\end{itemize}

\columnbreak

Olika sorters medlemmar i \Emph{Java}:
\begin{itemize}
\item variabel 
\item metod
\end{itemize}

\vspace{1em}

Variabler kan vara instansvariabler eller klassvariabler (nyckelord \jcode{static})

\end{multicols}

\pause
\begin{itemize}
\item Vid arv kan man överskugga \Eng{\code{override}} en medlem, så att medlemmen med samma namn i en subtyp får sin egen speciella skepnad.

\item När man konstruerar ett objektorienterat språk gäller det att man definierar sunda överskuggningsregler vid arv.
\end{itemize}
\end{Slide}


\begin{Slide}{Regler för \texttt{override} i Scala.} \SlideFontTiny
\label{slideW07:overriderules}
En medlem M1 i en supertyp får ersättas av en medlem M2 i en subtyp, givet reglerna:
\begin{enumerate}
\item M1 och M2 ska ha samma namn och typerna ska matcha.
\item \code{def} får bytas ut mot: \code{def}, \code{val}, \code{var}, \code{lazy val}
\item \code{val} får bytas ut mot: \code{val}, och om M1 är abstrakt mot en \code{lazy val}.
\item \code{var} får bara bytas ut mot en \code{var}.
\item \code{lazy val} får bara bytas ut mot en \code{lazy val}.
\item Om en medlem i en supertyp är abstrakt \emph{behöver} man inte använda nyckelordet \code{override} i subtypen. (Men det är bra att göra det ändå så att kompilatorn hjälper dig att kolla att du verkligen byter ut något.) 
\item Om en medlem i en supertyp är konkret \emph{måste} man använda nyckelordet \code{override} i subtypen, annars ges kompileringsfel.
\item M1 får inte vara \code{final}.
\item M1 får inte vara \code{private} eller \code{private[this]}, men kan vara \code{private[X]} om M2 också är \code{private[X]}, eller \code{private[Y]} om X innehåller Y.   
\item Om M1 är \code{protected} måste även M2 vara det.

\end{enumerate}
\end{Slide}


\begin{Slide}{Trait eller abstrakt klass?} 
\SlideFontSmall
\label{slideW07:traitorclass}
\begin{multicols}{2}
Använd en \Emph{trait} som supertyp om...
\begin{itemize}
\item ...du är osäker på vilket som är bäst. (Du kan alltid ändra till en abstrakt klass senare.)
\item ...du vill kunna mixa in din trait tillsammans med andra traits.
\item ...du bara har abstrakta medlemmar. 
\end{itemize}

\columnbreak

Använd en \Alert{abstract class} som supertyp om...
\begin{itemize}
\item ...du vill ge supertypen en parameter vid konstruktion.
\item ...du vill ärva supertypen från klasser skrivna i Java.
\item ...du vill minimera vad som behöver omkompileras vid ändringar. 
\end{itemize}


\end{multicols}
\end{Slide}



%\begin{Slide}{Designexempel: Klassen ???}\small
%TODO:
%  \begin{itemize} 
%  \item 
%  \end{itemize}
%\end{Slide}










